%\part{choco and visu}
\label{chocoandcpviz}
\hypertarget{chocoandcpviz}{}

\chapter{Choco and CP-Viz}\label{chocoandcpviz:chocoandcpviz}\hypertarget{chocoandcpviz:chocoandcpviz}{}

This chapter presents how to produce XML files for the \href{http://sourceforge.net/projects/cpviz/}{{CP-Viz}} system. CP-VIz is an open source visualization platform for CP. It provides multiple views to show the search tree, and the state of constraints and variables for a post- mortem analysis of a constraint program. It is system independent, uses XML format to exchange information between solvers and visualization tools.

\medskip
Find out more information about CP-Viz in  \cite{Simonis10}: \emph{A Generic Visualization Platform for CP} (\href{http://www.4c.ucc.ie/~hsimonis/cpviz-cp2010-paper.pdf}{{CP2010 paper}} and \href{http://www.4c.ucc.ie/~hsimonis/cpviz-cp2010-slides.pdf}{{presentation}}).

\medskip
Although visualization is really helpful, we do want the production of such files being easily unpluggable and not too much intrusive. That is why we have chosen \href{http://en.wikipedia.org/wiki/Aspect-oriented_programming}{{Aspect-oriented programming}}, more precisely the \href{http://www.eclipse.org/aspectj/}{{AspectJ}} library, and a small footprint logging system: \href{http://logback.qos.ch/}{{LOGBack}} to implement this fonctionnality in Choco. 

\section{Three step tutorial}\label{chocoandcpviz:3stepstuto}\hypertarget{chocoandcpviz:3stepstuto}{}

Producing CP-Viz traces in your Choco program can be done in three steps:

\begin{itemize}
\item first change the \mylst{choco-solver} JAR for the \mylst{choco-solver-cpviz} one in your classpath (it contains \mylst{choco-solver}, the tracer tools and the dependencies). This JAR file does not require any additionnal JARs in the classpath;
\item then, add the \mylst{Visualization} object and options and the optionnal \mylst{Visualizer} objects (see \hyperlink{chocoandcpviz:visualizationandvisualizers}{below});
\item and go to \mylst{your.cpviz.dir/viz/bin}, and run the following command:
\begin{lstlisting}
java ie.ucc.cccc.viz.Viz configuration.xml tree.xml visualization.xml
\end{lstlisting} 
\end{itemize}.

The two first steps produce XML files required by CP-Viz as parameter. The third step generates SVG and IDX files (see the CP-Viz documentation for more information). 

\begin{note}
Don't mix up :
\begin{itemize}
\item XML outputs, produced by Choco
\item and files generated by CP-Viz from XML outputs.
\end{itemize}
\end{note} 

\medskip


\section{Visualization and Visualizers}\label{chocoandcpviz:visualizationandvisualizers}\hypertarget{chocoandcpviz:visualizationandvisualizers}{}


\subsection{Visualization}\label{chocoandcpviz:visualization}\hypertarget{chocoandcpviz:visualization}
The \mylst{Visualization} object is mandatory to produce CP-Viz files. It creates, using the logging system, the 2 or 3 required files: \mylst{configuration-\{pbname\}.xml}, \mylst{tree-\{pbname\}.xml} and \mylst{visualization-\{pbname\}.xml}, where \mylst{\{pbname\}} is the problem name and can be specified at the \mylst{Visualization} creation. These files are created at the \underline{project root directory} (this can be changed by overriding \mylst{logback.xml} file, see \href{http://logback.qos.ch/}{LOGBack}). 

\medskip
\mylst{Visualization} has 2 constructors:
\begin{itemize}
\item \mylst{Visualization(String pbname, Solver solver, String dir)}: create a new instance of \mylst{Visualization}, where : 
	\begin{itemize}
	\item \mylst{pbname} is the name of the problem (and the suffix of the XML files), 
	\item \mylst{solver} is the \mylst{Solver} object used to solve the problem,
	\item \mylst{dir} is the path used by CP-Viz to generate SVG files and IDX file (see the CP-Viz documentation for more information). This must be an existing directory, otherwise CP-Viz can not create anything!
	\end{itemize}

\item \mylst{Visualization(Solver solver, String dir)}: identical to the previous one, except the \mylst{pbname} is generated using \mylst{new Random().nextLong()}.
\end{itemize}
and provide 6 services:
\begin{itemize}

\item \mylst{void createTree(String type, String display, String repeat, int width, int height)}: declare the tree search visualization, where :
	\begin{itemize}
	\item \mylst{type} should take its value in \mylst{\{"layout", "graph", "value"\}}
	\item \mylst{display} should take its value in \mylst{\{"compact", "expanded"\}}
	\item \mylst{repeat} should take its value in \mylst{\{"all", "final", "i", "-i"\}}
	\item \mylst{width} is the width of SVG canvas in screen pixels
	\item \mylst{height} is the height of SVG canvas in screen pixels. 
	\end{itemize}
These parameters are used by CP-Viz to create SVG files. They can be modified afterwards in the \mylst{configuration-\{pbname\}.xml} file. The generated file records informations about root node, try nodes, fail nodes and success nodes of the tree search. 
\textit{Constants can also be found in \mylst{CPVizConstant} class}.

\item \mylst{void createTree()}: default tree search declaration, with \mylst{type}$=$\mylst{"layout"}, \mylst{display}$=$\mylst{"compact"}, \mylst{repeat}$=$\mylst{"all"}, \mylst{width}$=500$, \mylst{height}$=500$.

\item \mylst{void createViz(String type, String display, String repeat, int width, int height)}: declare the constraint and variable visualizer container, where :
	\begin{itemize}
	\item \mylst{type} should take its value in \mylst{\{"layout"\}}
	\item \mylst{display} should take its value in \mylst{\{"compact", "expanded"\}}
	\item \mylst{repeat} should take its value in \mylst{\{"all", "final", "i", "-i"\}}
	\item \mylst{width} is the width of SVG canvas in screen pixels
	\item \mylst{height} is the height of SVG canvas in screen pixels. 
	\end{itemize}
These parameters are used by CP-Viz to create SVG files. They can be modified afterwards in the \mylst{configuration-\{pbname\}.xml} file.
\textit{Constants can also be found in \mylst{CPVizConstant} class}.

\item \mylst{void createViz()}: default constraint and variable visualizers container declaration, with \mylst{type}$=$\mylst{"layout"}, \mylst{display}$=$\mylst{"compact"}, \mylst{repeat}$=$\mylst{"all"}, \mylst{width}$=500$, \mylst{height}$=500$.

\item \mylst{void addVisualizer(Visualizer visualizer)}: add a \mylst{visualizer}  to the constraint and variable visualizers container. \textit{See the \hyperlink{chocoandcpviz:visualizers}{Visualizers} section for more details}.

\item \mylst{void close()}: safely close the XML files after the resolution --  \textit{required}.

\end{itemize}

\subsection{Visualizers}\label{chocoandcpviz:visualizers}\hypertarget{chocoandcpviz:visualizers}
A \mylst{Visualizer} is a specific object containing informations about the way some variables should be represented in CP-Viz. They describe the state of a set of variables all tree search long.  

\medskip
Each visalizer is a specific extension of the abstract \mylst{Visualizer} class. This class declares common parameters of visualizers and provides the following services:
\begin{itemize}
%\item \mylst{int getId()} : returns the identifier of the visualizer
\item \mylst{void setXY(int x, int y)} : sets the coordinates of the visualizer in the visualizer window (default is [0,0])
\item \mylst{void setGroup(String group)}: defines the group name of the visualizer
%\item \mylst{String getGroup()}: returns the group name of the visualizer
%\item \mylst{String getType()}: returns the type of the visualizer
%\item \mylst{String getDisplay()}: returns the display policy
%\item \mylst{int getWidth()}: returns the width of the visualizer in the visualizers window
%\item \mylst{int getHeight()}: returns the height of the visualizer in the visualizers window
\item \mylst{void setMinMax(int min, int max)}: sets the expected minimal and maximal value of any of the domains
\end{itemize}
 
 Although there are no real differences between them, we distinguish two types of visualizers: variable-oriented visualizers and constraint-oriented visualizers.
\medskip
\begin{notedef}\tt
\begin{itemize}
\item Variable-oriented visualizers : \hyperlink{binarymatrix:visu}{BinaryMatrix}, \hyperlink{binaryvector:visu}{BinaryVector}, \hyperlink{domainmatrix:visu}{DomainMatrix}, \hyperlink{vector:visu}{Vector}, \hyperlink{vectorsize:visu}{VectorSize}, \hyperlink{vectorwaterfall:visu}{VectorWaterfall},
\item Constraint-oriented visualizers : \hyperlink{alldiff:visu}{AllDifferent}, \hyperlink{alldiffmatrix:visu}{AllDifferentMatrix}, \hyperlink{binpacking:visu}{BinPacking}, \hyperlink{boolchan:visu}{BooleanChanneling}, \hyperlink{cumulative:visu}{Cumulative}, \hyperlink{element:visu}{Element}, \hyperlink{gcc:visu}{Gcc}, \hyperlink{inverse:visu}{Inverse}, \hyperlink{lexle:visu}{LexLe}, \hyperlink{lexlt:visu}{LexLt},    
\end{itemize}
\end{notedef}
Note that constraint-oriented visualizers are not based on any \mylst{Constraint} object but just \mylst{Var} object (\mylst{Solver} objects). This gives more freedom in visualizer declaration (for example, on reformulations).

\section{Example of code integration}\label{chocoandcpviz:codeintegration}\hypertarget{chocoandcpviz:codeintegration}{}
Let's see a visualization integration in the well-known \mylst{SendMoreMoney} problem:

\begin{lstlisting}[title=SendMoreMoney problem,captionpos=b]  
Model model;                                                                                                     
IntegerVariable S, E, N, D, M, O, R, Y;                                                                          
IntegerVariable[] SEND, MORE, MONEY;                                                                             
                                                                                                                 
model = new CPModel();                                                                                           
                                                                                                                 
S = makeIntVar("S", 0, 9);                                                                                       
E = makeIntVar("E", 0, 9);                                                                                       
N = makeIntVar("N", 0, 9);                                                                                       
D = makeIntVar("D", 0, 9);                                                                                       
M = makeIntVar("M", 0, 9);                                                                                       
O = makeIntVar("0", 0, 9);                                                                                       
R = makeIntVar("R", 0, 9);                                                                                       
Y = makeIntVar("Y", 0, 9);                                                                                       
SEND = new IntegerVariable[]{S, E, N, D};                                                                        
MORE = new IntegerVariable[]{M, O, R, E};                                                                        
MONEY = new IntegerVariable[]{M, O, N, E, Y};                                                                    
                                                                                                                 
model.addConstraints(neq(S, 0), neq(M, 0));                                                                      
model.addConstraint(allDifferent(S, E, N, D, M, O, R, Y));                                                       
model.addConstraints(                                                                                            
        eq(plus(scalar(new int[]{1000, 100, 10, 1}, SEND),                                                       
                scalar(new int[]{1000, 100, 10, 1}, MORE)),                                                      
                scalar(new int[]{10000, 1000, 100, 10, 1}, MONEY))                                               
);                                                                                                               
                                                                                                                 
                                                                                                                 
Solver solver = new CPSolver();                                                                                  
solver.read(model);                                                                                              
                                                                                                                 
//-------> Visualization declaration starts here <-------//                                                  
// create a new instance of Visualization                                                                        
Visualization visu = new Visualization("SendMoreMoney", solver, "path/to/out");                                   
                                                                                                                 
visu.createTree();  // declare tree search visualization                                                         
visu.createViz();   // declare visualizers container                                                             
                                                                                                                 
// create a new Vector visualizer                                                                                
Vector visualizer = new Vector(solver.getVar(S, E, N, D, M, O, R, Y), "expanded", 0, 0, 8, 10, "SENDMORY", 0, 9);
// add the vector to the visualizers container                                                                   
visu.addVisualizer(visualizer);                                                                                  
                                                                                                                 
// launch the resolution of the problem                                                                          
solver.solve();                                                                                                  
                                                                                                                 
// close the visualization                                                                                       
visu.close();                                                                                                    
//-------> Visualization declaration ends here <-------//                                                    
\end{lstlisting}

\medskip
The execution of this program results in 3 files:
\begin{lstlisting}[title=configuration-SendMoreMoney.xml, captionpos=b]
<configuration version="1.0" directory="/path/to/out" >
	<tool show="tree" type="layout" display="compact" repeat="all" width="500" height="500" fileroot="tree-SendMoreMoney" />
	<tool show="viz" type="layout" display="compact" repeat="all" width="500" height="500" fileroot="visualization-SendMoreMoney" />
</configuration>
\end{lstlisting}

\begin{lstlisting}[title=tree-SendMoreMoney.xml, captionpos=b]
<?xml version="1.0" encoding="UTF-8"?>
<!-- choco-solver -->
<tree version="1.0" >
	<root id="0" />
	<fail id="1" parent="0" name="E" size="4" value="4" />
	<try id="2" parent="0" name="E" size="1" value="5" />
	<succ id="2" />
</tree>
\end{lstlisting}

\begin{lstlisting}[title=visualization-SendMoreMoney.xml,captionpos=b]
<?xml version="1.0" encoding="UTF-8"?>
<!-- choco-solver -->
<visualization version="1.0" >
	<visualizer id="1" type="vector" display="expanded" width="8" height="10"  x="0" y="0" group="SENDMORY" min="0" max="9"/>
	<state id="1" tree_node="-1" >
		<visualizer_state id="1" >
			<dvar index="1" domain="0 1 2 3 4 5 6 7 8 9 " />
			<dvar index="2" domain="0 1 2 3 4 5 6 7 8 9 " />
			<dvar index="3" domain="0 1 2 3 4 5 6 7 8 9 " />
			<dvar index="4" domain="0 1 2 3 4 5 6 7 8 9 " />
			<dvar index="5" domain="0 1 2 3 4 5 6 7 8 9 " />
			<dvar index="6" domain="0 1 2 3 4 5 6 7 8 9 " />
			<dvar index="7" domain="0 1 2 3 4 5 6 7 8 9 " />
			<dvar index="8" domain="0 1 2 3 4 5 6 7 8 9 " />
		</visualizer_state>
	</state>
	<state id="2" tree_node="0" >
		<visualizer_state id="1" >
			<integer index="1" value="9" />
			<dvar index="2" domain="4 5 6 7 " />
			<dvar index="3" domain="5 6 7 8 " />
			<dvar index="4" domain="2 3 4 5 6 7 8 " />
			<integer index="5" value="1" />
			<integer index="6" value="0" />
			<dvar index="7" domain="2 3 4 5 6 7 8 " />
			<dvar index="8" domain="2 3 4 5 6 7 8 " />
		</visualizer_state>
	</state>
	<state id="3" tree_node="1" >
		<visualizer_state id="1" >
			<integer index="1" value="9" />
			<dvar index="2" domain="4 5 6 7 " />
			<dvar index="3" domain="5 6 7 8 " />
			<dvar index="4" domain="2 3 4 5 6 7 8 " />
			<integer index="5" value="1" />
			<integer index="6" value="0" />
			<dvar index="7" domain="2 3 4 5 6 7 8 " />
			<dvar index="8" domain="2 3 4 5 6 7 8 " />
			<failed index="2" group="SENDMORY" value="4"/>
		</visualizer_state>
	</state>
	<state id="4" tree_node="2" >
		<visualizer_state id="1" >
			<integer index="1" value="9" />
			<integer index="2" value="5" />
			<integer index="3" value="6" />
			<integer index="4" value="7" />
			<integer index="5" value="1" />
			<integer index="6" value="0" />
			<integer index="7" value="8" />
			<integer index="8" value="2" />
			<focus index="2" group="SENDMORY" type="vector"/>
		</visualizer_state>
	</state>
</visualization>
\end{lstlisting}

With these 3 XML files, CP-Viz can produce the following outputs:

\begin{figure}[!htp]
	\centerline{\Graph{media/tree-SendMoreMoney2.pdf}{width=0.5\linewidth}}
	\caption[]{Example of search tree output for SendMoreMoney resolution}\label{fig:media/tree-SendMoreMoney2.pdf}
\end{figure}

\begin{figure}[!htp]
	\centerline{\Graph{media/visualization-SendMoreMoney4.pdf}{width=0.5\linewidth}}
	\caption[]{Example of visualization output for SendMoreMoney resolution}\label{fig:media/visualization-SendMoreMoney4.pdf}
\end{figure}
