%!TEX root = ../../content-doc.tex
%\part{inversechanneling}
\label{inversechannelingwithinrange}
\hypertarget{inversechannelingwithinrange}{}

\section{inverseChannelingWithinRange (constraint)}\label{inversechannelingwithinrange:inversechannelingconstraintwithinrange}\hypertarget{inversechannelingwithinrange:inversechannelingconstraintwithinrange}{}
\begin{notedef}
  \texttt{inverseChannelingWithinRange}$(\collec{x_1}{x_n},\collec{y_1}{y_m})$ states that
%a channeling between two arrays  $x$ and $y$ of integer variables with the same domain.It enforces 
if $x_i$ is assigned to $j$ and if $j$ is less than or equal to the number of items of the collection Y then $y_j$ is assigned to $i$.
Conversely, if $y_j$ is assigned to $i$ and if $i$ is less than or equal to the number of items of the collection X then $x_i$ is assigned to $j$.

$$x_i = j \wedge j < |y| \quad\iff\quad y_j = i \wedge i < |x|,\qquad i \in[1,n],  j\in[1,m]$$
\end{notedef}
\begin{itemize}
	\item \textbf{API} : \mylst{inverseChanneling(IntegerVariable[] x, IntegerVariable[] y)}
	\item \textbf{return type} : \texttt{Constraint}
	\item \textbf{options} : \emph{no options}
	\item \textbf{favorite domain} : enumerated for x and y
	\item \textbf{references} :\\
      global constraint catalog: \href{http://www.emn.fr/z-info/sdemasse/gccat/Cinverse_within_range.html}{inverse\_within\_range}
\end{itemize}

\textbf{Example}:
\lstinputlisting{java/cinversechannelingwithinrange.j2t}
