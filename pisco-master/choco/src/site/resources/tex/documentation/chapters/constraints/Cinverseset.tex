%\part{inversechanneling}
\label{inverseset}
\hypertarget{inverseset}{}

\section{inverseSet (constraint)}\label{inverseset:inversesetconstraint}\hypertarget{inverseset:inversesetconstraint}{}
\begin{notedef}
  \texttt{inverseSet}$(\collec{x_1}{x_n},\collec{y_1}{y_m})$ states
that $x_i$ has (if an integer variable or contains, if a set variable) value $j$ if and only if $y_j$ contains value $i$:

If $x$ is a collection of integer variables:
$$x_i = j\quad\iff\quad i\in s_j,\qquad\forall i=0..n-1,j=0..m-1$$
Notice that this version induces that $y$ becomes a partition of the set of the indices of collection $x$.


If $x$ is a collection of set variables:
$$j \in x_i \quad\iff\quad i\in s_j,\qquad\forall i=0..n-1,j=0..m-1$$




\end{notedef}
\begin{itemize}
	\item \textbf{API} :
	\begin{itemize}
		\item \mylst{inverseSet(IntegerVariable[] x, SetVariable[] y)}
		\item \mylst{inverseSet(SetVariable[] x, SetVariable[] y)}
	\end{itemize}
	\item \textbf{return type} : \texttt{Constraint}
	\item \textbf{options} : \emph{no options}
	\item \textbf{favorite domain} : enumerated for x
	\item \textbf{references} :\\
      global constraint catalog: \href{http://www.emn.fr/x-info/sdemasse/gccat/Cinverse_set.html}{inverse\_set}
\end{itemize}

\textbf{Example}:
\lstinputlisting{java/cinverseset.j2t}
