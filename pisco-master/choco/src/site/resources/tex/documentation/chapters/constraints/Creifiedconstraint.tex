\section{reifiedConstraint (constraint)}\label{reifiedconstraint:reifiedconstraintconstraint}\hypertarget{reifiedconstraint:reifiedconstraintconstraint}{}
\begin{notedef}
  \begin{itemize}
  \item \texttt{reifiedConstraint}$(b,C)$ states that boolean $b$ is true if and only if constraint $C$ holds:
  $$(b=1)\ \iff\ C$$
  \item \texttt{reifiedConstraint}$(b,C_1,C_2)$ states that boolean $b$ is true if and only if $C_1$ holds, and $b$ is false if and only if $C_2$ holds ($C_2$ must be the negation of constraint of $C_1$):
$$(b\land C_1) \lor (\neg b \land C_2)$$
  \end{itemize}
\end{notedef}

\begin{itemize}
	\item \textbf{API} :
	\begin{itemize}
		\item \mylst{reifiedConstraint(IntegerVariable b, Constraint c)}
		\item \mylst{reifiedConstraint(IntegerVariable b, Constraint c1, Constraint c2)}
	\end{itemize}
	\item \textbf{return type} : \texttt{Constraint}
	\item \textbf{options} : \emph{n/a}
	\item \textbf{favorite domain} : \emph{n/a}
\end{itemize}

The constraint $C$ to reify has to provide its negation $\neg C$ (the negation is needed for propagation). 
Most basic constraints of Choco provides their negation by default, and can then be reified using the first API.
The second API attends to reify user-defined constraints as it allows the user to directly specify the negation constraint. 

The {\tt  reifiedConstraint} filter algorithm:
\begin{enumerate}
	\item if $b$ is instantiated to 1 (resp. to 0), then $C$ (resp. $\neg C$) is propagated
	\item otherwise
	\begin{enumerate}
		\item if $C$ is entailed, $b$ is set to 1
		\item else if $C$ is failed, $b$ is set to 0.
	\end{enumerate}
\end{enumerate}


\textbf{Example}:
\lstinputlisting{java/creifiedintconstraint.j2t}

