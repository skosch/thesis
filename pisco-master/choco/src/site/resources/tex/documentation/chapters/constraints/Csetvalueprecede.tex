%\part{setvalueprecede}
\label{setvalueprecede}
\hypertarget{setvalueprecede}{}

\section{setValuePrecede (constraint)}\label{setvalueprecede:setvalueprecedeconstraint}\hypertarget{setvalueprecede:setvalueprecedeconstraint}{}
\begin{notedef}
  \texttt{setValuePrecede}$(sv,s,t)$ states that if there exists a set variable $v_1 \in sv$ such that $s$ does not belong to $v$ and $t$ does,
  then there also exists a set variable $v_2 \in sv$ preceding $v_1$ in sequence $sv$ such that $s$ belongs to $v_2$ and $t$ does not. In other words,
  this constraint maintains a value precedence relation between $s$ and $t$ within the sequence of set variables $sv$.


  A useful application of this notion is in breaking symmetries of indistinguishable values, an important class of symmetries in practice.

\end{notedef}

\begin{itemize}
	\item \textbf{API} :
	\begin{itemize}
		\item \mylst{setValuePrecede(SetVariable[] sv, int s, int t)}
	\end{itemize}
	\item \textbf{return type} : \texttt{Constraint}
	\item \textbf{options} :\emph{n/a}
	\item \textbf{favorite domain} : \emph{to complete}
\end{itemize}

\textbf{Example}:
\lstinputlisting{java/csetvalueprecede.j2t}
