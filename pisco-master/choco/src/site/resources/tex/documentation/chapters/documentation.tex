\part{Documentation}\label{ch:doc}\hypertarget{ch:doc}{}

The documentation of Choco is organized as follows:
\begin{itemize}
\item 
The concise \hyperlink{doc:introduction}{introduction} provides some informations \hyperlink{introduction:aboutconstraintprogramming}{about constraint programming} concepts and a ``Hello world''-like \hyperlink{introduction:myfirstchocoprogram}{first Choco program}.
\item 
The \hyperlink{doc:model}{model} section gives informations on \hyperlink{doc:model}{how to create a model} and introduces \hyperlink{model:variables}{variables} and \hyperlink{model:constraints}{constraints}.
\item 
The \hyperlink{doc:solver}{solver} section gives informations on \hyperlink{doc:solver}{how to create a solver}, to \hyperlink{doc:solver}{read a model}, to define a \hyperlink{solver:searchstrategy}{search strategy}, and finally to \hyperlink{solver:solveaproblem}{solve a problem}.
\item 
The \hyperlink{doc:advanced}{advanced use} section explains how to define your own \hyperlink{advanced:defineyourownlimitsearchspace}{limit search space}, \hyperlink{advanced:defineyourownsearchstrategy}{search strategy}, \hyperlink{advanced:defineyourownconstraint}{constraint}, \hyperlink{advanced:defineyourownoperator}{operator}, \hyperlink{advanced:defineyourownvariable}{variable}, \hyperlink{advanced:backtrackablestructures}{backtrackable structure} and write \hyperlink{advanced:howtowriteloggingstatements}{logging statements}.
%The \hyperlink{doc:applications}{applications} section shows the use of Choco defined global constraints on \hyperlink{schedulinganduseofthecumulative:schedulinganduseofthecumulativeconstraint}{scheduling} or \hyperlink{geostdescription:placementanduseofthegeostconstraint}{placement} problems.
%\item 
%Lastly, the catalog of Choco defined \hyperlink{ch:constraints}{constraints} is presented.
\end{itemize}

%\section*{Beta}\label{documentation:beta}\hypertarget{documentation:beta}{}
%Here you can find short documentation concerning futur works, only available on the beta version or extension of the current jar:
%\begin{itemize}
%%	\item \hyperlink{chocoandgraphviz}{Choco and Graphviz} \emph{not yet available. Has been included in \hyperlink{chocoandvisu}{Choco and Visu}.}
%	\item \hyperlink{chocoandvisu}{Choco and Visu}
%\end{itemize}

%\section*{Material}\label{documentation:material}\hypertarget{documentation:material}{}
%Here you can find some materials at your disposal. If you create any, you can also send it to us, so we can add it to the list.

%\subsection{Presentations}\label{documentation:presentations}\hypertarget{documentation:presentations}{}
%\begin{itemize}
%	\item \textbf{August 2008} : The \emph{CHOCO : an Open Source Java Constraint Constraint Programming} white paper presentation send to the \href{http://www.cril.univ-artois.fr/cpai08/}{CSP'08 competition}. \href{media/pdf/choco-presentation.pdf}{PDF} of white paper presentation of Choco
%	\item \textbf{june 2008} : \emph{The CHOCO constraint programming solver} presentation held at the \href{https://projects.coin-or.org/events/wiki/cpaior2008}{Workshop on Open-Source Software for Integer and Constraint Programming} during the last \href{http://contraintes.inria.fr/cpaior08/}{CPAIOR} conference. \href{media/slides/cpaior-choco.pdf}{PDF slides} of the presentation given by Guillaume Rochart
%\end{itemize}

%!TEX root = ../content-doc.tex
\chapter{Introduction to constraint programming and Choco}\label{doc:introduction}\hypertarget{doc:introduction}{}

\section{About constraint programming}\label{introduction:aboutconstraintprogramming}\hypertarget{introduction:aboutconstraintprogramming}{}

\begin{myquote}
Constraint programming represents one of the closest approaches computer science has yet made to the Holy Grail of programming: the user states the problem, the computer solves it.
\begin{flushright}\bf E. C. Freuder, Constraints, 1997.\end{flushright}
\end{myquote}


Fast increasing computing power in the 1960s led to a wealth of works around problem solving, at the root of Operational Research, Numerical Analysis, Symbolic Computing, Scientific Computing, and a large part of Artificial Intelligence and programming languages. Constraint Programming is a discipline that gathers, interbreeds, and unifies ideas shared by all these domains to tackle decision support problems.

Constraint programming has been successfully applied in numerous domains. Recent applications include computer graphics (to express geometric coherence in the case of scene analysis), natural language processing (construction of efficient parsers), database systems (to ensure and/or restore consistency of the data), operations research problems (scheduling, routing), molecular biology (DNA sequencing), business applications (option trading), electrical engineering (to locate faults), circuit design (to compute layouts), etc.

Current research in this area deals with various fundamental issues, with implementation aspects and with new applications of constraint programming.

\subsection{Constraints}\label{introduction:constraints}\hypertarget{introduction:constraints}{}
A constraint is simply a logical relation among several unknowns (or variables), each taking a value in a given domain. A constraint thus restricts the possible values that variables can take, it represents some partial information about the variables of interest. For instance, the circle is inside the square relates two objects without precisely specifying their positions, i.e., their coordinates. Now, one may move the square or the circle and one is still able to maintain the relation between these two objects. Also, one may want to add another object, say a triangle, and to introduce another constraint, say the square is to the left of the triangle. From the user (human) point of view, everything remains absolutely transparent.

Constraints naturally meet several interesting properties:
\begin{itemize}
	\item constraints may specify partial information, i.e. constraint need not uniquely specify the values of its variables,
	\item constraints are non-directional, typically a constraint on (say) two variables $X, Y$ can be used to infer a constraint on $X$ given a constraint on $Y$ and vice versa,
	\item constraints are declarative, i.e. they specify what relationship must hold without specifying a computational procedure to enforce that relationship,
	\item constraints are additive, i.e. the order of imposition of constraints does not matter, all that matters at the end is that the conjunction of constraints is in effect,
	\item constraints are rarely independent, typically constraints in the constraint store share variables.
\end{itemize}

Constraints arise naturally in most areas of human endeavor. The three angles of a triangle sum to 180 degrees, the sum of the currents flowing into a node must equal zero, the position of the scroller in the window scrollbar must reflect the visible part of the underlying document, these are some examples of constraints which appear in the real world. Thus, constraints are a natural medium for people to express problems in many fields. 

\subsection{Constraint Programming}\label{introduction:constraintprogramming}\hypertarget{introduction:constraintprogramming}{}
Constraint programming is the study of computational systems based on constraints. The idea of constraint programming is to solve problems by stating constraints (conditions, properties) which must be satisfied by the solution.

Work in this area can be tracked back to research in Artificial Intelligence and Computer Graphics in the sixties and seventies. Only in the last decade, however, has there emerged a growing realization that these ideas provide the basis for a powerful approach to programming, modeling and problem solving and that different efforts to exploit these ideas can be unified under a common conceptual and practical framework, constraint programming. 

\begin{note}
If you know \textbf{sudoku}, then you know \textbf{constraint programming}. See why \hyperlink{sudokuandcp}{here}.
\end{note}


\section{Modeling with Constraint programming}\label{introduction:modelingwithconstraintprogramming}\hypertarget{introduction:modelingwithconstraintprogramming}{}
The formulation and the resolution of combinatorial problems are the two main goals of the constraint programming domain. This is an essential way to solve many interesting industrial problems such as scheduling, planning or design of timetables. 

\subsection{The Constraint Satisfaction Problem}\label{introduction:csp}\hypertarget{introduction:csp}{}

Constraint programming allows to solve combinatorial problems modeled by a Constraint Satisfaction Problem (CSP). Formally, a CSP is defined by a triplet $(X,D,C)$:
\begin{itemize}
	\item \textbf{Variables}: $X = \{X_1,X_2,\ldots,X_n\}$ is the set of variables of the problem.
	\item \textbf{Domains}: $D$ is a function which associates to each variable $X_i$ its domain $D(X_i)$, i.e. the set of possible values that can be assigned to $X_i$. The domain of a variable is usually a finite set of integers: $D(X_i)\subset\Z$ (\emph{integer variable}). But a domain can also be continuous ($D(X_i)\subseteq\R$ for a \emph{real variable}) or made of discrete set values ($D(X_i)\subseteq\mathcal{P}(\Z)$ for a \emph{set variable}).
	\item \textbf{Constraints}: $C = \{C_1,C_2,\ldots,C_m\}$ is the set of constraints. A constraint $C_j$ is a relation defined on a subset $X^j = \{X^j_1,X^j_2,\ldots,X^j_{n^j}\}\subseteq X$ of variables which restricts the possible tuples of values $(v_1,\ldots,v_{n^j})$ for these variables:
$$(v_1,\ldots,v_{n^j})\in C_j\cap (D(X^j_1)\times D(X^j_2)\times\cdots\times D(X^j_{n^j})).$$
Such a relation can be defined explicitely (ex: $(X_1,X_2)\in\{(0,1),(1,0)\}$) or implicitely (ex: $X_1+X_2\le 1$).
\end{itemize}

Solving a CSP consists in finding a tuple $v=(v_1,\ldots,v_{n})\in D(X)$ on the set of variables which satisfies all the constraints:
$$(v_1,\ldots,v_{n^j})\in C_j,\quad\forall j\in\{1,\ldots,m\}.$$

For optimization problems, one needs to define an \textbf{objective function} $f:D(X)\rightarrow\R$. An optimal solution is then a solution tuple of the CSP that minimizes (or maximizes) function $f$.

\subsection{Examples of CSP models}\label{introduction:examples}\hypertarget{introduction:examples}{}
This part provides three examples using different types of variables in different problems. These examples are used throughout this tutorial to illustrate their modeling with Choco.

\subsubsection{Example 1: the n-queens problem.}\label{introduction:example1:nqueens}\hypertarget{introduction:example1:nqueens}{}
Let us consider a chess board with $n$ rows and $n$ columns. A queen can move as far as she pleases, horizontally, vertically, or diagonally. The standard $n$-queens problem asks how to place $n$ queens on an $n$-ary chess board so that none of them can hit any other in one move.

The $n$-queens problem can be modeled by a CSP in the following way:
\begin{itemize}
	\item \textbf{Variables}: $X = \{X_{i}\ |\ i\in [1,n]\}$; one variable represents a column and the constraint "queens must be on different columns" is induced by the variables.
	\item \textbf{Domain}: for all variable $X_{i}\in X$, $D(X_{i}) = \{1,2,\ldots, n\}$.
	\item \textbf{Constraints}: the set of constraints is defined by the union of the three following constraints,
	\begin{itemize}
		\item queens have to be on distinct lines:
		\begin{itemize}
			\item $C_{lines} = \{X_{i}\neq X_{j}\ |\ i,j\in [1,n], i\neq j\}$.
		\end{itemize}
		\item queens have to be on distinct diagonals:
		\begin{itemize}
			\item $C_{diag1} = \{X_{i}\neq X_{j}+j-i\ |\ i,j\in [1,n], i\neq j\}$.
			\item $C_{diag2} = \{X_{i}\neq X_{j}+i-j\ |\ i,j\in [1,n], i\neq j\}$.
		\end{itemize}
	\end{itemize}
\end{itemize}

\subsubsection{Example 2: the ternary Steiner problem.}\label{introduction:example2:theternarysteinerproblem}\hypertarget{introduction:example2:theternarysteinerproblem}{}
A ternary Steiner system of order $n$ is a set of $n*(n-1)/6$ triplets of distinct elements taking their values in $[1,n]$, such that all the pairs included in two distinct triplets are different.
See \url{http://en.wikipedia.org/wiki/Steiner_system} for details. 

The ternary Steiner problem can be modeled by a CSP in the following way:
\begin{itemize}
	\item let $t = n*(n-1)/6$.
	\item \textbf{Variables}: $X = \{X_{i}\ |\ i\in [1,t]\}$.
	\item \textbf{Domain}: for all $i\in [1,t]$, $D(X_{i}) = \{1,...,n\}$.
	\item \textbf{Constraints}:
	\begin{itemize}
		\item every set variable $X_i$ has a cardinality of 3:
		\begin{itemize}
			\item for all $i\in [1,t]$, $|X_{i}| = 3$.
		\end{itemize}
		\item the cardinality of the intersection of every two distinct sets must not exceed 1:
		\begin{itemize}
			\item for all $i,j\in [1,t]$, $i\neq j$, $|X_{i}\cap X_{j}|\le 1$.
		\end{itemize}
	\end{itemize}
\end{itemize}

\subsubsection{Example 3: the CycloHexane problem.}\label{introduction:example3:thecyclohexaneproblem}\hypertarget{introduction:example3:thecyclohexaneproblem}{}
The problem consists in finding the 3D configuration of a cyclohexane molecule. It is described with a system of three non linear equations:
\begin{itemize}
	\item \textbf{Variables}: $x,y,z$.
	\item \textbf{Domain}: $]-\infty;+\infty[$.
	\item \textbf{Constraints}:
	\begin{align*}
		y^{2} * (1 + z^{2}) + z * (z - 24 * y) &= -13\\
		x^{2} * (1 + y^{2}) + y * (y - 24 * x) &= -13\\
		z^{2} * (1 + x^{2}) + x * (x - 24 * z) &= -13
	\end{align*}
\end{itemize}

\section{My first Choco program: the magic square}\label{introduction:myfirstchocoprogram}\hypertarget{introduction:myfirstchocoprogram}{}

\subsection{The magic square problem}\label{introduction:amagicsquareproblem}\hypertarget{introduction:amagicsquareproblem}{}
In the following, we will address the magic square problem of order 3 to illustrate step-by-step how to model and solve this problem using choco. 

\subsubsection{Definition:}
A magic square of order $n$ is an arrangement of $n^{2}$ numbers, usually distinct integers, in a square, such that the $n$ numbers in all rows, all columns, and both diagonals sum to the same constant. A standard magic square contains the integers from 1 to $n^{2}$.

The constant sum in every row, column and diagonal is called the magic constant or magic sum $M$. The magic constant of a classic magic square depends only on $n$ and has the value:
$M(n)=n(n^2 +1)/2$.

\href{http://en.wikipedia.org/wiki/Magic_square}{More details on the magic square problem.}


\subsection{A mathematical model}\label{introduction:mathematicalmodeling}\hypertarget{introduction:mathematicalmodeling}{}

Let $x_{ij}$ be the variable indicating the value of the $j^{th}$ cell of row $i$. 
Let $C$ be the set of constraints modeling the magic square as:
\begin{align*}
&x_{ij} \in [1,n^2],\ &&\forall i,j \in [1, n]\\
&x_{ij}\ne x_{kl},\ &&\forall i,j,k,l \in [1,n], i\ne k, j\ne l\\
&\sum_{j=1}^{n} x_{ij} = n(n^2 +1)/2,\ &&\forall i \in [1,n]\\
&\sum_{i=1}^{n} x_{ij} = n(n^2 +1)/2,\ &&\forall j \in [1,n]\\
&\sum_{i=1}^{n} x_{ii} = n(n^2 +1)/2&&\\
&\sum_{i=n}^{1} x_{i(n-i)} = n(n^2 +1)/2&&\\
\end{align*}

We have all the required information to model the problem with Choco.
\begin{note}
	For the moment, we just talk about \emph{model translation} from a mathematical representation to Choco.
	Choco can be used as a \emph{black box}, that means we just need to define the problem without knowing the way it will be solved. We can therefore focus on the modeling not on the solving.
\end{note}

\subsection{To Choco...}\label{introduction:inchoco}\hypertarget{introduction:inchoco}{}

First, we need to know some of the basic Choco objects:
\begin{itemize}
\item 
The \textbf{model} (object \texttt{Model} in Choco) is one of the central elements of a Choco program. Variables and constraints are associated to it.
\item
The \textbf{variables} (objects \texttt{IntegerVariable}, \texttt{SetVariable}, and \texttt{RealVariable} in Choco) are the \emph{unknowns} of the problem. Values of variables are taken from a \textbf{domain} which is defined by a set of values or quite often simply by a lower bound and an upper bound of the allowed values. The domain is given when creating the variable.
\begin{note}
Do not forget that we manipulate \textbf{variables} in the mathematical sense (as opposed to classical computer science). Their effective value will be known only once the problem has been solved.
\end{note}
\item
The \textbf{constraints} define relations to be satisfied between variables and constants.
In our first model, we only use the following constraints provided by Choco:
\begin{itemize}
	\item \texttt{eq(var1, var2)} which ensures that \texttt{var1} equals \texttt{var2}.
	\item \texttt{neq(var1, var2)} which ensures that \texttt{var1} is not equal to \texttt{var2}.
	\item \texttt{sum(var[])} which returns expression \texttt{var[0]+var[1]+...+var[n]}.
\end{itemize}
\end{itemize}

\subsection{The program}\label{introduction:theprogram}\hypertarget{introduction:theprogram}{}
After having created your java class file, import the Choco class to use the API:
\begin{lstlisting}
  import choco.Choco;
\end{lstlisting}
First of all, let's create a Model:
\lstinputlisting{java/imagicsquare1.j2t}
We create an instance of \texttt{CPModel()} for \textbf{C}onstraint \textbf{P}rogramming Model.
Do not forget to add the following imports:
\begin{lstlisting}
  import choco.cp.model.CPModel;
\end{lstlisting}
Then we declare the variables of the problem:
\lstinputlisting{java/imagicsquare2.j2t}
Add the import:
\begin{lstlisting}
  import choco.kernel.model.variables.integer.IntegerVariable;
\end{lstlisting}
We have defined the variable using the \texttt{makeIntVar} method which creates an enumerated domain: all the values are stored in the java object (beware, it is usually not necessary to store all the values and it is less efficient than to create a bounded domain).

\noindent Now, we are going to state a constraint ensuring that all variables must have a different value:
\lstinputlisting{java/imagicsquare3.j2t}
Add the import:
\begin{lstlisting}
  import choco.kernel.model.constraints.Constraint;
\end{lstlisting}
Then, we add the constraint ensuring that the magic sum is respected:
\lstinputlisting{java/imagicsquare4.j2t}
Then we define the constraint ensuring that each column is equal to the magic sum.
Actually, \texttt{var} just denotes the rows of the square. So we have to declare a temporary array of variables that defines the columns.
\lstinputlisting{java/imagicsquare5.j2t}
It is sometimes useful to define some temporary variables to keep the model simple or to reorder array of variables. That is why we also define two other temporary arrays for diagonals.
\lstinputlisting{java/imagicsquare6.j2t}
Now, we have defined the model. The next step is to solve it.
For that, we build a Solver object:
\lstinputlisting{java/imagicsquare7.j2t}

with the imports:
\begin{lstlisting}
  import choco.cp.solver.CPSolver;
\end{lstlisting}
We create an instance of \texttt{CPSolver()} for Constraint Programming Solver.
Then, the solver reads (translates) the model and solves it:
\lstinputlisting{java/imagicsquare8.j2t}
The only variables that need to be printed are the ones in \texttt{var} (all the others are only references to these ones). 
\begin{note}
We have to use the Solver to get the value of each variable of the model. The Model only declares the objects, the Solver finds their value.
\end{note}
We are done, we have created our first Choco program. 
%The complete source code can be found here: \href{media/zip/exmagicsquare.zip}{ExMagicSquare.zip}


\subsection{In summary}\label{introduction:whatisimportant}\hypertarget{introduction:whatisimportant}{}
\begin{itemize}
	\item A Choco Model is defined by a set of Variables with a given domain and a set of Constraints that link Variables:
it is necessary to add both Variables and Constraints to the Model.
	\item temporary Variables are useful to keep the Model readable, or necessary when reordering arrays.
	\item The value of a Variable can be known only once the Solver has found a solution.
	\item To keep the code readable, you can avoid the calls to the static methods of the Choco classes, by importing the static classes, i.e. instead of:
\begin{lstlisting}
  import choco.Choco;
  ...
  IntegerVariable v = Choco.makeIntVar("v", 1, 10);
  ...
  Constraint c = Choco.eq(v, 5);
\end{lstlisting}
you can use:
\begin{lstlisting}
  import static choco.Choco.*;
  ...
  IntegerVariable v = makeIntVar("v", 1, 10);
  ...
  Constraint c = eq(v, 5);
\end{lstlisting}
\end{itemize}

\section{Complete examples}\label{model:completeexamples}\hypertarget{model:completeexamples}{}
We provide now the complete Choco model for the three examples \hyperlink{introduction:examples}{previously described}.

\subsection{Example 1: the n-queens problem with Choco}\label{model:example1:nqueenschoco}\hypertarget{model:example1:nqueenschoco}{}
This first model for the \hyperlink{introduction:example1:nqueens}{n-queens problem} only involves binary constraints of differences between integer variables. One can immediately recognize the 4 main elements of any Choco code. First of all, create the model object. Then create the variables by using the Choco API (One variable per queen giving the row (or the column) where the queen will be placed). Finally, add the constraints and solve the problem. 

\lstinputlisting{java/inqueen.j2t}

\subsection{Example 2: the ternary Steiner problem with Choco}\label{model:example2:ternarysteinerchoco}\hypertarget{model:example2:ternarysteinerchoco}{}
The \hyperlink{introduction:example2:theternarysteinerproblem}{ternary Steiner problem} is entirely modeled using set variables and set constraints. 
\lstinputlisting{java/iternarysteiner.j2t}

\subsection{Example 3: the CycloHexane problem with Choco}\label{model:example3:thecyclohexaneproblemwithchoco}\hypertarget{model:example3:thecyclohexaneproblemwithchoco}{}
Real variables are illustrated on the problem of finding the 3D configuration of a cyclohexane molecule. 
\lstinputlisting{java/icyclohexane.j2t}



%!TEX root = ../content-doc.tex
\chapter{The model}\label{doc:model}\hypertarget{doc:model}{}

The {\tt Model}, along with the {\tt Solver}, is one of the two key elements of any Choco program. The Choco {\tt Model} allows to describe a problem in an easy and declarative way: it simply records the variables and the constraints defining the problem.

This section describes the large API provided by Choco to create different types of \hyperlink{model:variables}{variables} and \hyperlink{model:constraints}{constraints}.

%\begin{note}
\textbf{Note that a static import is required to use the Choco API:}
\begin{lstlisting}
  import static choco.Choco.*;
\end{lstlisting}
%It is mandatory in order to compile !
%\end{note}

%\section{How to create a model}\label{model:howtocreateamodel}\hypertarget{model:howtocreateamodel}{}
First of all, a {\tt Model} object is created as follows:
\begin{lstlisting}
Model model = new CPModel();
\end{lstlisting}
In that specific case, a Constraint Programming (CP) {\tt Model} object has been created. 


%%%%%%%%%%%%%%%%%%%%%%%%%%%%%%%%%%%%%%%%%%%%%%%%%%%%%%%%%%%%%%%%%%%%%%%%%%%%%%%%%%%%%%%%%%%%%%%%%%%%%%%%%%%%%%%%%%%%%%%%%%%%%%%%%%%%%%%%%%%%%%%%%%%
%%%%%%%%%%%%%%%%%%%%%%%%%%%%%%%%%%%%%%%%%%%%%%%%%%%%% VARIABLE %%%%%%%%%%%%%%%%%%%%%%%%%%%%%%%%%%%%%%%%%%%%%%%%%%%%%%%%%%%%%%%%%%%%%%%%%%%%%%%%%%%%
%%%%%%%%%%%%%%%%%%%%%%%%%%%%%%%%%%%%%%%%%%%%%%%%%%%%%%%%%%%%%%%%%%%%%%%%%%%%%%%%%%%%%%%%%%%%%%%%%%%%%%%%%%%%%%%%%%%%%%%%%%%%%%%%%%%%%%%%%%%%%%%%%%%


\section{Variables}\label{model:variables}\hypertarget{model:variables}{}

%Choco provides a large API to create different types of variables : \textbf{integer}, \textbf{real} and \textbf{set}. 

A Variable is defined by a type (\hyperlink{integervariable}{integer}, \hyperlink{realvariable}{real}, or \hyperlink{setvariable}{set} variable), a name, and the values of its domain. When creating a simple variable, some options can be set to specify its domain representation (\eg enumerated or bounded) within the {\tt Solver}.
%Some kinds of variables have options for their domain, it may have an effect on what kind of specific object is created when the model is read by the solver.
\begin{note}
The choice of the domain should be considered. The efficiency of the solver often depends on judicious choice of the domain type.
\end{note}
Variables can be combined as \hyperlink{model:expressionvariables}{expression variables} using operators.

One or more variables can be added to the model using the following methods of the \texttt{Model} class:
\lstinputlisting{java/mvariabledeclaration1.j2t}

\begin{note}
Explictly addition of variables is not mandatory. See \hyperlink{model:constraints}{\tt Constraint} for more details.
\end{note}

Specific role of variables \emph{var} can be defined with \emph{options}:  \hyperlink{model:decisionvariables}{non-decision} variables or  \hyperlink{model:objectivevariable}{objective} variable;
\lstinputlisting{java/mvariabledeclaration2.j2t}

%%%%%%%%%%%%%%%%%%%%%%%%%%%%%%%%%%%%%%%%%%%%%%%%%%%%%%%%%%%%%%%%%%%%%%%%%%%%%%%%%%%%%%%%%%%%%%%%%%%%%%%%%%%%%%%%%%%%%%%%%%%%%%%%%%%%%%%%%%%%%%%%%%%
%%%%%%%%%%%%%%%%%%%%%%%%%%%%%%%%%%%%%%%%%%%%%%%%%%%%% SIMPLE VARIABLE  %%%%%%%%%%%%%%%%%%%%%%%%%%%%%%%%%%%%%%%%%%%%%%%%%%%%%%%%%%%%%%%%%%%%%%%%%%%%

\subsection{Simple Variables}\label{model:simplevariables}\hypertarget{model:simplevariables}{}
See Section \hyperlink{ch:vars}{Variables} for details:

\begin{notedef}\tt
\hyperlink{integervariable}{IntegerVariable}, \hyperlink{setvariable}{SetVariable}, \hyperlink{realvariable}{RealVariable}
\end{notedef}

%%%%%%%%%%%%%%%%%%%%%%%%%%%%%%%%%%%%%%%%%%%%%%%%%%%%%%%%%%%%%%%%%%%%%%%%%%%%%%%%%%%%%%%%%%%%%%%%%%%%%%%%%%%%%%%%%%%%%%%%%%%%%%%%%%%%%%%%%%%%%%%%%%%
%%%%%%%%%%%%%%%%%%%%%%%%%%%%%%%%%%%%%%%%%%%%%%%%%%%%% CONSTANT VARIABLE  %%%%%%%%%%%%%%%%%%%%%%%%%%%%%%%%%%%%%%%%%%%%%%%%%%%%%%%%%%%%%%%%%%%%%%%%%%

\subsection{Constants}\label{model:constants}\hypertarget{model:constants}{}
A constant is a variable with a fixed domain. An \hyperlink{integervariable}{\tt IntegerVariable} declared with a unique value is automatically set as constant. A constant declared twice or more is only stored once in a model.

\lstinputlisting{java/mconstant.j2t}

%%%%%%%%%%%%%%%%%%%%%%%%%%%%%%%%%%%%%%%%%%%%%%%%%%%%%%%%%%%%%%%%%%%%%%%%%%%%%%%%%%%%%%%%%%%%%%%%%%%%%%%%%%%%%%%%%%%%%%%%%%%%%%%%%%%%%%%%%%%%%%%%%%%
%%%%%%%%%%%%%%%%%%%%%%%%%%%%%%%%%%%%%%%%%%%%%%%%%%%%% EXPRESSION VARIABLE  %%%%%%%%%%%%%%%%%%%%%%%%%%%%%%%%%%%%%%%%%%%%%%%%%%%%%%%%%%%%%%%%%%%%%%%%
\subsection{Expression variables and operators}\label{model:expressionvariables}\hypertarget{model:expressionvariables}{}
Expression variables represent the result of combinations between variables of the same type made by operators. Two types of expression variables exist : 
\begin{notedef}
\textbf{\tt IntegerExpressionVariable} and \textbf{\tt RealExpressionVariable}.
\end{notedef}
One can define an expression variable to define an operation, for example:
\lstinputlisting{java/mexpressionvariable.j2t}

%\section{Operators}\label{model:operators}\hypertarget{model:operators}{}

To construct expressions of variables, simple operators can be used. Each returns a \texttt{ExpressionVariable} object:
\begin{notedef}\tt
\begin{itemize}
\item Integer : \hyperlink{abs:absoperator}{abs}, \hyperlink{div:divoperator}{div}, \hyperlink{ifthenelse:ifthenelseoperator}{ifThenElse}, \hyperlink{max:maxoperator}{max}, \hyperlink{min:minoperator}{min}, \hyperlink{minus:minusoperator}{minus}, \hyperlink{mod:modoperator}{mod}, \hyperlink{mult:multoperator}{mult}, \hyperlink{neg:negoperator}{neg}, \hyperlink{plus:plusoperator}{plus}, \hyperlink{power:poweroperator}{power}, \hyperlink{scalar:scalaroperator}{scalar}, \hyperlink{sum:sumoperator}{sum},
\item Real : \hyperlink{cos:cosoperator}{cos}, \hyperlink{minus:minusoperator}{minus}, \hyperlink{mult:multoperator}{mult}, \hyperlink{plus:plusoperator}{plus}, \hyperlink{power:poweroperator}{power}, \hyperlink{sin:sinoperator}{sin}
\end{itemize}
\end{notedef}
Note that these operators are not considered as constraints: they do not return a \texttt{Constraint} objet but a \texttt{Variable} object.

%%%%%%%%%%%%%%%%%%%%%%%%%%%%%%%%%%%%%%%%%%%%%%%%%%%%%%%%%%%%%%%%%%%%%%%%%%%%%%%%%%%%%%%%%%%%%%%%%%%%%%%%%%%%%%%%%%%%%%%%%%%%%%%%%%%%%%%%%%%%%%%%%%%
%%%%%%%%%%%%%%%%%%%%%%%%%%%%%%%%%%%%%%%%%%%%%%%%%%%%% MULTIPLE VARIABLE  %%%%%%%%%%%%%%%%%%%%%%%%%%%%%%%%%%%%%%%%%%%%%%%%%%%%%%%%%%%%%%%%%%%%%%%%%%

\subsection{MultipleVariables}\label{model:multiplevariables}\hypertarget{model:multiplevariables}{}
These are syntaxic sugar. To make their declaration easier, \hyperlink{tree:treeconstraint}{\tt tree}, \hyperlink{geost:geostconstraint}{\tt geost}, and scheduling constraints allow or require to use multiple variables, like \texttt{TreeParametersObject}, \texttt{GeostObject} or \hyperlink{taskvariable}{\tt TaskVariable}.
See also the code examples for these constraints.

%%%%%%%%%%%%%%%%%%%%%%%%%%%%%%%%%%%%%%%%%%%%%%%%%%%%%%%%%%%%%%%%%%%%%%%%%%%%%%%%%%%%%%%%%%%%%%%%%%%%%%%%%%%%%%%%%%%%%%%%%%%%%%%%%%%%%%%%%%%%%%%%%%%
%%%%%%%%%%%%%%%%%%%%%%%%%%%%%%%%%%%%%%%%%%%%%%%%%%%%% OPTIONS %%%%%%%%%%%%%%%%%%%%%%%%%%%%%%%%%%%%%%%%%%%%%%%%%%%%%%%%%%%%%%%%%%%%%%%%%%%%%%%%%%%%%

\subsection{Decision/non-decision variables}\label{model:decisionvariables}\hypertarget{model:decisionvariables}{}

By default, each variable added to a model is a decision variable, \textit{i.e.} is included in the default search strategy. A variable can be stated as a non decision one if its value can be computed by side-effect. To specify non decision variables, one can 
\begin{itemize}
\item exclude them from the search strategies (see \hyperlink{solver:searchstrategy}{search strategy} for more details);
\item specify non-decision variables (adding \hyperlink{vnodecision:vnodecisionoptions}{\tt Options.V\_NO\_DECISION} to their options) and keep the default search strategy.
\end{itemize}
\lstinputlisting{java/mnodecision1.j2t}
Each of these options can also be set within a single instruction for a group of variables, as follows: 
\lstinputlisting{java/mnodecision2.j2t}

\begin{note}
 The declaration of a \hyperlink{solver:searchstrategy}{search strategy} will erase setting \hyperlink{vnodecision:vnodecisionoptions}{\tt Options.V\_NO\_DECISION}.
\end{note}
  \todo{more precise: user-defined/pre-defined, variable and/or value heuristics ?}

\subsection{Objective variable}\label{model:objectivevariable}\hypertarget{model:objectivevariable}{}
You can define an objective variable directly within the model, by using option \hyperlink{vobjective:vobjectiveoptions}{\tt Options.V\_OBJECTIVE}:
\lstinputlisting{java/mobjective.j2t}

Only one variable can be defined as an objective. If more than one objective variable is declared, then only the last one will be taken into account.

Note that optimization problems can be declared without defining an objective variable within the model (see the \hyperlink{solver:optimization}{optimization example}.)

%%%%%%%%%%%%%%%%%%%%%%%%%%%%%%%%%%%%%%%%%%%%%%%%%%%%%%%%%%%%%%%%%%%%%%%%%%%%%%%%%%%%%%%%%%%%%%%%%%%%%%%%%%%%%%%%%%%%%%%%%%%%%%%%%%%%%%%%%%%%%%%%%%%
%%%%%%%%%%%%%%%%%%%%%%%%%%%%%%%%%%%%%%%%%%%%%%%%%%%%% CONSTRAINT  %%%%%%%%%%%%%%%%%%%%%%%%%%%%%%%%%%%%%%%%%%%%%%%%%%%%%%%%%%%%%%%%%%%%%%%%%%%%%%%%%
%%%%%%%%%%%%%%%%%%%%%%%%%%%%%%%%%%%%%%%%%%%%%%%%%%%%%%%%%%%%%%%%%%%%%%%%%%%%%%%%%%%%%%%%%%%%%%%%%%%%%%%%%%%%%%%%%%%%%%%%%%%%%%%%%%%%%%%%%%%%%%%%%%%

\section{Constraints}\label{model:constraints}\hypertarget{model:constraints}{}
Choco provides a large number of simple and global constraints and allows the user to easily define its own new constraint.
% Either basic, global (a \hyperlink{constraints}{large set of global constraints} are available) or \hyperlink{advanced:defineyourownconstraint}{user-defined} constraints, they are used to specify conditions to be held on variables to the model. 
A constraint deals with one or more variables of the model and specify conditions to be held on these variables. 
A constraint is stated into the model by using the following methods available from the \texttt{Model} API: 

\lstinputlisting{java/mconstraintdeclaration1.j2t}

\begin{note}\
Adding a constraint automatically adds its variables to the model (explicit declaration of variables addition is not mandatory).
Thus, a variable not involved in any constraints will not be declared in the Solver during the reading step.
\end{note}


\subsubsection{Example:} adding a difference (disequality) constraint between two variables of the model

\lstinputlisting{java/mconstraintdeclaration2.j2t}

Available \emph{options} depend on the kind of constraint \emph{c} to add: they allow, for example, to choose the filtering algorithm to run during propagation. See \hyperlink{optionssettings}{Section options and settings} for more details, specific APIs exist for declaring options constraints.

This section presents the constraints available in the Choco API sorted by type or by domain. Related sections:
\begin{itemize}
\item a detailed description (with options, examples, references) of each constraint is given in Section \hyperlink{ch:constraints}{constraints}
%\item Section \hyperlink{doc:applications}{applications} shows how to apply some specific global constraints
\item Section \hyperlink{advanced:defineyourownconstraint}{user-defined constraint} explains how to create its own constraint.
\end{itemize}

%%%%%%%%%%%%%%%%%%%%%%%%%%%%%%%%%%%%%%%%%%%%%%%%%%%%%%%%%%%%%%%%%%%%%%%%%%%%%%%%%%%%%%%%%%%%%%%%%%%%%%%%%%%%%%%%%%%%%%%%%%%%%%%%%%%%%%%%%%%%%%%%%%%
%%%%%%%%%%%%%%%%%%%%%%%%%%%%%%%%%%%%%%%%%%%%%%%%%%%%% BINARY CONSTRAINT  %%%%%%%%%%%%%%%%%%%%%%%%%%%%%%%%%%%%%%%%%%%%%%%%%%%%%%%%%%%%%%%%%%%%%%%%%%

\subsection{Binary constraints}\label{model:comparisonconstraints}\hypertarget{model:comparisonconstraints}{}
%The simplest constraints are comparisons which are defined over expressions of variables such as linear combinations. The following comparison constraints can be accessed through the \texttt{Model} API:
Constraints involving two integer variables
\begin{notedef}\tt
  \begin{itemize}
  \item \hyperlink{eq:eqconstraint}{eq}, \hyperlink{geq:geqconstraint}{geq}, \hyperlink{gt:gtconstraint}{gt}, \hyperlink{leq:leqconstraint}{leq}, \hyperlink{lt:ltconstraint}{lt}, \hyperlink{neq:neqconstraint}{neq}
  \item \hyperlink{abs:absconstraint}{abs}, \hyperlink{oppositesign:oppositesignconstraint}{oppositeSign}, \hyperlink{samesign:samesignconstraint}{sameSign}
  \end{itemize}
\end{notedef}

%%%%%%%%%%%%%%%%%%%%%%%%%%%%%%%%%%%%%%%%%%%%%%%%%%%%%%%%%%%%%%%%%%%%%%%%%%%%%%%%%%%%%%%%%%%%%%%%%%%%%%%%%%%%%%%%%%%%%%%%%%%%%%%%%%%%%%%%%%%%%%%%%%%
%%%%%%%%%%%%%%%%%%%%%%%%%%%%%%%%%%%%%%%%%%%%%%%%%%%%% TERNARY CONSTRAINT  %%%%%%%%%%%%%%%%%%%%%%%%%%%%%%%%%%%%%%%%%%%%%%%%%%%%%%%%%%%%%%%%%%%%%%%%%

\subsection{Ternary constraints}\label{model:ternaryconstraints}\hypertarget{model:ternaryconstraints}{}
Constraints involving three integer variables
\begin{notedef}\tt
  \begin{itemize}
  \item \hyperlink{distanceeq:distanceeqconstraint}{distanceEQ}, \hyperlink{distanceneq:distanceneqconstraint}{distanceNEQ}, \hyperlink{distancegt:distancegtconstraint}{distanceGT}, \hyperlink{distancelt:distanceltconstraint}{distanceLT}
  \item \hyperlink{intdiv:intdivconstraint}{intDiv}, \hyperlink{mod:modconstraint}{mod}, \hyperlink{times:timesconstraint}{times}
  \end{itemize}
\end{notedef}

%%%%%%%%%%%%%%%%%%%%%%%%%%%%%%%%%%%%%%%%%%%%%%%%%%%%%%%%%%%%%%%%%%%%%%%%%%%%%%%%%%%%%%%%%%%%%%%%%%%%%%%%%%%%%%%%%%%%%%%%%%%%%%%%%%%%%%%%%%%%%%%%%%%
%%%%%%%%%%%%%%%%%%%%%%%%%%%%%%%%%%%%%%%%%%%%%%%%%%%%% REAL CONSTRAINT  %%%%%%%%%%%%%%%%%%%%%%%%%%%%%%%%%%%%%%%%%%%%%%%%%%%%%%%%%%%%%%%%%%%%%%%%%%%%

\subsection{Constraints involving real variables}\label{model:realconstraints}\hypertarget{model:realconstraints}{}
%The simplest constraints are comparisons which are defined over expressions of variables such as linear combinations. The following comparison constraints can be accessed through the \texttt{Model} API:
Constraints involving two real variables
\begin{notedef}\tt
  \begin{itemize}
  \item \hyperlink{eq:eqconstraint}{eq}, \hyperlink{geq:geqconstraint}{geq}, \hyperlink{leq:leqconstraint}{leq}
  \end{itemize}
\end{notedef}

%%%%%%%%%%%%%%%%%%%%%%%%%%%%%%%%%%%%%%%%%%%%%%%%%%%%%%%%%%%%%%%%%%%%%%%%%%%%%%%%%%%%%%%%%%%%%%%%%%%%%%%%%%%%%%%%%%%%%%%%%%%%%%%%%%%%%%%%%%%%%%%%%%%
%%%%%%%%%%%%%%%%%%%%%%%%%%%%%%%%%%%%%%%%%%%%%%%%%%%%% SET CONSTRAINT  %%%%%%%%%%%%%%%%%%%%%%%%%%%%%%%%%%%%%%%%%%%%%%%%%%%%%%%%%%%%%%%%%%%%%%%%%%%%%

\subsection{Constraints involving set variables}\label{model:setconstraints}\hypertarget{model:setconstraints}{}
%The simplest constraints are comparisons which are defined over expressions of variables such as linear combinations. The following comparison constraints can be accessed through the \texttt{Model} API:
Set constraints are illustrated on the \hyperlink{model:example2:ternarysteinerchoco}{ternary Steiner problem}. 
\begin{notedef}\tt
  \begin{itemize}
  \item \hyperlink{member:memberconstraint}{member}, \hyperlink{notmember:notmemberconstraint}{notMember}
  \item \hyperlink{eqcard:eqcardconstraint}{eqCard}, \hyperlink{geqcard:geqcardconstraint}{geqCard}, \hyperlink{leqcard:leqcardconstraint}{leqCard}, \hyperlink{neqcard}{neqCard}
  \item \hyperlink{eq}{eq}
  \item \hyperlink{isincluded:isincludedconstraint}{isIncluded}, \hyperlink{isnotincluded:isnotincludedconstraint}{isNotIncluded}
  \item \hyperlink{setinter:setinterconstraint}{setInter}
  \item \hyperlink{setdisjoint:setdisjointconstraint}{setDisjoint}, \hyperlink{setunion:setunionconstraint}{setUnion}
  \item \hyperlink{max:maxofaset}{max}, \hyperlink{min:minofaset}{min}
  \item \hyperlink{inverseset}{inverseSet}
  \item \hyperlink{among}{among}
  \item \hyperlink{pack:packconstraint}{pack}
  \end{itemize}
\end{notedef}

%\hyperlink{max:maxconstraint}{max}, \hyperlink{min:minconstraint}{min},

%%%%%%%%%%%%%%%%%%%%%%%%%%%%%%%%%%%%%%%%%%%%%%%%%%%%%%%%%%%%%%%%%%%%%%%%%%%%%%%%%%%%%%%%%%%%%%%%%%%%%%%%%%%%%%%%%%%%%%%%%%%%%%%%%%%%%%%%%%%%%%%%%%%
%%%%%%%%%%%%%%%%%%%%%%%%%%%%%%%%%%%%%%%%%%%%%%%%%%%%% CHANNELING CONSTRAINT  %%%%%%%%%%%%%%%%%%%%%%%%%%%%%%%%%%%%%%%%%%%%%%%%%%%%%%%%%%%%%%%%%%%%%%

\subsection{Channeling constraints}\label{model:channelingconstraints}\hypertarget{model:channelingconstraints}{}
The use of a redundant model, based on an alternative set of decision variables, is a frequent technique to strengthen propagation or to get more freedom to design dedicated search heuristics. 
The following constraints allow to ensure the integrity of two redundant models by linking (channeling) variable-value assignments in the first model to variable-value assignments in the second model:
\begin{notedef}\tt
  \begin{itemize}
  \item \hyperlink{boolchanneling:boolchannelingconstraint}{boolChanneling} $b_j=1 \iff x=j$, 
  \item \hyperlink{domainchanneling:domainchannelingconstraint}{domainChanneling} $b_j=1 \iff x=j$, $\forall j$, 
  \item \hyperlink{inversechanneling:inversechannelingconstraint}{inverseChanneling} $y_j=i \iff x_i=j$, $\forall i, j$, 
  \item \hyperlink{inversechannelingwithinrange:inversechannelingconstraintwithinrange}{inverseChannelingWithinRange} $y_j=i \wedge j \le |x| \iff x_i=j \wedge i \le |y|$, $\forall i, j$, 
  \item \hyperlink{inverseset:inversesetconstraint}{inverseSet} $i\in s_j \iff x_i=j$, $\forall i,j$, 
  \end{itemize}
\end{notedef}
In the n-queen problem, for example, a domain variable by column indicates the row $j$ to place a queen in column $i$. To enhance the propagation, a redundant model can be stated by defining a domain variable by row indicating the column $i$. As columns and rows can be interchanged, the same set of constraints applies to both models, then constraint \hyperlink{inversechanneling:inversechannelingconstraint}{inverseChanneling} is set to propagate between the two models.
\lstinputlisting{java/cinversechanneling.j2t}

Channeling constraints are also useful to compose a model made up of two parts as, for example, in a task-resources assignment problem where some constraints are set on the task set and some other constraints are set on the resource set.

\hyperlink{model:reifiedconstraints}{Reification} offers an other type of channeling, between a constraint and a boolean variable representing its truth value. 
More complex channeling can be done using reification and boolean operators although they are less effective. 
The reified constraint below states $b=1\iff x=y$:
\lstinputlisting{java/cchannelingreified.j2t}


%%%%%%%%%%%%%%%%%%%%%%%%%%%%%%%%%%%%%%%%%%%%%%%%%%%%%%%%%%%%%%%%%%%%%%%%%%%%%%%%%%%%%%%%%%%%%%%%%%%%%%%%%%%%%%%%%%%%%%%%%%%%%%%%%%%%%%%%%%%%%%%%%%%
%%%%%%%%%%%%%%%%%%%%%%%%%%%%%%%%%%%%%%%%%%%%%%%%%%%%% EXTENSIONS CONSTRAINT  %%%%%%%%%%%%%%%%%%%%%%%%%%%%%%%%%%%%%%%%%%%%%%%%%%%%%%%%%%%%%%%%%%%%%%

\subsection{Constraints in extension and relations}\label{model:arbitraryconstraintsinextension}\hypertarget{model:arbitraryconstraintsinextension}{}
Choco supports the statement of constraints defining arbitrary relations over two or more variables.
Such a relation may be defined by three means:
\begin{itemize}
	\item \textbf{feasible table:} the list of allowed tuples of values (that belong to the relation),
	\item \textbf{infeasible table:} the list of forbidden tuples of values (that do not belong to the relation),
	\item \textbf{predicate:} a method to be called in order to check whether a tuple of values belongs or not to the relation.
\end{itemize}
On the one hand, constraints based on tables may be rather memory consuming in case of large domains, although one relation table may be shared by several constraints. On the other hand, predicate constraints require little memory as they do not cache truth values, but imply some run-time overhead for calling the feasibility test. Table constraints are thus well suited for constraints over small domains; while predicate constraints are well suited for situations with large domains. 

Different levels of consistency can be enforced on constraints in extension (when selecting an API) and, for Arc Consistency, different filtering algorithms can be used (when selecting an option).
The Choco API for creating constraints in extension are as follows:

\begin{notedef}
  \begin{itemize}
  \item arc consistency (AC) for binary relations:\\
{\tt \hyperlink{feaspairac:feaspairacconstraint}{feasPairAC}, \hyperlink{infeaspairac:infeaspairacconstraint}{infeasPairAC}, \hyperlink{relationpairac:relationpairacconstraint}{relationPairAC}}
  \item arc consistency (AC) for n-ary relations:\\
{\tt \hyperlink{feastupleac:feastupleacconstraint}{feasTupleAC}, \hyperlink{infeastupleac:infeastupleacconstraint}{infeasTupleAC}, \hyperlink{relationtupleac:relationtupleacconstraint}{relationTupleAC}}
  \item weaker forward checking (FC) for n-ary relations:\\
{\tt \hyperlink{feastuplefc:feastuplefcconstraint}{feasTupleFC}, \hyperlink{infeastuplefc:infeastuplefcconstraint}{infeasTupleFC}, \hyperlink{relationtuplefc:relationtuplefcconstraint}{relationTupleFC}}
  \end{itemize}
\end{notedef}

\subsubsection{Relations.}
A same relation might be shared among several constraints, in this case it is highly recommended to create it first and then use the \hyperlink{relationpairac:relationpairacconstraint}{relationPairAC}, \hyperlink{relationtupleac:relationtupleacconstraint}{relationTupleAC}, or \hyperlink{relationtuplefc:relationtuplefcconstraint}{relationTupleFC} API  on the same relation for each constraint.

For binary relations, the following Choco API is provided:\\
\mylst{makeBinRelation(int[] min, int[] max, List<int[]>pairs, boolean feas)}

It returns a \texttt{BinRelation} giving a list of compatible (\texttt{feas=true}) or incompatible (\texttt{feas=false}) pairs of values. This relation can be applied to any pair of variables $(x_1,x_2)$ whose domains are included in the \texttt{min/max} intervals, i.e. such that:
$$\mathtt{min}[i] \le x_i.\mathtt{getInf}() \le x_i.\mathtt{getSup}() \le  \mathtt{max}[i],\quad \forall i.$$
Bounds \texttt{min/max} are mandatory in order to allow to compute the opposite of the relation if needed.

For n-ary relations, the corresponding Choco API is:\\
\mylst{makeLargeRelation(int[] min, int[] max, List<int[]> tuples, boolean feas);}

It returns a \texttt{LargeRelation}. If \texttt{feas=true}, the returned relation matches also the \texttt{IterLargeRelation} interface which provides constant time iteration abilities over tuples (for compatibility with the GAC algorithm used over feasible tuples).
\lstinputlisting{java/mlargerelation.j2t}

Lastly, some specific relations can be defined without storing the tuples, as in the following example (\texttt{TuplesTest} extends \texttt{LargeRelation}):
\lstinputlisting{java/mnotallequal.j2t}
Then, the \texttt{NotAllEqual} relation is stated as a constraint of a model:
\lstinputlisting{java/mrelationtuplefc.j2t}
%Again, for compatibility with the GAC algorithm invoked by relationTupleAC, such a relation has to match the \texttt{IterLargeRelation} interface for feasible tuples.


%%%%%%%%%%%%%%%%%%%%%%%%%%%%%%%%%%%%%%%%%%%%%%%%%%%%%%%%%%%%%%%%%%%%%%%%%%%%%%%%%%%%%%%%%%%%%%%%%%%%%%%%%%%%%%%%%%%%%%%%%%%%%%%%%%%%%%%%%%%%%%%%%%%
%%%%%%%%%%%%%%%%%%%%%%%%%%%%%%%%%%%%%%%%%%%%%%%%%%%%% REIFIED CONSTRAINT  %%%%%%%%%%%%%%%%%%%%%%%%%%%%%%%%%%%%%%%%%%%%%%%%%%%%%%%%%%%%%%%%%%%%%%%%%

\subsection{Reified constraints}\label{model:reifiedconstraints}\hypertarget{model:reifiedconstraints}{}
The \emph{truth value} of a constraint is a boolean that is true if and only if the constraint holds.
To \emph{reify} a constraint is to get its truth value. 

This mechanism can be used for example to model a MaxCSP problem where the number of satisfied constraints has to be maximized.
It is also intended to give the freedom to model complex constraints combining several reified constraints, using some logical operators on the truth values, such as in:
$(x \neq y) \lor (z \le 9)$.

Choco provides a generic constraint \hyperlink{reifiedconstraint:reifiedconstraintconstraint}{reifiedConstraint} to reify any constraint into a boolean variable expressing its truth value:
\begin{lstlisting}
  Constraint reifiedConstraint(IntegerVariable b, Constraint c);
  Constraint reifiedConstraint(IntegerVariable b, Constraint c1, Constraint c2);
\end{lstlisting}
Specific API are also provided to reify boolean constraints:  
\begin{notedef}\tt
  \begin{itemize}
  \item \hyperlink{reifiedconstraint:reifiedconstraintconstraint}{reifiedConstraint}, 
   \item \hyperlink{reifiedand:reifiedandconstraint}{reifiedAnd}, \hyperlink{reifiedleftimp:reifiedleftimpconstraint}{reifiedLeftImp}, \hyperlink{reifiednot:reifiednotconstraint}{reifiedNot}, \hyperlink{reifiedor:reifiedorconstraint}{reifiedOr}, \hyperlink{reifiedrightimp:reifiedrightimpconstraint}{reifiedRightImp}, \hyperlink{reifiedxnor:reifiedxnorconstraint}{reifiedXnor}, \hyperlink{reifiedxor:reifiedxorconstraint}{reifiedXor}
  \end{itemize}
\end{notedef}



\subsubsection{Handling complex expressions.}\label{model:handlingcomplexexpressions}\hypertarget{model:handlingcomplexexpressions}{}
In order to build complex combinations of constraints, Choco also provides a simpler and more direct API with the following logical meta-constraints taking constraints in arguments:
\begin{notedef}\tt
  \begin{itemize}
  \item \hyperlink{and:andconstraint}{and}, \hyperlink{or:orconstraint}{or}, \hyperlink{implies:impliesconstraint}{implies}, \hyperlink{ifonlyif:ifonlyifconstraint}{ifOnlyIf}, \hyperlink{ifthenelse:ifthenelseconstraint}{ifThenElse}, \hyperlink{not:notconstraint}{not}, \hyperlink{nand:nandconstraint}{nand}, \hyperlink{nor:norconstraint}{nor}
  \end{itemize}
\end{notedef}
For example, the following expression
$$((x = 10 * |y|) \lor (z \le 9))\quad \iff\quad \texttt{alldifferent}(a,b,c)$$
could be expressed in Choco by:
\begin{lstlisting}
	Constraint exp = ifOnlyIf( or( eq(x, mult(10, abs(y))), leq(z, 9) ), 
                               alldifferent(new IntegerVariable[]{a,b,c}) );
\end{lstlisting}
Such an expression is internally represented as a tree whose nodes are operators and leaves are variables, constants and constraints. Variables and constants can be combined as \texttt{ExpressionVariable} using \hyperlink{model:expressionvariables}{operators} (e.g, \texttt{mult(10,abs(w))}), or using simple constraints (e.g., \texttt{leq(z,9)}), or even using global constraints (e.g, \texttt{alldifferent(vars)}).
The language available on expressions currently matches the language used in the \href{http://cpai.ucc.ie/08/}{Constraint Solver Competition 2008} of the CPAI workshop.

At the solver level, there exists two different ways to represent expressions:
\begin{itemize}
\item \emph{by extension:} the first way is to handle expressions as \hyperlink{model:arbitraryconstraintsinextension}{constraints in extension}. The expression is then used to check a tuple in a dynamic way just like a n-ary relation that is defined without listing all the possible tuples. The expression is then propagated using the GAC3rm algorithm. This is very powerful as arc-consistency is achieved on the corresponding constraints.
\item \emph{by decomposition:} the second way is to decompose the expression automatically by introducing intermediate variables and possibly the generic \hyperlink{reifiedconstraint:reifiedconstraintconstraint}{\tt reifiedConstraint}. By doing so, the level of pruning decreases but expressions of larger arity involving large domains can be represented.
\end{itemize}
The way to represent expressions is decided at the modeling level. Representation \emph{by extension} is the default. Representation \emph{by decomposition} can be set instead by:
\begin{lstlisting}
  model.setDefaultExpressionDecomposition(true);
\end{lstlisting}

Representation \emph{by decomposition} can also be controlled individually for some expressions, by setting option \hyperlink{edecomp:edecompoptions}{\tt Options.E\_DECOMP} when adding the constraint.
For example, the following code tells the solver to decompose e1 but not e2 :
\begin{lstlisting}
	model.setDefaultExpressionDecomposition(false);
	IntegerVariable[] x = makeIntVarArray("x", 3, 1, 3, Options.V_BOUND);

	Constraint e1 = or(lt(x[0], x[1]), lt(x[1], x[0]));
	model.addConstraint(Options.E_DECOMP, e1);
	
	Constraint e2 = or(lt(x[1], x[2]), lt(x[2], x[1]));
	model.addConstraint(e2);
\end{lstlisting}

\subsubsection{When and how should I use expressions ?}\label{model:whenshouldiuseexpressions}\hypertarget{model:whenshouldiuseexpressions}{}
Expressions offer a slightly richer modeling language than the one available via standard constraints. However, expressions 
can not be handled as efficiently as constraints that embed a dedicated propagation algorithm. We therefore
recommend you to carefully check that you can not model the expression using the \emph{global constraints} of Choco before using
expressions.

Expressions represented \emph{in extension} should be used in the case of complex logical relationships that involve \textbf{few different variables}, each of \textbf{small domain}, and if \textbf{arc consistency} is desired on those variables.
In such a case, representation in extension can be much more effective than with decomposition. Imagine the following ``crazy'' example :
\begin{lstlisting}
 or( and( eq( abs(sub(div(x,50),div(y,50))),1), eq( abs(sub(mod(x,50),mod(y,50))),2)),
     and( eq( abs(sub(div(x,50),div(y,50))),2), eq( abs(sub(mod(x,50),mod(y,50))),1)))
\end{lstlisting}
This expression has a small arity: it involves only two variables $x$ and $y$.
Let assume that their domains has no more than 300 values, then such an expression should typically not be decomposed. Indeed, arc consistency will create many holes in the domains and filter much more than if the relation was decomposed.

Conversely, an expression should be decomposed as soon as it involves a large number of variables, or at least one variable with a large domain.

%%%%%%%%%%%%%%%%%%%%%%%%%%%%%%%%%%%%%%%%%%%%%%%%%%%%%%%%%%%%%%%%%%%%%%%%%%%%%%%%%%%%%%%%%%%%%%%%%%%%%%%%%%%%%%%%%%%%%%%%%%%%%%%%%%%%%%%%%%%%%%%%%%%
%%%%%%%%%%%%%%%%%%%%%%%%%%%%%%%%%%%%%%%%%%%%%%%%%%%%% GLOBAL CONSTRAINT  %%%%%%%%%%%%%%%%%%%%%%%%%%%%%%%%%%%%%%%%%%%%%%%%%%%%%%%%%%%%%%%%%%%%%%%%%%

\subsection{Global constraints}\label{model:advancedconstraints}\hypertarget{model:advancedconstraints}{}
Choco includes several \href{http://www.emn.fr/x-info/sdemasse/gccat/}{global constraints}. Those constraints accept any number of variables and offer dedicated filtering algorithms which are able to make deductions where a decomposed model would not.
For instance, constraint \texttt{alldifferent}$(a,b,c,d)$ with $a,b\in[1,4]$ and $c,d\in[3,4]$ allows to deduce that $a$ and $b$ cannot be instantiated to $3$ or $4$; such rule cannot be inferred by simple binary constraints. 

The up-to-date list of global constraints available in Choco can be found within the Javadoc API.
Most of these global constraints are listed below according to their application fields.
Details and examples can be found in Section \hyperlink{ch:constraints}{Elements of Choco/Constraints}.
\subsubsection{Value constraints}\label{model:valueconstraints}\hypertarget{model:valueconstraints}{}
Constraints that put a restriction on how values can be distributed among a collection of variables.
See also in Global Constraint Catalog: \href{http://www.emn.fr/x-info/sdemasse/gccat/Kvalue_constraint.html}{value constraint}.

\vspace{1em}\noindent\begin{notedef}\tt
  \begin{itemize}
  \item counting distinct values: 
\hyperlink{alldifferent:alldifferentconstraint}{allDifferent}, 
\hyperlink{atmostnvalue:atmostnvalueconstraint}{atMostNValue},
\hyperlink{increasingnvalue:increasingnvalueconstraint}{increasingNValue},
  \item counting values: 
\hyperlink{among:amongconstraint}{among},
\hyperlink{occurrence:occurrenceconstraint}{occurrence},
\hyperlink{occurrencemax:occurrencemaxconstraint}{occurrenceMax},
\hyperlink{occurrencemin:occurrenceminconstraint}{occurrenceMin},
\hyperlink{globalcardinality:globalcardinalityconstraint}{globalCardinality},
  \item indexing values: 
\hyperlink{nth:nthconstraint}{nth} (element),
\hyperlink{max:maxconstraint}{max},
\hyperlink{min:minconstraint}{min},
  \item ordering: 
\hyperlink{sorting:sortingconstraint}{sorting},
\hyperlink{increasingnvalue:increasingnvalueconstraint}{increasingNValue},
\hyperlink{increasingsum:increasingsumconstraint}{increasingSum},
\hyperlink{lex:lexconstraint}{lex}, 
\hyperlink{lexeq:lexeqconstraint}{lexeq},
\hyperlink{leximin:leximinconstraint}{leximin},
\hyperlink{lexchain:lexchainconstraint}{lexChain},
\hyperlink{lexchaineq:lexchaineqconstraint}{lexChainEq},
  \item tuple matching: 
\hyperlink{feastupleac:feastupleacconstraint}{feasTupleAC},
\hyperlink{feastuplefc:feastuplefcconstraint}{feasTupleFC},
\hyperlink{infeastupleac:infeastupleacconstraint}{infeasTupleAC},
\hyperlink{infeastuplefc:infeastuplefcconstraint}{infeasTupleFC},
\hyperlink{relationtupleac:relationtupleacconstraint}{relationTupleAC},
\hyperlink{relationtuplefc:relationtuplefcconstraint}{relationTupleFC},
  \item pattern matching: 
\hyperlink{regular:regularconstraint}{regular},
\hyperlink{costregular:costregularconstraint}{costRegular},
\hyperlink{multicostregular:multicostregularconstraint}{multiCostRegular}, 
%\hyperlink{stretchcyclic:stretchcyclicconstraint}{stretchCyclic}, 
\hyperlink{stretchpath:stretchpathconstraint}{stretchPath}, 
\hyperlink{tree:treeconstraint}{tree},
  \end{itemize}
\end{notedef}

\subsubsection{Boolean constraints}\label{modelglobal:logicconstraints}\hypertarget{modelglobal:logicconstraints}{}
Logical operations on boolean expressions.
See also in Global Constraint Catalog: \href{http://www.emn.fr/x-info/sdemasse/gccat/KBoolean_constraint.html}{boolean constraint}.

\vspace{1em}\noindent\begin{notedef}\tt
\hyperlink{and:andconstraint}{and},
\hyperlink{or:orconstraint}{or},
\hyperlink{clause:clauseconstraint}{clause},
\end{notedef}

\subsubsection{Channeling constraints}\label{modelglobal:channelingconstraints}\hypertarget{modelglobal:channelingconstraints}{}
Constraints linking two collections of variables (many-to-many) or indexing one among many variables (one-to-many).
See also Section \hyperlink{model:channelingconstraints}{Channeling} and in Global Constraint Catalog: \href{http://www.emn.fr/x-info/sdemasse/gccat/Kchannelling_constraint.html}{channelling constraint}.

 \vspace{1em}\noindent\begin{notedef}\tt
   \begin{itemize}
   \item one-to-many: 
 \hyperlink{domainchanneling:domainchannelingconstraint}{domainChanneling},
 \hyperlink{nth:nthconstraint}{nth} (element),
 \hyperlink{max:maxconstraint}{max},
 \hyperlink{min:minconstraint}{min},
   \item many-to-many: 
 \hyperlink{inversechanneling:inversechannelingconstraint}{inverseChanneling},
 \hyperlink{inverseset:inversesetconstraint}{inverseSet},
 \hyperlink{sorting:sortingconstraint}{sorting},
\hyperlink{pack:packconstraint}{pack},
 \end{itemize}
 \end{notedef}

\subsubsection{Optimization constraints}\label{model:optimizationconstraints}\hypertarget{model:optimizationconstraints}{}
Constraints channelling a variable to the sum of the weights of a collection of variable-value assignments.
See also in Global Constraint Catalog: \href{http://www.emn.fr/x-info/sdemasse/gccat/Kcost_filtering_constraint.html}{cost-filtering constraint}.
\vspace{1em}\noindent\begin{notedef}\tt
 \begin{itemize}
  \item one cost: 
\hyperlink{among:amongconstraint}{among},
\hyperlink{occurrence:occurrenceconstraint}{occurrence},
\hyperlink{occurrencemax:occurrencemaxconstraint}{occurrenceMax},
\hyperlink{occurrencemin:occurrenceminconstraint}{occurrenceMin},
\hyperlink{knapsackproblem:knapsackproblemconstraint}{knapsackProblem},
\hyperlink{equation:equationconstraint}{equation},
\hyperlink{costregular:costregularconstraint}{costRegular},
\hyperlink{tree:treeconstraint}{tree},
 \item several costs:
\hyperlink{globalcardinality:globalcardinalityconstraint}{globalCardinality},
\hyperlink{multicostregular:multicostregularconstraint}{multiCostRegular}, 
 \end{itemize}
\end{notedef}

\subsubsection{Packing constraints (capacitated resources)}\label{model:packingconstraints}\hypertarget{model:packingconstraints}{}
Constraints involving items to be packed in bins without overlapping. More generaly, any constraints modelling the concurrent assignment of objects to one or several capacitated resources.
See also in Global Constraint Catalog: \href{http://www.emn.fr/x-info/sdemasse/gccat/Kresource_constraint.html}{resource constraint}.

\vspace{1em}\noindent\begin{notedef}\tt
   \begin{itemize}
   \item packing problems: 
\hyperlink{equation:equationconstraint}{equation},
\hyperlink{knapsackproblem:knapsackproblemconstraint}{knapsackProblem},
\hyperlink{pack:packconstraint}{pack} (bin-packing),
   \item geometric placement problems: 
\hyperlink{geost:geostconstraint}{geost}, 
   \item scheduling problems: 
\hyperlink{disjoint}{disjoint (tasks)} 
\hyperlink{disjunctive:disjunctiveconstraint}{disjunctive}, 
\hyperlink{cumulative:cumulativeconstraint}{cumulative}, 
   \item timetabling problems: 
\hyperlink{costregular:costregularconstraint}{costRegular},
\hyperlink{multicostregular:multicostregularconstraint}{multiCostRegular}, 
 \end{itemize}
\end{notedef}

\subsubsection{Scheduling constraints (time assignment)}\label{model:schedulingconstraints}\hypertarget{model:schedulingconstraints}{}
Constraints involving tasks to be scheduled over a time horizon.
%See also \hyperlink{schedulinganduseofthecumulative:schedulinganduseofthecumulativeconstraint}{scheduling application} and 
See also in Global Constraint Catalog: \href{http://www.emn.fr/x-info/sdemasse/gccat/Kscheduling_constraint.html}{scheduling constraint}.

\vspace{1em}\noindent\begin{notedef}\tt
   \begin{itemize}
   \item temporal constraints:
\hyperlink{disjoint}{disjoint (tasks)} 
\hyperlink{precedence:precedenceconstraint}{precedence}, 
\hyperlink{precedencedisjoint:precedencedisjointconstraint}{precedenceDisjoint}, 
\hyperlink{precedenceimplied:precedenceimpliedconstraint}{precedenceImplied}, 
\hyperlink{precedencereified:precedencereifiedconstraint}{precedenceReified},
\hyperlink{forbiddeninterval:forbiddenintervalconstraint}{forbiddenInterval},
\hyperlink{tree:treeconstraint}{tree},
   \item resource constraints: 
\hyperlink{cumulative:cumulativeconstraint}{cumulative}, 
\hyperlink{disjunctive:disjunctiveconstraint}{disjunctive}, 
\hyperlink{geost:geostconstraint}{geost}, 
 \end{itemize}
\end{notedef}

\subsection{Things to know about \mylst{Model}, \mylst{Variable} and \mylst{Constraint}}

It is important to know the relation between \mylst{Model}s, \mylst{Variable}s and \mylst{Constraint}s. \mylst{Variable}s and \mylst{Constraints} are build without the help of a \mylst{Model}, so that they can be used natively in different \mylst{Model}s. That's why one need to add them to a \mylst{Model}, using \mylst{model.addVariable(Variable var)} and \mylst{model.addConstraint(Constraint cstr)}. On a variable addition, this one is added to the list of variables of the model. On a constraint addition, the constraint is added to the constraint network AND the constraint and its associated variables are linked. It means the constraint is now known from its variables, which was not the case before the addition.


Now, let see a short example: one declares two constraints involving the two same variables, and add them to five different models. This implies the following references:
\begin{itemize}
\item each model points to two constraint and two variables;
\item each constraint points the two variables;
\item each of the variables points to ten constraints: the very same constraint is considered as different from a model to another.  
\end{itemize}

This must be kept in mind while you write your program: whether a model is used or not, a shared variable stores references to the constraints it is involved in. 

\vspace{1cm}
\begin{notedef}
Do not count on the garbage collector to manage this (even if the \mylst{finalize()} method was declared for a \mylst{Model}): while a variable references a constraint declared in an obsolete model, neither the model nor the constraint can be safely destroyed. 

A good habit to have is to delete constraints of a model when this one is not used anymore, by calling \mylst{model.removeConstraints()}. This breaks links between constraint and variable of a model, prevents large memory consumption and can stabilize performances by reusing \mylst{Model}s.


Consider the following code, \mylst{model2} is faster and consumes less memory than \mylst{model1} for the same result.
\begin{lstlisting}
public static void main(String[] args) {
        for (int i = 0; i < 10; i++) {
            long t = -System.currentTimeMillis();
            model1(99999);
            t += System.currentTimeMillis();
            System.out.printf("%d ", t);
            t = -System.currentTimeMillis();
            model2(99999);
            t += System.currentTimeMillis();
            System.out.printf("%d\n", t);
        }
    }                                               
                                                 
private static void model1(int i) {              
    IntegerVariable v1 = makeIntVar("v1", 0, 10);
    IntegerVariable v2 = makeIntVar("v2", 0, 10);
    for (int j = 0; j < i; j++) {                
        CPModel m1 = new CPModel();              
        m1.addConstraint(eq(v1, v2));            
    }                                            
}                                                
                                                 
private static void model2(int i) {              
    IntegerVariable v1 = makeIntVar("v1", 0, 10);
    IntegerVariable v2 = makeIntVar("v2", 0, 10);
    CPModel m1 = new CPModel();                  
    for (int j = 0; j < i; j++) {                
        m1.addConstraint(eq(v1, v2));            
        m1.removeConstraints();                  
    }                                            
}                                                
\end{lstlisting}

\end{notedef}




% java ToTex ../../../../../../samples/src/main/java/samples/documentation/ ../../documentation/java/

%!TEX root = ../content-doc.tex
%\part{solver}
\label{solver}
\hypertarget{solver}{}


\chapter{The solver}\label{solver:thesolver}\hypertarget{solver:thesolver}{}

%\section{How to create a solver}\label{solver:howtocreateasolver}\hypertarget{solver:howtocreateasolver}{}


\newglossaryentry{Solver}{name={Solver},description={solver description}}
The \mylst{Solver}, along with the \mylst{Model}, is one of the two key elements of any Choco program. The Choco \mylst{Solver} is mainly focused on resolution part: reading the \mylst{Model}, defining the search strategies and the resolution policy.  

To create a \gls{Solver}, one just needs to create a new object as follow:
\begin{lstlisting}
Solver solver = new CPSolver();
\end{lstlisting}
This instruction creates a Constraint Programming (CP) {\tt Solver} object.

%\section{Read a model}\label{solver:readamodel}\hypertarget{solver:readamodel}{}
The solver gives an API to read a model. The reading of a model is compulsory and must be done after the entire definition of the model. 
\begin{lstlisting}
solver.read(model);
\end{lstlisting}
The reading is divided in two parts: \hyperlink{solver:variablesreading}{variables reading} and \hyperlink{solver:constraintsreading}{constraints reading}.

\section{Variables reading}\label{solver:variablesreading}\hypertarget{solver:variablesreading}{}
The variables are declared in a model with a given type \texttt{IntegerVariable, SetVariable, RealVariable} and, possibly, with a given domain type (e.g. bounded or enumerated domains for integer and set variables).
When reading the model, the solver iterates over the model variables, then creates the corresponding solver variables and domains data structures according to these types.

\begin{note}
\textbf{Bound variables} are related to large domains which are only represented by their lower and upper bounds. The domain is encoded in a space efficient way and propagation events only concern bound updates. Value removals between the bounds are therefore ignored (\emph{holes} are not considered). The level of consistency achieved by most constraints on these variables is called \emph{bound-consistency}.

On the contrary, the domain of an \textbf{enumerated variable} is explicitly represented and every value is considered while pruning. Basic constraints are therefore often able to achieve \emph{arc-consistency} on enumerated variables (except for NP-hard global constraint such as the cumulative constraint). Remember that switching from enumerated variables to bounded variables decreases the level of propagation achieved by the system.
\end{note}


%\begin{note}
\paragraph{Model variables and solver variables are distinct objects.} 
Model variables implement the \mylst{Variable} interface and solver variables implement the \mylst{Var} interface.
A model variable is defined by an abstract representation of its initial domain, while a solver variable encapsulates a concrete representation of the domain, and maintains its current state throughout the search.
Hence, one cannot access a variable value directly from a model variable but one can from its corresponding solver variable. The solver variables are anonymous but can be accessed from the corresponding model variables using the \texttt{Solver} API \mylst{getVar(Variable v)} and \mylst{getVar(Variable... v)}.
%To access to a model variable thanks to the solver, use the following 
%\end{note}

\subsection{from \texttt{IntegerVariable} to\texttt{IntDomainVar}}\label{solver:solverandintegervariables}\hypertarget{solver:solverandintegervariables}{}

For integer variables, the solver \textbf{\tt IntDomainVar} is counterpart to the model \hyperlink{integervariable}{\textbf{\tt IntegerVariable}}. 
Methods \textbf{\tt getVar(IntegerVariable var)} and \textbf{\tt getVar(IntegerVariable... vars)} of \texttt{Solver} return the objects \texttt{IntDomainVar} and \texttt{IntDomainVar[]} respectively corresponding to \texttt{var} and \texttt{vars}:
\begin{lstlisting}
  IntegerVariable x = Choco.makeEnumIntVar("x", 1, 100);  // model variable
  IntDomainVar xOnSolver = solver.getVar(x);  // solver variable
\end{lstlisting}

The state of an \texttt{IntDomainVar} can be accessed using these main public methods:

\noindent\begin{tabular}{p{.3\linewidth}p{.7\linewidth}}
  \hline
  \texttt{IntDomainVar} API &  description \\
  \hline
	\mylst{hasEnumeratedDomain()} &checks if the domain type is enumerated or bounded\\
	\mylst{getInf()} &returns the current lower bound of the variable\\
	\mylst{getSup()} &returns the current upper bound of the variable\\
	\mylst{getVal()} &returns the value of the variable if it is currently instantiated\\
	\mylst{isInstantiated()} &checks if the domain is currently reduced to a singleton\\
	\mylst{canBeInstantiatedTo(int v)} &checks if value \texttt{v} currently belongs to the domain of the variable\\
	\mylst{getDomainSize()} &returns the current size of the domain\\
  \hline\\
\end{tabular}

The data structure representing the current domain within the \texttt{IntDomainVar} object depends on the domain type (bounded, enumerated, boolean, constant, etc.) of the model variable. 
See \hyperlink{advanced}{advanced uses} for more informations on \texttt{IntDomainVar}.

\subsection{from \texttt{SetVariable} to \texttt{SetVar}}\label{solver:solverandsetvariables}\hypertarget{solver:solverandsetvariables}{}

For set variables, the solver \textbf{\tt SetVar} is counterpart to the model \hyperlink{setvariable}{\textbf{\tt SetVariable}}. 
Methods \textbf{\tt getVar(SetVariable var)} and \textbf{\tt getVar(SetVariable... vars)} of \texttt{Solver} return the objects \texttt{SetVar} and \texttt{SetVar[]} respectively corresponding to \texttt{var} and \texttt{vars}:
\begin{lstlisting}
	SetVariable x = Choco.makeBoundSetVar("x", 1, 40); // model variable
	SetVar xOnSolver = solver.getVar(x); // solver variable
\end{lstlisting}

Note that a set variable on integer values between $1$ and $n$ may have $2^{n}$ possible values, corresponding to every possible subsets of $\{1,2,\ldots,n\}$. Hence, the domain of a \texttt{SetVar} is encoded by these bounds only: the lower bound, called the \emph{kernel}, is the intersection of all possible set values, and the upper bound, called the \emph{envelope}, is the union of all possible set values. Furthermore, a \texttt{SetVar} encapsulates an \texttt{IntDomainVar} representing the cardinality of the set variable. The domain type of this variable (enumerated or bounded) depends on the option given at the construction of the \texttt{SetVariable}. 
%This makes an exponential number of values and the domain is represented with two bounds corresponding to the intersection of all possible sets (called the kernel) and the union of all possible sets (called the envelope) which are the possible candidate values for the variable.

The state of a \texttt{SetVar} can be accessed through these main public methods: \todo{Warning: Envelope is (french) spelled with two 'p' in the method name.}

\noindent\begin{tabular}{p{.3\linewidth}p{.7\linewidth}}
  \hline
  \texttt{SetVar} API &  description \\
  \hline
	\mylst{getCard()} &returns the current cardinality (an \texttt{IntDomainVar} object)\\
	\mylst{isInDomainKernel(int v)} &checks if value \texttt{v} belongs to the current kernel\\
	\mylst{isInDomainEnveloppe(int v)} &checks if value \texttt{v} belongs to the current envelope\\
	\mylst{getDomain()} &returns the current domain (a \texttt{SetDomain} object). Iterators on envelope or kernel can then be called\\
	\mylst{getKernelDomainSize()} &returns the current size of the kernel\\
	\mylst{getEnveloppeDomainSize()} &returns the current size of the envelope\\
	\mylst{getEnveloppeInf()} &returns the current smallest value of the envelope\\
	\mylst{getEnveloppeSup()} &returns the current largest value of the envelope\\
	\mylst{getKernelInf()} &returns the current smallest value of the kernel\\
	\mylst{getKernelSup()} &returns the current largest value of the kernel\\
	\mylst{getValue()} &returns the set value as a table of integers \texttt{int[]} when the variable is currently instantiated (kernel=envelope)\\
  \hline\\
\end{tabular}

\noindent See \hyperlink{advanced}{advanced uses} for more informations on \texttt{SetVar}.

\subsection{from \texttt{RealVariable} to \texttt{RealVar}}\label{solver:solverandrealvariables}\hypertarget{solver:solverandrealvariables}{}

\begin{note}
\emph{Real variables are still under development but can be used to solve toy problems such as small systems of equations.}
\end{note}
 
For real variables, the solver \textbf{\tt RealVar} is counterpart to the model \hyperlink{realvariable}{\textbf{\tt RealVariable}}. 
Methods \textbf{\tt getVar(RealVariable var)} and \textbf{\tt getVar(RealVariable... vars)} of \texttt{Solver} return the objects \texttt{RealVar} and \texttt{RealVar[]} respectively corresponding to \texttt{var} and \texttt{vars}:
\begin{lstlisting}
	RealVariable x = Choco.makeRealVar("x", 1.0, 3.0); // model variable
	RealVar xOnSolver = solver.getVar(x); // solver variable
\end{lstlisting}

Continuous variables are useful for non linear equation systems which are encountered in physics for example.
The state of a \texttt{RealVar} can be accessed through these main public methods:

\noindent\begin{tabular}{p{.3\linewidth}p{.7\linewidth}}
  \hline
  \texttt{RealVar} API &  description \\
  \hline
	\mylst{getInf()} &returns the current lower bound of the variable (\texttt{double})\\
	\mylst{getSup()} &returns the current upper bound of the variable (\texttt{double})\\
	\mylst{isInstantiated()} &checks if the domain is reduced to a canonical interval. A canonical interval indicates that the domain has reached the precision given by the user or the solver\\
  \hline\\
\end{tabular}

\noindent See \hyperlink{advanced}{advanced uses} for more informations on \texttt{RealVar}.

\subsection{from \texttt{TaskVariable} to \texttt{TaskVar}}\label{solver:solverandtaskvariables}\hypertarget{solver:solverandtaskvariables}{}

For task variables, the solver \textbf{\tt TaskVar} is counterpart to the model \hyperlink{taskvariable}{\textbf{\tt TaskVariable}}. 
Methods \textbf{\tt getVar(TaskVariable var)} and \textbf{\tt getVar(TaskVariable... vars)} of \texttt{Solver} return the objects \texttt{TaskVar} and \texttt{TaskVar[]} respectively corresponding to \texttt{var} and \texttt{vars}:
\begin{lstlisting}
	TaskVariable x = Choco.makeTaskVar("x", 0, 123, 18); // model variable
	TaskVar xOnSolver = solver.getVar(x); // solver variable
\end{lstlisting}

Task variables help at formulating scheduling problems where one has to determine the starting and ending time of a task. A task variable aggregates three integer variable: \texttt{start}, \texttt{end}, and \texttt{duration} and the implicit constraint \texttt{start}+\texttt{duration}=\texttt{end}. 

\todo{To complete.}
The state of a \texttt{TaskVar} can be accessed through these main public methods:

\noindent\begin{tabular}{p{.3\linewidth}p{.7\linewidth}}
  \hline
  \texttt{TaskVar} API &  description \\
  \hline
	\mylst{isInstantiated()} &checks if the three time integer variables are instantiated\\
  \hline\\
\end{tabular}

\noindent See \hyperlink{advanced}{advanced uses} for more informations on \texttt{TaskVar}.

\section{Constraints reading}\label{solver:constraintsreading}\hypertarget{solver:constraintsreading}{}
Once the solver variables are created when reading the model, the solver then iterates over the constraints of the model, and creates the solver \texttt{SConstraint} objects  by calling method \texttt{makeConstraint} of the \texttt{ConstraintManager} object associated to the model constraint type.
At this step, auxiliary solver variables and constraints may be generated. The created constraints are then added to the internal constraint network. 

Each solver constraint encapsulates a filtering algorithm which is called, during the search, when a propagation step occurs or when an external event (e.g., value removal or bound modification) happens on some variable of the constraint.

One can access the Solver representation of a Model constraint, using the \texttt{Solver} API \mylst{getCstr(Constraint c)}.

\section{Solve a problem}\label{solver:solveaproblem}\hypertarget{solver:solveaproblem}{}
%As Solver is the second element of a Choco program, the control of the search process without using predefined tools is made on the Solver.

Table below presents the different API offered by \texttt{Solver} to launch the problem resolution. All these methods return a \texttt{Boolean} object standing for the \emph{problem feasibility status} of the solver:
$$\begin{cases}
  \texttt{Boolean.TRUE} &\text{ if at least one feasible solution has been computed},\\
  \texttt{Boolean.FALSE} &\text{ if the problem is proved to be infeasible},\\
  \texttt{null} &\text{ otherwise, i.e. when a search limit has been reached before.}
\end{cases}$$

\noindent\begin{tabular}{p{.4\linewidth}p{.6\linewidth}}
  \hline
  \texttt{Solver} API & description \\
  \hline
      \mylst{solve()} or \mylst{solve(false)} &  runs backtracking until reaching \emph{a first feasible solution} (returns \mylst{Boolean.TRUE}) or the proof of infeasibility (returns \mylst{Boolean.FALSE}) or a search limit (returns \mylst{null}).\\[.3em]
      \hline\\
      \mylst{nextSolution()} &  Can only be called after a \texttt{solve()} or a \texttt{nextSolution()} call that has returned \mylst{Boolean.TRUE}. Runs backtracking, from the solution leaf reached by the previous \texttt{solve()} or \texttt{nextSolution()} call, until reaching \emph{a new feasible solution} (returns \mylst{Boolean.TRUE}), or proving no such new solution exists (returns \texttt{Boolean.FALSE}), or reaching a search limit (returns \mylst{null}).\\[.3em]
      \hline\\
      \mylst{isFeasible()} &  Returns the feasibility status of the solver.\\
      \hline\\
      \mylst{solveAll()} or \mylst{solve(true)} &  Runs backtracking until computing \emph{all feasible solutions}, or until proving infeasibility (returns \mylst{Boolean.FALSE}) or until reaching a search limit (returns \mylst{Boolean.TRUE} if at least one first solution was computed, and \mylst{null} otherwise). \\[.3em]
      \hline\\
      \mylst{maximize(Var obj, boolean restart),}\mylst{maximize(boolean restart)} &  Runs branch-and-bound until reaching \emph{a feasible solution that is proved to maximize objective} \mylst{obj},  or until proving infeasibility (returns \mylst{Boolean.FALSE}) or until reaching a search limit (returns \mylst{Boolean.TRUE} if at least one first solution was computed, and \mylst{null} otherwise). It proceeds by successive improvements of the best solution found so far: each time a feasible solution is found at a leaf of the tree search, then the search proceeds for a new solution with a greater objective, until it proves that no such improving solution exists.
Parameter \texttt{restart} is a boolean indicating whether the search continues from the solution leaf with a backtrack (if set to \mylst{false}) or if it is restarted from the root node (if set to \texttt{true}).\\
\hline\\
      \mylst{minimize(Var obj, boolean restart),}\mylst{minimize(boolean restart)} &  similar to \texttt{maximize} but for computing \emph{a feasible solution that is proved to minimize objective} \texttt{obj}.\\[.3em]      \hline\\
	\end{tabular}

The following  API are also useful to manipulate a \texttt{Solver} object:\\
\noindent\begin{tabular}{p{.4\linewidth}p{.6\linewidth}}
  \hline
  \texttt{Solver} API & description \\
  \hline
      \mylst{propagate()} &  Launchs propagation by running, in turn, the domain reduction algorithms of the constraints until it reaches a fix point. Throws a \texttt{ContradictionException} when a contradiction is detected, i.e. a variable domain is emptied. This method is called at each node of the tree search constructed by the solving methods above.\\[.3em]
      \hline\\
	\end{tabular}

\section{Storing et restoring solutions}\label{}\hypertarget{}{}
By default, the last solution found is stored and accessible once the search has ended.

\paragraph{Store solutions.} To store the solutions found during the resolution, one just needs to override the solution pool capacity (which is set to 1 by default).
\begin{lstlisting}
	int size = Integer.MAX_VALUE;
	solver.getConfiguration().putInt(Configuration.SOLUTION_POOL_CAPACITY, size);
	solver.solveAll();
	ISolutionPool pool = solver.getSearchStrategy().getSolutionPool();
\end{lstlisting}

Setting \textit{size} to \mylst{Integer.MAX\_VALUE} means ``store every solution'', otherwise you can set a finite value, for example 10 and the last 10 solutions will be stored.

A Solution object is made of arrays of values: an array of values of \mylst{IntVar}, another one for values of \mylst{SetVar} and the last one for the values of \mylst{RealVar}. One can retrieve the value of a variable by calling \mylst{getXXValue(int idx)}, where \mylst{XX} is the type of variable and \mylst{idx} is its index within the solver:

\begin{lstlisting}
	solution.getIntValue(solver.getIntVarIndex(ivar));
	solution.getSetValue(solver.getSetVarIndex(svar));
	solution.getRealValue(solver.getRealVarIndex(fvar));
\end{lstlisting}

And to make the link between a variable in a solution, simply retrieve the index of the variable in the solver:
\begin{lstlisting}
	int q0 = solver.getIntVarIndex(solver.getVar(vars[0])); // vars is an array of IntegerVariable
	solution.getIntValue(q0);
\end{lstlisting}

\paragraph{Restore solutions.} A solution can be restored within a solver. But to do that, some precautions must be made: before starting the search, the state of the solver must be backed up.
\begin{lstlisting}
	int rootworld = solver.getEnvironment().getWorldIndex();
	solver.worldPush();
	solver.solveAll();
\end{lstlisting}

And then, to restore, first restore the root world and backup another world:
\begin{lstlisting}
	solver.worldPopUntil(rootworld); // restore the original state, where domains were as declared (not yet instantiated)
	solver.worldPush(); // backup the current state of the solver, to allow other solution restoration
	solver.restoreSolution(solution); // restore the solution
	// do something
\end{lstlisting}
If you want to restore more than one solution, you just apply this 3 steps.

\section{Search Strategy}\label{solver:searchstrategy}\hypertarget{solver:searchstrategy}{}

\newglossaryentry{branching strategy}{name={branching strategy}, plural={branching strategies},description={heuristic controlling the execution of a search loop at a point where the control flow may be split between different branches}}
\newglossaryentry{search strategy}{name={search strategy}, plural={search strategies},description={composition of branching strategies}}

A key ingredient of any constraint approach is a clever \gls{search strategy}. 
In backtracking or branch-and-bound approaches, the search is organized as an enumeration tree, where each node corresponds to a subspace of the search, and each child node is a subdivision of its father node's space.
The tree is progressively constructed by applying a series of \glspl{branching strategy} that determine how to subdivise space at each node and in which order to explore the created child nodes. Branching strategies play the role of achieving intermediate goals in logic programming. 

This section presents how to define your own search strategy in Choco. 
\begin{note}
  Standard backtracking or branch-and-bound approaches in constraint programming develop the enumeration tree in a \textbf{Depth-First Search (DFS)} manner:
  \begin{enumerate}
  \item \emph{evaluate} a node: run propagation 
  \item if a failure occurs or if the search space cannot be separated then \emph{backtrack}: evaluate the next pending node
  \item otherwise \emph{branch}: divide the search space and evaluate the first child node.
  \end{enumerate}
  With Choco, the search process of the \texttt{CPSolver} does not currently allow to explore the tree in a different manner, using Best-First Search for example. 

  In addition, the common way of dividing the search space in CP-based backtracking/B\&B algorithms is \textbf{to assign a variable to a value or to forbid this assignment}. Choco provides such a branching strategy and the tools to easily customize the variable and value selection heuristics within this strategy. However, Choco makes possible to implement \textbf{more complex branching strategies} (e.g. constraint branching or dichotomy branching).
\end{note}

%The user may specify the sequence of branching strategies to be used to build the search tree. We will present in this section how to define your branching strategies.

\subsection{Overriding the default search strategy}\label{solver:overridethedefaultsearchstrategy}\hypertarget{solver:overridethedefaultsearchstrategy}{}

\newglossaryentry{value selector}{name={value selector}, plural={value selector},description={heuristic specifying how to choose a value from a chosen variable at a fix point}}
\newglossaryentry{value iterator}{name={value iterator}, plural={value iterators},description={heuristic specifying how to choose a value from a chosen variable, through an iterator, at a fix point}}
\newglossaryentry{variable selector}{name={variable selector}, plural={variable selectors},description={heuristic specifying how to choose a variable at a fix point}}


%Basically, a search strategy is the composition of three objects: a \gls{branching strategy}, a \gls{variable selector} and a \gls{value selector} (or a \gls{value iterator}). Some branching strategies simply assign a selected value to a selected variable, like \hyperlink{assignvar:assignvarbranchstrat}{AssignVar}, others branching strategies embed the variable selector, like \hyperlink{domoverwdeg:domoverwdegbranchstrat}{DomOverWDegBranchingNew}, or more, like  \hyperlink{impact:impactbranchstrat}{ImpactBasedBranching}.
\paragraph{Branching, variable selection and value selection strategies.}
Basically, a search strategy in Choco is a composition of \gls{branching strategy} objects, each defined on a given set of decision variables.
The most common branching strategies are based on the assignment of a selected variable to one or several selected values (one assignment in each branch). 
\begin{note}
Branching strategies apply to \texttt{Solver} variables (not \texttt{Model} variables).
\end{note}
The variable and value selection heuristics can be defined separately in their own objects: a \gls{variable selector} and a \gls{value selector} or a \gls{value iterator}. 
Branching strategy \hyperlink{assignvar:assignvarbranchstrat}{\tt AssignVar}, for example, can simply be customized via these embedded objects: the variable is first selected, then a value in the variable domain is selected. The two following instructions both create a $n$-ary branching strategy (\texttt{AssignVar}) selecting an integer decision variable of minimum domain size (\texttt{MinDomain} variable selector) and assigning it successively, in each branch, to one of its domain value, selected in increasing order (\texttt{IncreasingDomain} value iterator or \texttt{MinVal} value selector).
\begin{lstlisting}
  new AssignVar(new MinDomain(solver), new IncreasingDomain());
  new AssignVar(new MinDomain(solver), new MinVal());
\end{lstlisting}
Note that this usual strategy is pre-defined in \texttt{BranchingFactory}, and may then also be declared as follows:
\begin{lstlisting}
  BranchingFactory.minDomMinVal(solver);
\end{lstlisting}
Sometimes, the choice of the variable may also depend on the choice of the value, or it may require specific computations before or after branching. In this case, the variable selection heuristic can directly be implemented in the branching strategy object, like e.g. in \hyperlink{domoverwdeg:domoverwdegbranchstrat}{DomOverWDegBranchingNew}. Both variable and value selection heuristics can be implemented directly within the branching strategy, like e.g in \hyperlink{impact:impactbranchstrat}{ImpactBasedBranching}.

\paragraph{Default strategies.}
When no search strategy is specified, default search strategies apply to all the decision variables of the solver.
These strategies vary according to the variable types: 

\noindent\begin{tabular}{p{.25\linewidth}p{.7\linewidth}}
\hline
Variable type &  Default strategy \\
\hline
Set &   \hyperlink{assignsetvar:assignsetvarbranchstrat}{AssignSetVar} + \hyperlink{mindomset:mindomsetvarselector}{MinDomainSet} + \hyperlink{minenv:minenvvalselector}{MinEnv} \\
Integer & \hyperlink{domoverwdeg:domoverwdegbranchstrat}{DomOverWDegBranchingNew} +\hyperlink{increasingdomain:increasingdomainvaliterator}{IncreasingDomain}\\
 Real &  \hyperlink{assigninterval:assignintervalbranchstrat}{AssignInterval} + \hyperlink{cyclicrealvarselector:cyclicrealvarselectorvarselector}{CyclicRealVarSelector}+ \hyperlink{realincreasingdomain:realincreasingdomainvaliterator}{RealIncreasingDomain} \\
\hline\\
\end{tabular}
If the model has decision variables of different types, then these default branchings are evaluated in this order: first, the set decision variables are considered until they are all instantiated, then branching occurs on the pool of integer decision variables, and last on the pool of real decision variables.

\paragraph{Decision variables.}
Branchings apply to decision variables only. A branching can occur (i.e. the tree node can be separated according to this strategy) if and only if there exists a decision variable in its scope that is still not instantiated.
The non-decision variables are also called \emph{implied variables} because it is expected that, all variables -- including these -- will be instantiated (i.e. they will form a solution) by propagation as soon as all the decision variables will be instantiated. Consider for example, a problem with two sets of variables $x$ and $y$ linked by channeling or by some implication $x=S\implies y=T$ then the $x$ variables can be set as the decision variables, the $y$ will be instantiated by side-effect. 

By default, every solver variable belongs to the pool of decision variables, unless:
\begin{itemize}
\item it corresponds to a model variable created with flag \hyperlink{vnodecision:vnodecisionoptions}{\tt Options.V\_NO\_DECISION};
\item or it is internally created by the solver (e.g. when reading some model constraint) and explicitely excluded from the pool;
\item or the default branching strategies are overriden and the variable does not belong to the scope of one of the strategies specified by the user.
\end{itemize}

The scope of a branching strategy is defined at the creation of the strategy. For example,
\begin{lstlisting}
  new AssignSetVar(new MinDomSet(solver, solver.getVar(svars)), new MinEnv()));
\end{lstlisting}
defines a branching strategy that only applies to the solver variables corresponding to the model set variables \texttt{svars} (even if they were defined with flag \hyperlink{vnodecision:vnodecisionoptions}{\tt Options.V\_NO\_DECISION}).

Most branching strategies may be declared without specifying their scope. In this case, they apply to all the solver decision variables of the right type. For example, the branching strategy
\begin{lstlisting}
  new AssignSetVar(new MinDomSet(solver), new MinEnv()));
\end{lstlisting}
now applies to all the solver decision set variables.

If the default strategies of the solver are overridden by this strategy alone, then all other integer and real variables will automatically be removed from the decision pool: one has then to ensure that the instantiation of the set variables alone defines a complete solution.
If it is not the case, the branching strategy must be combined with additional branching strategies holding on the remaining unimplied variables.
\begin{note}
  If the default strategies are overriden, then the pool of decision variables is overriden by the union of the scopes of the user-specified branching strategies.
\end{note}
As the branching strategies are evaluated sequentially, a variable may belong to the scope of two different strategies, but it will only be considered by the first strategy, unless this first (user-defined) strategy let the variable un-instantiated.

\paragraph{Overriding the default search strategies.}
A branching strategy is added to the solver, as a goal, using the following API of \texttt{Solver}:
\begin{lstlisting}
  void addGoal(AbstractIntBranchingStrategy branching);
\end{lstlisting}
This method must be called on the solver object \emph{before} calling the solving method.
The initial list of goals is empty. If goals are specified, they are added to the list in the order of their declaration.
Otherwise, the list is initialized with the default goals (in the order: set, integer, real).

When one wants to relaunch the search, the list of goals of the solver can previously be reset using the following instruction:
\begin{lstlisting}
  solver.clearGoals();
\end{lstlisting} 

\paragraph{Complete example.}
The following example adds four branching objects to solver \texttt{s}. 
%on integer variables \texttt{vars1}, \texttt{vars2} and set variables \texttt{svars} . 
The first two branchings are both \texttt{AssignVar} strategies using different variable/value selection heuristics and applied to different scopes: the integer variables \texttt{vars1} and \texttt{vars2}, respectively. The third strategy applies to the set variables \texttt{svars}. The last random strategy applies to all the integer decision variables of the solver.
%  s.addGoal(new AssignVar(new MinDomain(s,s.getVar(vars1)), new IncreasingDomain()));
\begin{lstlisting}
  s.addGoal(BranchingFactory.minDomMinVal(s, s.getVar(vars1)));
  s.addGoal(new AssignVar(new DomOverDeg(s, s.getVar(vars2)), new DecreasingDomain()));
  s.addGoal(new AssignSetVar(new MinDomSet(s, s.getVar(svars)), new MinEnv()));
  s.addGoal(BranchingFactory.randomIntSearch(s, seed));
  s.solve();
\end{lstlisting}
The goals are evaluated in this order: first, variables \texttt{vars1} are considered until they are all instantiated, then branching occurs on variables \texttt{vars2}, then on variables \texttt{svars}. Finally, a random strategy is applied to all the integer decision variables of the solver that are not already instantiated, thereby excluding variables \texttt{vars1} and \texttt{vars2}.

\subsection{Pre-defined search strategies}\label{solver:predefinedsearchstrategy}\hypertarget{solver:predefinedsearchstrategy}{}

This section presents the strategies available in Choco. These objects are also detailed in Part \hyperlink{part:elements}{Elements of Choco}.
See Chapter \hyperlink{advanced}{advanced uses} for a description of how to write search strategies in Choco.

\paragraph{Branching strategy}\label{solver:branchstrat}\hypertarget{solver:branchstrat}{}
defines the way to branch from a tree search node.
  
\noindent The \textbf{branching strategies} currently available in Choco are the following: 
\begin{notedef}\tt
\hyperlink{assigninterval:assignintervalbranchstrat}{AssignInterval}, \hyperlink{assignorforbidintvarval:assignorforbidintvarvalbranchstrat}{AssignOrForbidIntVarVal}, \hyperlink{assignorforbidintvarvalpair:assignorforbidintvarvalpairbranchstrat}{AssignOrForbidIntVarValPair}, \hyperlink{assignsetvar:assignsetvarbranchstrat}{AssignSetVar}, \hyperlink{assignvar:assignvarbranchstrat}{AssignVar}, \hyperlink{domoverwdeg:domoverwdegbranchstrat}{DomOverWDegBranchingNew}, \hyperlink{domoverwdegbin:domoverwdegbinbranchstrat}{DomOverWDegBinBranchingNew}, \hyperlink{impact:impactbranchstrat}{ImpactBasedBranching}, \hyperlink{packdynremovals:packdynremovalsbranchstrat}{PackDynRemovals}, \hyperlink{settimes:settimesbranchstrat}{SetTimes}, \hyperlink{taskdomoverwdeg:taskdomoverwdegbranchstrat}{TaskOverWDegBinBranching}.
\end{notedef} 
They implement interface \texttt{BranchingStrategy}.   


\paragraph{Variable selector}\label{solver:variableselector}\hypertarget{solver:variableselector}{}
defines the way to choose a non instantiated variable on which the next decision will be made.

\noindent The \textbf{variable selectors} currently available in Choco are the following: 
\begin{itemize}
\item implementing interface \texttt{VarSelector<IntDomainVar>}:
\begin{notedef}\tt
\hyperlink{compositeintvarselector:compositeintvarselectorvarselector}{CompositeIntVarSelector}, \hyperlink{lexintvarselector:lexintvarselectorvarselector}{LexIntVarSelector}, \hyperlink{maxdomain:maxdomainvarselector}{MaxDomain}, \hyperlink{maxregret:maxregretvarselector}{MaxRegret}, \hyperlink{maxvaldomain:maxvaldomainvarselector}{MaxValueDomain}, \hyperlink{mindomain:mindomainvarselector}{MinDomain}, \hyperlink{minvaldomain:minvaldomainvarselector}{MinValueDomain}, \hyperlink{mostconstrained:mostconstrainedvarselector}{MostConstrained},  \hyperlink{randomvarint:randomvarintvarselector}{RandomIntVarSelector},  \hyperlink{staticvarorder:staticvarordervarselector}{StaticVarOrder}
\end{notedef}
%\noindent The \textbf{set variable selectors} currently available in Choco are the following: 
\item implementing interface \texttt{VarSelector<SetVar>}:   
\begin{notedef}\tt
\hyperlink{maxdomset:maxdomsetvarselector}{MaxDomSet}, \hyperlink{maxregretset:maxregretsetvarselector}{MaxRegretSet}, \hyperlink{maxvaldomset:maxvaldomsetvarselector}{MaxValueDomSet}, \hyperlink{mindomset:mindomsetvarselector}{MinDomSet}, \hyperlink{minvaldomset:minvaldomsetvarselector}{MinValueDomSet}, \hyperlink{mostconstrainedset:mostconstrainedsetvarselector}{MostConstrainedSet},  \hyperlink{randomvarset:randomvarsetvarselector}{RandomSetVarSelector},  \hyperlink{staticsetvarorder:staticsetvarordervarselector}{StaticSetVarOrder}
\end{notedef}
%\noindent The \textbf{real variable selector} currently available in Choco is the following: 
\item implementing interface \texttt{VarSelector<RealVar>}:
\begin{notedef}\tt
\hyperlink{cyclicrealvarselector:cyclicrealvarselectorvarselector}{CyclicRealVarSelector}
\end{notedef}
\end{itemize}

\subsubsection{Value iterator}\label{solver:valueiterator}\hypertarget{solver:valueiterator}{}
Once the variable has been choosen, the solver has to compute its value. The first way to do it is to schedule all the values once and to give an iterator to the solver.

\noindent The \textbf{value iterators} currently available in Choco are the following: 
\begin{itemize}
\item implementing interface \texttt{ValIterator<IntDomainVar>}:
\begin{notedef}\tt
\hyperlink{decreasingdomain:decreasingdomainvaliterator}{DecreasingDomain}, \hyperlink{increasingdomain:increasingdomainvaliterator}{IncreasingDomain}
\end{notedef}
\item implementing interface \texttt{ValIterator<RealVar>}:
\begin{notedef}\tt
\hyperlink{realincreasingdomain:realincreasingdomainvaliterator}{RealIncreasingDomain}
\end{notedef}
\end{itemize}

\subsubsection{Value selector}\label{solver:valueselector}\hypertarget{solver:valueselector}{}
The second way to do it is to compute the next value at each call.

\noindent The \textbf{integer value selector} currently available in Choco are the following: 
\begin{itemize}
\item implementing interface \texttt{ValSelector<IntDomainVar>}:
\begin{notedef}\tt
  \begin{itemize}
  \item \hyperlink{maxval:maxvalvalselector}{MaxVal}, \hyperlink{midval:midvalvalselector}{MidVal}, \hyperlink{minval:minvalvalselector}{MinVal}
  \item \hyperlink{bestfit:bestfitvalselector}{BestFit}, \hyperlink{costregularvalselector:costregularvalselectorvalselector}{CostRegularValSelector}, \hyperlink{fcostregularvalselector:fcostregularvalselectorvalselector}{FCostRegularValSelector}, \hyperlink{randomintvalselector:randomintvalselectorvalselector}{RandomIntValSelector}, %\hyperlink{mcrvalselector:mcrvalselectorvalselector}{MCRValSelector}, 
  \end{itemize}
\end{notedef}
\item implementing interface \texttt{ValSelector<SetVar>}:
  \begin{notedef}\tt
\hyperlink{minenv:minenvvalselector}{MinEnv}, \hyperlink{randomsetvalselector:randomsetvalselectorvalselector}{RandomSetValSelector}
\end{notedef}
\end{itemize}


\subsection{Why is it important to define a search strategy ?}\label{solver:whyisitimportanttodefineasearchstrategy}\hypertarget{solver:whyisitimportanttodefineasearchstrategy}{}

%At a In a partial instantiation, when a fix point has been reached, the Solver needs to take a decision to resume the search. The way decisions are chosen has a \textbf{real impact on the resolution step efficient}. 
\begin{note}
\emph{The search strategy should not be under-estimatimated!!}
A well-suited search strategy can reduce: the execution time, the number of expanded nodes, the number of backtracks.
\end{note}
Let see that small example:
\begin{lstlisting}
	Model m = new CPModel();
        int n = 1000;
        IntegerVariable var = Choco.makeIntVar("var", 0, 2);
        IntegerVariable[] bi = Choco.makeBooleanVarArray("b", n);
        m.addConstraint(Choco.eq(var, Choco.sum(bi)));

        Solver badStrat = new CPSolver();
        badStrat.read(m);
        badStrat.addGoal(
                new AssignVar(
                        new MinDomain(badStrat), 
                        new IncreasingDomain()
                ));
        badStrat.solve();
        badStrat.printRuntimeStatistics();

        Solver goodStrat = new CPSolver();
        goodStrat.read(m);
        goodStrat.addGoal(
                new AssignVar(
                        new MinDomain(goodStrat, goodStrat.getVar(new IntegerVariable[]{var})), 
                        new IncreasingDomain()
                ));
        goodStrat.solve();
        goodStrat.printRuntimeStatistics();
\end{lstlisting}

This model ensures that $var = b_{0} + b_{1} + \ldots + b_{1000}$ where $var$ is an integer variable with a small domain $[0,2]$ and $b_{i}$ are binary variables. No deduction arose from the propagation here, so a fix point is reached at the beginning of the search. A branching decision has to be taken, by selecting a variable and the first value to assign to it. Using the first strategy, the solver will find a solution after creating 1001 nodes: it iterates over all the variables, starting by assigning the 1000 binary variables $b_i$ (according to the \texttt{MinDomain} variable selector) to $0$ (according to the \texttt{IncreasingDomain} value iterator), variable $var$ is fixed to $0$ at the very last propagation. The second strategy finds the same solution with only two nodes: after branching first on $var=0$, propagation immediately fixes all the binary variables to $0$. 

\subsection{Restarts}\label{solver:restarts}\hypertarget{solver:restarts}{}

Restart means stopping the current tree search, then starting a new tree search from the root node.
Restarting makes sense only when coupled with randomized dynamic branching strategies ensuring that the same enumeration tree is not constructed twice. The branching strategies based on the past experience of the search, such as \texttt{DomOverWDegBranching}, \texttt{DomOverWDegBinBranching} and \texttt{ImpactBasedBranching}, are more accurate in combination with a restart approach.

Unless the number of allowed restarts is limited, a tree search with restarts is not complete anymore. It is a good strategy, though, when optimizing an NP-hard problem in a limited time.


Restarts can be set using the following API available on the \texttt{Solver}:
\begin{lstlisting}
setGeometricRestart(int base, double grow);
setGeometricRestart(int base, double grow, int restartLimit);
\end{lstlisting}
It performs a search with restarts controlled by the number of backtracks. 
Parameter \texttt{base} indicates the maximal number of backtracks allowed in the first search tree. Once this limit is reached, a restart occurs and the search continues until \texttt{base}*\texttt{grow} backtracks are done, and so on. After each restart, the limit number of backtracks is increased by the geometric factor \texttt{grow}. 
\texttt{restartLimit} states the maximum number of restarts.
\begin{lstlisting}
	CPSolver s = new CPSolver();
	s.read(model);
	
	s.setGeometricRestart(14, 1.5d);
	s.setFirstSolution(true);
	s.generateSearchStrategy();
	s.attachGoal(new DomOverWDegBranching(s, new IncreasingDomain()));
	s.launch();
\end{lstlisting}

The Luby's restart policy is an alternative to the geometric restart policy, and can be defined 
using the following API available on the \texttt{Solver}:
\begin{lstlisting}
setLubyRestart(int base);
setLubyRestart(int base, int grow);
setLubyRestart(int base, int grow, int restartLimit);
\end{lstlisting}
It performs a search with restarts controlled by the number of backtracks. 
The maximum number of backtracks allowed at a given restart iteration is given by \texttt{base} multiplied by the Las Vegas coefficient at this iteration. 
The sequence of these coefficients is defined recursively on its prefix subsequences: starting from the first prefix $1$, the $(k+1)$-th prefix is the $k$-th prefix repeated \texttt{grow} times and immediately followed by coefficient \texttt{grow}$^k$.
\begin{itemize}
	\item the first coefficients for \texttt{grow}=2 : [1, 1, 2, 1, 1, 2, 4, 1, 1, 2, 1, 1, 2, 4, 8, 1,...]
	\item the first coefficients for \texttt{grow}=3 : [1, 1, 1, 3, 1, 1, 1, 3, 1, 1, 1, 3, 9,...]
\end{itemize}

\begin{lstlisting}
	CPSolver s = new CPSolver();
	s.read(model);
	
	s.setLubyRestart(50, 2, 100);
	s.setFirstSolution(true);
	s.generateSearchStrategy();
	s.attachGoal(new DomOverWDegBranching(s, new IncreasingDomain()));
	s.launch();
\end{lstlisting}

\section{Limiting Search Space}\label{solver:limitingsearchspace}\hypertarget{solver:limitingsearchspace}{}
The \texttt{Solver} class provides ways to limit the tree search controlled by different criteria.
%Limits may be imposed on the search algorithm to avoid spending too much time in the exploration. 
These limits have to be specified before the resolution. They are updated and checked each time a new node is created. 
 Once a limit is reached, the search stops even if no solution is found.
\begin{description}
\item[time limit:] concerns the elapsed time from the beginning of the search (i.e. from the call to a resolution method).
A time limit is set using the \texttt{Solver} API \mylst{setTimeLimit(int timeLimit)}, where \textit{timeLimit} is in milliseconds. 
\mylst{getTimeCount()} returns the total solving time. 
\item[node limit:] concerns the number of opened nodes.
A node limit is set using the \texttt{Solver} API \mylst{setNodeLimit(int nodeLimit)}.
\mylst{getNodeCount()} returns the total number of explored nodes.
\item[backtrack limit:] concerns the number of performed backtracks. 
A backtrack limit is set using the \texttt{Solver} API \mylst{setBackTrackLimit(int backtrackLimit)}.
\mylst{getBackTrackCount()} return the total number of backtracks.
\item[fail limit:] concerns the number of contradiction encountered.
A fail limit is set using the \texttt{Solver} API \mylst{setFailLimit(int failLimit)}.
\mylst{getFailCount()} returns the total number of failures.
By default, the failure count is recorded, one should call \mylst{monitorFailLimit(true)} to activate it.
\end{description}
\todo{Define all these notions more precisely and add an example.}

\section{Logging the search}\label{solver:logs}\hypertarget{solver:logs}{}
A logging class is available to produce trace statements throughout search: \mylst{ChocoLogging}. 

\subsection{Architecture }\label{solver:logarchitecture}\hypertarget{solver:logarchitecture}{}

Choco logging system is based on the \mylst{java.util.logging} package and located in the package \mylst{common.logging}.
Most Choco abstract classes or interfaces propose a static field \mylst{LOGGER}.
The following figures present the architecture of the logging system with the default verbosity.

\insertGraphique{.7\linewidth}{media/logger-default.png}{Logger Tree with the default verbosity}

The shape of the node depicts the type of logger:
\begin{itemize}
	\item The \emph{house} loggers represent private loggers. Do not use these loggers directly because their level are low and all messages would always be displayed.
	\item The \emph{octagon} loggers represent critical loggers. These loggers are provided in the variables, constraints and search classes and could have a huge impact on the global performance.
	\item The \emph{box} loggers are provided for dev and users.
\end{itemize}
The color of the node gives its logging level with DEFAULT verbosity:
\texttt{Level.FINEST} (\textcolor{yellow}{gold}),
\texttt{Level.INFO} (\textcolor{orange}{orange}),
\texttt{Level.WARNING} (\textcolor{red}{red}).

\subsection{Verbosities and messages}\label{solver:verbosityandmessages}\hypertarget{solver:verbosityandmessages}{}

The verbosity level of the solver can be set by the following static method :

\begin{lstlisting}
	// Before the resolution
	ChocoLogging.toVerbose();
	//... resolution declaration
	solver.solve();
	// And after the resolution
	ChocoLogging.flushLogs();
\end{lstlisting}


The following table summarizes the verbosities available in choco: 
\begin{itemize}
	\item \textbf{OFF -- level 0:} Disable logging.	 

	\vspace{0.2cm} 
	\textit{Usage} :  \mylst{ChocoLogging.setVerbosity(Verbosity.OFF)}
	
	\item \textbf{SILENT -- level 1:} Display only severe messages.
	
	\vspace{0.2cm} 
	\textit{Usage} : 
		\begin{itemize}
		\item \mylst{ChocoLogging.toSilent()}
		\item \mylst{ChocoLogging.setVerbosity(Verbosity.SILENT)}
		\end{itemize}
	
	\item \textbf{DEFAULT -- level 2:} Display informations on final search state.
		\begin{itemize}
			\item ON START
				\lstset{language={sh},columns=fixed}
\begin{lstlisting}
 ** CHOCO : Constraint Programming Solver
 ** CHOCO v2.1.1 (April, 2009), Copyleft (c) 1999-2010
 \end{lstlisting}
			\item ON COMPLETE SEARCH:
				\begin{lstlisting}
- Search completed -
 [Maximize		: {0},]
 [Minimize		: {1},]
  Solutions		: {2},
  Times (ms)	: {3},
  Nodes			: {4},
  Backtracks	: {5},
  Restarts		: {6}.
  \end{lstlisting}
	brackets [\textit{line}] indicate \textit{line} is optional,\\
 	\texttt{Maximize} --resp. \texttt{Minimize}-- indicates the best known value before exiting of the objective value in \textit{maximize()} --resp. \textit{minimize()}-- strategy.

			\item ON COMPLETE SEARCH WITHOUT SOLUTIONS :
				\begin{lstlisting}
- Search completed - No solutions
 [Maximize		: {0},]
 [Minimize		: {1},]
  Solutions		: {2},
  Times (ms)	: {3},
  Nodes			: {4},
  Backtracks	: {5},
  Restarts		: {6}.
\end{lstlisting}
	brackets [\textit{line}] indicate \textit{line} is optional,\\
 	\texttt{Maximize} --resp. \texttt{Minimize}-- indicates the best known value before exiting of the objective value in \textit{maximize()} --resp. \textit{minimize()}-- strategy.

			\item ON INCOMPLETE SEARCH:
				\begin{lstlisting}
- Search incompleted - Exiting on limit reached
  Limit			: {0},
 [Maximize		: {1},]
 [Minimize		: {2},]
  Solutions		: {3},
  Times (ms)	: {4},
  Nodes			: {5},
  Backtracks	: {6},
  Restarts		: {7}.
  
  \end{lstlisting}
	brackets [\textit{line}] indicate \textit{line} is optional,\\
 	\texttt{Maximize} --resp. \texttt{Minimize}-- indicates the best known value before exiting of the objective value in \textit{maximize()} --resp. \textit{minimize()}-- strategy.
		\end{itemize}			

	\textit{Usage} : 
		\begin{itemize}
		\item \mylst{ChocoLogging.toDefault()}
		\item \mylst{ChocoLogging.setVerbosity(Verbosity.DEFAULT)}
		\end{itemize}

	\item \textbf{VERBOSE -- level 3:} Display informations on search state.
		\begin{itemize}
			\item EVERY X (default=1000) NODES:
			\begin{lstlisting}
- Partial search - [Objective : {0}, ]{1} solutions, {2} Time (ms), {3} Nodes, {4} Backtracks, {5} Restarts.
			\end{lstlisting}
			\texttt{Objective} indicates the best known value.

			\item ON RESTART : 
			\begin{lstlisting}
- Restarting search - {0} Restarts.
			\end{lstlisting}
		\end{itemize}
		
		\textit{Usage} : 
		\begin{itemize}
		\item \mylst{ChocoLogging.toVerbose()}
		\item \mylst{ChocoLogging.setVerbosity(Verbosity.VERBOSE)}
		\end{itemize}

	\item \textbf{SOLUTION -- level 4:} display all solutions.
		\begin{itemize}
			\item AT EACH SOLUTION:
			\begin{lstlisting}
- Solution #{0} found. [Objective: {0}, ]{1} Solutions, {2} Time (ms), {3} Nodes, {4} Backtracks, {5} Restarts.
  X_1:v1, x_2:v2...
			\end{lstlisting}
		\end{itemize}
		
		\textit{Usage} : 
		\begin{itemize}
		\item \mylst{ChocoLogging.toSolution()}
		\item \mylst{ChocoLogging.setVerbosity(Verbosity.SOLUTION)}
		\end{itemize}

	\item \textbf{SEARCH -- level 5:} Display the search tree.
		\begin{itemize}
			\item AT EACH NODE, ON DOWN BRANCH:
			\begin{lstlisting}
...[w] down branch X==v branch b
			\end{lstlisting}
where \texttt{w} is the current world index, \texttt{X} the branching variable, \texttt{v} the branching value and \texttt{b} the branch index. This message can be adapted on variable type and search strategy.

			\item AT EACH NODE, ON UP BRANCH:
			\begin{lstlisting}
...[w] up branch X==v branch b
			\end{lstlisting}
where \texttt{w} is the current world index, \texttt{X} the branching variable, \texttt{v} the branching value and \texttt{b} the branch index. 
		\end{itemize}
		
		\textit{Usage} : 
		\begin{itemize}
		\item \mylst{ChocoLogging.toSearch()}
		\item \mylst{ChocoLogging.setVerbosity(Verbosity.SEARCH)}
		\end{itemize}

	\item \textbf{FINEST -- level 6:} display all logs.
	
	\vspace{0.2cm} 
	\textit{Usage} :  \mylst{ChocoLogging.setVerbosity(Verbosity.FINEST)}

\end{itemize}

More precisely, if the verbosity level is greater than DEFAULT, then the verbosity levels of the model and of the solver are increased to INFO, and the verbosity levels of the search and of the branching are slightly modified to display the solution(s) and search messages.

\subsection{Basic settings}\label{solver:logbasicsettings}\hypertarget{solver:logbasicsettings}{}

Note that in the case of a verbosity greater or equals to \texttt{toVerbose()}, the regular search information step is set to 1000, by default. You can change this value, using:
\begin{lstlisting}
  ChocoLogging.setEveryXNodes(20000);
\end{lstlisting}
 

Note that in the case of verbosity \texttt{toSearch()}, trace statements are printed up to a maximal depth in the search tree. The default value is set to 25, but you can change the value of this threshold, say to 10, with the following setter method:
\begin{lstlisting}
  ChocoLogging.setLoggingMaxDepth(10);
\end{lstlisting}


\section{Clean a Solver}\label{solver:clean}\hypertarget{solver:clean}{}

Although it is simple and secure to create new instance of \mylst{Solver}, sometimes it  is more obvious and efficient to reuse a \mylst{Solver}. 
It is recommended to reuse instance of \mylst{Solver} when a problem is being solved more than once without deep modifications between two resolutions:
\begin{itemize}
\item a problem is being resolved with different search strategies,
\item a problem is being modified (by adding or removing constraints) through multiple resolutions.
\end{itemize}

\begin{note}
Reusing a \mylst{Solver} must be prepared and well thought out. A backup of the initial state of the \mylst{Solver} may be necessary to recover initial domains and constraints internal structures. In the case where new constraints are created and added, this must be done manipulating \mylst{Solver} objects (SConstraint for example). So, this required knowledge in advanced uses of \mylst{Solver}.
\end{note}

What are the methods to clean up a Solver ?
\begin{itemize}
\item \underline{reset the search strategy :} this is done by calling \mylst{resetSearchStrategy()} on a \mylst{CPSolver}.
A call to this method clears the defined branching strategies (safely removes previous ones) and sets the current search strategy to \mylst{null}.
User's defined branching strategies must be defined again.
User's defined limits will be reset for the next search.

\item \underline{cancel restarts :} this is done by calling \mylst{cancelRestarts(Solver solver)} on \mylst{RestartFactory}.
Sets the restarts parameters to the default ones. This must be done if a restart strategy has been declared.

\end{itemize}

\subsection{What about simply calling \mylst{solver.clear()}?}

A call to \mylst{solver.clear()} will reset every internal structures of a \mylst{Solver}: it clears the variable list, the constraint list, the environment, the propagation engine, the model read, etc.
This set the solver in the same state as it was just after its creation.

\subsection{Things to know about \mylst{Solver} reusability}

There are few things to know:
\begin{itemize}
\item a variable cannot be removed from a \mylst{Solver},
\item a statically posted constraint (added using \mylst{solver.postCut(SConstraint c)}) can be removed \textit{at any time} using \mylst{void eraseConstraint(SConstraint c)},
\item a dynamically posted constraint (added using \mylst{solver.post(SConstraint c)}) can be removed \textit{at root node} using \mylst{void eraseConstraint(SConstraint c)}.
\end{itemize}


%\subsection{Optimization}\label{solver:optimization}\hypertarget{solver:optimization}{}
%\todo{to introduce}
%\begin{lstlisting}
%  Model m = new CPModel();
%  IntegerVariable obj1 = makeEnumIntVar("obj1", 0, 7);
%  IntegerVariable obj2 = makeEnumIntVar("obj1", 0, 5);
%  IntegerVariable obj3 = makeEnumIntVar("obj1", 0, 3);
%  IntegerVariable cost = makeBoundIntVar("cout", 0, 1000000);
%  int capacity = 34;
%  int[] volumes = new int[]{7, 5, 3};
%  int[] energy = new int[]{6, 4, 2};
%  // capacity constraint
%  m.addConstraint(leq(scalar(volumes, new IntegerVariable[]{obj1, obj2, obj3}), capacity));
%	
%  // objective function
%  m.addConstraint(eq(scalar(energy, new IntegerVariable[]{obj1, obj2, obj3}), cost));
%  
%  Solver s = new CPSolver();
%  s.read(m);
%  
%  s.maximize(s.getVar(cost), false);
%\end{lstlisting}
\label{doc:solver}\hypertarget{doc:solver}{}
%\input{chapters/constraints.tex}\label{doc:constraints}\hypertarget{doc:constraints}{}
%!TEX root = ../content-doc.tex
%\part{advanced}
\label{advanced}
\hypertarget{advanced}{}


\chapter{Advanced uses of Choco}\label{advanced:advancedusesofchoco}\hypertarget{advanced:advancedusesofchoco}{}

\section{Environment}\label{advanced:environment}\hypertarget{advanced:environment}{}

Environment is a central object of the backtracking system. It defines the notion of \textit{world}. A world contains values of storable objects or operations that permit to \textit{backtrack} to its state. The environment \textit{pushes} and \textit{pops} worlds.

There are \textit{primitive} data types (\mylst{IStateBitSet, IStateBool, IStateDouble, IStateInt, IStateLong}) and \textit{objects} data types (\mylst{IStateBinarytree, IStateIntInterval, IStateIntProcedure, IStateIntVector, IStateObject, IStateVector}).

There are two different environments: \textit{EnvironmentTrailing} and \textit{EnvironmentCopying}.

\subsection{Copying}\label{advanced:copying}\hypertarget{advanced:copying}{}
In that environment, each data type is defined by a value (primitive or object) and a timestamp. Every time a world is pushed, each value is copied in an array (one array per data type), with finite indice. When a world is popped, every value is restored. 

\subsection{Trailing}\label{advanced:trailing}\hypertarget{advanced:trailing}{}
In that environment, data types are defined by its value. Every operation applied to a data type is pushed in a \textit{trailer}. When a world is pushed, the indice of the last operation is stored. When a world is popped, these operations are popped and \textit{unapplied} until reaching the last operation of the previous world.\\\textit{Default one in CPSolver}

\subsection{Backtrackable structures}\label{advanced:backtrackablestructures}\hypertarget{advanced:backtrackablestructures}{}
\todo{to complete}
\section{How does the propagation engine work ?}

Once the \mylst{Model} and \mylst{Solver} have been defined, the resolution can start. It is based on decisions and filtering orders, this is the propagation engine. In this part, we're going to present how the resolution is guided in Choco. 

A resolution instruction (\mylst{solve()}, \mylst{solveAll()}, \mylst{maximize(...)} or \mylst{minimize(...)}) always starts by setting options based on resolution policy, then generates the search strategies and ends by running the search loop. 

\subsection{How does a search loop work ?}\label{advanced:howdoesasearchloopwork}\hypertarget{advanced:howdoesasearchloopwork}{}
The search loop is the \textit{conductor} of the engine. It goes down and up in the branches in order to cover the tree search, call the filtering algorithm, etc.
Here is the organigram of the search loop. 

%\insertGraphique{0.8\linewidth}{media/searchloop.pdf}{Organigram of the search loop}

\begin{figure}[!htp]
	\centerline{\Graph{media/searchloop.pdf}{width=1\linewidth}}
	\caption[]{Organigram of the search loop}\label{fig:media/searchloop.pdf}
\end{figure}

Basically, the search loop is divided in 5 steps: \mylst{INITIAL PROPAGATION} (highlighted in red), \mylst{OPEN NODE} (highlighted in green), \mylst{DOWN BRANCH} (highlighted in violet), \mylst{UP BRANCH} (highlighted in orange) and \mylst{RESTART} (highlighted in blue). 

\subsection{Propagate}\label{advanced:propagate}\hypertarget{advanced:propagate}{}

%% fix point A fix point is reached when there is no more event to treat
\newglossaryentry{fix point}{name={fix point},description={definition of a fix point}}

The main and unique \mylst{PropagationEngine} of Choco is \mylst{ChocoEngine}. This engine stores events occurring on variables, \textit{variable events}, and specific calls to constraint filtering algorithm, \textit{constraint events}, in order to reach a \gls{fix point} or to detect contradictions. Events are stored in queues (FIFO).  

On a call to \mylst{Solver.propagate()} or during a resolution step, the consistency of a model is computed: stored events are popped and propagated (apply side-effects). The propagation of a single event can create new ones, feeding the system until fix point or contradiction. 



\begin{figure}[!htp]
	\centerline{\Graph{media/propagationloop.pdf}{width=0.5\linewidth}}
	\caption[]{Organigram of the propagation loop}\label{fig:media/propagationloop.pdf}
\end{figure}

 
If the propagation of an event leads to a contradiction, the propagation engine stop the process. In both case, the search loop take up with the new state.

\subsubsection{Seven priorities}

Before going further, it is important to know that events declare a parameter named \textit{priority}. The priority of a constraint's event depends on the constraint priority (required in the super class constructor). And the priority of a variable's event is given by the maximum priority of the variable's constraints. 

To each priority corresponds a \textit{rank}. In the propagation engine, there are as many queues as ranks. 
During the propagation loop, the rank is used to push the event on the corresponding queue. Events are \underline{always} popped from the smallest ranked queue to the largest ranked queue.

There are seven priorities :  \mylst{UNARY}, \mylst{BINARY}, \mylst{TERNARY}, \mylst{LINEAR}, \mylst{QUADRATIC}, \mylst{CUBIC} and \mylst{VERY_SLOW}, each of these qualifies the "cost" of a constraint (related to its internal filtering algorithm). 

By default, priorities and ranks are defined as follow:
\begin{tabular}{|l|r|}
\hline
Priority & Rank \\
\hline
unary & 1\\
binary & 2\\
ternary & 3\\
linear & 4\\
quadratic & 5\\
cubic & 6\\
very slow & 7\\
\hline
\end{tabular}
\vspace{0.2cm}

Although the priority of a constraint cannot be changed, the rank of priority can be reconsider by setting another value to \mylst{Configuration.VEQ_ORDER} or \mylst{Configuration.CEQ_ORDER}. 

\vspace{0.5cm}

\begin{lstlisting}
Configuration conf = new Configuration();
conf.putInt(Configuration.VEQ_ORDER, 7654321); // default value is 1234567
conf.putInt(Configuration.CEQ_ORDER, 1111744); // default value is 1234567
Solver solver = new CPSolver(conf);
\end{lstlisting}

In that example, priorities ranks are reversed for the variable queues, and totally rearranged for the constraint queue.


\subsubsection{Constraint event}

At the very beginning of resolution, when constraint filtering algorithms have not been called once, a call to the \mylst{awake()} method is planned by posting constraint event to the propagation engine. For some \textit{expensive} constraints (like \hyperlink{geost:geostconstraint}{Geost}), a call to the main filtering algorithm (described in \mylst{propagate()}) can be added during the resolution by posting a constraint event to the propagation engine, in that case a call to the \mylst{propagate()} method is planned. 
This event is added to the list of constraint events to be treated by the propagation engine.

\subsubsection{Variable event}

The resolution goal is to instantiate variables in order to find solutions. Instantiation of a variable is done applying modification on its domain. 
Each time a modification is applied on a domain, an event is posted, storing informations about the action done (event type, variable, values, etc.). This event will be given to the related constraints of the modified variable, to check consistency and propagate this new information to the other variables.

Depending on the type of domain, events existing in Choco are:

\noindent\begin{tabular}{lp{.6\linewidth}}
\hline
Event &  description \\
\hline
\multicolumn{2}{l}{Integer variable \mylst{IntDomainVar}}\\
  \hline
  \mylst{REMVAL} & Remove a single value from the domain.\\
  \mylst{INCINF} &  Increase the lower bound of the domain. \\
  \mylst{DECSUP} &  Decrease the upper bound of the domain. \\
  \mylst{INSTINT} &  Instantiate the domain, i.e. reduce it to a single value. \\   
\hline
\multicolumn{2}{l}{Set variable \mylst{SetVar}}\\
\hline
  \mylst{REMENV} & Remove a single value from the envelope domain \\
  \mylst{ADDKER} &  Add a single value to the kernel domain\\
  \mylst{INSTSET} &  Instantiate the domain, i.e. set values to both kernel and envelope domain. \\   
\hline  
\multicolumn{2}{l}{Real variable \mylst{RealVar}}\\
\hline
  \mylst{INCINF} &  Increase the lower bound of the domain\\
  \mylst{DECSUP} &  Decrease the upper bound of the domain\\
\hline  
\end{tabular}

An event given as a parameter to the engine is then pushed into a unique queue, waiting to be treated. There are seven different queues where an event can be pushed, it depends on the priority of the event.
   

Beware, an event is not automatically pushed in a queue: if an other event based on the same variable is already present in the queue, the two events are merged into one. 
\begin{note}
Due to promotion, variables' events can be treated in a different order than creation one!

Let X and Y, two integer variables with priorities of the same rank. During the propagation of a constraint, the lower bound of a integer variable X is updated to 3 (this event is added to the queue Q), then Y is instantiated to 4 (this other event is added to the same queue Q) and finally the upper bound of X is updated to 4 (the already pushed event on X is updated with this new information). At that point, there are only 2 events in the queue Q: one on X and one on Y. 
As X and Y have the same rank, events will be treated by creation order: event on X first ant event on Y then. But the event on X contains 2 informations, one on the lower bound modification and another on the upper bound modification. So, although the instantiation of Y has been created first, the treatment of the upper bound modification of X will be treated first.
\end{note}


\begin{note}
Original event can be \textit{promoted}: for example removing the last but one value of an integer variable is promoted to instantiation.

Promotion are:

\noindent\begin{tabular}{lp{.6\linewidth}}
\hline
Original event &  can be promoted to \\
\hline
\multicolumn{2}{l}{Integer variable \mylst{IntDomainVar}}\\
  \hline
  \mylst{REMVAL} & \mylst{INCINF} or \mylst{DECSUP} or \mylst{INSTINT}\\
  \mylst{INCINF} &  \mylst{INSTINT}\\
  \mylst{DECSUP} &  \mylst{INSTINT} \\
  \mylst{INSTINT} &  \textit{none} \\   
\hline
\multicolumn{2}{l}{Set variable \mylst{SetVar}}\\
\hline
  \mylst{REMENV} & \mylst{INSTSET} \\
  \mylst{ADDKER} &  \mylst{INSTSET}\\
  \mylst{INSTSET} &  \textit{none} \\   
\hline  
\multicolumn{2}{l}{Real variable \mylst{RealVar}}\\
\hline
  \mylst{INCINF} &  \textit{none}\\
  \mylst{DECSUP} &  \textit{none}\\
\hline  
\end{tabular}

 \end{note}


\section{Define your own search strategy}\label{advanced:defineyourownsearchstrategy}\hypertarget{advanced:defineyourownsearchstrategy}{}
%A key ingredient of any constraint approach is a clever branching strategy. The construction of the search tree is done according to a series of Branching objects (that plays the role of achieving intermediate goals in logic programming). The user may specify the sequence of branching objects to be used to build the search tree. 
Section~\hyperlink{solver:searchstrategy}{Search strategy} presented the default branching strategies available in Choco and showed how to post them or to compose them as goals.
In this section, we will start with a very simple and common way to branch by choosing values for variables and specially how to define its own variable/value selection strategy. We will then focus on more complex branching such as dichotomic or n-ary choices. Finally we will show how to control the search space in more details with well known strategy such as LDS (Limited discrepancy search).

Reminder : \hyperlink{advanced:howdoesasearchloopwork}{How does the search loop work?}

For integer variables, the variable and value selection strategy objects are based on the following interfaces:
\begin{itemize}
	\item \mylst{AbstractIntBranchingStrategy}: abstract class for the branching strategy,
	\item \mylst{VarSelector<V>} : Interface for the variable selection (\mylst{V extends Var}),
	\item \mylst{ValIterator<V>} : Interface to describes an iteration scheme on the domain of a variable,
	\item \mylst{ValSelector<V>} : Interface for a value selection.
\end{itemize}

Concrete examples of these interfaces are respectively,  \hyperlink{assignvar:assignvarbranchstrat}{AssignVar}, \hyperlink{mindomain:mindomainvarselector}{MinDomain}, \hyperlink{increasingdomain:increasingdomainvaliterator}{IncreasingDomain}, \hyperlink{maxval:maxvalvalselector}{MaxVal}.

\subsection{How to define your own Branching object}\label{advanced:beyondvariable/valueselection,howtodefineyourownbranchingobject}\hypertarget{advanced:beyondvariable/valueselection,howtodefineyourownbranchingobject}{}

When you need a specific branching strategy that can't be expressed with the ones already existing, you can define your own concrete class of:

\noindent{\begin{tabular}{ll}
\hline
  Default class to implement &  definition \\
  \hline
  \mylst{AbstractBinIntBranchingStrategy} &  abstract class defining a binary tree search \\
  \mylst{AbstractLargeIntSConstraint} &  abstract class defining a n-ary tree search. \\
  \hline\\
\end{tabular}}


\insertGraphique{\linewidth}{media/branching.pdf}{Branching strategy: interfaces and abstract classes.}

We give here two examples of implementations of these classes, first for a binary branching, then for a n-ary branching. 
\begin{lstlisting}   
/**
 * A class for branching schemes that consider two branches: 
 * - one assigning a value to an IntVar (X == v) 
 * - and the other forbidding this assignment (X != v)
 */
public class AssignOrForbid extends AbstractBinIntBranchingStrategy {

    protected VarSelector<IntDomainVar> varSelector;

    protected ValSelector<IntDomainVar> valSelector;

    public AssignOrForbid(VarSelector<IntDomainVar> varSelector,
                          ValSelector<IntDomainVar> valSelector) {
        super();
        this.valSelector = valSelector;
        this.varSelector = varSelector;
    }

    /**
     * Select the variable to constrained
     *
     * @return the branching object
     */
    public Object selectBranchingObject() throws ContradictionException {
        return varSelector.selectVar();
    }

    /**
     * Select the value to assign, and set it in the decision object in parameter
     * @param decision the next decision to apply
     */
    @Override
    public void setFirstBranch(final IntBranchingDecision decision) {
        decision.setBranchingValue(valSelector.getBestVal(decision.getBranchingIntVar()));
    }


    /**
     * Create and return the message to print, in case of strong verbosity
     * @param decision current decision
     * @return pretty print of the current decision
     */
    @Override
    public String getDecisionLogMessage(final IntBranchingDecision decision) {
        return decision.getBranchingObject() +  (decision.getBranchIndex() == 0 ? "==" : "=/=") + decision.getBranchingValue();
    }


    /**
     * Execution action based on the couple: {decision, branching index}.
     * As <code>this</code> build a binary branching, there are only 2 branching indices:
     * 0 -- assignment, the variable is instantiated to the value
     * 1 -- forbidance, the value is removed from the domain of the variable
     *
     * @throws ContradictionException if the decision leads to an incoherence
     */
    @Override
    public void goDownBranch(final IntBranchingDecision decision) throws ContradictionException {
        if (decision.getBranchIndex() == 0) {
            decision.setIntVal();
        } else {
            decision.remIntVal();
        }
    }
}
\end{lstlisting}

\begin{lstlisting}   
/**
 * A class for branching schemes that consider n branches: 
 * -  assigning a value v_i to an variable (X == v_i)
 */
public class Assign extends AbstractLargeIntBranchingStrategy {

    protected final VarSelector<IntDomainVar> varSelector;

    protected ValIterator<IntDomainVar> valIterator;

    public Assign(VarSelector<IntDomainVar> varSelector, ValIterator<IntDomainVar> valIterator) {
        this.varSelector = varSelector;
        this.valIterator = valIterator;
    }

    /**
     * Select the variable to constrained
     *
     * @return the branching object
     */
    public Object selectBranchingObject() throws ContradictionException {
        return varSelector.selectVar();
    }

    /**
     * Select the first value to assign, and set it in the decision object in parameter
     *
     * @param decision the first decision to apply
     */
    public void setFirstBranch(final IntBranchingDecision decision) {
        decision.setBranchingValue(valIterator.getFirstVal(decision.getBranchingIntVar()));
    }

    /**
     * Select the next value to assign, and set it in the decision object in parameter
     *
     * @param decision the next decision to apply
     */
    public void setNextBranch(final IntBranchingDecision decision) {
        decision.setBranchingValue(valIterator.getNextVal(decision.getBranchingIntVar(), decision.getBranchingValue()));
    }

    /**
     * Check wether there is still a value to assign
     *
     * @param decision the last decision applied
     * @return <code>false</code> if there is still a branching to do
     */
    public boolean finishedBranching(final IntBranchingDecision decision) {
        return !valIterator.hasNextVal(decision.getBranchingIntVar(), decision.getBranchingValue());
    }

    /**
     * Apply the computed decision building the i^th branch.
     * --> assignment: the variable is instantiated to the value
     * 
     * 
     * @param decision the decision to apply.
     * @throws ContradictionException if the decision leads to an incoherence
     */
    @Override
    public void goDownBranch(final IntBranchingDecision decision) throws ContradictionException {
        decision.setIntVal();
    }

    /**
     * Reconsider the computed decision, destroying the i^th branch
     * --> forbiddance: the value is removed from the domain of the variable
     * 
     * @param decision the decision that has been set at the father choice point
     * @throws ContradictionException if the non-decision leads to an incoherence
     */
    @Override
    public void goUpBranch(final IntBranchingDecision decision) throws ContradictionException {
        decision.remIntVal();
    }

    /**
     * Create and return the message to print, in case of strong verbosity
     * @param decision current decision
     * @return pretty print of the current decision
     */
    @Override
    public String getDecisionLogMessage(IntBranchingDecision decision) {
        return decision.getBranchingObject() + "==" + decision.getBranchingValue();
    }
}
\end{lstlisting}

\subsection{Define your own variable selection}\label{advanced:defineyourownvariableselection}\hypertarget{advanced:defineyourownvariableselection}{}

\insertGraphique{.5\linewidth}{media/varselector-s.pdf}{Variable selector: interface}

You may extend this small library of branching schemes and heuristics by defining your own concrete classes of \mylst{AbstractIntVarSelector}. We give here an example of an \mylst{VarSelector<IntDomainVar>} with the implementation of a static variable ordering :
\begin{lstlisting}
/**
 * A variable selector selecting the first non instantiated variable according to a given static order
 */
public class StaticVarOrder extends AbstractIntVarSelector {

    private final IStateInt last;

    public StaticVarOrder(Solver solver) {
        super(solver);
        this.last = solver.getEnvironment().makeInt(0);
    }

    public StaticVarOrder(Solver solver, IntDomainVar[] vars) {
        super(solver, vars);
        this.last = solver.getEnvironment().makeInt(0);
    }

    /**
     * Select the next uninstantiated variable, according to the define policy: input order
     * @return the selected variable if exists, <code>null</code> otherwise
     */
    public IntDomainVar selectVar() {
        for (int i = last.get(); i < vars.length; i++) {
            if (!vars[i].isInstantiated()) {
                last.set(i);
                return vars[i];

            }
        }
        return null;
    }
}
\end{lstlisting}

Notice on this example that you only need to implement method \mylst{selectVar()} which belongs to the contract of \mylst{VarSelector}. This method should return a non instantiated variable or \mylst{null}. Once the branching is finished, the next branching (if one exists) is taken by the search algorithm to continue the search, otherwise, the search stops as all variable are instantiated. To avoid the loop over the variables of the branching, a backtrackable integer (\mylst{IStateInt}) could be used to remember the last instantiated variable and to directly select the next one in the table. Notice that backtrackable structures could be used in any of the code presented in this chapter to speedup the computation of dynamic choices.

\insertGraphique{.8\linewidth}{media/varselector-a.pdf}{Variable selector: interface and abstract classes}

If you need an integer variable selector that can be used as a parameter of \hyperlink{lexintvarselector:lexintvarselectorvarselector}{LexIntVarSelector}, it should extend \mylst{IntHeuristicIntVarSelector} or \mylst{DoubleHeuristicIntVarSelector}. These two abstract classes only require to implement one method : \mylst{getHeuristic(IntDomainVar v)} which computes and returns a criterion (\mylst{int} or \mylst{double}). The criteria are used in a master class to select the smallest crtierion's variable 

We give here an other example of an \mylst{VarSelector<IntDomainVar>}, this one extends \mylst{IntHeuristicIntVarSelector} and choose the variable with the smallest domain :
\begin{lstlisting}
public class MinDomain extends IntHeuristicIntVarSelector {

	public MinDomain(Solver solver) {
		super(solver);
	}

	public MinDomain(Solver solver, IntDomainVar[] vs) {
		super(solver, vs);
	}

    /**
     * Compute the criterion, according to the define policy: smallest domain size
     * @return the selected variable if exists, <code>null</code> otherwise
     */
	@Override
	public int getHeuristic(IntDomainVar v) {
		return v.getDomainSize();
	}

}
\end{lstlisting}


You can add your variable selector as a part of a search strategy, using \mylst{solver.addGoal()}.

\subsection{Define your own value selection}\label{advanced:defineyourownvalueselection}\hypertarget{advanced:defineyourownvalueselection}{}
You may also define your own concrete classes of \mylst{ValIterator} or \mylst{ValSelector}. 

\subsubsection{Value selector}\label{advanced:valueselector}\hypertarget{advanced:valueselector}{}

\insertGraphique{.3\linewidth}{media/valselector.pdf}{Value selector: interface}

We give here an example of an \mylst{IntValSelector} with the implementation of a minimum value selecting:
\begin{lstlisting}
public class MinVal implements ValSelector<IntDomainVar> {
  /**
   * selecting the lowest value in the domain
   *
   * @param x the variable under consideration
   * @return what seems the most interesting value for branching
   */
  public int getBestVal(IntDomainVar x) {
    return x.getInf();
  }
}
\end{lstlisting}
Only \mylst{getBestVal()} method must be implemented, returning the best value \emph{in the domain} according to the heuristic.

You can add your value selector as a part of a search strategy, using \mylst{solver.addGoal()}.

\begin{note}
Using a value selector with bounded domain variable is strongly inadvised, except if it pick up bounds value. If the value selector pick up a value that is not a bound, when it goes up in the tree search, that value could be not removed and picked twice (or more)!
\end{note} 

\subsubsection{Values iterator}\label{advanced:valuesiterator}\hypertarget{advanced:valuesiterator}{}

\insertGraphique{.3\linewidth}{media/valiterator.pdf}{Value iterator: interface}

We give here an example of an \mylst{ValIterator} with the implementation of an increasing domain iterator:
\begin{lstlisting}
public class IncreasingDomain implements ValIterator<IntDomainVar> {

  /**
   * testing whether more branches can be considered after branch i, on the alternative associated to variable x
   *
   * @param x the variable under scrutiny
   * @param i the index of the last branch explored
   * @return true if more branches can be expanded after branch i
   */
  public boolean hasNextVal(IntDomainVar x, int i) {
    return (i < x.getSup());
  }

  /**
   * Accessing the index of the first branch for variable x
   *
   * @param x the variable under scrutiny
   * @return the index of the first branch (such as the first value to be assigned to the variable)
   */
  public int getFirstVal(IntDomainVar x) {
    return x.getInf();
  }

  /**
   * generates the index of the next branch after branch i, on the alternative associated to variable x
   *
   * @param x the variable under scrutiny
   * @param i the index of the last branch explored
   * @return the index of the next branch to be expanded after branch i
   */
  public int getNextVal(IntDomainVar x, int i) {
    return x.getNextDomainValue(i);
  }
}
\end{lstlisting}
%Works as an basic \mylst{Iterator} object, implementing the three main methods \mylst{hasNextVal()}, \mylst{getFirstVal()} and \mylst{getNextVal()}.

You can add your value iterator as a part of a search strategy, using \mylst{solver.addGoal()}.

%\todo{under development} See \href{http://choco-solver.net/index.phptitle=userguide:beyondvariable.2fvalueselection.2chowtodefineyourownbranchingobject}{old version}

\section{Define your own limit search space}\label{advanced:defineyourownlimitsearchspace}\hypertarget{advanced:defineyourownlimitsearchspace}{}

To define your own limits/statistics (notice that a limit object can be used only to get statistics about the search), you can create a limit object by extending the \mylst{AbstractGlobalSearchLimit} class or implementing directly the interface \mylst{IGlobalSearchLimit}. Limits are managed at each node of the tree search and are updated each time a node is open or closed. Notice that limits are therefore time consuming. Implementing its own limit need only to specify to the following interface :

\begin{lstlisting}
	/**
	 * The interface of objects limiting the global search exploration
	 */
	public interface GlobalSearchLimit {

	  /**
	   * resets the limit (the counter run from now on)
	   * @param first true for the very first initialization, false for subsequent ones
	   */
	  public void reset(boolean first);
	
	  /**
	   * notify the limit object whenever a new node is created in the search tree
	   * @param solver the controller of the search exploration, managing the limit
	   * @return true if the limit accepts the creation of the new node, false otherwise
	   */
	  public boolean newNode(AbstractGlobalSearchSolver solver);
	
	  /**
	   * notify the limit object whenever the search closes a node in the search tree
	   * @param solver the controller of the search exploration, managing the limit
	   * @return true if the limit accepts the death of the new node, false otherwise
	   */
	  public boolean endNode(AbstractGlobalSearchSolver solver);
	}
\end{lstlisting}

Look at the following example to see a concrete implementation of the previous interface. We define here a limit on the depth of the search (which is not found by default in choco). The \mylst{getWorldIndex()} is used to get the current world, i.e the current depth of the search or the number of choices which have been done from baseWorld. 

\begin{lstlisting}
	public class DepthLimit extends AbstractGlobalSearchLimit {
	
	  public DepthLimit(AbstractGlobalSearchSolver theSolver, int theLimit) {
	    super(theSolver,theLimit);
	    unit = "deep";
	  }
	
	  public boolean newNode(AbstractGlobalSearchSolver solver) {
	    nb = Math.max(nb, this.getProblem().getWorldIndex() –
	    this.getProblem().getSolver().getSearchSolver().baseWorld);
	    return (nb < nbMax);
	  }
	
	  public boolean endNode(AbstractGlobalSearchSolver solver) {
	    return true;
	  }
	
	  public void reset(boolean first) {
	   if (first) {
	    nbTot = 0;
	   } else {
	    nbTot = Math.max(nbTot, nb);
	   }
	   nb = 0;
	  }
\end{lstlisting}

Once you have implemented your own limit, you need to tell the search solver to take it into account. Instead of using a call to the \mylst{solve()} method, you have to create the search solver by yourself and add the limit to its limits list such as in the following code :
\begin{lstlisting}
	Solver s = new CPSolver();
	s.read(model);
	s.setFirstSolution(true);
	s.generateSearchStrategy();
	s.getSearchStrategy().limits.add(new DepthLimit(s.getSearchStrategy(),10));
	s.launch();
\end{lstlisting}

%\subsubsection{Search loop with recomputation}\label{advanced:searchloopwithrecomputation}\hypertarget{advanced:searchloopwithrecomputation}{}

\section{Define your own constraint}\label{advanced:defineyourownconstraint}\hypertarget{advanced:defineyourownconstraint}{}

This section describes how to add you own constraint, with specific propagation algorithms. Note that this section is only useful in case you want to express a constraint for which the basic propagation algorithms (using tables of tuples, or boolean predicates) are not efficient enough to propagate the constraint.

The general process consists in defining a new constraint class and implementing the various propagation methods. We recommend the user to follow the examples of existing constraint classes (for instance, such as \mylst{GreaterOrEqualXYC} for a binary inequality) 

\subsection{The constraint hierarchy}\label{advanced:theconstrainthierarchy}\hypertarget{advanced:theconstrainthierarchy}{}

Each new constraint must be represented by an object implementing the \mylst{SConstraint} interface (\mylst{S} for solver constraint). To help the user defining new constraint classes, several abstract classes defining \texttt{SConstraint} have been implemented. These abstract classes provide the user with a management of the constraint network and the propagation engineering. They should be used as much as possible.

For constraints on integer variables, the easiest way to implement your own constraint is to inherit from one of the following classes, depending of the number of solver integer variables (\texttt{IntDomainVar}) involved:

\centerline{\begin{tabular}{ll}
      \hline
  Default class to implement &  number of solver integer variables \\
  \hline
  \mylst{AbstractUnIntSConstraint} &  \textbf{one} variable \\
  \mylst{AbstractBinIntSConstraint} &  \textbf{two} variables \\
  \mylst{AbstractTernIntSConstraint} &  \textbf{three} variables \\
  \mylst{AbstractLargeIntSConstraint} &  any number of variables. \\
  \hline\\
\end{tabular}}

\noindent Constraints over integers must implement the following methods (grouped in the \texttt{IntSConstraint} interface):

\noindent\begin{tabular}{lp{.6\linewidth}}
  \hline
  Method to implement &  description \\
  \hline
  \mylst{pretty()} &Returns a pretty print of the constraint \\
  \mylst{propagate()} &The main propagation method (propagation from scratch). Propagating the constraint until local consistency is reached. \\
  \mylst{awake()} &Propagating the constraint for the very first time until local consistency is reached. The awake is meant to initialize the data structures contrary to the propagate. Specially, it is important to avoid initializing the data structures in the constructor. \\
  \mylst{awakeOnInst(int x)} &Default propagation on instantiation: full constraint re-propagation. \\
  \mylst{awakeOnBounds(int x)} &Default propagation on improved bounds: propagation on domain revision. \\
  \mylst{awakeOnRemovals(int x, IntIterator v)} &Default propagation on mutliple values removal: propagation on domain revision. The iterator allow to iterate over the values that have been removed. \\
&\\
\hline
\multicolumn{2}{l}{Methods \texttt{awakeOnBounds} and \texttt{awakeOnRemovals} can be replaced by more fine grained methods:}\\
\hline
%Alternative Method &  description \\
%  \hline
  \mylst{awakeOnInf(int x)} &Default propagation on improved lower bound: propagation on domain revision. \\
  \mylst{awakeOnSup(int x)} &Default propagation on improved upper bound: propagation on domain revision. \\
  \mylst{awakeOnRem(int x, int v)} &Default propagation on one value removal: propagation on domain revision.  \\
&\\
  \hline
\multicolumn{2}{l}{To use the constraint in expressions or reification, the following minimum API is mandatory:}\\
  \hline
  \mylst{isSatisfied(int[] x)} &Tests if the constraint is satisfied when the variables are instantiated. \\
	\mylst{isEntailed()} &Checks if the constraint must be checked or must fail. It returns true if the constraint is known to be satisfied whatever happend on the variable from now on, false if it is violated. \\
	\mylst{opposite()} &It returns an AbstractSConstraint that is the opposite of the current constraint. \\
    \hline\\
	\end{tabular}

In the same way, a \textbf{set constraint} can inherit from \texttt{AbstractUnSetSConstraint}, \texttt{AbstractBinSetSConstraint}, \texttt{AbstractTernSetSConstraint} or \texttt{AbstractLargeSetSConstraint}.

A \textbf{real constraint} can inherit from \texttt{AbstractUnRealSConstraint}, \texttt{AbstractBinRealSConstraint} or \texttt{AbstractLargeRealSConstraint}.

A mixed constraint between \textbf{set and integer variables} can inherit from \texttt{AbstractBinSetIntSConstraint} or \texttt{AbstractLargeSetIntSConstraint}.

\begin{note}
A simple way to implement its own constraint is to:
\begin{itemize}
	\item create an empty constraint with only \texttt{propagate()} method implemented and every \texttt{awakeOnXxx()} ones set to \texttt{this.constAwake(false);}
	\item when the propagation filter is sure, separate it into the \texttt{awakeOnXxx()} methods in order to have finer granularity
	\item finally, if necessary, use backtrackables objects to improve the efficient of your constraint
\end{itemize}

\end{note}

\subsubsection{Interact with variables}\label{advanced:interactwithvariables}\hypertarget{advanced:interactwithvariables}{}

One of the constraint function is to remove forbidden values from domain variable (\textit{filtering}). \mylst{Variable}s provide services to allow constraint to interact with their domain.

\vspace{0.5cm}
\textbf{IntDomainVar}

\begin{description}
\item[ ] \mylst{boolean removeVal(int x, final SConstraint cause, final boolean forceAwake)}

Update the domain of the integer variable by removing \mylst{x} from the domain. \mylst{cause} is the constraint at the origin of the event, \mylst{forceAwake} indicates wether or not the \mylst{cause} constraint must be informed of this event. The result of such call can be \mylst{true}, the value has been removed without any trouble, \mylst{false} the value was not present in the domain. A \mylst{ContradictionException} is thrown if it empties the domain of the variable

\item[ ] \mylst{boolean removeInterval(int a, int b, final SConstraint cause, final boolean forceAwake)}

Update the domain of the integer variable by removing all values contained in the interval $[a,b]$ from the domain. \mylst{cause} is the constraint at the origin of the event, \mylst{forceAwake} indicates wether or not the \mylst{cause} constraint must be informed of this event. The result of such call can be \mylst{true}, the values has been removed without any trouble, \mylst{false} if the intersection of the current domain and $[a,b]$ was empty. A \mylst{ContradictionException} is thrown if it empties the domain of the variable.

\item[ ] \mylst{boolean updateInf(int x, final SConstraint cause, final boolean forceAwake)} 

Update the domain of the integer variable by removing all values strictly below \mylst{x} from the domain. \mylst{cause} is the constraint at the origin of the event, \mylst{forceAwake} indicates wether or not the \mylst{cause} constraint must be informed of this event. The result of such call can be \mylst{true}, the lower bound has been updated without any trouble, \mylst{false} the new lower bound was smaller or equal to the actual one. A \mylst{ContradictionException} is thrown if it empties the domain of the variable.

\item[ ] \mylst{boolean updateSup(int x, final SConstraint cause, final boolean forceAwake)} 

Update the domain of the integer variable by removing all values strictly above \mylst{x} from the domain. \mylst{cause} is the constraint at the origin of the event, \mylst{forceAwake} indicates wether or not the \mylst{cause} constraint must be informed of this event. The result of such call can be \mylst{true}, the upper bound has been updated without any trouble, \mylst{false} the new upper bound was greater or equal to the actual one. A \mylst{ContradictionException} is thrown if it empties the domain of the variable. 

\item[ ] \mylst{boolean instantiate(int x, final SConstraint cause, final boolean forceAwake)} 

Update the domain of the integer variable by removing all values but \mylst{x} from the domain. \mylst{cause} is the constraint at the origin of the event, \mylst{forceAwake} indicates wether or not the \mylst{cause} constraint must be informed of this event. The result of such call can be \mylst{true}, the domain has been reduced to a singleton without any trouble, \mylst{false} the domain was already instantiated to the same value . A \mylst{ContradictionException} is thrown if \mylst{x} is out of the domain or if the domain was already instantiated to another value.

\end{description}

\vspace{0.5cm}
\textbf{SetVar}

\begin{description}
\item[ ] \mylst{boolean remFromEnveloppe(int x, final SConstraint cause, final boolean forceAwake)}

Update the domain of the set variable by removing \mylst{x} from the envelope's domain. \mylst{cause} is the constraint at the origin of the event, \mylst{forceAwake} indicates wether or not the \mylst{cause} constraint must be informed of this event. The result of such call can be \mylst{true}, the value has been removed without any trouble, \mylst{false} the value was not present in the envelope. A \mylst{ContradictionException} is thrown if \mylst{x} is present in the kernel's domain.

\item[ ] \mylst{boolean addToKernel(int x, final SConstraint cause, final boolean forceAwake)}

Update the domain of the set variable by adding \mylst{x} into the kernel's domain. \mylst{cause} is the constraint at the origin of the event, \mylst{forceAwake} indicates wether or not the \mylst{cause} constraint must be informed of this event. The result of such call can be \mylst{true}, the value has been added without any trouble, \mylst{false} the value was already present in the kernel. A \mylst{ContradictionException} is thrown if \mylst{x} is not present in the envelope's domain.

\item[ ] \mylst{boolean instantiate(int[] xs, final SConstraint cause, final boolean forceAwake)} 

Update the domain of the set variable by removing every values but those in \mylst{xs} from the envelope's domain and by adding every values of \mylst{xs} into the kernel's domain. A set variable is known as \textit{instantiated} when $E \cap K \ne \emptyset $ and $E \bigtriangleup K = \emptyset$. 
\mylst{cause} is the constraint at the origin of the event, \mylst{forceAwake} indicates wether or not the \mylst{cause} constraint must be informed of this event. The result of such call can be \mylst{true}, the envelope or the kernel have been updated without trouble, \mylst{false} the envelope and the kernel were already equal to \mylst{xs} . A \mylst{ContradictionException} is thrown if the at least one value from \mylst{xs} is not present in the kernel's domain or in the envelope's domain.
 
\end{description}

\vspace{0.5cm}
\textbf{RealVar}

\begin{description}
\item[ ] \mylst{void intersect(RealInterval interval)}

Update the domain of the real variable by intersecting the domain with \mylst{interval} (define at least by two doubles, known as lower and upper bound). A \mylst{ContradictionException} is thrown the resulting interval is incoherent (the new lower bound is greater than the new upper bound).

\end{description}


\subsubsection{How do I add my constraint to the Model ?}\label{advanced:howdoiaddmyconstrainttothemodel}\hypertarget{advanced:howdoiaddmyconstrainttothemodel}{}

Adding your constraint to the model requires you to definite a specific constraint manager (that can be a inner class of your Constraint).
This manager need to implement:
\begin{lstlisting}
makeConstraint(Solver s, Variable[] vars, Object params, HashSet<String> options)
\end{lstlisting}
This method allows the Solver to create an instance of your constraint, with your parameters and Solver objects.

\begin{note}
If you create your constraint manager as an inner class, you must declare this class as \textbf{public and static}.
If you don't, the solver can't instantiate your manager.
\end{note}

Once this manager has been implemented, you simply add your constraint to the model using the \texttt{addConstraint()} API with a \texttt{ComponentConstraint} object:
\begin{lstlisting}
  model.addConstraint( new ComponentConstraint(MyConstraintManager.class, params, vars) );
  // OR
  model.addConstraint( new ComponentConstraint("package.of.MyConstraint", params, vars) );
\end{lstlisting}
Where \emph{params} is whatever you want (\texttt{Object[], int, String},...) and \emph{vars} is an array of Model Variables (or more specific) objects.

\subsection{Example: implement and add the \texttt{IsOdd} constraint}
One creates the constraint by implementing the \texttt{AbstractUnIntSConstraint} (one integer variable) class:
\lstinputlisting{java/isodd.j2t}

To add the constraint to the model, one creates the following class (or inner class):
\lstinputlisting{java/isoddmanager.j2t}
It calls the constructor of the constraint, with every \emph{vars}, \emph{params} and \emph{options} needed.

Then, the constraint can be added to a model as follows:
\begin{lstlisting}
	// Creation of the model
	Model m = new CPModel();
	
	// Declaration of the variable
	IntegerVariable aVar = Choco.makeIntVar("a_variable", 0, 10);
	
	// Adding the constraint to the model, 1st solution:
	m.addConstraint(new ComponentConstraint(IsOddManager.class, null, new IntegerVariable[]{aVar}));
	// OR 2nd solution:
	m.addConstraint(new ComponentConstraint("myPackage.Constraint.IsOddManager", null, new IntegerVariable[]{aVar}));
	
	Solver s = new CPSolver();
	s.read(m);
	s.solve();
\end{lstlisting}
And that's it!!

\subsection{Example of an empty constraint}\label{advanced:anexempleofemptyconstraint}\hypertarget{advanced:anexempleofemptyconstraint}{}

%See the complete code: \href{media/zip/constraintpattern.zip}{ConstraintPattern.zip}

\begin{lstlisting}
  public class ConstraintPattern extends AbstractLargeIntSConstraint {
      
      public ConstraintPattern(IntDomainVar[] vars) {
          super(vars);
      }
	
      /**
      * pretty print. The String is not constant and may depend on the context.
      * @return a readable string representation of the object
      */
      public String pretty() {
          return null;
      }
	
      /**
      * check whether the tuple satisfies the constraint
      * @param tuple values
      * @return true if satisfied
      */
      public boolean isSatisfied(int[] tuple) {
          return false;
      }

      /**
      * propagate until local consistency is reached
      */
      public void propagate() throws ContradictionException {
          // elementary method to implement
      }
	    
      /**
      * propagate for the very first time until local consistency is reached.
      */
      public void awake() throws ContradictionException {
          constAwake(false);        // change if necessary
      }
	
	
      /**
      * default propagation on instantiation: full constraint re-propagation
      * @param var index of the variable to reduce
      */
      public void awakeOnInst(int var) throws ContradictionException {
          constAwake(false);        // change if necessary
      }
	
      /**
      * default propagation on improved lower bound: propagation on domain revision
      * @param var index of the variable to reduce
      */
      public void awakeOnInf(int var) throws ContradictionException {
          constAwake(false);        // change if necessary
      }
	
	
      /**
      * default propagation on improved upper bound: propagation on domain revision
      * @param var index of the variable to reduce
      */
      public void awakeOnSup(int var) throws ContradictionException {
          constAwake(false);        // change if necessary
      }
	
      /**
      * default propagation on improve bounds: propagation on domain revision
      * @param var index of the variable to reduce
      */
      public void awakeOnBounds(int var) throws ContradictionException {
          constAwake(false);        // change if necessary
      }
	
      /**
      * default propagation on one value removal: propagation on domain revision
      * @param var index of the variable to reduce
      * @param val the removed value
      */
      public void awakeOnRem(int var, int val) throws ContradictionException {
          constAwake(false);        // change if necessary
      }
	
      /**
      * default propagation on one value removal: propagation on domain revision
      * @param var index of the variable to reduce
      * @param delta iterator over remove values
      */
      public void awakeOnRemovals(int var, IntIterator delta) throws ContradictionException {
          constAwake(false);        // change if necessary
      }
  }
\end{lstlisting}

The first step to create a constraint in Choco is to implement all \texttt{awakeOn...} methods with \texttt{constAwake(false)} and to put your propagation algorithm in the \texttt{propagate()} method. 

A constraint can choose not to react to fine grained events such as the removal of a value of a given variable but instead delay its propagation at the end of the fix point reached by ``fine grained events'' and fast constraints that deal with them incrementally (that's the purpose of the constraints events queue). 

To do that, you can use \texttt{constAwake(false)} that tells the solver that you want this constraint to be called only once the variables events queue is empty. This is done so that heavy propagators can delay their action after the fast one to avoid doing a heavy processing at each single little modification of domains.

\section{Define your own operator}\label{advanced:defineyourownoperator}\hypertarget{advanced:defineyourownoperator}{}
%\todo{to complete}

%%%%%%%%%%%%%%%%%%
%%%%%%%%%%%%%%%%%%
%%%%%%%%%%%%%%%%%%
%%%%%%%%%%%%%%%%%%
%%%%%%%%%%%%%%%%%%

There are 2 types of operators: \textbf{boolean} and \textbf{arithmetic}. These operators are based on integer variable and/or integer constants. 
Let's take 2 examples \textit{plus} and \textit{eq}.

The operator \textit{plus} is an arithmetic one (see  \mylst{PlusNode.java} class for details) that computes the sum of two variables (and/or constants). It extends  \mylst{INode} (it can be part of an expression object) and implements  \mylst{ArithmNode} (it can be evaluated). So, required services are:

\begin{itemize}
\item a constructor. The type of operator should be defined using \mylst{CUSTOM}.
\begin{lstlisting}
public PlusNode(INode[] subt) {
	super(subt, NodeType.CUSTOM);
}
\end{lstlisting}
\item  \mylst{pretty()} : a pretty print of the operator
\begin{lstlisting}
public String pretty() {
        return "("+subtrees[0].pretty()+" + "+subtrees[1].pretty()+")";
    }
\end{lstlisting}

\item  \mylst{eval(int[] tuple)} : evaluation of the operator with the given tuple. An arithmetic evaluation of the subtrees is done (based on the given tuple) to compute the sub expression before evaluating the current operator \textit{plus}. This allows tree-like representation of an expression.
\begin{lstlisting}
public int eval(int[] tuple) {
		return ((ArithmNode) subtrees[0]).eval(tuple) + ((ArithmNode) subtrees[1]).eval(tuple);
}
\end{lstlisting}
\end{itemize}


The operator \textit{eq} is a boolean one (see  \mylst{EqNode.java} class for more details) that checks if two variables (and/or constants) are equal. It extends  \mylst{AbstractBooleanNode} (that can be checked). So required services are:
\begin{itemize}
\item a constructor
\begin{lstlisting}
public EqNode(INode[] subt) {
        super(subt, NodeType.CUSTOM);
    }
\end{lstlisting}
\item  \mylst{pretty()} : a pretty print of the operator
\begin{lstlisting}
public String pretty() {
        return "("+subtrees[0].pretty()+"="+subtrees[1].pretty()+")";
}
\end{lstlisting} 

\item  \mylst{eval(int[] tuple)}: an expression checker, based on the given tuple. An arithmetic evaluation of the subtrees is done (based on the given tuple) in order to check the operator \textit{eq}.
\begin{lstlisting}
public boolean checkTuple(int[] tuple) {
		return ((ArithmNode) subtrees[0]).eval(tuple)
		        ==  ((ArithmNode) subtrees[1]).eval(tuple);
	}
\end{lstlisting}

\item  \mylst{extractConstraint(Solver s)} : extracts the corresponding constraint in intension constraint without reification.
\begin{lstlisting}
public SConstraint extractConstraint(Solver s) {
        IntDomainVar v1 = subtrees[0].extractResult(s);
		IntDomainVar v2 = subtrees[1].extractResult(s);
		return s.eq(v1,v2);
    }
\end{lstlisting}
\end{itemize}
Now let's see how to use this operator in a Model.

To do that, create your own manager implementing \mylst{ExpressionManager}, which makes the link between the model and the solver.

Then to use your operator in your model, you can define a static method to simplify the calls.

Let's sum up in a short example based on the \textit{plus} operator. 

The manager would be:
\begin{lstlisting}
public class PlusManager implements ExpressionManager {
    @Override
    public INode makeNode(Solver solver, Constraint[] cstrs, Variable[] vars) {
        if(solver instanceof CPSolver){
            CPSolver s = (CPSolver)solver;
            if(vars.length == 1){
                INode[] nodes = new INode[vars.length];
                for(int v = 0; v < vars.length; v++){
                    nodes[v] = vars[v].getExpressionManager().makeNode(s, vars[v].getConstraints(), vars[v].getVariables());
                }
                return new PlusNode(nodes);
            }
        }
        throw new ModelException("Could not found a node manager in " + this.getClass() + " !");
    }
}
\end{lstlisting}


A main class would be:
\begin{lstlisting}
public class Sandbox {

    public static void main(String[] args) {
        model1();
    }

    public static IntegerExpressionVariable plus(IntegerVariable x){
        return new IntegerExpressionVariable(null, "package.of.PlusManager", x);
    }

    private static void model1() {
        Model m = new CPModel();
        IntegerVariable x = Choco.makeIntVar("x", 0, 5);
        IntegerVariable y = Choco.makeIntVar("y", 4, 8);
        IntegerVariable z = Choco.makeIntVar("z", 0, 10);
	
        // declare an expression using my operator
        IntegerExpressionVariable xx = plus(x,y);

        // and use it in common constraint
        m.addConstraint(Choco.eq(z, xx));

        Solver s = new CPSolver();
        s.read(m);

        ChocoLogging.toSolution();
        s.solveAll();
    }
}
\end{lstlisting}

Keep in mind that you can not define operators for set and real. 

%%%%%%%%%%%%%%%%%%
%%%%%%%%%%%%%%%%%%
%%%%%%%%%%%%%%%%%%
%%%%%%%%%%%%%%%%%%
%%%%%%%%%%%%%%%%%%

\section{Define your own variable}\label{advanced:defineyourownvariable}\hypertarget{advanced:defineyourownvariable}{}
\todo{to complete}

\section{Model and Solver detectors}\label{advanced:detectors}\hypertarget{advanced:detectors}{}

Sometimes, on automatic code generation or during benchmarking, it could be useful to apply generic rules to analyze a \mylst{Model} and detect lacks of modeling and apply patchs. With Choco, this is possible using \mylst{ModelDetectorFactory} and \mylst{PreProcessCPSolver}.

\subsection{Model detector}\label{advanced:modeldetector}\hypertarget{advanced:modeldetector}{}

The analysis of a \mylst{Model} is done using the \mylst{ModelDetectorFactory}. First one declares the list of rules to apply, then they are applied to a specific model. 
This is done by using the follwing API:
\begin{lstlisting}
ModelDetectorFactory.run(CPModel model, AbstractDetector... detectors)
\end{lstlisting}
An \mylst{AbstractDetector} object describes the pattern to detect within the model and rules to apply. Applying a rule is commonly to refactor a given model, by adding or removing variables and constraints. 

\begin{note}
Calling \mylst{ModelDetectorFactory.run(..)} will produce a modified copy of the current model. 
Very few \mylst{AbstractDetector} just analyze the model, without any side effects.
\end{note}

\textbf{Detectors}


\begin{itemize}
\item[] \mylst{AbstractDetector analysis(CPModel m)} 
Analyze the model \mylst{m}, and print out messages about general statistics: very large domain variables, inappropriate domain type variables, free variables (variables not involved in any constraints), etc. 
\item[] \mylst{AbstractDetector intVarEqDet(CPModel m)}
\item[] \mylst{AbstractDetector taskVarEqDet(CPModel m)}
\item[] \mylst{AbstractDetector cliqueDetector(CPModel m, boolean breakSymetries)}
\item[] \mylst{AbstractDetector disjFromCumulDetector(CPModel m)}
\item[] \mylst{AbstractDetector precFromImpliedDetector(CPModel m, DisjunctiveModel disjMod)}
\item[] \mylst{AbstractDetector precFromReifiedDetector(CPModel m, DisjunctiveModel disjMod)}
\item[] \mylst{AbstractDetector precFromTimeWindowDetector(CPModel m, DisjunctiveModel disjMod)}
\item[] \mylst{AbstractDetector precFromDisjointDetector(CPModel m, DisjunctiveModel disjMod)}
\item[] \mylst{AbstractDetector disjointDetector(CPModel m, DisjunctiveModel disjMod)}
\item[] \mylst{AbstractDetector disjointFromDisjDetector(CPModel m, DisjunctiveModel disjMod)}
\item[] \mylst{AbstractDetector disjointFromCumulDetector(CPModel m, DisjunctiveModel disjMod)}
\item[] \mylst{AbstractDetector rmDisjDetector(final CPModel m)}
\item[] \mylst{AbstractDetector[] disjunctiveModelDetectors(CPModel m, DisjunctiveModel disjMod)}
\item[] \mylst{AbstractDetector[] schedulingModelDetectors(final CPModel m, DisjunctiveModel disjMod)}
\end{itemize}



\begin{note}
Analysing a \mylst{Model} can be time consuming. It should be used carefully.
\end{note}
 
 

\todo{to complete}


\subsection{Preprocess solver}\label{advanced:preprocesssolver}\hypertarget{advanced:preprocesssolver}{}

One may want to keep the \mylst{Model} unchanged but apply detection directly on the \mylst{Solver}. This can be done during the \textit{reading} step: the model is analyses on the fly and the rules are applied directly on the solver. 

To do that, simply replaced the \mylst{CPSolver} declaration by:
\begin{lstlisting}
Solver solver = new PreProcessCPSolver(); // new CPSolver();
\end{lstlisting} 

The rules that will be applied can be selected by updating the \mylst{PreProcessConfiguration} file.
\todo{to complete}


\section{Logging statements}\label{advanced:loggingstatements}\hypertarget{advanced:loggingstatements}{}

\subsection{Define your own logger.}\label{advanced:defineyourownlogger}\hypertarget{advanced:defineyourownlogger}{}
\begin{lstlisting}
ChocoLogging.makeUserLogger(String suffix);
\end{lstlisting}


\subsection{How to write logging statements ?}\label{advanced:howtowriteloggingstatements}\hypertarget{advanced:howtowriteloggingstatements}{}

\begin{itemize}
	\item Critical Loggers are provided to display error or warning. Displaying too much message really \textbf{impacts the performances}.
	\item Check the logging level before creating arrays or strings.
	\item Avoid multiple calls to \texttt{Logger} functions. Prefer to build a \texttt{StringBuilder} then call the \texttt{Logger} function.
	\item Use the \texttt{Logger.log} function instead of building string in \texttt{Logger.info()}.
\end{itemize}

\subsection{Handlers.}\label{advanced:handlers}\hypertarget{advanced:handlers}{}
Logs are displayed on \texttt{System.out} but warnings and severe messages are also displayed on \texttt{System.err}.
\texttt{ChocoLogging.java} also provides utility functions to easily change handlers:
\begin{itemize}
	\item Functions \texttt{set...Handler} remove current handlers and replace them by a new handler.
	\item Functions \texttt{add...Handler} add new handlers but do not touch existing handlers.
\end{itemize}

% \subsubsection{Figure source}\label{advanced:figuresource}\hypertarget{advanced:figuresource}{}
% \begin{lstlisting}
%   digraph G {
%       node [style=filled, shape=box];
%       choco [shape=house,fillcolor=gold];
	
%       kernel [shape=house,fillcolor=gold];
%       engine [shape=octagon,fillcolor=indianred];
%       search [shape=octagon,fillcolor=darkorange];
%       branching [shape=octagon,fillcolor=indianred];
	
%       api [shape=house,fillcolor=gold];
%       model [fillcolor=indianred];
%       solver [fillcolor=indianred];
%       parser [fillcolor=darkorange];
      
%       user [fillcolor=darkorange];
%       samples [fillcolor=darkorange];
      
%       test [fillcolor=indianred];
      
%       choco -> kernel;
%       choco -> API;
%       choco -> user;
%       choco -> test;
      
%       kernel -> engine;
%       kernel -> search;
      
%       api -> model;
%       api -> solver
%       api -> parser;
      
%       user  -> samples;
      
%       search -> branching;
% 	}
% \end{lstlisting}
\label{doc:advanced}\hypertarget{doc:advanced}{}
\input{chapters/Choco_and_CP_Viz.tex}