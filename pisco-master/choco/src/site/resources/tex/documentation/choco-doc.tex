% Time-stamp: <[choco-doc.tex] last changed 28-10-2010 13:18 by sofdem>
\documentclass[a4paper,10pt]{book}
\usepackage{hyperref}
\usepackage{fancyhdr,lastpage,fullpage,fancybox}
\usepackage{amsmath,amssymb}

%%%% Glossary  %%%%%%%%%%%%%%%%%%%%%%%%%%%%%%%%%%%%%%%%%%%%%%%%%%%%%%%%%
\usepackage[acronym]{glossaries} % make a separate list of acronyms
% to use call
% $> latex choco-doc 
% $> makeglossaries choco-doc 
% $> latex choco-doc 

\makeglossaries

%%%% Web Tralics  %%%%%%%%%%%%%%%%%%%%%%%%%%%%%%%%%%%%%%%%%%%%%%%%%%%%%%%%%
\newif\ifweb
%\webtrue
\webfalse

\ifweb
\else
%%%% Refs, formats, styles  %%%%%%%%%%%%%%%%%%%%%%%%%%%%%%%%%%%%%%%%%%%%%%%%%%%%%%%%%
\hypersetup{plainpages=false, colorlinks=true, urlcolor=blue, linkcolor=cyan, citecolor=orange, filecolor=red}
\fancypagestyle{plain}{\fancyhf{}
    \fancyhead[LE]{\slshape \leftmark} 
    \fancyhead[RO]{\slshape \rightmark} 
    \lfoot{\footnotesize\emph{CHOCO solver documentation \\BSD licence \number\year}}
    \cfoot{\footnotesize\textsf{-\thepage/\pageref{LastPage}-}}
    \rfoot{\footnotesize\emph{\number\day/\number\month/\number\year}}}
\fancypagestyle{fancyfront}{\fancyhf{}
    \fancyhead[LE]{\slshape \leftmark} 
    \fancyhead[RO]{\slshape \rightmark} 
    \lfoot{\footnotesize\emph{CHOCO solver documentation \\BSD licence \number\year}}
    \cfoot{\footnotesize\textsf{-\thepage-}}
    \rfoot{\footnotesize\emph{\number\day/\number\month/\number\year}}}
%    \renewcommand{\headrulewidth}{0pt}}
\addtolength{\headsep}{\baselineskip} 

\makeatletter 
\def\cleardoublepage{\clearpage\if@twoside \ifodd\c@page\else 
\hbox{} 
\thispagestyle{empty} 
\newpage 
\if@twocolumn\hbox{}\newpage\fi\fi\fi} 
\makeatother 

%%%% Fonts, Figures, Colors %%%%%%%%%%%%%%%%%%%%%%%%%%%%%%%%%%%%%%%%%%%%%%%%%%%%%%%%%
\newcommand{\algofont}{\fontencoding{U}\fontfamily{eur}\fontseries{b}\fontshape{n}\selectfont}
\usepackage[usenames]{color}
\usepackage[pdftex]{graphicx,xcolor} %\pdfcompresslevel=9  %\pdfimageresolution=600
\def\Graphique#1#2{\includegraphics[#2]{#1.pdf}}
\def\Graph#1#2{\includegraphics[#2]{#1}}
%\def\insertLogo#1{\includegraphics{#1}}
\newcommand{\insertGraphique}[3]{\begin{figure}[htp]\centerline{\Graph{#2}{width=#1}}\caption[]{#3}\label{fig:#2}\end{figure}}
\newcommand{\newrgbcolor}[2]{\definecolor{#1}{rgb}{#2}}
\definecolor{darkred}{rgb}{0.5372549, 0.27843137, 0.27058824}
\definecolor{darkblue}{rgb}{0.27843137, 0.27058824, 0.5372549}
\newcommand{\gr}[1]{{\color{lightgray} #1}}
\newcommand{\dr}[1]{{\color{darkblue} #1}}
%%%% listings, env  %%%%%%%%%%%%%%%%%%%%%%%%%%%%%%%%%%%%%%%%%%%%%%%%%%%%%%%%%%%%%%%%%%%%%%%
\usepackage{listings}
\definecolor{Brown}{cmyk}{0,0.81,1,0.60}
\definecolor{OliveGreen}{cmyk}{0.64,0,0.95,0.40}
\definecolor{CadetBlue}{cmyk}{0.62,0.57,0.23,0}
\lstset{language=java,frame=ltrb,framesep=5pt,
    numbers=none,
    tabsize=3,
    breaklines=true,
    aboveskip=1ex,
    belowskip=1ex,
    basicstyle=\small\ttfamily,
    backgroundcolor=\color{gray!25},
    columns=fullflexible,
    keywordstyle=\ttfamily\color{OliveGreen},
    identifierstyle=\ttfamily\color{CadetBlue}\bfseries, 
    commentstyle=\color{Brown},
    stringstyle=\ttfamily,
    showstringspaces=true}

\newcommand{\mylst}[1]{\lstinline|#1|}

\newenvironment{note}{%
\setlength{\fboxsep}{15pt}%
\Sbox% 
\minipage{.7\linewidth}% 
}% 
{\endminipage\endSbox% 
\begin{center}\doublebox{\TheSbox}\end{center}}

\newenvironment{notedef}{%
\setlength{\fboxsep}{15pt}%
\Sbox% 
\minipage{.95\linewidth}% 
}% 
{\endminipage\endSbox% 
\par\hspace{-1em}\shadowbox{\TheSbox}\par\noindent}

\newenvironment{myquote}{\begin{notedef}}{\end{notedef}}
%\newenvironment{notedef}{\begin{note}}{\end{note}}
\fi
%%%%%%%%%%%%%%%%%%%%%%%%%%%%%%%%%%%%%%%%%%%%%%%%%%%%%%%%%%%%%%%%%%%%%%%%%%%%%%%%%%%%%
\newcommand{\collec}[2]{\ensuremath{\langle #1,..,#2\rangle}}
\newcommand{\coll}[1]{\ensuremath{\langle #1\rangle}}
\newcommand{\N}{\ensuremath{\mathbb{N}}}
\newcommand{\Z}{\ensuremath{\mathbb{Z}}}
\newcommand{\R}{\ensuremath{\mathbb{R}}}
\newcommand{\defarrow}{\ensuremath{\stackrel{\text{\tiny def}}{\longleftrightarrow}}}
\usepackage{theorem}
\newtheorem{definition}{Definition}
\newenvironment{proof}{\par\noindent\textbf{Proof.} }{\ \rule{0.5em}{0.5em}\par\bigskip}
\newtheorem{theorem}{Theorem}
\newtheorem{lemma}[theorem]{Lemma}
\newtheorem{proposition}[theorem]{Proposition}
\newtheorem{corollary}[theorem]{Corollary}
\newtheorem{example}{Example}
\newtheorem{question}{Question}

\newcommand{\gui}[1]{\flqq\,#1\,\frqq}
\newcommand{\todo}[1]{\textcolor{orange}{#1}}

\newcommand{\HRule}{\rule{\linewidth}{0.5mm}}

\newcommand{\eg}{\textit{e.g.},~}
\newcommand{\ie}{\textit{i.e.},~}
%%%%%%%%%%%%%%%%%%%%%%%%%%%%%%%%%%%%%%%%%%%%%%%%%%%%%%%%%%%%%%%%%%%%%%%%%%%%%%%%%%%%%

\begin{document}

\pagestyle{fancyfront}
\frontmatter

\input{title-doc}
\tableofcontents

\mainmatter
\pagestyle{plain}

%!TEX root = choco-doc.tex
\addcontentsline{toc}{chapter}{Preface}
\chapter*{Preface}
Choco is a java library for constraint satisfaction problems (CSP) and constraint programming (CP). It is built on a event-based propagation mechanism with backtrackable structures.
Choco is an open-source software, distributed under a \textbf{BSD licence} and hosted by \href{http://sourceforge.net/projects/choco/}{sourceforge.net}.
For any informations visit \url{http://choco.emn.fr}.
\bigskip

\noindent This document is organized as follows:
\begin{itemize}
\item \hyperlink{ch:doc}{Documentation} is the user-guide of Choco. After a short \hyperlink{doc:introduction}{introduction} to constraint programming and to the Choco solver, it presents the basics of \hyperlink{doc:model}{modeling} and \hyperlink{doc:solver}{solving} with Choco, and some \hyperlink{doc:advanced}{advanced usages} (customizing propagation and search).
\item \hyperlink{part:elements}{Elements of Choco} gives a detailed description of the \hyperlink{ch:vars}{variables}, \hyperlink{ch:operators}{operators}, \hyperlink{ch:constraints}{constraints} currently available in Choco.
\item \hyperlink{ch:extra}{Extras} presents future works, only available on the beta version or extension of the current jar, such as the \hyperlink{chocoandvisu:chocoandvisu}{visualization module of Choco}. The section dedicated to \hyperlink{sudokuandcp:sudokuandconstraintprogramming}{Sudoku} aims at explaining the basic principles of Constraint Programming (propagation and search) on this famous game.


\end{itemize}

\part{Documentation}\label{ch:doc}\hypertarget{ch:doc}{}

The documentation of Choco is organized as follows:
\begin{itemize}
\item 
The concise \hyperlink{doc:introduction}{introduction} provides some informations \hyperlink{introduction:aboutconstraintprogramming}{about constraint programming} concepts and a ``Hello world''-like \hyperlink{introduction:myfirstchocoprogram}{first Choco program}.
\item 
The \hyperlink{doc:model}{model} section gives informations on \hyperlink{doc:model}{how to create a model} and introduces \hyperlink{model:variables}{variables} and \hyperlink{model:constraints}{constraints}.
\item 
The \hyperlink{doc:solver}{solver} section gives informations on \hyperlink{doc:solver}{how to create a solver}, to \hyperlink{doc:solver}{read a model}, to define a \hyperlink{solver:searchstrategy}{search strategy}, and finally to \hyperlink{solver:solveaproblem}{solve a problem}.
\item 
The \hyperlink{doc:advanced}{advanced use} section explains how to define your own \hyperlink{advanced:defineyourownlimitsearchspace}{limit search space}, \hyperlink{advanced:defineyourownsearchstrategy}{search strategy}, \hyperlink{advanced:defineyourownconstraint}{constraint}, \hyperlink{advanced:defineyourownoperator}{operator}, \hyperlink{advanced:defineyourownvariable}{variable}, \hyperlink{advanced:backtrackablestructures}{backtrackable structure} and write \hyperlink{advanced:howtowriteloggingstatements}{logging statements}.
%The \hyperlink{doc:applications}{applications} section shows the use of Choco defined global constraints on \hyperlink{schedulinganduseofthecumulative:schedulinganduseofthecumulativeconstraint}{scheduling} or \hyperlink{geostdescription:placementanduseofthegeostconstraint}{placement} problems.
%\item 
%Lastly, the catalog of Choco defined \hyperlink{ch:constraints}{constraints} is presented.
\end{itemize}

%\section*{Beta}\label{documentation:beta}\hypertarget{documentation:beta}{}
%Here you can find short documentation concerning futur works, only available on the beta version or extension of the current jar:
%\begin{itemize}
%%	\item \hyperlink{chocoandgraphviz}{Choco and Graphviz} \emph{not yet available. Has been included in \hyperlink{chocoandvisu}{Choco and Visu}.}
%	\item \hyperlink{chocoandvisu}{Choco and Visu}
%\end{itemize}

%\section*{Material}\label{documentation:material}\hypertarget{documentation:material}{}
%Here you can find some materials at your disposal. If you create any, you can also send it to us, so we can add it to the list.

%\subsection{Presentations}\label{documentation:presentations}\hypertarget{documentation:presentations}{}
%\begin{itemize}
%	\item \textbf{August 2008} : The \emph{CHOCO : an Open Source Java Constraint Constraint Programming} white paper presentation send to the \href{http://www.cril.univ-artois.fr/cpai08/}{CSP'08 competition}. \href{media/pdf/choco-presentation.pdf}{PDF} of white paper presentation of Choco
%	\item \textbf{june 2008} : \emph{The CHOCO constraint programming solver} presentation held at the \href{https://projects.coin-or.org/events/wiki/cpaior2008}{Workshop on Open-Source Software for Integer and Constraint Programming} during the last \href{http://contraintes.inria.fr/cpaior08/}{CPAIOR} conference. \href{media/slides/cpaior-choco.pdf}{PDF slides} of the presentation given by Guillaume Rochart
%\end{itemize}

%!TEX root = ../content-doc.tex
\chapter{Introduction to constraint programming and Choco}\label{doc:introduction}\hypertarget{doc:introduction}{}

\section{About constraint programming}\label{introduction:aboutconstraintprogramming}\hypertarget{introduction:aboutconstraintprogramming}{}

\begin{myquote}
Constraint programming represents one of the closest approaches computer science has yet made to the Holy Grail of programming: the user states the problem, the computer solves it.
\begin{flushright}\bf E. C. Freuder, Constraints, 1997.\end{flushright}
\end{myquote}


Fast increasing computing power in the 1960s led to a wealth of works around problem solving, at the root of Operational Research, Numerical Analysis, Symbolic Computing, Scientific Computing, and a large part of Artificial Intelligence and programming languages. Constraint Programming is a discipline that gathers, interbreeds, and unifies ideas shared by all these domains to tackle decision support problems.

Constraint programming has been successfully applied in numerous domains. Recent applications include computer graphics (to express geometric coherence in the case of scene analysis), natural language processing (construction of efficient parsers), database systems (to ensure and/or restore consistency of the data), operations research problems (scheduling, routing), molecular biology (DNA sequencing), business applications (option trading), electrical engineering (to locate faults), circuit design (to compute layouts), etc.

Current research in this area deals with various fundamental issues, with implementation aspects and with new applications of constraint programming.

\subsection{Constraints}\label{introduction:constraints}\hypertarget{introduction:constraints}{}
A constraint is simply a logical relation among several unknowns (or variables), each taking a value in a given domain. A constraint thus restricts the possible values that variables can take, it represents some partial information about the variables of interest. For instance, the circle is inside the square relates two objects without precisely specifying their positions, i.e., their coordinates. Now, one may move the square or the circle and one is still able to maintain the relation between these two objects. Also, one may want to add another object, say a triangle, and to introduce another constraint, say the square is to the left of the triangle. From the user (human) point of view, everything remains absolutely transparent.

Constraints naturally meet several interesting properties:
\begin{itemize}
	\item constraints may specify partial information, i.e. constraint need not uniquely specify the values of its variables,
	\item constraints are non-directional, typically a constraint on (say) two variables $X, Y$ can be used to infer a constraint on $X$ given a constraint on $Y$ and vice versa,
	\item constraints are declarative, i.e. they specify what relationship must hold without specifying a computational procedure to enforce that relationship,
	\item constraints are additive, i.e. the order of imposition of constraints does not matter, all that matters at the end is that the conjunction of constraints is in effect,
	\item constraints are rarely independent, typically constraints in the constraint store share variables.
\end{itemize}

Constraints arise naturally in most areas of human endeavor. The three angles of a triangle sum to 180 degrees, the sum of the currents flowing into a node must equal zero, the position of the scroller in the window scrollbar must reflect the visible part of the underlying document, these are some examples of constraints which appear in the real world. Thus, constraints are a natural medium for people to express problems in many fields. 

\subsection{Constraint Programming}\label{introduction:constraintprogramming}\hypertarget{introduction:constraintprogramming}{}
Constraint programming is the study of computational systems based on constraints. The idea of constraint programming is to solve problems by stating constraints (conditions, properties) which must be satisfied by the solution.

Work in this area can be tracked back to research in Artificial Intelligence and Computer Graphics in the sixties and seventies. Only in the last decade, however, has there emerged a growing realization that these ideas provide the basis for a powerful approach to programming, modeling and problem solving and that different efforts to exploit these ideas can be unified under a common conceptual and practical framework, constraint programming. 

\begin{note}
If you know \textbf{sudoku}, then you know \textbf{constraint programming}. See why \hyperlink{sudokuandcp}{here}.
\end{note}


\section{Modeling with Constraint programming}\label{introduction:modelingwithconstraintprogramming}\hypertarget{introduction:modelingwithconstraintprogramming}{}
The formulation and the resolution of combinatorial problems are the two main goals of the constraint programming domain. This is an essential way to solve many interesting industrial problems such as scheduling, planning or design of timetables. 

\subsection{The Constraint Satisfaction Problem}\label{introduction:csp}\hypertarget{introduction:csp}{}

Constraint programming allows to solve combinatorial problems modeled by a Constraint Satisfaction Problem (CSP). Formally, a CSP is defined by a triplet $(X,D,C)$:
\begin{itemize}
	\item \textbf{Variables}: $X = \{X_1,X_2,\ldots,X_n\}$ is the set of variables of the problem.
	\item \textbf{Domains}: $D$ is a function which associates to each variable $X_i$ its domain $D(X_i)$, i.e. the set of possible values that can be assigned to $X_i$. The domain of a variable is usually a finite set of integers: $D(X_i)\subset\Z$ (\emph{integer variable}). But a domain can also be continuous ($D(X_i)\subseteq\R$ for a \emph{real variable}) or made of discrete set values ($D(X_i)\subseteq\mathcal{P}(\Z)$ for a \emph{set variable}).
	\item \textbf{Constraints}: $C = \{C_1,C_2,\ldots,C_m\}$ is the set of constraints. A constraint $C_j$ is a relation defined on a subset $X^j = \{X^j_1,X^j_2,\ldots,X^j_{n^j}\}\subseteq X$ of variables which restricts the possible tuples of values $(v_1,\ldots,v_{n^j})$ for these variables:
$$(v_1,\ldots,v_{n^j})\in C_j\cap (D(X^j_1)\times D(X^j_2)\times\cdots\times D(X^j_{n^j})).$$
Such a relation can be defined explicitely (ex: $(X_1,X_2)\in\{(0,1),(1,0)\}$) or implicitely (ex: $X_1+X_2\le 1$).
\end{itemize}

Solving a CSP consists in finding a tuple $v=(v_1,\ldots,v_{n})\in D(X)$ on the set of variables which satisfies all the constraints:
$$(v_1,\ldots,v_{n^j})\in C_j,\quad\forall j\in\{1,\ldots,m\}.$$

For optimization problems, one needs to define an \textbf{objective function} $f:D(X)\rightarrow\R$. An optimal solution is then a solution tuple of the CSP that minimizes (or maximizes) function $f$.

\subsection{Examples of CSP models}\label{introduction:examples}\hypertarget{introduction:examples}{}
This part provides three examples using different types of variables in different problems. These examples are used throughout this tutorial to illustrate their modeling with Choco.

\subsubsection{Example 1: the n-queens problem.}\label{introduction:example1:nqueens}\hypertarget{introduction:example1:nqueens}{}
Let us consider a chess board with $n$ rows and $n$ columns. A queen can move as far as she pleases, horizontally, vertically, or diagonally. The standard $n$-queens problem asks how to place $n$ queens on an $n$-ary chess board so that none of them can hit any other in one move.

The $n$-queens problem can be modeled by a CSP in the following way:
\begin{itemize}
	\item \textbf{Variables}: $X = \{X_{i}\ |\ i\in [1,n]\}$; one variable represents a column and the constraint "queens must be on different columns" is induced by the variables.
	\item \textbf{Domain}: for all variable $X_{i}\in X$, $D(X_{i}) = \{1,2,\ldots, n\}$.
	\item \textbf{Constraints}: the set of constraints is defined by the union of the three following constraints,
	\begin{itemize}
		\item queens have to be on distinct lines:
		\begin{itemize}
			\item $C_{lines} = \{X_{i}\neq X_{j}\ |\ i,j\in [1,n], i\neq j\}$.
		\end{itemize}
		\item queens have to be on distinct diagonals:
		\begin{itemize}
			\item $C_{diag1} = \{X_{i}\neq X_{j}+j-i\ |\ i,j\in [1,n], i\neq j\}$.
			\item $C_{diag2} = \{X_{i}\neq X_{j}+i-j\ |\ i,j\in [1,n], i\neq j\}$.
		\end{itemize}
	\end{itemize}
\end{itemize}

\subsubsection{Example 2: the ternary Steiner problem.}\label{introduction:example2:theternarysteinerproblem}\hypertarget{introduction:example2:theternarysteinerproblem}{}
A ternary Steiner system of order $n$ is a set of $n*(n-1)/6$ triplets of distinct elements taking their values in $[1,n]$, such that all the pairs included in two distinct triplets are different.
See \url{http://en.wikipedia.org/wiki/Steiner_system} for details. 

The ternary Steiner problem can be modeled by a CSP in the following way:
\begin{itemize}
	\item let $t = n*(n-1)/6$.
	\item \textbf{Variables}: $X = \{X_{i}\ |\ i\in [1,t]\}$.
	\item \textbf{Domain}: for all $i\in [1,t]$, $D(X_{i}) = \{1,...,n\}$.
	\item \textbf{Constraints}:
	\begin{itemize}
		\item every set variable $X_i$ has a cardinality of 3:
		\begin{itemize}
			\item for all $i\in [1,t]$, $|X_{i}| = 3$.
		\end{itemize}
		\item the cardinality of the intersection of every two distinct sets must not exceed 1:
		\begin{itemize}
			\item for all $i,j\in [1,t]$, $i\neq j$, $|X_{i}\cap X_{j}|\le 1$.
		\end{itemize}
	\end{itemize}
\end{itemize}

\subsubsection{Example 3: the CycloHexane problem.}\label{introduction:example3:thecyclohexaneproblem}\hypertarget{introduction:example3:thecyclohexaneproblem}{}
The problem consists in finding the 3D configuration of a cyclohexane molecule. It is described with a system of three non linear equations:
\begin{itemize}
	\item \textbf{Variables}: $x,y,z$.
	\item \textbf{Domain}: $]-\infty;+\infty[$.
	\item \textbf{Constraints}:
	\begin{align*}
		y^{2} * (1 + z^{2}) + z * (z - 24 * y) &= -13\\
		x^{2} * (1 + y^{2}) + y * (y - 24 * x) &= -13\\
		z^{2} * (1 + x^{2}) + x * (x - 24 * z) &= -13
	\end{align*}
\end{itemize}

\section{My first Choco program: the magic square}\label{introduction:myfirstchocoprogram}\hypertarget{introduction:myfirstchocoprogram}{}

\subsection{The magic square problem}\label{introduction:amagicsquareproblem}\hypertarget{introduction:amagicsquareproblem}{}
In the following, we will address the magic square problem of order 3 to illustrate step-by-step how to model and solve this problem using choco. 

\subsubsection{Definition:}
A magic square of order $n$ is an arrangement of $n^{2}$ numbers, usually distinct integers, in a square, such that the $n$ numbers in all rows, all columns, and both diagonals sum to the same constant. A standard magic square contains the integers from 1 to $n^{2}$.

The constant sum in every row, column and diagonal is called the magic constant or magic sum $M$. The magic constant of a classic magic square depends only on $n$ and has the value:
$M(n)=n(n^2 +1)/2$.

\href{http://en.wikipedia.org/wiki/Magic_square}{More details on the magic square problem.}


\subsection{A mathematical model}\label{introduction:mathematicalmodeling}\hypertarget{introduction:mathematicalmodeling}{}

Let $x_{ij}$ be the variable indicating the value of the $j^{th}$ cell of row $i$. 
Let $C$ be the set of constraints modeling the magic square as:
\begin{align*}
&x_{ij} \in [1,n^2],\ &&\forall i,j \in [1, n]\\
&x_{ij}\ne x_{kl},\ &&\forall i,j,k,l \in [1,n], i\ne k, j\ne l\\
&\sum_{j=1}^{n} x_{ij} = n(n^2 +1)/2,\ &&\forall i \in [1,n]\\
&\sum_{i=1}^{n} x_{ij} = n(n^2 +1)/2,\ &&\forall j \in [1,n]\\
&\sum_{i=1}^{n} x_{ii} = n(n^2 +1)/2&&\\
&\sum_{i=n}^{1} x_{i(n-i)} = n(n^2 +1)/2&&\\
\end{align*}

We have all the required information to model the problem with Choco.
\begin{note}
	For the moment, we just talk about \emph{model translation} from a mathematical representation to Choco.
	Choco can be used as a \emph{black box}, that means we just need to define the problem without knowing the way it will be solved. We can therefore focus on the modeling not on the solving.
\end{note}

\subsection{To Choco...}\label{introduction:inchoco}\hypertarget{introduction:inchoco}{}

First, we need to know some of the basic Choco objects:
\begin{itemize}
\item 
The \textbf{model} (object \texttt{Model} in Choco) is one of the central elements of a Choco program. Variables and constraints are associated to it.
\item
The \textbf{variables} (objects \texttt{IntegerVariable}, \texttt{SetVariable}, and \texttt{RealVariable} in Choco) are the \emph{unknowns} of the problem. Values of variables are taken from a \textbf{domain} which is defined by a set of values or quite often simply by a lower bound and an upper bound of the allowed values. The domain is given when creating the variable.
\begin{note}
Do not forget that we manipulate \textbf{variables} in the mathematical sense (as opposed to classical computer science). Their effective value will be known only once the problem has been solved.
\end{note}
\item
The \textbf{constraints} define relations to be satisfied between variables and constants.
In our first model, we only use the following constraints provided by Choco:
\begin{itemize}
	\item \texttt{eq(var1, var2)} which ensures that \texttt{var1} equals \texttt{var2}.
	\item \texttt{neq(var1, var2)} which ensures that \texttt{var1} is not equal to \texttt{var2}.
	\item \texttt{sum(var[])} which returns expression \texttt{var[0]+var[1]+...+var[n]}.
\end{itemize}
\end{itemize}

\subsection{The program}\label{introduction:theprogram}\hypertarget{introduction:theprogram}{}
After having created your java class file, import the Choco class to use the API:
\begin{lstlisting}
  import choco.Choco;
\end{lstlisting}
First of all, let's create a Model:
\lstinputlisting{java/imagicsquare1.j2t}
We create an instance of \texttt{CPModel()} for \textbf{C}onstraint \textbf{P}rogramming Model.
Do not forget to add the following imports:
\begin{lstlisting}
  import choco.cp.model.CPModel;
\end{lstlisting}
Then we declare the variables of the problem:
\lstinputlisting{java/imagicsquare2.j2t}
Add the import:
\begin{lstlisting}
  import choco.kernel.model.variables.integer.IntegerVariable;
\end{lstlisting}
We have defined the variable using the \texttt{makeIntVar} method which creates an enumerated domain: all the values are stored in the java object (beware, it is usually not necessary to store all the values and it is less efficient than to create a bounded domain).

\noindent Now, we are going to state a constraint ensuring that all variables must have a different value:
\lstinputlisting{java/imagicsquare3.j2t}
Add the import:
\begin{lstlisting}
  import choco.kernel.model.constraints.Constraint;
\end{lstlisting}
Then, we add the constraint ensuring that the magic sum is respected:
\lstinputlisting{java/imagicsquare4.j2t}
Then we define the constraint ensuring that each column is equal to the magic sum.
Actually, \texttt{var} just denotes the rows of the square. So we have to declare a temporary array of variables that defines the columns.
\lstinputlisting{java/imagicsquare5.j2t}
It is sometimes useful to define some temporary variables to keep the model simple or to reorder array of variables. That is why we also define two other temporary arrays for diagonals.
\lstinputlisting{java/imagicsquare6.j2t}
Now, we have defined the model. The next step is to solve it.
For that, we build a Solver object:
\lstinputlisting{java/imagicsquare7.j2t}

with the imports:
\begin{lstlisting}
  import choco.cp.solver.CPSolver;
\end{lstlisting}
We create an instance of \texttt{CPSolver()} for Constraint Programming Solver.
Then, the solver reads (translates) the model and solves it:
\lstinputlisting{java/imagicsquare8.j2t}
The only variables that need to be printed are the ones in \texttt{var} (all the others are only references to these ones). 
\begin{note}
We have to use the Solver to get the value of each variable of the model. The Model only declares the objects, the Solver finds their value.
\end{note}
We are done, we have created our first Choco program. 
%The complete source code can be found here: \href{media/zip/exmagicsquare.zip}{ExMagicSquare.zip}


\subsection{In summary}\label{introduction:whatisimportant}\hypertarget{introduction:whatisimportant}{}
\begin{itemize}
	\item A Choco Model is defined by a set of Variables with a given domain and a set of Constraints that link Variables:
it is necessary to add both Variables and Constraints to the Model.
	\item temporary Variables are useful to keep the Model readable, or necessary when reordering arrays.
	\item The value of a Variable can be known only once the Solver has found a solution.
	\item To keep the code readable, you can avoid the calls to the static methods of the Choco classes, by importing the static classes, i.e. instead of:
\begin{lstlisting}
  import choco.Choco;
  ...
  IntegerVariable v = Choco.makeIntVar("v", 1, 10);
  ...
  Constraint c = Choco.eq(v, 5);
\end{lstlisting}
you can use:
\begin{lstlisting}
  import static choco.Choco.*;
  ...
  IntegerVariable v = makeIntVar("v", 1, 10);
  ...
  Constraint c = eq(v, 5);
\end{lstlisting}
\end{itemize}

\section{Complete examples}\label{model:completeexamples}\hypertarget{model:completeexamples}{}
We provide now the complete Choco model for the three examples \hyperlink{introduction:examples}{previously described}.

\subsection{Example 1: the n-queens problem with Choco}\label{model:example1:nqueenschoco}\hypertarget{model:example1:nqueenschoco}{}
This first model for the \hyperlink{introduction:example1:nqueens}{n-queens problem} only involves binary constraints of differences between integer variables. One can immediately recognize the 4 main elements of any Choco code. First of all, create the model object. Then create the variables by using the Choco API (One variable per queen giving the row (or the column) where the queen will be placed). Finally, add the constraints and solve the problem. 

\lstinputlisting{java/inqueen.j2t}

\subsection{Example 2: the ternary Steiner problem with Choco}\label{model:example2:ternarysteinerchoco}\hypertarget{model:example2:ternarysteinerchoco}{}
The \hyperlink{introduction:example2:theternarysteinerproblem}{ternary Steiner problem} is entirely modeled using set variables and set constraints. 
\lstinputlisting{java/iternarysteiner.j2t}

\subsection{Example 3: the CycloHexane problem with Choco}\label{model:example3:thecyclohexaneproblemwithchoco}\hypertarget{model:example3:thecyclohexaneproblemwithchoco}{}
Real variables are illustrated on the problem of finding the 3D configuration of a cyclohexane molecule. 
\lstinputlisting{java/icyclohexane.j2t}



%!TEX root = ../content-doc.tex
\chapter{The model}\label{doc:model}\hypertarget{doc:model}{}

The {\tt Model}, along with the {\tt Solver}, is one of the two key elements of any Choco program. The Choco {\tt Model} allows to describe a problem in an easy and declarative way: it simply records the variables and the constraints defining the problem.

This section describes the large API provided by Choco to create different types of \hyperlink{model:variables}{variables} and \hyperlink{model:constraints}{constraints}.

%\begin{note}
\textbf{Note that a static import is required to use the Choco API:}
\begin{lstlisting}
  import static choco.Choco.*;
\end{lstlisting}
%It is mandatory in order to compile !
%\end{note}

%\section{How to create a model}\label{model:howtocreateamodel}\hypertarget{model:howtocreateamodel}{}
First of all, a {\tt Model} object is created as follows:
\begin{lstlisting}
Model model = new CPModel();
\end{lstlisting}
In that specific case, a Constraint Programming (CP) {\tt Model} object has been created. 


%%%%%%%%%%%%%%%%%%%%%%%%%%%%%%%%%%%%%%%%%%%%%%%%%%%%%%%%%%%%%%%%%%%%%%%%%%%%%%%%%%%%%%%%%%%%%%%%%%%%%%%%%%%%%%%%%%%%%%%%%%%%%%%%%%%%%%%%%%%%%%%%%%%
%%%%%%%%%%%%%%%%%%%%%%%%%%%%%%%%%%%%%%%%%%%%%%%%%%%%% VARIABLE %%%%%%%%%%%%%%%%%%%%%%%%%%%%%%%%%%%%%%%%%%%%%%%%%%%%%%%%%%%%%%%%%%%%%%%%%%%%%%%%%%%%
%%%%%%%%%%%%%%%%%%%%%%%%%%%%%%%%%%%%%%%%%%%%%%%%%%%%%%%%%%%%%%%%%%%%%%%%%%%%%%%%%%%%%%%%%%%%%%%%%%%%%%%%%%%%%%%%%%%%%%%%%%%%%%%%%%%%%%%%%%%%%%%%%%%


\section{Variables}\label{model:variables}\hypertarget{model:variables}{}

%Choco provides a large API to create different types of variables : \textbf{integer}, \textbf{real} and \textbf{set}. 

A Variable is defined by a type (\hyperlink{integervariable}{integer}, \hyperlink{realvariable}{real}, or \hyperlink{setvariable}{set} variable), a name, and the values of its domain. When creating a simple variable, some options can be set to specify its domain representation (\eg enumerated or bounded) within the {\tt Solver}.
%Some kinds of variables have options for their domain, it may have an effect on what kind of specific object is created when the model is read by the solver.
\begin{note}
The choice of the domain should be considered. The efficiency of the solver often depends on judicious choice of the domain type.
\end{note}
Variables can be combined as \hyperlink{model:expressionvariables}{expression variables} using operators.

One or more variables can be added to the model using the following methods of the \texttt{Model} class:
\lstinputlisting{java/mvariabledeclaration1.j2t}

\begin{note}
Explictly addition of variables is not mandatory. See \hyperlink{model:constraints}{\tt Constraint} for more details.
\end{note}

Specific role of variables \emph{var} can be defined with \emph{options}:  \hyperlink{model:decisionvariables}{non-decision} variables or  \hyperlink{model:objectivevariable}{objective} variable;
\lstinputlisting{java/mvariabledeclaration2.j2t}

%%%%%%%%%%%%%%%%%%%%%%%%%%%%%%%%%%%%%%%%%%%%%%%%%%%%%%%%%%%%%%%%%%%%%%%%%%%%%%%%%%%%%%%%%%%%%%%%%%%%%%%%%%%%%%%%%%%%%%%%%%%%%%%%%%%%%%%%%%%%%%%%%%%
%%%%%%%%%%%%%%%%%%%%%%%%%%%%%%%%%%%%%%%%%%%%%%%%%%%%% SIMPLE VARIABLE  %%%%%%%%%%%%%%%%%%%%%%%%%%%%%%%%%%%%%%%%%%%%%%%%%%%%%%%%%%%%%%%%%%%%%%%%%%%%

\subsection{Simple Variables}\label{model:simplevariables}\hypertarget{model:simplevariables}{}
See Section \hyperlink{ch:vars}{Variables} for details:

\begin{notedef}\tt
\hyperlink{integervariable}{IntegerVariable}, \hyperlink{setvariable}{SetVariable}, \hyperlink{realvariable}{RealVariable}
\end{notedef}

%%%%%%%%%%%%%%%%%%%%%%%%%%%%%%%%%%%%%%%%%%%%%%%%%%%%%%%%%%%%%%%%%%%%%%%%%%%%%%%%%%%%%%%%%%%%%%%%%%%%%%%%%%%%%%%%%%%%%%%%%%%%%%%%%%%%%%%%%%%%%%%%%%%
%%%%%%%%%%%%%%%%%%%%%%%%%%%%%%%%%%%%%%%%%%%%%%%%%%%%% CONSTANT VARIABLE  %%%%%%%%%%%%%%%%%%%%%%%%%%%%%%%%%%%%%%%%%%%%%%%%%%%%%%%%%%%%%%%%%%%%%%%%%%

\subsection{Constants}\label{model:constants}\hypertarget{model:constants}{}
A constant is a variable with a fixed domain. An \hyperlink{integervariable}{\tt IntegerVariable} declared with a unique value is automatically set as constant. A constant declared twice or more is only stored once in a model.

\lstinputlisting{java/mconstant.j2t}

%%%%%%%%%%%%%%%%%%%%%%%%%%%%%%%%%%%%%%%%%%%%%%%%%%%%%%%%%%%%%%%%%%%%%%%%%%%%%%%%%%%%%%%%%%%%%%%%%%%%%%%%%%%%%%%%%%%%%%%%%%%%%%%%%%%%%%%%%%%%%%%%%%%
%%%%%%%%%%%%%%%%%%%%%%%%%%%%%%%%%%%%%%%%%%%%%%%%%%%%% EXPRESSION VARIABLE  %%%%%%%%%%%%%%%%%%%%%%%%%%%%%%%%%%%%%%%%%%%%%%%%%%%%%%%%%%%%%%%%%%%%%%%%
\subsection{Expression variables and operators}\label{model:expressionvariables}\hypertarget{model:expressionvariables}{}
Expression variables represent the result of combinations between variables of the same type made by operators. Two types of expression variables exist : 
\begin{notedef}
\textbf{\tt IntegerExpressionVariable} and \textbf{\tt RealExpressionVariable}.
\end{notedef}
One can define an expression variable to define an operation, for example:
\lstinputlisting{java/mexpressionvariable.j2t}

%\section{Operators}\label{model:operators}\hypertarget{model:operators}{}

To construct expressions of variables, simple operators can be used. Each returns a \texttt{ExpressionVariable} object:
\begin{notedef}\tt
\begin{itemize}
\item Integer : \hyperlink{abs:absoperator}{abs}, \hyperlink{div:divoperator}{div}, \hyperlink{ifthenelse:ifthenelseoperator}{ifThenElse}, \hyperlink{max:maxoperator}{max}, \hyperlink{min:minoperator}{min}, \hyperlink{minus:minusoperator}{minus}, \hyperlink{mod:modoperator}{mod}, \hyperlink{mult:multoperator}{mult}, \hyperlink{neg:negoperator}{neg}, \hyperlink{plus:plusoperator}{plus}, \hyperlink{power:poweroperator}{power}, \hyperlink{scalar:scalaroperator}{scalar}, \hyperlink{sum:sumoperator}{sum},
\item Real : \hyperlink{cos:cosoperator}{cos}, \hyperlink{minus:minusoperator}{minus}, \hyperlink{mult:multoperator}{mult}, \hyperlink{plus:plusoperator}{plus}, \hyperlink{power:poweroperator}{power}, \hyperlink{sin:sinoperator}{sin}
\end{itemize}
\end{notedef}
Note that these operators are not considered as constraints: they do not return a \texttt{Constraint} objet but a \texttt{Variable} object.

%%%%%%%%%%%%%%%%%%%%%%%%%%%%%%%%%%%%%%%%%%%%%%%%%%%%%%%%%%%%%%%%%%%%%%%%%%%%%%%%%%%%%%%%%%%%%%%%%%%%%%%%%%%%%%%%%%%%%%%%%%%%%%%%%%%%%%%%%%%%%%%%%%%
%%%%%%%%%%%%%%%%%%%%%%%%%%%%%%%%%%%%%%%%%%%%%%%%%%%%% MULTIPLE VARIABLE  %%%%%%%%%%%%%%%%%%%%%%%%%%%%%%%%%%%%%%%%%%%%%%%%%%%%%%%%%%%%%%%%%%%%%%%%%%

\subsection{MultipleVariables}\label{model:multiplevariables}\hypertarget{model:multiplevariables}{}
These are syntaxic sugar. To make their declaration easier, \hyperlink{tree:treeconstraint}{\tt tree}, \hyperlink{geost:geostconstraint}{\tt geost}, and scheduling constraints allow or require to use multiple variables, like \texttt{TreeParametersObject}, \texttt{GeostObject} or \hyperlink{taskvariable}{\tt TaskVariable}.
See also the code examples for these constraints.

%%%%%%%%%%%%%%%%%%%%%%%%%%%%%%%%%%%%%%%%%%%%%%%%%%%%%%%%%%%%%%%%%%%%%%%%%%%%%%%%%%%%%%%%%%%%%%%%%%%%%%%%%%%%%%%%%%%%%%%%%%%%%%%%%%%%%%%%%%%%%%%%%%%
%%%%%%%%%%%%%%%%%%%%%%%%%%%%%%%%%%%%%%%%%%%%%%%%%%%%% OPTIONS %%%%%%%%%%%%%%%%%%%%%%%%%%%%%%%%%%%%%%%%%%%%%%%%%%%%%%%%%%%%%%%%%%%%%%%%%%%%%%%%%%%%%

\subsection{Decision/non-decision variables}\label{model:decisionvariables}\hypertarget{model:decisionvariables}{}

By default, each variable added to a model is a decision variable, \textit{i.e.} is included in the default search strategy. A variable can be stated as a non decision one if its value can be computed by side-effect. To specify non decision variables, one can 
\begin{itemize}
\item exclude them from the search strategies (see \hyperlink{solver:searchstrategy}{search strategy} for more details);
\item specify non-decision variables (adding \hyperlink{vnodecision:vnodecisionoptions}{\tt Options.V\_NO\_DECISION} to their options) and keep the default search strategy.
\end{itemize}
\lstinputlisting{java/mnodecision1.j2t}
Each of these options can also be set within a single instruction for a group of variables, as follows: 
\lstinputlisting{java/mnodecision2.j2t}

\begin{note}
 The declaration of a \hyperlink{solver:searchstrategy}{search strategy} will erase setting \hyperlink{vnodecision:vnodecisionoptions}{\tt Options.V\_NO\_DECISION}.
\end{note}
  \todo{more precise: user-defined/pre-defined, variable and/or value heuristics ?}

\subsection{Objective variable}\label{model:objectivevariable}\hypertarget{model:objectivevariable}{}
You can define an objective variable directly within the model, by using option \hyperlink{vobjective:vobjectiveoptions}{\tt Options.V\_OBJECTIVE}:
\lstinputlisting{java/mobjective.j2t}

Only one variable can be defined as an objective. If more than one objective variable is declared, then only the last one will be taken into account.

Note that optimization problems can be declared without defining an objective variable within the model (see the \hyperlink{solver:optimization}{optimization example}.)

%%%%%%%%%%%%%%%%%%%%%%%%%%%%%%%%%%%%%%%%%%%%%%%%%%%%%%%%%%%%%%%%%%%%%%%%%%%%%%%%%%%%%%%%%%%%%%%%%%%%%%%%%%%%%%%%%%%%%%%%%%%%%%%%%%%%%%%%%%%%%%%%%%%
%%%%%%%%%%%%%%%%%%%%%%%%%%%%%%%%%%%%%%%%%%%%%%%%%%%%% CONSTRAINT  %%%%%%%%%%%%%%%%%%%%%%%%%%%%%%%%%%%%%%%%%%%%%%%%%%%%%%%%%%%%%%%%%%%%%%%%%%%%%%%%%
%%%%%%%%%%%%%%%%%%%%%%%%%%%%%%%%%%%%%%%%%%%%%%%%%%%%%%%%%%%%%%%%%%%%%%%%%%%%%%%%%%%%%%%%%%%%%%%%%%%%%%%%%%%%%%%%%%%%%%%%%%%%%%%%%%%%%%%%%%%%%%%%%%%

\section{Constraints}\label{model:constraints}\hypertarget{model:constraints}{}
Choco provides a large number of simple and global constraints and allows the user to easily define its own new constraint.
% Either basic, global (a \hyperlink{constraints}{large set of global constraints} are available) or \hyperlink{advanced:defineyourownconstraint}{user-defined} constraints, they are used to specify conditions to be held on variables to the model. 
A constraint deals with one or more variables of the model and specify conditions to be held on these variables. 
A constraint is stated into the model by using the following methods available from the \texttt{Model} API: 

\lstinputlisting{java/mconstraintdeclaration1.j2t}

\begin{note}\
Adding a constraint automatically adds its variables to the model (explicit declaration of variables addition is not mandatory).
Thus, a variable not involved in any constraints will not be declared in the Solver during the reading step.
\end{note}


\subsubsection{Example:} adding a difference (disequality) constraint between two variables of the model

\lstinputlisting{java/mconstraintdeclaration2.j2t}

Available \emph{options} depend on the kind of constraint \emph{c} to add: they allow, for example, to choose the filtering algorithm to run during propagation. See \hyperlink{optionssettings}{Section options and settings} for more details, specific APIs exist for declaring options constraints.

This section presents the constraints available in the Choco API sorted by type or by domain. Related sections:
\begin{itemize}
\item a detailed description (with options, examples, references) of each constraint is given in Section \hyperlink{ch:constraints}{constraints}
%\item Section \hyperlink{doc:applications}{applications} shows how to apply some specific global constraints
\item Section \hyperlink{advanced:defineyourownconstraint}{user-defined constraint} explains how to create its own constraint.
\end{itemize}

%%%%%%%%%%%%%%%%%%%%%%%%%%%%%%%%%%%%%%%%%%%%%%%%%%%%%%%%%%%%%%%%%%%%%%%%%%%%%%%%%%%%%%%%%%%%%%%%%%%%%%%%%%%%%%%%%%%%%%%%%%%%%%%%%%%%%%%%%%%%%%%%%%%
%%%%%%%%%%%%%%%%%%%%%%%%%%%%%%%%%%%%%%%%%%%%%%%%%%%%% BINARY CONSTRAINT  %%%%%%%%%%%%%%%%%%%%%%%%%%%%%%%%%%%%%%%%%%%%%%%%%%%%%%%%%%%%%%%%%%%%%%%%%%

\subsection{Binary constraints}\label{model:comparisonconstraints}\hypertarget{model:comparisonconstraints}{}
%The simplest constraints are comparisons which are defined over expressions of variables such as linear combinations. The following comparison constraints can be accessed through the \texttt{Model} API:
Constraints involving two integer variables
\begin{notedef}\tt
  \begin{itemize}
  \item \hyperlink{eq:eqconstraint}{eq}, \hyperlink{geq:geqconstraint}{geq}, \hyperlink{gt:gtconstraint}{gt}, \hyperlink{leq:leqconstraint}{leq}, \hyperlink{lt:ltconstraint}{lt}, \hyperlink{neq:neqconstraint}{neq}
  \item \hyperlink{abs:absconstraint}{abs}, \hyperlink{oppositesign:oppositesignconstraint}{oppositeSign}, \hyperlink{samesign:samesignconstraint}{sameSign}
  \end{itemize}
\end{notedef}

%%%%%%%%%%%%%%%%%%%%%%%%%%%%%%%%%%%%%%%%%%%%%%%%%%%%%%%%%%%%%%%%%%%%%%%%%%%%%%%%%%%%%%%%%%%%%%%%%%%%%%%%%%%%%%%%%%%%%%%%%%%%%%%%%%%%%%%%%%%%%%%%%%%
%%%%%%%%%%%%%%%%%%%%%%%%%%%%%%%%%%%%%%%%%%%%%%%%%%%%% TERNARY CONSTRAINT  %%%%%%%%%%%%%%%%%%%%%%%%%%%%%%%%%%%%%%%%%%%%%%%%%%%%%%%%%%%%%%%%%%%%%%%%%

\subsection{Ternary constraints}\label{model:ternaryconstraints}\hypertarget{model:ternaryconstraints}{}
Constraints involving three integer variables
\begin{notedef}\tt
  \begin{itemize}
  \item \hyperlink{distanceeq:distanceeqconstraint}{distanceEQ}, \hyperlink{distanceneq:distanceneqconstraint}{distanceNEQ}, \hyperlink{distancegt:distancegtconstraint}{distanceGT}, \hyperlink{distancelt:distanceltconstraint}{distanceLT}
  \item \hyperlink{intdiv:intdivconstraint}{intDiv}, \hyperlink{mod:modconstraint}{mod}, \hyperlink{times:timesconstraint}{times}
  \end{itemize}
\end{notedef}

%%%%%%%%%%%%%%%%%%%%%%%%%%%%%%%%%%%%%%%%%%%%%%%%%%%%%%%%%%%%%%%%%%%%%%%%%%%%%%%%%%%%%%%%%%%%%%%%%%%%%%%%%%%%%%%%%%%%%%%%%%%%%%%%%%%%%%%%%%%%%%%%%%%
%%%%%%%%%%%%%%%%%%%%%%%%%%%%%%%%%%%%%%%%%%%%%%%%%%%%% REAL CONSTRAINT  %%%%%%%%%%%%%%%%%%%%%%%%%%%%%%%%%%%%%%%%%%%%%%%%%%%%%%%%%%%%%%%%%%%%%%%%%%%%

\subsection{Constraints involving real variables}\label{model:realconstraints}\hypertarget{model:realconstraints}{}
%The simplest constraints are comparisons which are defined over expressions of variables such as linear combinations. The following comparison constraints can be accessed through the \texttt{Model} API:
Constraints involving two real variables
\begin{notedef}\tt
  \begin{itemize}
  \item \hyperlink{eq:eqconstraint}{eq}, \hyperlink{geq:geqconstraint}{geq}, \hyperlink{leq:leqconstraint}{leq}
  \end{itemize}
\end{notedef}

%%%%%%%%%%%%%%%%%%%%%%%%%%%%%%%%%%%%%%%%%%%%%%%%%%%%%%%%%%%%%%%%%%%%%%%%%%%%%%%%%%%%%%%%%%%%%%%%%%%%%%%%%%%%%%%%%%%%%%%%%%%%%%%%%%%%%%%%%%%%%%%%%%%
%%%%%%%%%%%%%%%%%%%%%%%%%%%%%%%%%%%%%%%%%%%%%%%%%%%%% SET CONSTRAINT  %%%%%%%%%%%%%%%%%%%%%%%%%%%%%%%%%%%%%%%%%%%%%%%%%%%%%%%%%%%%%%%%%%%%%%%%%%%%%

\subsection{Constraints involving set variables}\label{model:setconstraints}\hypertarget{model:setconstraints}{}
%The simplest constraints are comparisons which are defined over expressions of variables such as linear combinations. The following comparison constraints can be accessed through the \texttt{Model} API:
Set constraints are illustrated on the \hyperlink{model:example2:ternarysteinerchoco}{ternary Steiner problem}. 
\begin{notedef}\tt
  \begin{itemize}
  \item \hyperlink{member:memberconstraint}{member}, \hyperlink{notmember:notmemberconstraint}{notMember}
  \item \hyperlink{eqcard:eqcardconstraint}{eqCard}, \hyperlink{geqcard:geqcardconstraint}{geqCard}, \hyperlink{leqcard:leqcardconstraint}{leqCard}, \hyperlink{neqcard}{neqCard}
  \item \hyperlink{eq}{eq}
  \item \hyperlink{isincluded:isincludedconstraint}{isIncluded}, \hyperlink{isnotincluded:isnotincludedconstraint}{isNotIncluded}
  \item \hyperlink{setinter:setinterconstraint}{setInter}
  \item \hyperlink{setdisjoint:setdisjointconstraint}{setDisjoint}, \hyperlink{setunion:setunionconstraint}{setUnion}
  \item \hyperlink{max:maxofaset}{max}, \hyperlink{min:minofaset}{min}
  \item \hyperlink{inverseset}{inverseSet}
  \item \hyperlink{among}{among}
  \item \hyperlink{pack:packconstraint}{pack}
  \end{itemize}
\end{notedef}

%\hyperlink{max:maxconstraint}{max}, \hyperlink{min:minconstraint}{min},

%%%%%%%%%%%%%%%%%%%%%%%%%%%%%%%%%%%%%%%%%%%%%%%%%%%%%%%%%%%%%%%%%%%%%%%%%%%%%%%%%%%%%%%%%%%%%%%%%%%%%%%%%%%%%%%%%%%%%%%%%%%%%%%%%%%%%%%%%%%%%%%%%%%
%%%%%%%%%%%%%%%%%%%%%%%%%%%%%%%%%%%%%%%%%%%%%%%%%%%%% CHANNELING CONSTRAINT  %%%%%%%%%%%%%%%%%%%%%%%%%%%%%%%%%%%%%%%%%%%%%%%%%%%%%%%%%%%%%%%%%%%%%%

\subsection{Channeling constraints}\label{model:channelingconstraints}\hypertarget{model:channelingconstraints}{}
The use of a redundant model, based on an alternative set of decision variables, is a frequent technique to strengthen propagation or to get more freedom to design dedicated search heuristics. 
The following constraints allow to ensure the integrity of two redundant models by linking (channeling) variable-value assignments in the first model to variable-value assignments in the second model:
\begin{notedef}\tt
  \begin{itemize}
  \item \hyperlink{boolchanneling:boolchannelingconstraint}{boolChanneling} $b_j=1 \iff x=j$, 
  \item \hyperlink{domainchanneling:domainchannelingconstraint}{domainChanneling} $b_j=1 \iff x=j$, $\forall j$, 
  \item \hyperlink{inversechanneling:inversechannelingconstraint}{inverseChanneling} $y_j=i \iff x_i=j$, $\forall i, j$, 
  \item \hyperlink{inversechannelingwithinrange:inversechannelingconstraintwithinrange}{inverseChannelingWithinRange} $y_j=i \wedge j \le |x| \iff x_i=j \wedge i \le |y|$, $\forall i, j$, 
  \item \hyperlink{inverseset:inversesetconstraint}{inverseSet} $i\in s_j \iff x_i=j$, $\forall i,j$, 
  \end{itemize}
\end{notedef}
In the n-queen problem, for example, a domain variable by column indicates the row $j$ to place a queen in column $i$. To enhance the propagation, a redundant model can be stated by defining a domain variable by row indicating the column $i$. As columns and rows can be interchanged, the same set of constraints applies to both models, then constraint \hyperlink{inversechanneling:inversechannelingconstraint}{inverseChanneling} is set to propagate between the two models.
\lstinputlisting{java/cinversechanneling.j2t}

Channeling constraints are also useful to compose a model made up of two parts as, for example, in a task-resources assignment problem where some constraints are set on the task set and some other constraints are set on the resource set.

\hyperlink{model:reifiedconstraints}{Reification} offers an other type of channeling, between a constraint and a boolean variable representing its truth value. 
More complex channeling can be done using reification and boolean operators although they are less effective. 
The reified constraint below states $b=1\iff x=y$:
\lstinputlisting{java/cchannelingreified.j2t}


%%%%%%%%%%%%%%%%%%%%%%%%%%%%%%%%%%%%%%%%%%%%%%%%%%%%%%%%%%%%%%%%%%%%%%%%%%%%%%%%%%%%%%%%%%%%%%%%%%%%%%%%%%%%%%%%%%%%%%%%%%%%%%%%%%%%%%%%%%%%%%%%%%%
%%%%%%%%%%%%%%%%%%%%%%%%%%%%%%%%%%%%%%%%%%%%%%%%%%%%% EXTENSIONS CONSTRAINT  %%%%%%%%%%%%%%%%%%%%%%%%%%%%%%%%%%%%%%%%%%%%%%%%%%%%%%%%%%%%%%%%%%%%%%

\subsection{Constraints in extension and relations}\label{model:arbitraryconstraintsinextension}\hypertarget{model:arbitraryconstraintsinextension}{}
Choco supports the statement of constraints defining arbitrary relations over two or more variables.
Such a relation may be defined by three means:
\begin{itemize}
	\item \textbf{feasible table:} the list of allowed tuples of values (that belong to the relation),
	\item \textbf{infeasible table:} the list of forbidden tuples of values (that do not belong to the relation),
	\item \textbf{predicate:} a method to be called in order to check whether a tuple of values belongs or not to the relation.
\end{itemize}
On the one hand, constraints based on tables may be rather memory consuming in case of large domains, although one relation table may be shared by several constraints. On the other hand, predicate constraints require little memory as they do not cache truth values, but imply some run-time overhead for calling the feasibility test. Table constraints are thus well suited for constraints over small domains; while predicate constraints are well suited for situations with large domains. 

Different levels of consistency can be enforced on constraints in extension (when selecting an API) and, for Arc Consistency, different filtering algorithms can be used (when selecting an option).
The Choco API for creating constraints in extension are as follows:

\begin{notedef}
  \begin{itemize}
  \item arc consistency (AC) for binary relations:\\
{\tt \hyperlink{feaspairac:feaspairacconstraint}{feasPairAC}, \hyperlink{infeaspairac:infeaspairacconstraint}{infeasPairAC}, \hyperlink{relationpairac:relationpairacconstraint}{relationPairAC}}
  \item arc consistency (AC) for n-ary relations:\\
{\tt \hyperlink{feastupleac:feastupleacconstraint}{feasTupleAC}, \hyperlink{infeastupleac:infeastupleacconstraint}{infeasTupleAC}, \hyperlink{relationtupleac:relationtupleacconstraint}{relationTupleAC}}
  \item weaker forward checking (FC) for n-ary relations:\\
{\tt \hyperlink{feastuplefc:feastuplefcconstraint}{feasTupleFC}, \hyperlink{infeastuplefc:infeastuplefcconstraint}{infeasTupleFC}, \hyperlink{relationtuplefc:relationtuplefcconstraint}{relationTupleFC}}
  \end{itemize}
\end{notedef}

\subsubsection{Relations.}
A same relation might be shared among several constraints, in this case it is highly recommended to create it first and then use the \hyperlink{relationpairac:relationpairacconstraint}{relationPairAC}, \hyperlink{relationtupleac:relationtupleacconstraint}{relationTupleAC}, or \hyperlink{relationtuplefc:relationtuplefcconstraint}{relationTupleFC} API  on the same relation for each constraint.

For binary relations, the following Choco API is provided:\\
\mylst{makeBinRelation(int[] min, int[] max, List<int[]>pairs, boolean feas)}

It returns a \texttt{BinRelation} giving a list of compatible (\texttt{feas=true}) or incompatible (\texttt{feas=false}) pairs of values. This relation can be applied to any pair of variables $(x_1,x_2)$ whose domains are included in the \texttt{min/max} intervals, i.e. such that:
$$\mathtt{min}[i] \le x_i.\mathtt{getInf}() \le x_i.\mathtt{getSup}() \le  \mathtt{max}[i],\quad \forall i.$$
Bounds \texttt{min/max} are mandatory in order to allow to compute the opposite of the relation if needed.

For n-ary relations, the corresponding Choco API is:\\
\mylst{makeLargeRelation(int[] min, int[] max, List<int[]> tuples, boolean feas);}

It returns a \texttt{LargeRelation}. If \texttt{feas=true}, the returned relation matches also the \texttt{IterLargeRelation} interface which provides constant time iteration abilities over tuples (for compatibility with the GAC algorithm used over feasible tuples).
\lstinputlisting{java/mlargerelation.j2t}

Lastly, some specific relations can be defined without storing the tuples, as in the following example (\texttt{TuplesTest} extends \texttt{LargeRelation}):
\lstinputlisting{java/mnotallequal.j2t}
Then, the \texttt{NotAllEqual} relation is stated as a constraint of a model:
\lstinputlisting{java/mrelationtuplefc.j2t}
%Again, for compatibility with the GAC algorithm invoked by relationTupleAC, such a relation has to match the \texttt{IterLargeRelation} interface for feasible tuples.


%%%%%%%%%%%%%%%%%%%%%%%%%%%%%%%%%%%%%%%%%%%%%%%%%%%%%%%%%%%%%%%%%%%%%%%%%%%%%%%%%%%%%%%%%%%%%%%%%%%%%%%%%%%%%%%%%%%%%%%%%%%%%%%%%%%%%%%%%%%%%%%%%%%
%%%%%%%%%%%%%%%%%%%%%%%%%%%%%%%%%%%%%%%%%%%%%%%%%%%%% REIFIED CONSTRAINT  %%%%%%%%%%%%%%%%%%%%%%%%%%%%%%%%%%%%%%%%%%%%%%%%%%%%%%%%%%%%%%%%%%%%%%%%%

\subsection{Reified constraints}\label{model:reifiedconstraints}\hypertarget{model:reifiedconstraints}{}
The \emph{truth value} of a constraint is a boolean that is true if and only if the constraint holds.
To \emph{reify} a constraint is to get its truth value. 

This mechanism can be used for example to model a MaxCSP problem where the number of satisfied constraints has to be maximized.
It is also intended to give the freedom to model complex constraints combining several reified constraints, using some logical operators on the truth values, such as in:
$(x \neq y) \lor (z \le 9)$.

Choco provides a generic constraint \hyperlink{reifiedconstraint:reifiedconstraintconstraint}{reifiedConstraint} to reify any constraint into a boolean variable expressing its truth value:
\begin{lstlisting}
  Constraint reifiedConstraint(IntegerVariable b, Constraint c);
  Constraint reifiedConstraint(IntegerVariable b, Constraint c1, Constraint c2);
\end{lstlisting}
Specific API are also provided to reify boolean constraints:  
\begin{notedef}\tt
  \begin{itemize}
  \item \hyperlink{reifiedconstraint:reifiedconstraintconstraint}{reifiedConstraint}, 
   \item \hyperlink{reifiedand:reifiedandconstraint}{reifiedAnd}, \hyperlink{reifiedleftimp:reifiedleftimpconstraint}{reifiedLeftImp}, \hyperlink{reifiednot:reifiednotconstraint}{reifiedNot}, \hyperlink{reifiedor:reifiedorconstraint}{reifiedOr}, \hyperlink{reifiedrightimp:reifiedrightimpconstraint}{reifiedRightImp}, \hyperlink{reifiedxnor:reifiedxnorconstraint}{reifiedXnor}, \hyperlink{reifiedxor:reifiedxorconstraint}{reifiedXor}
  \end{itemize}
\end{notedef}



\subsubsection{Handling complex expressions.}\label{model:handlingcomplexexpressions}\hypertarget{model:handlingcomplexexpressions}{}
In order to build complex combinations of constraints, Choco also provides a simpler and more direct API with the following logical meta-constraints taking constraints in arguments:
\begin{notedef}\tt
  \begin{itemize}
  \item \hyperlink{and:andconstraint}{and}, \hyperlink{or:orconstraint}{or}, \hyperlink{implies:impliesconstraint}{implies}, \hyperlink{ifonlyif:ifonlyifconstraint}{ifOnlyIf}, \hyperlink{ifthenelse:ifthenelseconstraint}{ifThenElse}, \hyperlink{not:notconstraint}{not}, \hyperlink{nand:nandconstraint}{nand}, \hyperlink{nor:norconstraint}{nor}
  \end{itemize}
\end{notedef}
For example, the following expression
$$((x = 10 * |y|) \lor (z \le 9))\quad \iff\quad \texttt{alldifferent}(a,b,c)$$
could be expressed in Choco by:
\begin{lstlisting}
	Constraint exp = ifOnlyIf( or( eq(x, mult(10, abs(y))), leq(z, 9) ), 
                               alldifferent(new IntegerVariable[]{a,b,c}) );
\end{lstlisting}
Such an expression is internally represented as a tree whose nodes are operators and leaves are variables, constants and constraints. Variables and constants can be combined as \texttt{ExpressionVariable} using \hyperlink{model:expressionvariables}{operators} (e.g, \texttt{mult(10,abs(w))}), or using simple constraints (e.g., \texttt{leq(z,9)}), or even using global constraints (e.g, \texttt{alldifferent(vars)}).
The language available on expressions currently matches the language used in the \href{http://cpai.ucc.ie/08/}{Constraint Solver Competition 2008} of the CPAI workshop.

At the solver level, there exists two different ways to represent expressions:
\begin{itemize}
\item \emph{by extension:} the first way is to handle expressions as \hyperlink{model:arbitraryconstraintsinextension}{constraints in extension}. The expression is then used to check a tuple in a dynamic way just like a n-ary relation that is defined without listing all the possible tuples. The expression is then propagated using the GAC3rm algorithm. This is very powerful as arc-consistency is achieved on the corresponding constraints.
\item \emph{by decomposition:} the second way is to decompose the expression automatically by introducing intermediate variables and possibly the generic \hyperlink{reifiedconstraint:reifiedconstraintconstraint}{\tt reifiedConstraint}. By doing so, the level of pruning decreases but expressions of larger arity involving large domains can be represented.
\end{itemize}
The way to represent expressions is decided at the modeling level. Representation \emph{by extension} is the default. Representation \emph{by decomposition} can be set instead by:
\begin{lstlisting}
  model.setDefaultExpressionDecomposition(true);
\end{lstlisting}

Representation \emph{by decomposition} can also be controlled individually for some expressions, by setting option \hyperlink{edecomp:edecompoptions}{\tt Options.E\_DECOMP} when adding the constraint.
For example, the following code tells the solver to decompose e1 but not e2 :
\begin{lstlisting}
	model.setDefaultExpressionDecomposition(false);
	IntegerVariable[] x = makeIntVarArray("x", 3, 1, 3, Options.V_BOUND);

	Constraint e1 = or(lt(x[0], x[1]), lt(x[1], x[0]));
	model.addConstraint(Options.E_DECOMP, e1);
	
	Constraint e2 = or(lt(x[1], x[2]), lt(x[2], x[1]));
	model.addConstraint(e2);
\end{lstlisting}

\subsubsection{When and how should I use expressions ?}\label{model:whenshouldiuseexpressions}\hypertarget{model:whenshouldiuseexpressions}{}
Expressions offer a slightly richer modeling language than the one available via standard constraints. However, expressions 
can not be handled as efficiently as constraints that embed a dedicated propagation algorithm. We therefore
recommend you to carefully check that you can not model the expression using the \emph{global constraints} of Choco before using
expressions.

Expressions represented \emph{in extension} should be used in the case of complex logical relationships that involve \textbf{few different variables}, each of \textbf{small domain}, and if \textbf{arc consistency} is desired on those variables.
In such a case, representation in extension can be much more effective than with decomposition. Imagine the following ``crazy'' example :
\begin{lstlisting}
 or( and( eq( abs(sub(div(x,50),div(y,50))),1), eq( abs(sub(mod(x,50),mod(y,50))),2)),
     and( eq( abs(sub(div(x,50),div(y,50))),2), eq( abs(sub(mod(x,50),mod(y,50))),1)))
\end{lstlisting}
This expression has a small arity: it involves only two variables $x$ and $y$.
Let assume that their domains has no more than 300 values, then such an expression should typically not be decomposed. Indeed, arc consistency will create many holes in the domains and filter much more than if the relation was decomposed.

Conversely, an expression should be decomposed as soon as it involves a large number of variables, or at least one variable with a large domain.

%%%%%%%%%%%%%%%%%%%%%%%%%%%%%%%%%%%%%%%%%%%%%%%%%%%%%%%%%%%%%%%%%%%%%%%%%%%%%%%%%%%%%%%%%%%%%%%%%%%%%%%%%%%%%%%%%%%%%%%%%%%%%%%%%%%%%%%%%%%%%%%%%%%
%%%%%%%%%%%%%%%%%%%%%%%%%%%%%%%%%%%%%%%%%%%%%%%%%%%%% GLOBAL CONSTRAINT  %%%%%%%%%%%%%%%%%%%%%%%%%%%%%%%%%%%%%%%%%%%%%%%%%%%%%%%%%%%%%%%%%%%%%%%%%%

\subsection{Global constraints}\label{model:advancedconstraints}\hypertarget{model:advancedconstraints}{}
Choco includes several \href{http://www.emn.fr/x-info/sdemasse/gccat/}{global constraints}. Those constraints accept any number of variables and offer dedicated filtering algorithms which are able to make deductions where a decomposed model would not.
For instance, constraint \texttt{alldifferent}$(a,b,c,d)$ with $a,b\in[1,4]$ and $c,d\in[3,4]$ allows to deduce that $a$ and $b$ cannot be instantiated to $3$ or $4$; such rule cannot be inferred by simple binary constraints. 

The up-to-date list of global constraints available in Choco can be found within the Javadoc API.
Most of these global constraints are listed below according to their application fields.
Details and examples can be found in Section \hyperlink{ch:constraints}{Elements of Choco/Constraints}.
\subsubsection{Value constraints}\label{model:valueconstraints}\hypertarget{model:valueconstraints}{}
Constraints that put a restriction on how values can be distributed among a collection of variables.
See also in Global Constraint Catalog: \href{http://www.emn.fr/x-info/sdemasse/gccat/Kvalue_constraint.html}{value constraint}.

\vspace{1em}\noindent\begin{notedef}\tt
  \begin{itemize}
  \item counting distinct values: 
\hyperlink{alldifferent:alldifferentconstraint}{allDifferent}, 
\hyperlink{atmostnvalue:atmostnvalueconstraint}{atMostNValue},
\hyperlink{increasingnvalue:increasingnvalueconstraint}{increasingNValue},
  \item counting values: 
\hyperlink{among:amongconstraint}{among},
\hyperlink{occurrence:occurrenceconstraint}{occurrence},
\hyperlink{occurrencemax:occurrencemaxconstraint}{occurrenceMax},
\hyperlink{occurrencemin:occurrenceminconstraint}{occurrenceMin},
\hyperlink{globalcardinality:globalcardinalityconstraint}{globalCardinality},
  \item indexing values: 
\hyperlink{nth:nthconstraint}{nth} (element),
\hyperlink{max:maxconstraint}{max},
\hyperlink{min:minconstraint}{min},
  \item ordering: 
\hyperlink{sorting:sortingconstraint}{sorting},
\hyperlink{increasingnvalue:increasingnvalueconstraint}{increasingNValue},
\hyperlink{increasingsum:increasingsumconstraint}{increasingSum},
\hyperlink{lex:lexconstraint}{lex}, 
\hyperlink{lexeq:lexeqconstraint}{lexeq},
\hyperlink{leximin:leximinconstraint}{leximin},
\hyperlink{lexchain:lexchainconstraint}{lexChain},
\hyperlink{lexchaineq:lexchaineqconstraint}{lexChainEq},
  \item tuple matching: 
\hyperlink{feastupleac:feastupleacconstraint}{feasTupleAC},
\hyperlink{feastuplefc:feastuplefcconstraint}{feasTupleFC},
\hyperlink{infeastupleac:infeastupleacconstraint}{infeasTupleAC},
\hyperlink{infeastuplefc:infeastuplefcconstraint}{infeasTupleFC},
\hyperlink{relationtupleac:relationtupleacconstraint}{relationTupleAC},
\hyperlink{relationtuplefc:relationtuplefcconstraint}{relationTupleFC},
  \item pattern matching: 
\hyperlink{regular:regularconstraint}{regular},
\hyperlink{costregular:costregularconstraint}{costRegular},
\hyperlink{multicostregular:multicostregularconstraint}{multiCostRegular}, 
%\hyperlink{stretchcyclic:stretchcyclicconstraint}{stretchCyclic}, 
\hyperlink{stretchpath:stretchpathconstraint}{stretchPath}, 
\hyperlink{tree:treeconstraint}{tree},
  \end{itemize}
\end{notedef}

\subsubsection{Boolean constraints}\label{modelglobal:logicconstraints}\hypertarget{modelglobal:logicconstraints}{}
Logical operations on boolean expressions.
See also in Global Constraint Catalog: \href{http://www.emn.fr/x-info/sdemasse/gccat/KBoolean_constraint.html}{boolean constraint}.

\vspace{1em}\noindent\begin{notedef}\tt
\hyperlink{and:andconstraint}{and},
\hyperlink{or:orconstraint}{or},
\hyperlink{clause:clauseconstraint}{clause},
\end{notedef}

\subsubsection{Channeling constraints}\label{modelglobal:channelingconstraints}\hypertarget{modelglobal:channelingconstraints}{}
Constraints linking two collections of variables (many-to-many) or indexing one among many variables (one-to-many).
See also Section \hyperlink{model:channelingconstraints}{Channeling} and in Global Constraint Catalog: \href{http://www.emn.fr/x-info/sdemasse/gccat/Kchannelling_constraint.html}{channelling constraint}.

 \vspace{1em}\noindent\begin{notedef}\tt
   \begin{itemize}
   \item one-to-many: 
 \hyperlink{domainchanneling:domainchannelingconstraint}{domainChanneling},
 \hyperlink{nth:nthconstraint}{nth} (element),
 \hyperlink{max:maxconstraint}{max},
 \hyperlink{min:minconstraint}{min},
   \item many-to-many: 
 \hyperlink{inversechanneling:inversechannelingconstraint}{inverseChanneling},
 \hyperlink{inverseset:inversesetconstraint}{inverseSet},
 \hyperlink{sorting:sortingconstraint}{sorting},
\hyperlink{pack:packconstraint}{pack},
 \end{itemize}
 \end{notedef}

\subsubsection{Optimization constraints}\label{model:optimizationconstraints}\hypertarget{model:optimizationconstraints}{}
Constraints channelling a variable to the sum of the weights of a collection of variable-value assignments.
See also in Global Constraint Catalog: \href{http://www.emn.fr/x-info/sdemasse/gccat/Kcost_filtering_constraint.html}{cost-filtering constraint}.
\vspace{1em}\noindent\begin{notedef}\tt
 \begin{itemize}
  \item one cost: 
\hyperlink{among:amongconstraint}{among},
\hyperlink{occurrence:occurrenceconstraint}{occurrence},
\hyperlink{occurrencemax:occurrencemaxconstraint}{occurrenceMax},
\hyperlink{occurrencemin:occurrenceminconstraint}{occurrenceMin},
\hyperlink{knapsackproblem:knapsackproblemconstraint}{knapsackProblem},
\hyperlink{equation:equationconstraint}{equation},
\hyperlink{costregular:costregularconstraint}{costRegular},
\hyperlink{tree:treeconstraint}{tree},
 \item several costs:
\hyperlink{globalcardinality:globalcardinalityconstraint}{globalCardinality},
\hyperlink{multicostregular:multicostregularconstraint}{multiCostRegular}, 
 \end{itemize}
\end{notedef}

\subsubsection{Packing constraints (capacitated resources)}\label{model:packingconstraints}\hypertarget{model:packingconstraints}{}
Constraints involving items to be packed in bins without overlapping. More generaly, any constraints modelling the concurrent assignment of objects to one or several capacitated resources.
See also in Global Constraint Catalog: \href{http://www.emn.fr/x-info/sdemasse/gccat/Kresource_constraint.html}{resource constraint}.

\vspace{1em}\noindent\begin{notedef}\tt
   \begin{itemize}
   \item packing problems: 
\hyperlink{equation:equationconstraint}{equation},
\hyperlink{knapsackproblem:knapsackproblemconstraint}{knapsackProblem},
\hyperlink{pack:packconstraint}{pack} (bin-packing),
   \item geometric placement problems: 
\hyperlink{geost:geostconstraint}{geost}, 
   \item scheduling problems: 
\hyperlink{disjoint}{disjoint (tasks)} 
\hyperlink{disjunctive:disjunctiveconstraint}{disjunctive}, 
\hyperlink{cumulative:cumulativeconstraint}{cumulative}, 
   \item timetabling problems: 
\hyperlink{costregular:costregularconstraint}{costRegular},
\hyperlink{multicostregular:multicostregularconstraint}{multiCostRegular}, 
 \end{itemize}
\end{notedef}

\subsubsection{Scheduling constraints (time assignment)}\label{model:schedulingconstraints}\hypertarget{model:schedulingconstraints}{}
Constraints involving tasks to be scheduled over a time horizon.
%See also \hyperlink{schedulinganduseofthecumulative:schedulinganduseofthecumulativeconstraint}{scheduling application} and 
See also in Global Constraint Catalog: \href{http://www.emn.fr/x-info/sdemasse/gccat/Kscheduling_constraint.html}{scheduling constraint}.

\vspace{1em}\noindent\begin{notedef}\tt
   \begin{itemize}
   \item temporal constraints:
\hyperlink{disjoint}{disjoint (tasks)} 
\hyperlink{precedence:precedenceconstraint}{precedence}, 
\hyperlink{precedencedisjoint:precedencedisjointconstraint}{precedenceDisjoint}, 
\hyperlink{precedenceimplied:precedenceimpliedconstraint}{precedenceImplied}, 
\hyperlink{precedencereified:precedencereifiedconstraint}{precedenceReified},
\hyperlink{forbiddeninterval:forbiddenintervalconstraint}{forbiddenInterval},
\hyperlink{tree:treeconstraint}{tree},
   \item resource constraints: 
\hyperlink{cumulative:cumulativeconstraint}{cumulative}, 
\hyperlink{disjunctive:disjunctiveconstraint}{disjunctive}, 
\hyperlink{geost:geostconstraint}{geost}, 
 \end{itemize}
\end{notedef}

\subsection{Things to know about \mylst{Model}, \mylst{Variable} and \mylst{Constraint}}

It is important to know the relation between \mylst{Model}s, \mylst{Variable}s and \mylst{Constraint}s. \mylst{Variable}s and \mylst{Constraints} are build without the help of a \mylst{Model}, so that they can be used natively in different \mylst{Model}s. That's why one need to add them to a \mylst{Model}, using \mylst{model.addVariable(Variable var)} and \mylst{model.addConstraint(Constraint cstr)}. On a variable addition, this one is added to the list of variables of the model. On a constraint addition, the constraint is added to the constraint network AND the constraint and its associated variables are linked. It means the constraint is now known from its variables, which was not the case before the addition.


Now, let see a short example: one declares two constraints involving the two same variables, and add them to five different models. This implies the following references:
\begin{itemize}
\item each model points to two constraint and two variables;
\item each constraint points the two variables;
\item each of the variables points to ten constraints: the very same constraint is considered as different from a model to another.  
\end{itemize}

This must be kept in mind while you write your program: whether a model is used or not, a shared variable stores references to the constraints it is involved in. 

\vspace{1cm}
\begin{notedef}
Do not count on the garbage collector to manage this (even if the \mylst{finalize()} method was declared for a \mylst{Model}): while a variable references a constraint declared in an obsolete model, neither the model nor the constraint can be safely destroyed. 

A good habit to have is to delete constraints of a model when this one is not used anymore, by calling \mylst{model.removeConstraints()}. This breaks links between constraint and variable of a model, prevents large memory consumption and can stabilize performances by reusing \mylst{Model}s.


Consider the following code, \mylst{model2} is faster and consumes less memory than \mylst{model1} for the same result.
\begin{lstlisting}
public static void main(String[] args) {
        for (int i = 0; i < 10; i++) {
            long t = -System.currentTimeMillis();
            model1(99999);
            t += System.currentTimeMillis();
            System.out.printf("%d ", t);
            t = -System.currentTimeMillis();
            model2(99999);
            t += System.currentTimeMillis();
            System.out.printf("%d\n", t);
        }
    }                                               
                                                 
private static void model1(int i) {              
    IntegerVariable v1 = makeIntVar("v1", 0, 10);
    IntegerVariable v2 = makeIntVar("v2", 0, 10);
    for (int j = 0; j < i; j++) {                
        CPModel m1 = new CPModel();              
        m1.addConstraint(eq(v1, v2));            
    }                                            
}                                                
                                                 
private static void model2(int i) {              
    IntegerVariable v1 = makeIntVar("v1", 0, 10);
    IntegerVariable v2 = makeIntVar("v2", 0, 10);
    CPModel m1 = new CPModel();                  
    for (int j = 0; j < i; j++) {                
        m1.addConstraint(eq(v1, v2));            
        m1.removeConstraints();                  
    }                                            
}                                                
\end{lstlisting}

\end{notedef}




% java ToTex ../../../../../../samples/src/main/java/samples/documentation/ ../../documentation/java/

%!TEX root = ../content-doc.tex
%\part{solver}
\label{solver}
\hypertarget{solver}{}


\chapter{The solver}\label{solver:thesolver}\hypertarget{solver:thesolver}{}

%\section{How to create a solver}\label{solver:howtocreateasolver}\hypertarget{solver:howtocreateasolver}{}


\newglossaryentry{Solver}{name={Solver},description={solver description}}
The \mylst{Solver}, along with the \mylst{Model}, is one of the two key elements of any Choco program. The Choco \mylst{Solver} is mainly focused on resolution part: reading the \mylst{Model}, defining the search strategies and the resolution policy.  

To create a \gls{Solver}, one just needs to create a new object as follow:
\begin{lstlisting}
Solver solver = new CPSolver();
\end{lstlisting}
This instruction creates a Constraint Programming (CP) {\tt Solver} object.

%\section{Read a model}\label{solver:readamodel}\hypertarget{solver:readamodel}{}
The solver gives an API to read a model. The reading of a model is compulsory and must be done after the entire definition of the model. 
\begin{lstlisting}
solver.read(model);
\end{lstlisting}
The reading is divided in two parts: \hyperlink{solver:variablesreading}{variables reading} and \hyperlink{solver:constraintsreading}{constraints reading}.

\section{Variables reading}\label{solver:variablesreading}\hypertarget{solver:variablesreading}{}
The variables are declared in a model with a given type \texttt{IntegerVariable, SetVariable, RealVariable} and, possibly, with a given domain type (e.g. bounded or enumerated domains for integer and set variables).
When reading the model, the solver iterates over the model variables, then creates the corresponding solver variables and domains data structures according to these types.

\begin{note}
\textbf{Bound variables} are related to large domains which are only represented by their lower and upper bounds. The domain is encoded in a space efficient way and propagation events only concern bound updates. Value removals between the bounds are therefore ignored (\emph{holes} are not considered). The level of consistency achieved by most constraints on these variables is called \emph{bound-consistency}.

On the contrary, the domain of an \textbf{enumerated variable} is explicitly represented and every value is considered while pruning. Basic constraints are therefore often able to achieve \emph{arc-consistency} on enumerated variables (except for NP-hard global constraint such as the cumulative constraint). Remember that switching from enumerated variables to bounded variables decreases the level of propagation achieved by the system.
\end{note}


%\begin{note}
\paragraph{Model variables and solver variables are distinct objects.} 
Model variables implement the \mylst{Variable} interface and solver variables implement the \mylst{Var} interface.
A model variable is defined by an abstract representation of its initial domain, while a solver variable encapsulates a concrete representation of the domain, and maintains its current state throughout the search.
Hence, one cannot access a variable value directly from a model variable but one can from its corresponding solver variable. The solver variables are anonymous but can be accessed from the corresponding model variables using the \texttt{Solver} API \mylst{getVar(Variable v)} and \mylst{getVar(Variable... v)}.
%To access to a model variable thanks to the solver, use the following 
%\end{note}

\subsection{from \texttt{IntegerVariable} to\texttt{IntDomainVar}}\label{solver:solverandintegervariables}\hypertarget{solver:solverandintegervariables}{}

For integer variables, the solver \textbf{\tt IntDomainVar} is counterpart to the model \hyperlink{integervariable}{\textbf{\tt IntegerVariable}}. 
Methods \textbf{\tt getVar(IntegerVariable var)} and \textbf{\tt getVar(IntegerVariable... vars)} of \texttt{Solver} return the objects \texttt{IntDomainVar} and \texttt{IntDomainVar[]} respectively corresponding to \texttt{var} and \texttt{vars}:
\begin{lstlisting}
  IntegerVariable x = Choco.makeEnumIntVar("x", 1, 100);  // model variable
  IntDomainVar xOnSolver = solver.getVar(x);  // solver variable
\end{lstlisting}

The state of an \texttt{IntDomainVar} can be accessed using these main public methods:

\noindent\begin{tabular}{p{.3\linewidth}p{.7\linewidth}}
  \hline
  \texttt{IntDomainVar} API &  description \\
  \hline
	\mylst{hasEnumeratedDomain()} &checks if the domain type is enumerated or bounded\\
	\mylst{getInf()} &returns the current lower bound of the variable\\
	\mylst{getSup()} &returns the current upper bound of the variable\\
	\mylst{getVal()} &returns the value of the variable if it is currently instantiated\\
	\mylst{isInstantiated()} &checks if the domain is currently reduced to a singleton\\
	\mylst{canBeInstantiatedTo(int v)} &checks if value \texttt{v} currently belongs to the domain of the variable\\
	\mylst{getDomainSize()} &returns the current size of the domain\\
  \hline\\
\end{tabular}

The data structure representing the current domain within the \texttt{IntDomainVar} object depends on the domain type (bounded, enumerated, boolean, constant, etc.) of the model variable. 
See \hyperlink{advanced}{advanced uses} for more informations on \texttt{IntDomainVar}.

\subsection{from \texttt{SetVariable} to \texttt{SetVar}}\label{solver:solverandsetvariables}\hypertarget{solver:solverandsetvariables}{}

For set variables, the solver \textbf{\tt SetVar} is counterpart to the model \hyperlink{setvariable}{\textbf{\tt SetVariable}}. 
Methods \textbf{\tt getVar(SetVariable var)} and \textbf{\tt getVar(SetVariable... vars)} of \texttt{Solver} return the objects \texttt{SetVar} and \texttt{SetVar[]} respectively corresponding to \texttt{var} and \texttt{vars}:
\begin{lstlisting}
	SetVariable x = Choco.makeBoundSetVar("x", 1, 40); // model variable
	SetVar xOnSolver = solver.getVar(x); // solver variable
\end{lstlisting}

Note that a set variable on integer values between $1$ and $n$ may have $2^{n}$ possible values, corresponding to every possible subsets of $\{1,2,\ldots,n\}$. Hence, the domain of a \texttt{SetVar} is encoded by these bounds only: the lower bound, called the \emph{kernel}, is the intersection of all possible set values, and the upper bound, called the \emph{envelope}, is the union of all possible set values. Furthermore, a \texttt{SetVar} encapsulates an \texttt{IntDomainVar} representing the cardinality of the set variable. The domain type of this variable (enumerated or bounded) depends on the option given at the construction of the \texttt{SetVariable}. 
%This makes an exponential number of values and the domain is represented with two bounds corresponding to the intersection of all possible sets (called the kernel) and the union of all possible sets (called the envelope) which are the possible candidate values for the variable.

The state of a \texttt{SetVar} can be accessed through these main public methods: \todo{Warning: Envelope is (french) spelled with two 'p' in the method name.}

\noindent\begin{tabular}{p{.3\linewidth}p{.7\linewidth}}
  \hline
  \texttt{SetVar} API &  description \\
  \hline
	\mylst{getCard()} &returns the current cardinality (an \texttt{IntDomainVar} object)\\
	\mylst{isInDomainKernel(int v)} &checks if value \texttt{v} belongs to the current kernel\\
	\mylst{isInDomainEnveloppe(int v)} &checks if value \texttt{v} belongs to the current envelope\\
	\mylst{getDomain()} &returns the current domain (a \texttt{SetDomain} object). Iterators on envelope or kernel can then be called\\
	\mylst{getKernelDomainSize()} &returns the current size of the kernel\\
	\mylst{getEnveloppeDomainSize()} &returns the current size of the envelope\\
	\mylst{getEnveloppeInf()} &returns the current smallest value of the envelope\\
	\mylst{getEnveloppeSup()} &returns the current largest value of the envelope\\
	\mylst{getKernelInf()} &returns the current smallest value of the kernel\\
	\mylst{getKernelSup()} &returns the current largest value of the kernel\\
	\mylst{getValue()} &returns the set value as a table of integers \texttt{int[]} when the variable is currently instantiated (kernel=envelope)\\
  \hline\\
\end{tabular}

\noindent See \hyperlink{advanced}{advanced uses} for more informations on \texttt{SetVar}.

\subsection{from \texttt{RealVariable} to \texttt{RealVar}}\label{solver:solverandrealvariables}\hypertarget{solver:solverandrealvariables}{}

\begin{note}
\emph{Real variables are still under development but can be used to solve toy problems such as small systems of equations.}
\end{note}
 
For real variables, the solver \textbf{\tt RealVar} is counterpart to the model \hyperlink{realvariable}{\textbf{\tt RealVariable}}. 
Methods \textbf{\tt getVar(RealVariable var)} and \textbf{\tt getVar(RealVariable... vars)} of \texttt{Solver} return the objects \texttt{RealVar} and \texttt{RealVar[]} respectively corresponding to \texttt{var} and \texttt{vars}:
\begin{lstlisting}
	RealVariable x = Choco.makeRealVar("x", 1.0, 3.0); // model variable
	RealVar xOnSolver = solver.getVar(x); // solver variable
\end{lstlisting}

Continuous variables are useful for non linear equation systems which are encountered in physics for example.
The state of a \texttt{RealVar} can be accessed through these main public methods:

\noindent\begin{tabular}{p{.3\linewidth}p{.7\linewidth}}
  \hline
  \texttt{RealVar} API &  description \\
  \hline
	\mylst{getInf()} &returns the current lower bound of the variable (\texttt{double})\\
	\mylst{getSup()} &returns the current upper bound of the variable (\texttt{double})\\
	\mylst{isInstantiated()} &checks if the domain is reduced to a canonical interval. A canonical interval indicates that the domain has reached the precision given by the user or the solver\\
  \hline\\
\end{tabular}

\noindent See \hyperlink{advanced}{advanced uses} for more informations on \texttt{RealVar}.

\subsection{from \texttt{TaskVariable} to \texttt{TaskVar}}\label{solver:solverandtaskvariables}\hypertarget{solver:solverandtaskvariables}{}

For task variables, the solver \textbf{\tt TaskVar} is counterpart to the model \hyperlink{taskvariable}{\textbf{\tt TaskVariable}}. 
Methods \textbf{\tt getVar(TaskVariable var)} and \textbf{\tt getVar(TaskVariable... vars)} of \texttt{Solver} return the objects \texttt{TaskVar} and \texttt{TaskVar[]} respectively corresponding to \texttt{var} and \texttt{vars}:
\begin{lstlisting}
	TaskVariable x = Choco.makeTaskVar("x", 0, 123, 18); // model variable
	TaskVar xOnSolver = solver.getVar(x); // solver variable
\end{lstlisting}

Task variables help at formulating scheduling problems where one has to determine the starting and ending time of a task. A task variable aggregates three integer variable: \texttt{start}, \texttt{end}, and \texttt{duration} and the implicit constraint \texttt{start}+\texttt{duration}=\texttt{end}. 

\todo{To complete.}
The state of a \texttt{TaskVar} can be accessed through these main public methods:

\noindent\begin{tabular}{p{.3\linewidth}p{.7\linewidth}}
  \hline
  \texttt{TaskVar} API &  description \\
  \hline
	\mylst{isInstantiated()} &checks if the three time integer variables are instantiated\\
  \hline\\
\end{tabular}

\noindent See \hyperlink{advanced}{advanced uses} for more informations on \texttt{TaskVar}.

\section{Constraints reading}\label{solver:constraintsreading}\hypertarget{solver:constraintsreading}{}
Once the solver variables are created when reading the model, the solver then iterates over the constraints of the model, and creates the solver \texttt{SConstraint} objects  by calling method \texttt{makeConstraint} of the \texttt{ConstraintManager} object associated to the model constraint type.
At this step, auxiliary solver variables and constraints may be generated. The created constraints are then added to the internal constraint network. 

Each solver constraint encapsulates a filtering algorithm which is called, during the search, when a propagation step occurs or when an external event (e.g., value removal or bound modification) happens on some variable of the constraint.

One can access the Solver representation of a Model constraint, using the \texttt{Solver} API \mylst{getCstr(Constraint c)}.

\section{Solve a problem}\label{solver:solveaproblem}\hypertarget{solver:solveaproblem}{}
%As Solver is the second element of a Choco program, the control of the search process without using predefined tools is made on the Solver.

Table below presents the different API offered by \texttt{Solver} to launch the problem resolution. All these methods return a \texttt{Boolean} object standing for the \emph{problem feasibility status} of the solver:
$$\begin{cases}
  \texttt{Boolean.TRUE} &\text{ if at least one feasible solution has been computed},\\
  \texttt{Boolean.FALSE} &\text{ if the problem is proved to be infeasible},\\
  \texttt{null} &\text{ otherwise, i.e. when a search limit has been reached before.}
\end{cases}$$

\noindent\begin{tabular}{p{.4\linewidth}p{.6\linewidth}}
  \hline
  \texttt{Solver} API & description \\
  \hline
      \mylst{solve()} or \mylst{solve(false)} &  runs backtracking until reaching \emph{a first feasible solution} (returns \mylst{Boolean.TRUE}) or the proof of infeasibility (returns \mylst{Boolean.FALSE}) or a search limit (returns \mylst{null}).\\[.3em]
      \hline\\
      \mylst{nextSolution()} &  Can only be called after a \texttt{solve()} or a \texttt{nextSolution()} call that has returned \mylst{Boolean.TRUE}. Runs backtracking, from the solution leaf reached by the previous \texttt{solve()} or \texttt{nextSolution()} call, until reaching \emph{a new feasible solution} (returns \mylst{Boolean.TRUE}), or proving no such new solution exists (returns \texttt{Boolean.FALSE}), or reaching a search limit (returns \mylst{null}).\\[.3em]
      \hline\\
      \mylst{isFeasible()} &  Returns the feasibility status of the solver.\\
      \hline\\
      \mylst{solveAll()} or \mylst{solve(true)} &  Runs backtracking until computing \emph{all feasible solutions}, or until proving infeasibility (returns \mylst{Boolean.FALSE}) or until reaching a search limit (returns \mylst{Boolean.TRUE} if at least one first solution was computed, and \mylst{null} otherwise). \\[.3em]
      \hline\\
      \mylst{maximize(Var obj, boolean restart),}\mylst{maximize(boolean restart)} &  Runs branch-and-bound until reaching \emph{a feasible solution that is proved to maximize objective} \mylst{obj},  or until proving infeasibility (returns \mylst{Boolean.FALSE}) or until reaching a search limit (returns \mylst{Boolean.TRUE} if at least one first solution was computed, and \mylst{null} otherwise). It proceeds by successive improvements of the best solution found so far: each time a feasible solution is found at a leaf of the tree search, then the search proceeds for a new solution with a greater objective, until it proves that no such improving solution exists.
Parameter \texttt{restart} is a boolean indicating whether the search continues from the solution leaf with a backtrack (if set to \mylst{false}) or if it is restarted from the root node (if set to \texttt{true}).\\
\hline\\
      \mylst{minimize(Var obj, boolean restart),}\mylst{minimize(boolean restart)} &  similar to \texttt{maximize} but for computing \emph{a feasible solution that is proved to minimize objective} \texttt{obj}.\\[.3em]      \hline\\
	\end{tabular}

The following  API are also useful to manipulate a \texttt{Solver} object:\\
\noindent\begin{tabular}{p{.4\linewidth}p{.6\linewidth}}
  \hline
  \texttt{Solver} API & description \\
  \hline
      \mylst{propagate()} &  Launchs propagation by running, in turn, the domain reduction algorithms of the constraints until it reaches a fix point. Throws a \texttt{ContradictionException} when a contradiction is detected, i.e. a variable domain is emptied. This method is called at each node of the tree search constructed by the solving methods above.\\[.3em]
      \hline\\
	\end{tabular}

\section{Storing et restoring solutions}\label{}\hypertarget{}{}
By default, the last solution found is stored and accessible once the search has ended.

\paragraph{Store solutions.} To store the solutions found during the resolution, one just needs to override the solution pool capacity (which is set to 1 by default).
\begin{lstlisting}
	int size = Integer.MAX_VALUE;
	solver.getConfiguration().putInt(Configuration.SOLUTION_POOL_CAPACITY, size);
	solver.solveAll();
	ISolutionPool pool = solver.getSearchStrategy().getSolutionPool();
\end{lstlisting}

Setting \textit{size} to \mylst{Integer.MAX\_VALUE} means ``store every solution'', otherwise you can set a finite value, for example 10 and the last 10 solutions will be stored.

A Solution object is made of arrays of values: an array of values of \mylst{IntVar}, another one for values of \mylst{SetVar} and the last one for the values of \mylst{RealVar}. One can retrieve the value of a variable by calling \mylst{getXXValue(int idx)}, where \mylst{XX} is the type of variable and \mylst{idx} is its index within the solver:

\begin{lstlisting}
	solution.getIntValue(solver.getIntVarIndex(ivar));
	solution.getSetValue(solver.getSetVarIndex(svar));
	solution.getRealValue(solver.getRealVarIndex(fvar));
\end{lstlisting}

And to make the link between a variable in a solution, simply retrieve the index of the variable in the solver:
\begin{lstlisting}
	int q0 = solver.getIntVarIndex(solver.getVar(vars[0])); // vars is an array of IntegerVariable
	solution.getIntValue(q0);
\end{lstlisting}

\paragraph{Restore solutions.} A solution can be restored within a solver. But to do that, some precautions must be made: before starting the search, the state of the solver must be backed up.
\begin{lstlisting}
	int rootworld = solver.getEnvironment().getWorldIndex();
	solver.worldPush();
	solver.solveAll();
\end{lstlisting}

And then, to restore, first restore the root world and backup another world:
\begin{lstlisting}
	solver.worldPopUntil(rootworld); // restore the original state, where domains were as declared (not yet instantiated)
	solver.worldPush(); // backup the current state of the solver, to allow other solution restoration
	solver.restoreSolution(solution); // restore the solution
	// do something
\end{lstlisting}
If you want to restore more than one solution, you just apply this 3 steps.

\section{Search Strategy}\label{solver:searchstrategy}\hypertarget{solver:searchstrategy}{}

\newglossaryentry{branching strategy}{name={branching strategy}, plural={branching strategies},description={heuristic controlling the execution of a search loop at a point where the control flow may be split between different branches}}
\newglossaryentry{search strategy}{name={search strategy}, plural={search strategies},description={composition of branching strategies}}

A key ingredient of any constraint approach is a clever \gls{search strategy}. 
In backtracking or branch-and-bound approaches, the search is organized as an enumeration tree, where each node corresponds to a subspace of the search, and each child node is a subdivision of its father node's space.
The tree is progressively constructed by applying a series of \glspl{branching strategy} that determine how to subdivise space at each node and in which order to explore the created child nodes. Branching strategies play the role of achieving intermediate goals in logic programming. 

This section presents how to define your own search strategy in Choco. 
\begin{note}
  Standard backtracking or branch-and-bound approaches in constraint programming develop the enumeration tree in a \textbf{Depth-First Search (DFS)} manner:
  \begin{enumerate}
  \item \emph{evaluate} a node: run propagation 
  \item if a failure occurs or if the search space cannot be separated then \emph{backtrack}: evaluate the next pending node
  \item otherwise \emph{branch}: divide the search space and evaluate the first child node.
  \end{enumerate}
  With Choco, the search process of the \texttt{CPSolver} does not currently allow to explore the tree in a different manner, using Best-First Search for example. 

  In addition, the common way of dividing the search space in CP-based backtracking/B\&B algorithms is \textbf{to assign a variable to a value or to forbid this assignment}. Choco provides such a branching strategy and the tools to easily customize the variable and value selection heuristics within this strategy. However, Choco makes possible to implement \textbf{more complex branching strategies} (e.g. constraint branching or dichotomy branching).
\end{note}

%The user may specify the sequence of branching strategies to be used to build the search tree. We will present in this section how to define your branching strategies.

\subsection{Overriding the default search strategy}\label{solver:overridethedefaultsearchstrategy}\hypertarget{solver:overridethedefaultsearchstrategy}{}

\newglossaryentry{value selector}{name={value selector}, plural={value selector},description={heuristic specifying how to choose a value from a chosen variable at a fix point}}
\newglossaryentry{value iterator}{name={value iterator}, plural={value iterators},description={heuristic specifying how to choose a value from a chosen variable, through an iterator, at a fix point}}
\newglossaryentry{variable selector}{name={variable selector}, plural={variable selectors},description={heuristic specifying how to choose a variable at a fix point}}


%Basically, a search strategy is the composition of three objects: a \gls{branching strategy}, a \gls{variable selector} and a \gls{value selector} (or a \gls{value iterator}). Some branching strategies simply assign a selected value to a selected variable, like \hyperlink{assignvar:assignvarbranchstrat}{AssignVar}, others branching strategies embed the variable selector, like \hyperlink{domoverwdeg:domoverwdegbranchstrat}{DomOverWDegBranchingNew}, or more, like  \hyperlink{impact:impactbranchstrat}{ImpactBasedBranching}.
\paragraph{Branching, variable selection and value selection strategies.}
Basically, a search strategy in Choco is a composition of \gls{branching strategy} objects, each defined on a given set of decision variables.
The most common branching strategies are based on the assignment of a selected variable to one or several selected values (one assignment in each branch). 
\begin{note}
Branching strategies apply to \texttt{Solver} variables (not \texttt{Model} variables).
\end{note}
The variable and value selection heuristics can be defined separately in their own objects: a \gls{variable selector} and a \gls{value selector} or a \gls{value iterator}. 
Branching strategy \hyperlink{assignvar:assignvarbranchstrat}{\tt AssignVar}, for example, can simply be customized via these embedded objects: the variable is first selected, then a value in the variable domain is selected. The two following instructions both create a $n$-ary branching strategy (\texttt{AssignVar}) selecting an integer decision variable of minimum domain size (\texttt{MinDomain} variable selector) and assigning it successively, in each branch, to one of its domain value, selected in increasing order (\texttt{IncreasingDomain} value iterator or \texttt{MinVal} value selector).
\begin{lstlisting}
  new AssignVar(new MinDomain(solver), new IncreasingDomain());
  new AssignVar(new MinDomain(solver), new MinVal());
\end{lstlisting}
Note that this usual strategy is pre-defined in \texttt{BranchingFactory}, and may then also be declared as follows:
\begin{lstlisting}
  BranchingFactory.minDomMinVal(solver);
\end{lstlisting}
Sometimes, the choice of the variable may also depend on the choice of the value, or it may require specific computations before or after branching. In this case, the variable selection heuristic can directly be implemented in the branching strategy object, like e.g. in \hyperlink{domoverwdeg:domoverwdegbranchstrat}{DomOverWDegBranchingNew}. Both variable and value selection heuristics can be implemented directly within the branching strategy, like e.g in \hyperlink{impact:impactbranchstrat}{ImpactBasedBranching}.

\paragraph{Default strategies.}
When no search strategy is specified, default search strategies apply to all the decision variables of the solver.
These strategies vary according to the variable types: 

\noindent\begin{tabular}{p{.25\linewidth}p{.7\linewidth}}
\hline
Variable type &  Default strategy \\
\hline
Set &   \hyperlink{assignsetvar:assignsetvarbranchstrat}{AssignSetVar} + \hyperlink{mindomset:mindomsetvarselector}{MinDomainSet} + \hyperlink{minenv:minenvvalselector}{MinEnv} \\
Integer & \hyperlink{domoverwdeg:domoverwdegbranchstrat}{DomOverWDegBranchingNew} +\hyperlink{increasingdomain:increasingdomainvaliterator}{IncreasingDomain}\\
 Real &  \hyperlink{assigninterval:assignintervalbranchstrat}{AssignInterval} + \hyperlink{cyclicrealvarselector:cyclicrealvarselectorvarselector}{CyclicRealVarSelector}+ \hyperlink{realincreasingdomain:realincreasingdomainvaliterator}{RealIncreasingDomain} \\
\hline\\
\end{tabular}
If the model has decision variables of different types, then these default branchings are evaluated in this order: first, the set decision variables are considered until they are all instantiated, then branching occurs on the pool of integer decision variables, and last on the pool of real decision variables.

\paragraph{Decision variables.}
Branchings apply to decision variables only. A branching can occur (i.e. the tree node can be separated according to this strategy) if and only if there exists a decision variable in its scope that is still not instantiated.
The non-decision variables are also called \emph{implied variables} because it is expected that, all variables -- including these -- will be instantiated (i.e. they will form a solution) by propagation as soon as all the decision variables will be instantiated. Consider for example, a problem with two sets of variables $x$ and $y$ linked by channeling or by some implication $x=S\implies y=T$ then the $x$ variables can be set as the decision variables, the $y$ will be instantiated by side-effect. 

By default, every solver variable belongs to the pool of decision variables, unless:
\begin{itemize}
\item it corresponds to a model variable created with flag \hyperlink{vnodecision:vnodecisionoptions}{\tt Options.V\_NO\_DECISION};
\item or it is internally created by the solver (e.g. when reading some model constraint) and explicitely excluded from the pool;
\item or the default branching strategies are overriden and the variable does not belong to the scope of one of the strategies specified by the user.
\end{itemize}

The scope of a branching strategy is defined at the creation of the strategy. For example,
\begin{lstlisting}
  new AssignSetVar(new MinDomSet(solver, solver.getVar(svars)), new MinEnv()));
\end{lstlisting}
defines a branching strategy that only applies to the solver variables corresponding to the model set variables \texttt{svars} (even if they were defined with flag \hyperlink{vnodecision:vnodecisionoptions}{\tt Options.V\_NO\_DECISION}).

Most branching strategies may be declared without specifying their scope. In this case, they apply to all the solver decision variables of the right type. For example, the branching strategy
\begin{lstlisting}
  new AssignSetVar(new MinDomSet(solver), new MinEnv()));
\end{lstlisting}
now applies to all the solver decision set variables.

If the default strategies of the solver are overridden by this strategy alone, then all other integer and real variables will automatically be removed from the decision pool: one has then to ensure that the instantiation of the set variables alone defines a complete solution.
If it is not the case, the branching strategy must be combined with additional branching strategies holding on the remaining unimplied variables.
\begin{note}
  If the default strategies are overriden, then the pool of decision variables is overriden by the union of the scopes of the user-specified branching strategies.
\end{note}
As the branching strategies are evaluated sequentially, a variable may belong to the scope of two different strategies, but it will only be considered by the first strategy, unless this first (user-defined) strategy let the variable un-instantiated.

\paragraph{Overriding the default search strategies.}
A branching strategy is added to the solver, as a goal, using the following API of \texttt{Solver}:
\begin{lstlisting}
  void addGoal(AbstractIntBranchingStrategy branching);
\end{lstlisting}
This method must be called on the solver object \emph{before} calling the solving method.
The initial list of goals is empty. If goals are specified, they are added to the list in the order of their declaration.
Otherwise, the list is initialized with the default goals (in the order: set, integer, real).

When one wants to relaunch the search, the list of goals of the solver can previously be reset using the following instruction:
\begin{lstlisting}
  solver.clearGoals();
\end{lstlisting} 

\paragraph{Complete example.}
The following example adds four branching objects to solver \texttt{s}. 
%on integer variables \texttt{vars1}, \texttt{vars2} and set variables \texttt{svars} . 
The first two branchings are both \texttt{AssignVar} strategies using different variable/value selection heuristics and applied to different scopes: the integer variables \texttt{vars1} and \texttt{vars2}, respectively. The third strategy applies to the set variables \texttt{svars}. The last random strategy applies to all the integer decision variables of the solver.
%  s.addGoal(new AssignVar(new MinDomain(s,s.getVar(vars1)), new IncreasingDomain()));
\begin{lstlisting}
  s.addGoal(BranchingFactory.minDomMinVal(s, s.getVar(vars1)));
  s.addGoal(new AssignVar(new DomOverDeg(s, s.getVar(vars2)), new DecreasingDomain()));
  s.addGoal(new AssignSetVar(new MinDomSet(s, s.getVar(svars)), new MinEnv()));
  s.addGoal(BranchingFactory.randomIntSearch(s, seed));
  s.solve();
\end{lstlisting}
The goals are evaluated in this order: first, variables \texttt{vars1} are considered until they are all instantiated, then branching occurs on variables \texttt{vars2}, then on variables \texttt{svars}. Finally, a random strategy is applied to all the integer decision variables of the solver that are not already instantiated, thereby excluding variables \texttt{vars1} and \texttt{vars2}.

\subsection{Pre-defined search strategies}\label{solver:predefinedsearchstrategy}\hypertarget{solver:predefinedsearchstrategy}{}

This section presents the strategies available in Choco. These objects are also detailed in Part \hyperlink{part:elements}{Elements of Choco}.
See Chapter \hyperlink{advanced}{advanced uses} for a description of how to write search strategies in Choco.

\paragraph{Branching strategy}\label{solver:branchstrat}\hypertarget{solver:branchstrat}{}
defines the way to branch from a tree search node.
  
\noindent The \textbf{branching strategies} currently available in Choco are the following: 
\begin{notedef}\tt
\hyperlink{assigninterval:assignintervalbranchstrat}{AssignInterval}, \hyperlink{assignorforbidintvarval:assignorforbidintvarvalbranchstrat}{AssignOrForbidIntVarVal}, \hyperlink{assignorforbidintvarvalpair:assignorforbidintvarvalpairbranchstrat}{AssignOrForbidIntVarValPair}, \hyperlink{assignsetvar:assignsetvarbranchstrat}{AssignSetVar}, \hyperlink{assignvar:assignvarbranchstrat}{AssignVar}, \hyperlink{domoverwdeg:domoverwdegbranchstrat}{DomOverWDegBranchingNew}, \hyperlink{domoverwdegbin:domoverwdegbinbranchstrat}{DomOverWDegBinBranchingNew}, \hyperlink{impact:impactbranchstrat}{ImpactBasedBranching}, \hyperlink{packdynremovals:packdynremovalsbranchstrat}{PackDynRemovals}, \hyperlink{settimes:settimesbranchstrat}{SetTimes}, \hyperlink{taskdomoverwdeg:taskdomoverwdegbranchstrat}{TaskOverWDegBinBranching}.
\end{notedef} 
They implement interface \texttt{BranchingStrategy}.   


\paragraph{Variable selector}\label{solver:variableselector}\hypertarget{solver:variableselector}{}
defines the way to choose a non instantiated variable on which the next decision will be made.

\noindent The \textbf{variable selectors} currently available in Choco are the following: 
\begin{itemize}
\item implementing interface \texttt{VarSelector<IntDomainVar>}:
\begin{notedef}\tt
\hyperlink{compositeintvarselector:compositeintvarselectorvarselector}{CompositeIntVarSelector}, \hyperlink{lexintvarselector:lexintvarselectorvarselector}{LexIntVarSelector}, \hyperlink{maxdomain:maxdomainvarselector}{MaxDomain}, \hyperlink{maxregret:maxregretvarselector}{MaxRegret}, \hyperlink{maxvaldomain:maxvaldomainvarselector}{MaxValueDomain}, \hyperlink{mindomain:mindomainvarselector}{MinDomain}, \hyperlink{minvaldomain:minvaldomainvarselector}{MinValueDomain}, \hyperlink{mostconstrained:mostconstrainedvarselector}{MostConstrained},  \hyperlink{randomvarint:randomvarintvarselector}{RandomIntVarSelector},  \hyperlink{staticvarorder:staticvarordervarselector}{StaticVarOrder}
\end{notedef}
%\noindent The \textbf{set variable selectors} currently available in Choco are the following: 
\item implementing interface \texttt{VarSelector<SetVar>}:   
\begin{notedef}\tt
\hyperlink{maxdomset:maxdomsetvarselector}{MaxDomSet}, \hyperlink{maxregretset:maxregretsetvarselector}{MaxRegretSet}, \hyperlink{maxvaldomset:maxvaldomsetvarselector}{MaxValueDomSet}, \hyperlink{mindomset:mindomsetvarselector}{MinDomSet}, \hyperlink{minvaldomset:minvaldomsetvarselector}{MinValueDomSet}, \hyperlink{mostconstrainedset:mostconstrainedsetvarselector}{MostConstrainedSet},  \hyperlink{randomvarset:randomvarsetvarselector}{RandomSetVarSelector},  \hyperlink{staticsetvarorder:staticsetvarordervarselector}{StaticSetVarOrder}
\end{notedef}
%\noindent The \textbf{real variable selector} currently available in Choco is the following: 
\item implementing interface \texttt{VarSelector<RealVar>}:
\begin{notedef}\tt
\hyperlink{cyclicrealvarselector:cyclicrealvarselectorvarselector}{CyclicRealVarSelector}
\end{notedef}
\end{itemize}

\subsubsection{Value iterator}\label{solver:valueiterator}\hypertarget{solver:valueiterator}{}
Once the variable has been choosen, the solver has to compute its value. The first way to do it is to schedule all the values once and to give an iterator to the solver.

\noindent The \textbf{value iterators} currently available in Choco are the following: 
\begin{itemize}
\item implementing interface \texttt{ValIterator<IntDomainVar>}:
\begin{notedef}\tt
\hyperlink{decreasingdomain:decreasingdomainvaliterator}{DecreasingDomain}, \hyperlink{increasingdomain:increasingdomainvaliterator}{IncreasingDomain}
\end{notedef}
\item implementing interface \texttt{ValIterator<RealVar>}:
\begin{notedef}\tt
\hyperlink{realincreasingdomain:realincreasingdomainvaliterator}{RealIncreasingDomain}
\end{notedef}
\end{itemize}

\subsubsection{Value selector}\label{solver:valueselector}\hypertarget{solver:valueselector}{}
The second way to do it is to compute the next value at each call.

\noindent The \textbf{integer value selector} currently available in Choco are the following: 
\begin{itemize}
\item implementing interface \texttt{ValSelector<IntDomainVar>}:
\begin{notedef}\tt
  \begin{itemize}
  \item \hyperlink{maxval:maxvalvalselector}{MaxVal}, \hyperlink{midval:midvalvalselector}{MidVal}, \hyperlink{minval:minvalvalselector}{MinVal}
  \item \hyperlink{bestfit:bestfitvalselector}{BestFit}, \hyperlink{costregularvalselector:costregularvalselectorvalselector}{CostRegularValSelector}, \hyperlink{fcostregularvalselector:fcostregularvalselectorvalselector}{FCostRegularValSelector}, \hyperlink{randomintvalselector:randomintvalselectorvalselector}{RandomIntValSelector}, %\hyperlink{mcrvalselector:mcrvalselectorvalselector}{MCRValSelector}, 
  \end{itemize}
\end{notedef}
\item implementing interface \texttt{ValSelector<SetVar>}:
  \begin{notedef}\tt
\hyperlink{minenv:minenvvalselector}{MinEnv}, \hyperlink{randomsetvalselector:randomsetvalselectorvalselector}{RandomSetValSelector}
\end{notedef}
\end{itemize}


\subsection{Why is it important to define a search strategy ?}\label{solver:whyisitimportanttodefineasearchstrategy}\hypertarget{solver:whyisitimportanttodefineasearchstrategy}{}

%At a In a partial instantiation, when a fix point has been reached, the Solver needs to take a decision to resume the search. The way decisions are chosen has a \textbf{real impact on the resolution step efficient}. 
\begin{note}
\emph{The search strategy should not be under-estimatimated!!}
A well-suited search strategy can reduce: the execution time, the number of expanded nodes, the number of backtracks.
\end{note}
Let see that small example:
\begin{lstlisting}
	Model m = new CPModel();
        int n = 1000;
        IntegerVariable var = Choco.makeIntVar("var", 0, 2);
        IntegerVariable[] bi = Choco.makeBooleanVarArray("b", n);
        m.addConstraint(Choco.eq(var, Choco.sum(bi)));

        Solver badStrat = new CPSolver();
        badStrat.read(m);
        badStrat.addGoal(
                new AssignVar(
                        new MinDomain(badStrat), 
                        new IncreasingDomain()
                ));
        badStrat.solve();
        badStrat.printRuntimeStatistics();

        Solver goodStrat = new CPSolver();
        goodStrat.read(m);
        goodStrat.addGoal(
                new AssignVar(
                        new MinDomain(goodStrat, goodStrat.getVar(new IntegerVariable[]{var})), 
                        new IncreasingDomain()
                ));
        goodStrat.solve();
        goodStrat.printRuntimeStatistics();
\end{lstlisting}

This model ensures that $var = b_{0} + b_{1} + \ldots + b_{1000}$ where $var$ is an integer variable with a small domain $[0,2]$ and $b_{i}$ are binary variables. No deduction arose from the propagation here, so a fix point is reached at the beginning of the search. A branching decision has to be taken, by selecting a variable and the first value to assign to it. Using the first strategy, the solver will find a solution after creating 1001 nodes: it iterates over all the variables, starting by assigning the 1000 binary variables $b_i$ (according to the \texttt{MinDomain} variable selector) to $0$ (according to the \texttt{IncreasingDomain} value iterator), variable $var$ is fixed to $0$ at the very last propagation. The second strategy finds the same solution with only two nodes: after branching first on $var=0$, propagation immediately fixes all the binary variables to $0$. 

\subsection{Restarts}\label{solver:restarts}\hypertarget{solver:restarts}{}

Restart means stopping the current tree search, then starting a new tree search from the root node.
Restarting makes sense only when coupled with randomized dynamic branching strategies ensuring that the same enumeration tree is not constructed twice. The branching strategies based on the past experience of the search, such as \texttt{DomOverWDegBranching}, \texttt{DomOverWDegBinBranching} and \texttt{ImpactBasedBranching}, are more accurate in combination with a restart approach.

Unless the number of allowed restarts is limited, a tree search with restarts is not complete anymore. It is a good strategy, though, when optimizing an NP-hard problem in a limited time.


Restarts can be set using the following API available on the \texttt{Solver}:
\begin{lstlisting}
setGeometricRestart(int base, double grow);
setGeometricRestart(int base, double grow, int restartLimit);
\end{lstlisting}
It performs a search with restarts controlled by the number of backtracks. 
Parameter \texttt{base} indicates the maximal number of backtracks allowed in the first search tree. Once this limit is reached, a restart occurs and the search continues until \texttt{base}*\texttt{grow} backtracks are done, and so on. After each restart, the limit number of backtracks is increased by the geometric factor \texttt{grow}. 
\texttt{restartLimit} states the maximum number of restarts.
\begin{lstlisting}
	CPSolver s = new CPSolver();
	s.read(model);
	
	s.setGeometricRestart(14, 1.5d);
	s.setFirstSolution(true);
	s.generateSearchStrategy();
	s.attachGoal(new DomOverWDegBranching(s, new IncreasingDomain()));
	s.launch();
\end{lstlisting}

The Luby's restart policy is an alternative to the geometric restart policy, and can be defined 
using the following API available on the \texttt{Solver}:
\begin{lstlisting}
setLubyRestart(int base);
setLubyRestart(int base, int grow);
setLubyRestart(int base, int grow, int restartLimit);
\end{lstlisting}
It performs a search with restarts controlled by the number of backtracks. 
The maximum number of backtracks allowed at a given restart iteration is given by \texttt{base} multiplied by the Las Vegas coefficient at this iteration. 
The sequence of these coefficients is defined recursively on its prefix subsequences: starting from the first prefix $1$, the $(k+1)$-th prefix is the $k$-th prefix repeated \texttt{grow} times and immediately followed by coefficient \texttt{grow}$^k$.
\begin{itemize}
	\item the first coefficients for \texttt{grow}=2 : [1, 1, 2, 1, 1, 2, 4, 1, 1, 2, 1, 1, 2, 4, 8, 1,...]
	\item the first coefficients for \texttt{grow}=3 : [1, 1, 1, 3, 1, 1, 1, 3, 1, 1, 1, 3, 9,...]
\end{itemize}

\begin{lstlisting}
	CPSolver s = new CPSolver();
	s.read(model);
	
	s.setLubyRestart(50, 2, 100);
	s.setFirstSolution(true);
	s.generateSearchStrategy();
	s.attachGoal(new DomOverWDegBranching(s, new IncreasingDomain()));
	s.launch();
\end{lstlisting}

\section{Limiting Search Space}\label{solver:limitingsearchspace}\hypertarget{solver:limitingsearchspace}{}
The \texttt{Solver} class provides ways to limit the tree search controlled by different criteria.
%Limits may be imposed on the search algorithm to avoid spending too much time in the exploration. 
These limits have to be specified before the resolution. They are updated and checked each time a new node is created. 
 Once a limit is reached, the search stops even if no solution is found.
\begin{description}
\item[time limit:] concerns the elapsed time from the beginning of the search (i.e. from the call to a resolution method).
A time limit is set using the \texttt{Solver} API \mylst{setTimeLimit(int timeLimit)}, where \textit{timeLimit} is in milliseconds. 
\mylst{getTimeCount()} returns the total solving time. 
\item[node limit:] concerns the number of opened nodes.
A node limit is set using the \texttt{Solver} API \mylst{setNodeLimit(int nodeLimit)}.
\mylst{getNodeCount()} returns the total number of explored nodes.
\item[backtrack limit:] concerns the number of performed backtracks. 
A backtrack limit is set using the \texttt{Solver} API \mylst{setBackTrackLimit(int backtrackLimit)}.
\mylst{getBackTrackCount()} return the total number of backtracks.
\item[fail limit:] concerns the number of contradiction encountered.
A fail limit is set using the \texttt{Solver} API \mylst{setFailLimit(int failLimit)}.
\mylst{getFailCount()} returns the total number of failures.
By default, the failure count is recorded, one should call \mylst{monitorFailLimit(true)} to activate it.
\end{description}
\todo{Define all these notions more precisely and add an example.}

\section{Logging the search}\label{solver:logs}\hypertarget{solver:logs}{}
A logging class is available to produce trace statements throughout search: \mylst{ChocoLogging}. 

\subsection{Architecture }\label{solver:logarchitecture}\hypertarget{solver:logarchitecture}{}

Choco logging system is based on the \mylst{java.util.logging} package and located in the package \mylst{common.logging}.
Most Choco abstract classes or interfaces propose a static field \mylst{LOGGER}.
The following figures present the architecture of the logging system with the default verbosity.

\insertGraphique{.7\linewidth}{media/logger-default.png}{Logger Tree with the default verbosity}

The shape of the node depicts the type of logger:
\begin{itemize}
	\item The \emph{house} loggers represent private loggers. Do not use these loggers directly because their level are low and all messages would always be displayed.
	\item The \emph{octagon} loggers represent critical loggers. These loggers are provided in the variables, constraints and search classes and could have a huge impact on the global performance.
	\item The \emph{box} loggers are provided for dev and users.
\end{itemize}
The color of the node gives its logging level with DEFAULT verbosity:
\texttt{Level.FINEST} (\textcolor{yellow}{gold}),
\texttt{Level.INFO} (\textcolor{orange}{orange}),
\texttt{Level.WARNING} (\textcolor{red}{red}).

\subsection{Verbosities and messages}\label{solver:verbosityandmessages}\hypertarget{solver:verbosityandmessages}{}

The verbosity level of the solver can be set by the following static method :

\begin{lstlisting}
	// Before the resolution
	ChocoLogging.toVerbose();
	//... resolution declaration
	solver.solve();
	// And after the resolution
	ChocoLogging.flushLogs();
\end{lstlisting}


The following table summarizes the verbosities available in choco: 
\begin{itemize}
	\item \textbf{OFF -- level 0:} Disable logging.	 

	\vspace{0.2cm} 
	\textit{Usage} :  \mylst{ChocoLogging.setVerbosity(Verbosity.OFF)}
	
	\item \textbf{SILENT -- level 1:} Display only severe messages.
	
	\vspace{0.2cm} 
	\textit{Usage} : 
		\begin{itemize}
		\item \mylst{ChocoLogging.toSilent()}
		\item \mylst{ChocoLogging.setVerbosity(Verbosity.SILENT)}
		\end{itemize}
	
	\item \textbf{DEFAULT -- level 2:} Display informations on final search state.
		\begin{itemize}
			\item ON START
				\lstset{language={sh},columns=fixed}
\begin{lstlisting}
 ** CHOCO : Constraint Programming Solver
 ** CHOCO v2.1.1 (April, 2009), Copyleft (c) 1999-2010
 \end{lstlisting}
			\item ON COMPLETE SEARCH:
				\begin{lstlisting}
- Search completed -
 [Maximize		: {0},]
 [Minimize		: {1},]
  Solutions		: {2},
  Times (ms)	: {3},
  Nodes			: {4},
  Backtracks	: {5},
  Restarts		: {6}.
  \end{lstlisting}
	brackets [\textit{line}] indicate \textit{line} is optional,\\
 	\texttt{Maximize} --resp. \texttt{Minimize}-- indicates the best known value before exiting of the objective value in \textit{maximize()} --resp. \textit{minimize()}-- strategy.

			\item ON COMPLETE SEARCH WITHOUT SOLUTIONS :
				\begin{lstlisting}
- Search completed - No solutions
 [Maximize		: {0},]
 [Minimize		: {1},]
  Solutions		: {2},
  Times (ms)	: {3},
  Nodes			: {4},
  Backtracks	: {5},
  Restarts		: {6}.
\end{lstlisting}
	brackets [\textit{line}] indicate \textit{line} is optional,\\
 	\texttt{Maximize} --resp. \texttt{Minimize}-- indicates the best known value before exiting of the objective value in \textit{maximize()} --resp. \textit{minimize()}-- strategy.

			\item ON INCOMPLETE SEARCH:
				\begin{lstlisting}
- Search incompleted - Exiting on limit reached
  Limit			: {0},
 [Maximize		: {1},]
 [Minimize		: {2},]
  Solutions		: {3},
  Times (ms)	: {4},
  Nodes			: {5},
  Backtracks	: {6},
  Restarts		: {7}.
  
  \end{lstlisting}
	brackets [\textit{line}] indicate \textit{line} is optional,\\
 	\texttt{Maximize} --resp. \texttt{Minimize}-- indicates the best known value before exiting of the objective value in \textit{maximize()} --resp. \textit{minimize()}-- strategy.
		\end{itemize}			

	\textit{Usage} : 
		\begin{itemize}
		\item \mylst{ChocoLogging.toDefault()}
		\item \mylst{ChocoLogging.setVerbosity(Verbosity.DEFAULT)}
		\end{itemize}

	\item \textbf{VERBOSE -- level 3:} Display informations on search state.
		\begin{itemize}
			\item EVERY X (default=1000) NODES:
			\begin{lstlisting}
- Partial search - [Objective : {0}, ]{1} solutions, {2} Time (ms), {3} Nodes, {4} Backtracks, {5} Restarts.
			\end{lstlisting}
			\texttt{Objective} indicates the best known value.

			\item ON RESTART : 
			\begin{lstlisting}
- Restarting search - {0} Restarts.
			\end{lstlisting}
		\end{itemize}
		
		\textit{Usage} : 
		\begin{itemize}
		\item \mylst{ChocoLogging.toVerbose()}
		\item \mylst{ChocoLogging.setVerbosity(Verbosity.VERBOSE)}
		\end{itemize}

	\item \textbf{SOLUTION -- level 4:} display all solutions.
		\begin{itemize}
			\item AT EACH SOLUTION:
			\begin{lstlisting}
- Solution #{0} found. [Objective: {0}, ]{1} Solutions, {2} Time (ms), {3} Nodes, {4} Backtracks, {5} Restarts.
  X_1:v1, x_2:v2...
			\end{lstlisting}
		\end{itemize}
		
		\textit{Usage} : 
		\begin{itemize}
		\item \mylst{ChocoLogging.toSolution()}
		\item \mylst{ChocoLogging.setVerbosity(Verbosity.SOLUTION)}
		\end{itemize}

	\item \textbf{SEARCH -- level 5:} Display the search tree.
		\begin{itemize}
			\item AT EACH NODE, ON DOWN BRANCH:
			\begin{lstlisting}
...[w] down branch X==v branch b
			\end{lstlisting}
where \texttt{w} is the current world index, \texttt{X} the branching variable, \texttt{v} the branching value and \texttt{b} the branch index. This message can be adapted on variable type and search strategy.

			\item AT EACH NODE, ON UP BRANCH:
			\begin{lstlisting}
...[w] up branch X==v branch b
			\end{lstlisting}
where \texttt{w} is the current world index, \texttt{X} the branching variable, \texttt{v} the branching value and \texttt{b} the branch index. 
		\end{itemize}
		
		\textit{Usage} : 
		\begin{itemize}
		\item \mylst{ChocoLogging.toSearch()}
		\item \mylst{ChocoLogging.setVerbosity(Verbosity.SEARCH)}
		\end{itemize}

	\item \textbf{FINEST -- level 6:} display all logs.
	
	\vspace{0.2cm} 
	\textit{Usage} :  \mylst{ChocoLogging.setVerbosity(Verbosity.FINEST)}

\end{itemize}

More precisely, if the verbosity level is greater than DEFAULT, then the verbosity levels of the model and of the solver are increased to INFO, and the verbosity levels of the search and of the branching are slightly modified to display the solution(s) and search messages.

\subsection{Basic settings}\label{solver:logbasicsettings}\hypertarget{solver:logbasicsettings}{}

Note that in the case of a verbosity greater or equals to \texttt{toVerbose()}, the regular search information step is set to 1000, by default. You can change this value, using:
\begin{lstlisting}
  ChocoLogging.setEveryXNodes(20000);
\end{lstlisting}
 

Note that in the case of verbosity \texttt{toSearch()}, trace statements are printed up to a maximal depth in the search tree. The default value is set to 25, but you can change the value of this threshold, say to 10, with the following setter method:
\begin{lstlisting}
  ChocoLogging.setLoggingMaxDepth(10);
\end{lstlisting}


\section{Clean a Solver}\label{solver:clean}\hypertarget{solver:clean}{}

Although it is simple and secure to create new instance of \mylst{Solver}, sometimes it  is more obvious and efficient to reuse a \mylst{Solver}. 
It is recommended to reuse instance of \mylst{Solver} when a problem is being solved more than once without deep modifications between two resolutions:
\begin{itemize}
\item a problem is being resolved with different search strategies,
\item a problem is being modified (by adding or removing constraints) through multiple resolutions.
\end{itemize}

\begin{note}
Reusing a \mylst{Solver} must be prepared and well thought out. A backup of the initial state of the \mylst{Solver} may be necessary to recover initial domains and constraints internal structures. In the case where new constraints are created and added, this must be done manipulating \mylst{Solver} objects (SConstraint for example). So, this required knowledge in advanced uses of \mylst{Solver}.
\end{note}

What are the methods to clean up a Solver ?
\begin{itemize}
\item \underline{reset the search strategy :} this is done by calling \mylst{resetSearchStrategy()} on a \mylst{CPSolver}.
A call to this method clears the defined branching strategies (safely removes previous ones) and sets the current search strategy to \mylst{null}.
User's defined branching strategies must be defined again.
User's defined limits will be reset for the next search.

\item \underline{cancel restarts :} this is done by calling \mylst{cancelRestarts(Solver solver)} on \mylst{RestartFactory}.
Sets the restarts parameters to the default ones. This must be done if a restart strategy has been declared.

\end{itemize}

\subsection{What about simply calling \mylst{solver.clear()}?}

A call to \mylst{solver.clear()} will reset every internal structures of a \mylst{Solver}: it clears the variable list, the constraint list, the environment, the propagation engine, the model read, etc.
This set the solver in the same state as it was just after its creation.

\subsection{Things to know about \mylst{Solver} reusability}

There are few things to know:
\begin{itemize}
\item a variable cannot be removed from a \mylst{Solver},
\item a statically posted constraint (added using \mylst{solver.postCut(SConstraint c)}) can be removed \textit{at any time} using \mylst{void eraseConstraint(SConstraint c)},
\item a dynamically posted constraint (added using \mylst{solver.post(SConstraint c)}) can be removed \textit{at root node} using \mylst{void eraseConstraint(SConstraint c)}.
\end{itemize}


%\subsection{Optimization}\label{solver:optimization}\hypertarget{solver:optimization}{}
%\todo{to introduce}
%\begin{lstlisting}
%  Model m = new CPModel();
%  IntegerVariable obj1 = makeEnumIntVar("obj1", 0, 7);
%  IntegerVariable obj2 = makeEnumIntVar("obj1", 0, 5);
%  IntegerVariable obj3 = makeEnumIntVar("obj1", 0, 3);
%  IntegerVariable cost = makeBoundIntVar("cout", 0, 1000000);
%  int capacity = 34;
%  int[] volumes = new int[]{7, 5, 3};
%  int[] energy = new int[]{6, 4, 2};
%  // capacity constraint
%  m.addConstraint(leq(scalar(volumes, new IntegerVariable[]{obj1, obj2, obj3}), capacity));
%	
%  // objective function
%  m.addConstraint(eq(scalar(energy, new IntegerVariable[]{obj1, obj2, obj3}), cost));
%  
%  Solver s = new CPSolver();
%  s.read(m);
%  
%  s.maximize(s.getVar(cost), false);
%\end{lstlisting}
\label{doc:solver}\hypertarget{doc:solver}{}
%\input{chapters/constraints.tex}\label{doc:constraints}\hypertarget{doc:constraints}{}
%!TEX root = ../content-doc.tex
%\part{advanced}
\label{advanced}
\hypertarget{advanced}{}


\chapter{Advanced uses of Choco}\label{advanced:advancedusesofchoco}\hypertarget{advanced:advancedusesofchoco}{}

\section{Environment}\label{advanced:environment}\hypertarget{advanced:environment}{}

Environment is a central object of the backtracking system. It defines the notion of \textit{world}. A world contains values of storable objects or operations that permit to \textit{backtrack} to its state. The environment \textit{pushes} and \textit{pops} worlds.

There are \textit{primitive} data types (\mylst{IStateBitSet, IStateBool, IStateDouble, IStateInt, IStateLong}) and \textit{objects} data types (\mylst{IStateBinarytree, IStateIntInterval, IStateIntProcedure, IStateIntVector, IStateObject, IStateVector}).

There are two different environments: \textit{EnvironmentTrailing} and \textit{EnvironmentCopying}.

\subsection{Copying}\label{advanced:copying}\hypertarget{advanced:copying}{}
In that environment, each data type is defined by a value (primitive or object) and a timestamp. Every time a world is pushed, each value is copied in an array (one array per data type), with finite indice. When a world is popped, every value is restored. 

\subsection{Trailing}\label{advanced:trailing}\hypertarget{advanced:trailing}{}
In that environment, data types are defined by its value. Every operation applied to a data type is pushed in a \textit{trailer}. When a world is pushed, the indice of the last operation is stored. When a world is popped, these operations are popped and \textit{unapplied} until reaching the last operation of the previous world.\\\textit{Default one in CPSolver}

\subsection{Backtrackable structures}\label{advanced:backtrackablestructures}\hypertarget{advanced:backtrackablestructures}{}
\todo{to complete}
\section{How does the propagation engine work ?}

Once the \mylst{Model} and \mylst{Solver} have been defined, the resolution can start. It is based on decisions and filtering orders, this is the propagation engine. In this part, we're going to present how the resolution is guided in Choco. 

A resolution instruction (\mylst{solve()}, \mylst{solveAll()}, \mylst{maximize(...)} or \mylst{minimize(...)}) always starts by setting options based on resolution policy, then generates the search strategies and ends by running the search loop. 

\subsection{How does a search loop work ?}\label{advanced:howdoesasearchloopwork}\hypertarget{advanced:howdoesasearchloopwork}{}
The search loop is the \textit{conductor} of the engine. It goes down and up in the branches in order to cover the tree search, call the filtering algorithm, etc.
Here is the organigram of the search loop. 

%\insertGraphique{0.8\linewidth}{media/searchloop.pdf}{Organigram of the search loop}

\begin{figure}[!htp]
	\centerline{\Graph{media/searchloop.pdf}{width=1\linewidth}}
	\caption[]{Organigram of the search loop}\label{fig:media/searchloop.pdf}
\end{figure}

Basically, the search loop is divided in 5 steps: \mylst{INITIAL PROPAGATION} (highlighted in red), \mylst{OPEN NODE} (highlighted in green), \mylst{DOWN BRANCH} (highlighted in violet), \mylst{UP BRANCH} (highlighted in orange) and \mylst{RESTART} (highlighted in blue). 

\subsection{Propagate}\label{advanced:propagate}\hypertarget{advanced:propagate}{}

%% fix point A fix point is reached when there is no more event to treat
\newglossaryentry{fix point}{name={fix point},description={definition of a fix point}}

The main and unique \mylst{PropagationEngine} of Choco is \mylst{ChocoEngine}. This engine stores events occurring on variables, \textit{variable events}, and specific calls to constraint filtering algorithm, \textit{constraint events}, in order to reach a \gls{fix point} or to detect contradictions. Events are stored in queues (FIFO).  

On a call to \mylst{Solver.propagate()} or during a resolution step, the consistency of a model is computed: stored events are popped and propagated (apply side-effects). The propagation of a single event can create new ones, feeding the system until fix point or contradiction. 



\begin{figure}[!htp]
	\centerline{\Graph{media/propagationloop.pdf}{width=0.5\linewidth}}
	\caption[]{Organigram of the propagation loop}\label{fig:media/propagationloop.pdf}
\end{figure}

 
If the propagation of an event leads to a contradiction, the propagation engine stop the process. In both case, the search loop take up with the new state.

\subsubsection{Seven priorities}

Before going further, it is important to know that events declare a parameter named \textit{priority}. The priority of a constraint's event depends on the constraint priority (required in the super class constructor). And the priority of a variable's event is given by the maximum priority of the variable's constraints. 

To each priority corresponds a \textit{rank}. In the propagation engine, there are as many queues as ranks. 
During the propagation loop, the rank is used to push the event on the corresponding queue. Events are \underline{always} popped from the smallest ranked queue to the largest ranked queue.

There are seven priorities :  \mylst{UNARY}, \mylst{BINARY}, \mylst{TERNARY}, \mylst{LINEAR}, \mylst{QUADRATIC}, \mylst{CUBIC} and \mylst{VERY_SLOW}, each of these qualifies the "cost" of a constraint (related to its internal filtering algorithm). 

By default, priorities and ranks are defined as follow:
\begin{tabular}{|l|r|}
\hline
Priority & Rank \\
\hline
unary & 1\\
binary & 2\\
ternary & 3\\
linear & 4\\
quadratic & 5\\
cubic & 6\\
very slow & 7\\
\hline
\end{tabular}
\vspace{0.2cm}

Although the priority of a constraint cannot be changed, the rank of priority can be reconsider by setting another value to \mylst{Configuration.VEQ_ORDER} or \mylst{Configuration.CEQ_ORDER}. 

\vspace{0.5cm}

\begin{lstlisting}
Configuration conf = new Configuration();
conf.putInt(Configuration.VEQ_ORDER, 7654321); // default value is 1234567
conf.putInt(Configuration.CEQ_ORDER, 1111744); // default value is 1234567
Solver solver = new CPSolver(conf);
\end{lstlisting}

In that example, priorities ranks are reversed for the variable queues, and totally rearranged for the constraint queue.


\subsubsection{Constraint event}

At the very beginning of resolution, when constraint filtering algorithms have not been called once, a call to the \mylst{awake()} method is planned by posting constraint event to the propagation engine. For some \textit{expensive} constraints (like \hyperlink{geost:geostconstraint}{Geost}), a call to the main filtering algorithm (described in \mylst{propagate()}) can be added during the resolution by posting a constraint event to the propagation engine, in that case a call to the \mylst{propagate()} method is planned. 
This event is added to the list of constraint events to be treated by the propagation engine.

\subsubsection{Variable event}

The resolution goal is to instantiate variables in order to find solutions. Instantiation of a variable is done applying modification on its domain. 
Each time a modification is applied on a domain, an event is posted, storing informations about the action done (event type, variable, values, etc.). This event will be given to the related constraints of the modified variable, to check consistency and propagate this new information to the other variables.

Depending on the type of domain, events existing in Choco are:

\noindent\begin{tabular}{lp{.6\linewidth}}
\hline
Event &  description \\
\hline
\multicolumn{2}{l}{Integer variable \mylst{IntDomainVar}}\\
  \hline
  \mylst{REMVAL} & Remove a single value from the domain.\\
  \mylst{INCINF} &  Increase the lower bound of the domain. \\
  \mylst{DECSUP} &  Decrease the upper bound of the domain. \\
  \mylst{INSTINT} &  Instantiate the domain, i.e. reduce it to a single value. \\   
\hline
\multicolumn{2}{l}{Set variable \mylst{SetVar}}\\
\hline
  \mylst{REMENV} & Remove a single value from the envelope domain \\
  \mylst{ADDKER} &  Add a single value to the kernel domain\\
  \mylst{INSTSET} &  Instantiate the domain, i.e. set values to both kernel and envelope domain. \\   
\hline  
\multicolumn{2}{l}{Real variable \mylst{RealVar}}\\
\hline
  \mylst{INCINF} &  Increase the lower bound of the domain\\
  \mylst{DECSUP} &  Decrease the upper bound of the domain\\
\hline  
\end{tabular}

An event given as a parameter to the engine is then pushed into a unique queue, waiting to be treated. There are seven different queues where an event can be pushed, it depends on the priority of the event.
   

Beware, an event is not automatically pushed in a queue: if an other event based on the same variable is already present in the queue, the two events are merged into one. 
\begin{note}
Due to promotion, variables' events can be treated in a different order than creation one!

Let X and Y, two integer variables with priorities of the same rank. During the propagation of a constraint, the lower bound of a integer variable X is updated to 3 (this event is added to the queue Q), then Y is instantiated to 4 (this other event is added to the same queue Q) and finally the upper bound of X is updated to 4 (the already pushed event on X is updated with this new information). At that point, there are only 2 events in the queue Q: one on X and one on Y. 
As X and Y have the same rank, events will be treated by creation order: event on X first ant event on Y then. But the event on X contains 2 informations, one on the lower bound modification and another on the upper bound modification. So, although the instantiation of Y has been created first, the treatment of the upper bound modification of X will be treated first.
\end{note}


\begin{note}
Original event can be \textit{promoted}: for example removing the last but one value of an integer variable is promoted to instantiation.

Promotion are:

\noindent\begin{tabular}{lp{.6\linewidth}}
\hline
Original event &  can be promoted to \\
\hline
\multicolumn{2}{l}{Integer variable \mylst{IntDomainVar}}\\
  \hline
  \mylst{REMVAL} & \mylst{INCINF} or \mylst{DECSUP} or \mylst{INSTINT}\\
  \mylst{INCINF} &  \mylst{INSTINT}\\
  \mylst{DECSUP} &  \mylst{INSTINT} \\
  \mylst{INSTINT} &  \textit{none} \\   
\hline
\multicolumn{2}{l}{Set variable \mylst{SetVar}}\\
\hline
  \mylst{REMENV} & \mylst{INSTSET} \\
  \mylst{ADDKER} &  \mylst{INSTSET}\\
  \mylst{INSTSET} &  \textit{none} \\   
\hline  
\multicolumn{2}{l}{Real variable \mylst{RealVar}}\\
\hline
  \mylst{INCINF} &  \textit{none}\\
  \mylst{DECSUP} &  \textit{none}\\
\hline  
\end{tabular}

 \end{note}


\section{Define your own search strategy}\label{advanced:defineyourownsearchstrategy}\hypertarget{advanced:defineyourownsearchstrategy}{}
%A key ingredient of any constraint approach is a clever branching strategy. The construction of the search tree is done according to a series of Branching objects (that plays the role of achieving intermediate goals in logic programming). The user may specify the sequence of branching objects to be used to build the search tree. 
Section~\hyperlink{solver:searchstrategy}{Search strategy} presented the default branching strategies available in Choco and showed how to post them or to compose them as goals.
In this section, we will start with a very simple and common way to branch by choosing values for variables and specially how to define its own variable/value selection strategy. We will then focus on more complex branching such as dichotomic or n-ary choices. Finally we will show how to control the search space in more details with well known strategy such as LDS (Limited discrepancy search).

Reminder : \hyperlink{advanced:howdoesasearchloopwork}{How does the search loop work?}

For integer variables, the variable and value selection strategy objects are based on the following interfaces:
\begin{itemize}
	\item \mylst{AbstractIntBranchingStrategy}: abstract class for the branching strategy,
	\item \mylst{VarSelector<V>} : Interface for the variable selection (\mylst{V extends Var}),
	\item \mylst{ValIterator<V>} : Interface to describes an iteration scheme on the domain of a variable,
	\item \mylst{ValSelector<V>} : Interface for a value selection.
\end{itemize}

Concrete examples of these interfaces are respectively,  \hyperlink{assignvar:assignvarbranchstrat}{AssignVar}, \hyperlink{mindomain:mindomainvarselector}{MinDomain}, \hyperlink{increasingdomain:increasingdomainvaliterator}{IncreasingDomain}, \hyperlink{maxval:maxvalvalselector}{MaxVal}.

\subsection{How to define your own Branching object}\label{advanced:beyondvariable/valueselection,howtodefineyourownbranchingobject}\hypertarget{advanced:beyondvariable/valueselection,howtodefineyourownbranchingobject}{}

When you need a specific branching strategy that can't be expressed with the ones already existing, you can define your own concrete class of:

\noindent{\begin{tabular}{ll}
\hline
  Default class to implement &  definition \\
  \hline
  \mylst{AbstractBinIntBranchingStrategy} &  abstract class defining a binary tree search \\
  \mylst{AbstractLargeIntSConstraint} &  abstract class defining a n-ary tree search. \\
  \hline\\
\end{tabular}}


\insertGraphique{\linewidth}{media/branching.pdf}{Branching strategy: interfaces and abstract classes.}

We give here two examples of implementations of these classes, first for a binary branching, then for a n-ary branching. 
\begin{lstlisting}   
/**
 * A class for branching schemes that consider two branches: 
 * - one assigning a value to an IntVar (X == v) 
 * - and the other forbidding this assignment (X != v)
 */
public class AssignOrForbid extends AbstractBinIntBranchingStrategy {

    protected VarSelector<IntDomainVar> varSelector;

    protected ValSelector<IntDomainVar> valSelector;

    public AssignOrForbid(VarSelector<IntDomainVar> varSelector,
                          ValSelector<IntDomainVar> valSelector) {
        super();
        this.valSelector = valSelector;
        this.varSelector = varSelector;
    }

    /**
     * Select the variable to constrained
     *
     * @return the branching object
     */
    public Object selectBranchingObject() throws ContradictionException {
        return varSelector.selectVar();
    }

    /**
     * Select the value to assign, and set it in the decision object in parameter
     * @param decision the next decision to apply
     */
    @Override
    public void setFirstBranch(final IntBranchingDecision decision) {
        decision.setBranchingValue(valSelector.getBestVal(decision.getBranchingIntVar()));
    }


    /**
     * Create and return the message to print, in case of strong verbosity
     * @param decision current decision
     * @return pretty print of the current decision
     */
    @Override
    public String getDecisionLogMessage(final IntBranchingDecision decision) {
        return decision.getBranchingObject() +  (decision.getBranchIndex() == 0 ? "==" : "=/=") + decision.getBranchingValue();
    }


    /**
     * Execution action based on the couple: {decision, branching index}.
     * As <code>this</code> build a binary branching, there are only 2 branching indices:
     * 0 -- assignment, the variable is instantiated to the value
     * 1 -- forbidance, the value is removed from the domain of the variable
     *
     * @throws ContradictionException if the decision leads to an incoherence
     */
    @Override
    public void goDownBranch(final IntBranchingDecision decision) throws ContradictionException {
        if (decision.getBranchIndex() == 0) {
            decision.setIntVal();
        } else {
            decision.remIntVal();
        }
    }
}
\end{lstlisting}

\begin{lstlisting}   
/**
 * A class for branching schemes that consider n branches: 
 * -  assigning a value v_i to an variable (X == v_i)
 */
public class Assign extends AbstractLargeIntBranchingStrategy {

    protected final VarSelector<IntDomainVar> varSelector;

    protected ValIterator<IntDomainVar> valIterator;

    public Assign(VarSelector<IntDomainVar> varSelector, ValIterator<IntDomainVar> valIterator) {
        this.varSelector = varSelector;
        this.valIterator = valIterator;
    }

    /**
     * Select the variable to constrained
     *
     * @return the branching object
     */
    public Object selectBranchingObject() throws ContradictionException {
        return varSelector.selectVar();
    }

    /**
     * Select the first value to assign, and set it in the decision object in parameter
     *
     * @param decision the first decision to apply
     */
    public void setFirstBranch(final IntBranchingDecision decision) {
        decision.setBranchingValue(valIterator.getFirstVal(decision.getBranchingIntVar()));
    }

    /**
     * Select the next value to assign, and set it in the decision object in parameter
     *
     * @param decision the next decision to apply
     */
    public void setNextBranch(final IntBranchingDecision decision) {
        decision.setBranchingValue(valIterator.getNextVal(decision.getBranchingIntVar(), decision.getBranchingValue()));
    }

    /**
     * Check wether there is still a value to assign
     *
     * @param decision the last decision applied
     * @return <code>false</code> if there is still a branching to do
     */
    public boolean finishedBranching(final IntBranchingDecision decision) {
        return !valIterator.hasNextVal(decision.getBranchingIntVar(), decision.getBranchingValue());
    }

    /**
     * Apply the computed decision building the i^th branch.
     * --> assignment: the variable is instantiated to the value
     * 
     * 
     * @param decision the decision to apply.
     * @throws ContradictionException if the decision leads to an incoherence
     */
    @Override
    public void goDownBranch(final IntBranchingDecision decision) throws ContradictionException {
        decision.setIntVal();
    }

    /**
     * Reconsider the computed decision, destroying the i^th branch
     * --> forbiddance: the value is removed from the domain of the variable
     * 
     * @param decision the decision that has been set at the father choice point
     * @throws ContradictionException if the non-decision leads to an incoherence
     */
    @Override
    public void goUpBranch(final IntBranchingDecision decision) throws ContradictionException {
        decision.remIntVal();
    }

    /**
     * Create and return the message to print, in case of strong verbosity
     * @param decision current decision
     * @return pretty print of the current decision
     */
    @Override
    public String getDecisionLogMessage(IntBranchingDecision decision) {
        return decision.getBranchingObject() + "==" + decision.getBranchingValue();
    }
}
\end{lstlisting}

\subsection{Define your own variable selection}\label{advanced:defineyourownvariableselection}\hypertarget{advanced:defineyourownvariableselection}{}

\insertGraphique{.5\linewidth}{media/varselector-s.pdf}{Variable selector: interface}

You may extend this small library of branching schemes and heuristics by defining your own concrete classes of \mylst{AbstractIntVarSelector}. We give here an example of an \mylst{VarSelector<IntDomainVar>} with the implementation of a static variable ordering :
\begin{lstlisting}
/**
 * A variable selector selecting the first non instantiated variable according to a given static order
 */
public class StaticVarOrder extends AbstractIntVarSelector {

    private final IStateInt last;

    public StaticVarOrder(Solver solver) {
        super(solver);
        this.last = solver.getEnvironment().makeInt(0);
    }

    public StaticVarOrder(Solver solver, IntDomainVar[] vars) {
        super(solver, vars);
        this.last = solver.getEnvironment().makeInt(0);
    }

    /**
     * Select the next uninstantiated variable, according to the define policy: input order
     * @return the selected variable if exists, <code>null</code> otherwise
     */
    public IntDomainVar selectVar() {
        for (int i = last.get(); i < vars.length; i++) {
            if (!vars[i].isInstantiated()) {
                last.set(i);
                return vars[i];

            }
        }
        return null;
    }
}
\end{lstlisting}

Notice on this example that you only need to implement method \mylst{selectVar()} which belongs to the contract of \mylst{VarSelector}. This method should return a non instantiated variable or \mylst{null}. Once the branching is finished, the next branching (if one exists) is taken by the search algorithm to continue the search, otherwise, the search stops as all variable are instantiated. To avoid the loop over the variables of the branching, a backtrackable integer (\mylst{IStateInt}) could be used to remember the last instantiated variable and to directly select the next one in the table. Notice that backtrackable structures could be used in any of the code presented in this chapter to speedup the computation of dynamic choices.

\insertGraphique{.8\linewidth}{media/varselector-a.pdf}{Variable selector: interface and abstract classes}

If you need an integer variable selector that can be used as a parameter of \hyperlink{lexintvarselector:lexintvarselectorvarselector}{LexIntVarSelector}, it should extend \mylst{IntHeuristicIntVarSelector} or \mylst{DoubleHeuristicIntVarSelector}. These two abstract classes only require to implement one method : \mylst{getHeuristic(IntDomainVar v)} which computes and returns a criterion (\mylst{int} or \mylst{double}). The criteria are used in a master class to select the smallest crtierion's variable 

We give here an other example of an \mylst{VarSelector<IntDomainVar>}, this one extends \mylst{IntHeuristicIntVarSelector} and choose the variable with the smallest domain :
\begin{lstlisting}
public class MinDomain extends IntHeuristicIntVarSelector {

	public MinDomain(Solver solver) {
		super(solver);
	}

	public MinDomain(Solver solver, IntDomainVar[] vs) {
		super(solver, vs);
	}

    /**
     * Compute the criterion, according to the define policy: smallest domain size
     * @return the selected variable if exists, <code>null</code> otherwise
     */
	@Override
	public int getHeuristic(IntDomainVar v) {
		return v.getDomainSize();
	}

}
\end{lstlisting}


You can add your variable selector as a part of a search strategy, using \mylst{solver.addGoal()}.

\subsection{Define your own value selection}\label{advanced:defineyourownvalueselection}\hypertarget{advanced:defineyourownvalueselection}{}
You may also define your own concrete classes of \mylst{ValIterator} or \mylst{ValSelector}. 

\subsubsection{Value selector}\label{advanced:valueselector}\hypertarget{advanced:valueselector}{}

\insertGraphique{.3\linewidth}{media/valselector.pdf}{Value selector: interface}

We give here an example of an \mylst{IntValSelector} with the implementation of a minimum value selecting:
\begin{lstlisting}
public class MinVal implements ValSelector<IntDomainVar> {
  /**
   * selecting the lowest value in the domain
   *
   * @param x the variable under consideration
   * @return what seems the most interesting value for branching
   */
  public int getBestVal(IntDomainVar x) {
    return x.getInf();
  }
}
\end{lstlisting}
Only \mylst{getBestVal()} method must be implemented, returning the best value \emph{in the domain} according to the heuristic.

You can add your value selector as a part of a search strategy, using \mylst{solver.addGoal()}.

\begin{note}
Using a value selector with bounded domain variable is strongly inadvised, except if it pick up bounds value. If the value selector pick up a value that is not a bound, when it goes up in the tree search, that value could be not removed and picked twice (or more)!
\end{note} 

\subsubsection{Values iterator}\label{advanced:valuesiterator}\hypertarget{advanced:valuesiterator}{}

\insertGraphique{.3\linewidth}{media/valiterator.pdf}{Value iterator: interface}

We give here an example of an \mylst{ValIterator} with the implementation of an increasing domain iterator:
\begin{lstlisting}
public class IncreasingDomain implements ValIterator<IntDomainVar> {

  /**
   * testing whether more branches can be considered after branch i, on the alternative associated to variable x
   *
   * @param x the variable under scrutiny
   * @param i the index of the last branch explored
   * @return true if more branches can be expanded after branch i
   */
  public boolean hasNextVal(IntDomainVar x, int i) {
    return (i < x.getSup());
  }

  /**
   * Accessing the index of the first branch for variable x
   *
   * @param x the variable under scrutiny
   * @return the index of the first branch (such as the first value to be assigned to the variable)
   */
  public int getFirstVal(IntDomainVar x) {
    return x.getInf();
  }

  /**
   * generates the index of the next branch after branch i, on the alternative associated to variable x
   *
   * @param x the variable under scrutiny
   * @param i the index of the last branch explored
   * @return the index of the next branch to be expanded after branch i
   */
  public int getNextVal(IntDomainVar x, int i) {
    return x.getNextDomainValue(i);
  }
}
\end{lstlisting}
%Works as an basic \mylst{Iterator} object, implementing the three main methods \mylst{hasNextVal()}, \mylst{getFirstVal()} and \mylst{getNextVal()}.

You can add your value iterator as a part of a search strategy, using \mylst{solver.addGoal()}.

%\todo{under development} See \href{http://choco-solver.net/index.phptitle=userguide:beyondvariable.2fvalueselection.2chowtodefineyourownbranchingobject}{old version}

\section{Define your own limit search space}\label{advanced:defineyourownlimitsearchspace}\hypertarget{advanced:defineyourownlimitsearchspace}{}

To define your own limits/statistics (notice that a limit object can be used only to get statistics about the search), you can create a limit object by extending the \mylst{AbstractGlobalSearchLimit} class or implementing directly the interface \mylst{IGlobalSearchLimit}. Limits are managed at each node of the tree search and are updated each time a node is open or closed. Notice that limits are therefore time consuming. Implementing its own limit need only to specify to the following interface :

\begin{lstlisting}
	/**
	 * The interface of objects limiting the global search exploration
	 */
	public interface GlobalSearchLimit {

	  /**
	   * resets the limit (the counter run from now on)
	   * @param first true for the very first initialization, false for subsequent ones
	   */
	  public void reset(boolean first);
	
	  /**
	   * notify the limit object whenever a new node is created in the search tree
	   * @param solver the controller of the search exploration, managing the limit
	   * @return true if the limit accepts the creation of the new node, false otherwise
	   */
	  public boolean newNode(AbstractGlobalSearchSolver solver);
	
	  /**
	   * notify the limit object whenever the search closes a node in the search tree
	   * @param solver the controller of the search exploration, managing the limit
	   * @return true if the limit accepts the death of the new node, false otherwise
	   */
	  public boolean endNode(AbstractGlobalSearchSolver solver);
	}
\end{lstlisting}

Look at the following example to see a concrete implementation of the previous interface. We define here a limit on the depth of the search (which is not found by default in choco). The \mylst{getWorldIndex()} is used to get the current world, i.e the current depth of the search or the number of choices which have been done from baseWorld. 

\begin{lstlisting}
	public class DepthLimit extends AbstractGlobalSearchLimit {
	
	  public DepthLimit(AbstractGlobalSearchSolver theSolver, int theLimit) {
	    super(theSolver,theLimit);
	    unit = "deep";
	  }
	
	  public boolean newNode(AbstractGlobalSearchSolver solver) {
	    nb = Math.max(nb, this.getProblem().getWorldIndex() –
	    this.getProblem().getSolver().getSearchSolver().baseWorld);
	    return (nb < nbMax);
	  }
	
	  public boolean endNode(AbstractGlobalSearchSolver solver) {
	    return true;
	  }
	
	  public void reset(boolean first) {
	   if (first) {
	    nbTot = 0;
	   } else {
	    nbTot = Math.max(nbTot, nb);
	   }
	   nb = 0;
	  }
\end{lstlisting}

Once you have implemented your own limit, you need to tell the search solver to take it into account. Instead of using a call to the \mylst{solve()} method, you have to create the search solver by yourself and add the limit to its limits list such as in the following code :
\begin{lstlisting}
	Solver s = new CPSolver();
	s.read(model);
	s.setFirstSolution(true);
	s.generateSearchStrategy();
	s.getSearchStrategy().limits.add(new DepthLimit(s.getSearchStrategy(),10));
	s.launch();
\end{lstlisting}

%\subsubsection{Search loop with recomputation}\label{advanced:searchloopwithrecomputation}\hypertarget{advanced:searchloopwithrecomputation}{}

\section{Define your own constraint}\label{advanced:defineyourownconstraint}\hypertarget{advanced:defineyourownconstraint}{}

This section describes how to add you own constraint, with specific propagation algorithms. Note that this section is only useful in case you want to express a constraint for which the basic propagation algorithms (using tables of tuples, or boolean predicates) are not efficient enough to propagate the constraint.

The general process consists in defining a new constraint class and implementing the various propagation methods. We recommend the user to follow the examples of existing constraint classes (for instance, such as \mylst{GreaterOrEqualXYC} for a binary inequality) 

\subsection{The constraint hierarchy}\label{advanced:theconstrainthierarchy}\hypertarget{advanced:theconstrainthierarchy}{}

Each new constraint must be represented by an object implementing the \mylst{SConstraint} interface (\mylst{S} for solver constraint). To help the user defining new constraint classes, several abstract classes defining \texttt{SConstraint} have been implemented. These abstract classes provide the user with a management of the constraint network and the propagation engineering. They should be used as much as possible.

For constraints on integer variables, the easiest way to implement your own constraint is to inherit from one of the following classes, depending of the number of solver integer variables (\texttt{IntDomainVar}) involved:

\centerline{\begin{tabular}{ll}
      \hline
  Default class to implement &  number of solver integer variables \\
  \hline
  \mylst{AbstractUnIntSConstraint} &  \textbf{one} variable \\
  \mylst{AbstractBinIntSConstraint} &  \textbf{two} variables \\
  \mylst{AbstractTernIntSConstraint} &  \textbf{three} variables \\
  \mylst{AbstractLargeIntSConstraint} &  any number of variables. \\
  \hline\\
\end{tabular}}

\noindent Constraints over integers must implement the following methods (grouped in the \texttt{IntSConstraint} interface):

\noindent\begin{tabular}{lp{.6\linewidth}}
  \hline
  Method to implement &  description \\
  \hline
  \mylst{pretty()} &Returns a pretty print of the constraint \\
  \mylst{propagate()} &The main propagation method (propagation from scratch). Propagating the constraint until local consistency is reached. \\
  \mylst{awake()} &Propagating the constraint for the very first time until local consistency is reached. The awake is meant to initialize the data structures contrary to the propagate. Specially, it is important to avoid initializing the data structures in the constructor. \\
  \mylst{awakeOnInst(int x)} &Default propagation on instantiation: full constraint re-propagation. \\
  \mylst{awakeOnBounds(int x)} &Default propagation on improved bounds: propagation on domain revision. \\
  \mylst{awakeOnRemovals(int x, IntIterator v)} &Default propagation on mutliple values removal: propagation on domain revision. The iterator allow to iterate over the values that have been removed. \\
&\\
\hline
\multicolumn{2}{l}{Methods \texttt{awakeOnBounds} and \texttt{awakeOnRemovals} can be replaced by more fine grained methods:}\\
\hline
%Alternative Method &  description \\
%  \hline
  \mylst{awakeOnInf(int x)} &Default propagation on improved lower bound: propagation on domain revision. \\
  \mylst{awakeOnSup(int x)} &Default propagation on improved upper bound: propagation on domain revision. \\
  \mylst{awakeOnRem(int x, int v)} &Default propagation on one value removal: propagation on domain revision.  \\
&\\
  \hline
\multicolumn{2}{l}{To use the constraint in expressions or reification, the following minimum API is mandatory:}\\
  \hline
  \mylst{isSatisfied(int[] x)} &Tests if the constraint is satisfied when the variables are instantiated. \\
	\mylst{isEntailed()} &Checks if the constraint must be checked or must fail. It returns true if the constraint is known to be satisfied whatever happend on the variable from now on, false if it is violated. \\
	\mylst{opposite()} &It returns an AbstractSConstraint that is the opposite of the current constraint. \\
    \hline\\
	\end{tabular}

In the same way, a \textbf{set constraint} can inherit from \texttt{AbstractUnSetSConstraint}, \texttt{AbstractBinSetSConstraint}, \texttt{AbstractTernSetSConstraint} or \texttt{AbstractLargeSetSConstraint}.

A \textbf{real constraint} can inherit from \texttt{AbstractUnRealSConstraint}, \texttt{AbstractBinRealSConstraint} or \texttt{AbstractLargeRealSConstraint}.

A mixed constraint between \textbf{set and integer variables} can inherit from \texttt{AbstractBinSetIntSConstraint} or \texttt{AbstractLargeSetIntSConstraint}.

\begin{note}
A simple way to implement its own constraint is to:
\begin{itemize}
	\item create an empty constraint with only \texttt{propagate()} method implemented and every \texttt{awakeOnXxx()} ones set to \texttt{this.constAwake(false);}
	\item when the propagation filter is sure, separate it into the \texttt{awakeOnXxx()} methods in order to have finer granularity
	\item finally, if necessary, use backtrackables objects to improve the efficient of your constraint
\end{itemize}

\end{note}

\subsubsection{Interact with variables}\label{advanced:interactwithvariables}\hypertarget{advanced:interactwithvariables}{}

One of the constraint function is to remove forbidden values from domain variable (\textit{filtering}). \mylst{Variable}s provide services to allow constraint to interact with their domain.

\vspace{0.5cm}
\textbf{IntDomainVar}

\begin{description}
\item[ ] \mylst{boolean removeVal(int x, final SConstraint cause, final boolean forceAwake)}

Update the domain of the integer variable by removing \mylst{x} from the domain. \mylst{cause} is the constraint at the origin of the event, \mylst{forceAwake} indicates wether or not the \mylst{cause} constraint must be informed of this event. The result of such call can be \mylst{true}, the value has been removed without any trouble, \mylst{false} the value was not present in the domain. A \mylst{ContradictionException} is thrown if it empties the domain of the variable

\item[ ] \mylst{boolean removeInterval(int a, int b, final SConstraint cause, final boolean forceAwake)}

Update the domain of the integer variable by removing all values contained in the interval $[a,b]$ from the domain. \mylst{cause} is the constraint at the origin of the event, \mylst{forceAwake} indicates wether or not the \mylst{cause} constraint must be informed of this event. The result of such call can be \mylst{true}, the values has been removed without any trouble, \mylst{false} if the intersection of the current domain and $[a,b]$ was empty. A \mylst{ContradictionException} is thrown if it empties the domain of the variable.

\item[ ] \mylst{boolean updateInf(int x, final SConstraint cause, final boolean forceAwake)} 

Update the domain of the integer variable by removing all values strictly below \mylst{x} from the domain. \mylst{cause} is the constraint at the origin of the event, \mylst{forceAwake} indicates wether or not the \mylst{cause} constraint must be informed of this event. The result of such call can be \mylst{true}, the lower bound has been updated without any trouble, \mylst{false} the new lower bound was smaller or equal to the actual one. A \mylst{ContradictionException} is thrown if it empties the domain of the variable.

\item[ ] \mylst{boolean updateSup(int x, final SConstraint cause, final boolean forceAwake)} 

Update the domain of the integer variable by removing all values strictly above \mylst{x} from the domain. \mylst{cause} is the constraint at the origin of the event, \mylst{forceAwake} indicates wether or not the \mylst{cause} constraint must be informed of this event. The result of such call can be \mylst{true}, the upper bound has been updated without any trouble, \mylst{false} the new upper bound was greater or equal to the actual one. A \mylst{ContradictionException} is thrown if it empties the domain of the variable. 

\item[ ] \mylst{boolean instantiate(int x, final SConstraint cause, final boolean forceAwake)} 

Update the domain of the integer variable by removing all values but \mylst{x} from the domain. \mylst{cause} is the constraint at the origin of the event, \mylst{forceAwake} indicates wether or not the \mylst{cause} constraint must be informed of this event. The result of such call can be \mylst{true}, the domain has been reduced to a singleton without any trouble, \mylst{false} the domain was already instantiated to the same value . A \mylst{ContradictionException} is thrown if \mylst{x} is out of the domain or if the domain was already instantiated to another value.

\end{description}

\vspace{0.5cm}
\textbf{SetVar}

\begin{description}
\item[ ] \mylst{boolean remFromEnveloppe(int x, final SConstraint cause, final boolean forceAwake)}

Update the domain of the set variable by removing \mylst{x} from the envelope's domain. \mylst{cause} is the constraint at the origin of the event, \mylst{forceAwake} indicates wether or not the \mylst{cause} constraint must be informed of this event. The result of such call can be \mylst{true}, the value has been removed without any trouble, \mylst{false} the value was not present in the envelope. A \mylst{ContradictionException} is thrown if \mylst{x} is present in the kernel's domain.

\item[ ] \mylst{boolean addToKernel(int x, final SConstraint cause, final boolean forceAwake)}

Update the domain of the set variable by adding \mylst{x} into the kernel's domain. \mylst{cause} is the constraint at the origin of the event, \mylst{forceAwake} indicates wether or not the \mylst{cause} constraint must be informed of this event. The result of such call can be \mylst{true}, the value has been added without any trouble, \mylst{false} the value was already present in the kernel. A \mylst{ContradictionException} is thrown if \mylst{x} is not present in the envelope's domain.

\item[ ] \mylst{boolean instantiate(int[] xs, final SConstraint cause, final boolean forceAwake)} 

Update the domain of the set variable by removing every values but those in \mylst{xs} from the envelope's domain and by adding every values of \mylst{xs} into the kernel's domain. A set variable is known as \textit{instantiated} when $E \cap K \ne \emptyset $ and $E \bigtriangleup K = \emptyset$. 
\mylst{cause} is the constraint at the origin of the event, \mylst{forceAwake} indicates wether or not the \mylst{cause} constraint must be informed of this event. The result of such call can be \mylst{true}, the envelope or the kernel have been updated without trouble, \mylst{false} the envelope and the kernel were already equal to \mylst{xs} . A \mylst{ContradictionException} is thrown if the at least one value from \mylst{xs} is not present in the kernel's domain or in the envelope's domain.
 
\end{description}

\vspace{0.5cm}
\textbf{RealVar}

\begin{description}
\item[ ] \mylst{void intersect(RealInterval interval)}

Update the domain of the real variable by intersecting the domain with \mylst{interval} (define at least by two doubles, known as lower and upper bound). A \mylst{ContradictionException} is thrown the resulting interval is incoherent (the new lower bound is greater than the new upper bound).

\end{description}


\subsubsection{How do I add my constraint to the Model ?}\label{advanced:howdoiaddmyconstrainttothemodel}\hypertarget{advanced:howdoiaddmyconstrainttothemodel}{}

Adding your constraint to the model requires you to definite a specific constraint manager (that can be a inner class of your Constraint).
This manager need to implement:
\begin{lstlisting}
makeConstraint(Solver s, Variable[] vars, Object params, HashSet<String> options)
\end{lstlisting}
This method allows the Solver to create an instance of your constraint, with your parameters and Solver objects.

\begin{note}
If you create your constraint manager as an inner class, you must declare this class as \textbf{public and static}.
If you don't, the solver can't instantiate your manager.
\end{note}

Once this manager has been implemented, you simply add your constraint to the model using the \texttt{addConstraint()} API with a \texttt{ComponentConstraint} object:
\begin{lstlisting}
  model.addConstraint( new ComponentConstraint(MyConstraintManager.class, params, vars) );
  // OR
  model.addConstraint( new ComponentConstraint("package.of.MyConstraint", params, vars) );
\end{lstlisting}
Where \emph{params} is whatever you want (\texttt{Object[], int, String},...) and \emph{vars} is an array of Model Variables (or more specific) objects.

\subsection{Example: implement and add the \texttt{IsOdd} constraint}
One creates the constraint by implementing the \texttt{AbstractUnIntSConstraint} (one integer variable) class:
\lstinputlisting{java/isodd.j2t}

To add the constraint to the model, one creates the following class (or inner class):
\lstinputlisting{java/isoddmanager.j2t}
It calls the constructor of the constraint, with every \emph{vars}, \emph{params} and \emph{options} needed.

Then, the constraint can be added to a model as follows:
\begin{lstlisting}
	// Creation of the model
	Model m = new CPModel();
	
	// Declaration of the variable
	IntegerVariable aVar = Choco.makeIntVar("a_variable", 0, 10);
	
	// Adding the constraint to the model, 1st solution:
	m.addConstraint(new ComponentConstraint(IsOddManager.class, null, new IntegerVariable[]{aVar}));
	// OR 2nd solution:
	m.addConstraint(new ComponentConstraint("myPackage.Constraint.IsOddManager", null, new IntegerVariable[]{aVar}));
	
	Solver s = new CPSolver();
	s.read(m);
	s.solve();
\end{lstlisting}
And that's it!!

\subsection{Example of an empty constraint}\label{advanced:anexempleofemptyconstraint}\hypertarget{advanced:anexempleofemptyconstraint}{}

%See the complete code: \href{media/zip/constraintpattern.zip}{ConstraintPattern.zip}

\begin{lstlisting}
  public class ConstraintPattern extends AbstractLargeIntSConstraint {
      
      public ConstraintPattern(IntDomainVar[] vars) {
          super(vars);
      }
	
      /**
      * pretty print. The String is not constant and may depend on the context.
      * @return a readable string representation of the object
      */
      public String pretty() {
          return null;
      }
	
      /**
      * check whether the tuple satisfies the constraint
      * @param tuple values
      * @return true if satisfied
      */
      public boolean isSatisfied(int[] tuple) {
          return false;
      }

      /**
      * propagate until local consistency is reached
      */
      public void propagate() throws ContradictionException {
          // elementary method to implement
      }
	    
      /**
      * propagate for the very first time until local consistency is reached.
      */
      public void awake() throws ContradictionException {
          constAwake(false);        // change if necessary
      }
	
	
      /**
      * default propagation on instantiation: full constraint re-propagation
      * @param var index of the variable to reduce
      */
      public void awakeOnInst(int var) throws ContradictionException {
          constAwake(false);        // change if necessary
      }
	
      /**
      * default propagation on improved lower bound: propagation on domain revision
      * @param var index of the variable to reduce
      */
      public void awakeOnInf(int var) throws ContradictionException {
          constAwake(false);        // change if necessary
      }
	
	
      /**
      * default propagation on improved upper bound: propagation on domain revision
      * @param var index of the variable to reduce
      */
      public void awakeOnSup(int var) throws ContradictionException {
          constAwake(false);        // change if necessary
      }
	
      /**
      * default propagation on improve bounds: propagation on domain revision
      * @param var index of the variable to reduce
      */
      public void awakeOnBounds(int var) throws ContradictionException {
          constAwake(false);        // change if necessary
      }
	
      /**
      * default propagation on one value removal: propagation on domain revision
      * @param var index of the variable to reduce
      * @param val the removed value
      */
      public void awakeOnRem(int var, int val) throws ContradictionException {
          constAwake(false);        // change if necessary
      }
	
      /**
      * default propagation on one value removal: propagation on domain revision
      * @param var index of the variable to reduce
      * @param delta iterator over remove values
      */
      public void awakeOnRemovals(int var, IntIterator delta) throws ContradictionException {
          constAwake(false);        // change if necessary
      }
  }
\end{lstlisting}

The first step to create a constraint in Choco is to implement all \texttt{awakeOn...} methods with \texttt{constAwake(false)} and to put your propagation algorithm in the \texttt{propagate()} method. 

A constraint can choose not to react to fine grained events such as the removal of a value of a given variable but instead delay its propagation at the end of the fix point reached by ``fine grained events'' and fast constraints that deal with them incrementally (that's the purpose of the constraints events queue). 

To do that, you can use \texttt{constAwake(false)} that tells the solver that you want this constraint to be called only once the variables events queue is empty. This is done so that heavy propagators can delay their action after the fast one to avoid doing a heavy processing at each single little modification of domains.

\section{Define your own operator}\label{advanced:defineyourownoperator}\hypertarget{advanced:defineyourownoperator}{}
%\todo{to complete}

%%%%%%%%%%%%%%%%%%
%%%%%%%%%%%%%%%%%%
%%%%%%%%%%%%%%%%%%
%%%%%%%%%%%%%%%%%%
%%%%%%%%%%%%%%%%%%

There are 2 types of operators: \textbf{boolean} and \textbf{arithmetic}. These operators are based on integer variable and/or integer constants. 
Let's take 2 examples \textit{plus} and \textit{eq}.

The operator \textit{plus} is an arithmetic one (see  \mylst{PlusNode.java} class for details) that computes the sum of two variables (and/or constants). It extends  \mylst{INode} (it can be part of an expression object) and implements  \mylst{ArithmNode} (it can be evaluated). So, required services are:

\begin{itemize}
\item a constructor. The type of operator should be defined using \mylst{CUSTOM}.
\begin{lstlisting}
public PlusNode(INode[] subt) {
	super(subt, NodeType.CUSTOM);
}
\end{lstlisting}
\item  \mylst{pretty()} : a pretty print of the operator
\begin{lstlisting}
public String pretty() {
        return "("+subtrees[0].pretty()+" + "+subtrees[1].pretty()+")";
    }
\end{lstlisting}

\item  \mylst{eval(int[] tuple)} : evaluation of the operator with the given tuple. An arithmetic evaluation of the subtrees is done (based on the given tuple) to compute the sub expression before evaluating the current operator \textit{plus}. This allows tree-like representation of an expression.
\begin{lstlisting}
public int eval(int[] tuple) {
		return ((ArithmNode) subtrees[0]).eval(tuple) + ((ArithmNode) subtrees[1]).eval(tuple);
}
\end{lstlisting}
\end{itemize}


The operator \textit{eq} is a boolean one (see  \mylst{EqNode.java} class for more details) that checks if two variables (and/or constants) are equal. It extends  \mylst{AbstractBooleanNode} (that can be checked). So required services are:
\begin{itemize}
\item a constructor
\begin{lstlisting}
public EqNode(INode[] subt) {
        super(subt, NodeType.CUSTOM);
    }
\end{lstlisting}
\item  \mylst{pretty()} : a pretty print of the operator
\begin{lstlisting}
public String pretty() {
        return "("+subtrees[0].pretty()+"="+subtrees[1].pretty()+")";
}
\end{lstlisting} 

\item  \mylst{eval(int[] tuple)}: an expression checker, based on the given tuple. An arithmetic evaluation of the subtrees is done (based on the given tuple) in order to check the operator \textit{eq}.
\begin{lstlisting}
public boolean checkTuple(int[] tuple) {
		return ((ArithmNode) subtrees[0]).eval(tuple)
		        ==  ((ArithmNode) subtrees[1]).eval(tuple);
	}
\end{lstlisting}

\item  \mylst{extractConstraint(Solver s)} : extracts the corresponding constraint in intension constraint without reification.
\begin{lstlisting}
public SConstraint extractConstraint(Solver s) {
        IntDomainVar v1 = subtrees[0].extractResult(s);
		IntDomainVar v2 = subtrees[1].extractResult(s);
		return s.eq(v1,v2);
    }
\end{lstlisting}
\end{itemize}
Now let's see how to use this operator in a Model.

To do that, create your own manager implementing \mylst{ExpressionManager}, which makes the link between the model and the solver.

Then to use your operator in your model, you can define a static method to simplify the calls.

Let's sum up in a short example based on the \textit{plus} operator. 

The manager would be:
\begin{lstlisting}
public class PlusManager implements ExpressionManager {
    @Override
    public INode makeNode(Solver solver, Constraint[] cstrs, Variable[] vars) {
        if(solver instanceof CPSolver){
            CPSolver s = (CPSolver)solver;
            if(vars.length == 1){
                INode[] nodes = new INode[vars.length];
                for(int v = 0; v < vars.length; v++){
                    nodes[v] = vars[v].getExpressionManager().makeNode(s, vars[v].getConstraints(), vars[v].getVariables());
                }
                return new PlusNode(nodes);
            }
        }
        throw new ModelException("Could not found a node manager in " + this.getClass() + " !");
    }
}
\end{lstlisting}


A main class would be:
\begin{lstlisting}
public class Sandbox {

    public static void main(String[] args) {
        model1();
    }

    public static IntegerExpressionVariable plus(IntegerVariable x){
        return new IntegerExpressionVariable(null, "package.of.PlusManager", x);
    }

    private static void model1() {
        Model m = new CPModel();
        IntegerVariable x = Choco.makeIntVar("x", 0, 5);
        IntegerVariable y = Choco.makeIntVar("y", 4, 8);
        IntegerVariable z = Choco.makeIntVar("z", 0, 10);
	
        // declare an expression using my operator
        IntegerExpressionVariable xx = plus(x,y);

        // and use it in common constraint
        m.addConstraint(Choco.eq(z, xx));

        Solver s = new CPSolver();
        s.read(m);

        ChocoLogging.toSolution();
        s.solveAll();
    }
}
\end{lstlisting}

Keep in mind that you can not define operators for set and real. 

%%%%%%%%%%%%%%%%%%
%%%%%%%%%%%%%%%%%%
%%%%%%%%%%%%%%%%%%
%%%%%%%%%%%%%%%%%%
%%%%%%%%%%%%%%%%%%

\section{Define your own variable}\label{advanced:defineyourownvariable}\hypertarget{advanced:defineyourownvariable}{}
\todo{to complete}

\section{Model and Solver detectors}\label{advanced:detectors}\hypertarget{advanced:detectors}{}

Sometimes, on automatic code generation or during benchmarking, it could be useful to apply generic rules to analyze a \mylst{Model} and detect lacks of modeling and apply patchs. With Choco, this is possible using \mylst{ModelDetectorFactory} and \mylst{PreProcessCPSolver}.

\subsection{Model detector}\label{advanced:modeldetector}\hypertarget{advanced:modeldetector}{}

The analysis of a \mylst{Model} is done using the \mylst{ModelDetectorFactory}. First one declares the list of rules to apply, then they are applied to a specific model. 
This is done by using the follwing API:
\begin{lstlisting}
ModelDetectorFactory.run(CPModel model, AbstractDetector... detectors)
\end{lstlisting}
An \mylst{AbstractDetector} object describes the pattern to detect within the model and rules to apply. Applying a rule is commonly to refactor a given model, by adding or removing variables and constraints. 

\begin{note}
Calling \mylst{ModelDetectorFactory.run(..)} will produce a modified copy of the current model. 
Very few \mylst{AbstractDetector} just analyze the model, without any side effects.
\end{note}

\textbf{Detectors}


\begin{itemize}
\item[] \mylst{AbstractDetector analysis(CPModel m)} 
Analyze the model \mylst{m}, and print out messages about general statistics: very large domain variables, inappropriate domain type variables, free variables (variables not involved in any constraints), etc. 
\item[] \mylst{AbstractDetector intVarEqDet(CPModel m)}
\item[] \mylst{AbstractDetector taskVarEqDet(CPModel m)}
\item[] \mylst{AbstractDetector cliqueDetector(CPModel m, boolean breakSymetries)}
\item[] \mylst{AbstractDetector disjFromCumulDetector(CPModel m)}
\item[] \mylst{AbstractDetector precFromImpliedDetector(CPModel m, DisjunctiveModel disjMod)}
\item[] \mylst{AbstractDetector precFromReifiedDetector(CPModel m, DisjunctiveModel disjMod)}
\item[] \mylst{AbstractDetector precFromTimeWindowDetector(CPModel m, DisjunctiveModel disjMod)}
\item[] \mylst{AbstractDetector precFromDisjointDetector(CPModel m, DisjunctiveModel disjMod)}
\item[] \mylst{AbstractDetector disjointDetector(CPModel m, DisjunctiveModel disjMod)}
\item[] \mylst{AbstractDetector disjointFromDisjDetector(CPModel m, DisjunctiveModel disjMod)}
\item[] \mylst{AbstractDetector disjointFromCumulDetector(CPModel m, DisjunctiveModel disjMod)}
\item[] \mylst{AbstractDetector rmDisjDetector(final CPModel m)}
\item[] \mylst{AbstractDetector[] disjunctiveModelDetectors(CPModel m, DisjunctiveModel disjMod)}
\item[] \mylst{AbstractDetector[] schedulingModelDetectors(final CPModel m, DisjunctiveModel disjMod)}
\end{itemize}



\begin{note}
Analysing a \mylst{Model} can be time consuming. It should be used carefully.
\end{note}
 
 

\todo{to complete}


\subsection{Preprocess solver}\label{advanced:preprocesssolver}\hypertarget{advanced:preprocesssolver}{}

One may want to keep the \mylst{Model} unchanged but apply detection directly on the \mylst{Solver}. This can be done during the \textit{reading} step: the model is analyses on the fly and the rules are applied directly on the solver. 

To do that, simply replaced the \mylst{CPSolver} declaration by:
\begin{lstlisting}
Solver solver = new PreProcessCPSolver(); // new CPSolver();
\end{lstlisting} 

The rules that will be applied can be selected by updating the \mylst{PreProcessConfiguration} file.
\todo{to complete}


\section{Logging statements}\label{advanced:loggingstatements}\hypertarget{advanced:loggingstatements}{}

\subsection{Define your own logger.}\label{advanced:defineyourownlogger}\hypertarget{advanced:defineyourownlogger}{}
\begin{lstlisting}
ChocoLogging.makeUserLogger(String suffix);
\end{lstlisting}


\subsection{How to write logging statements ?}\label{advanced:howtowriteloggingstatements}\hypertarget{advanced:howtowriteloggingstatements}{}

\begin{itemize}
	\item Critical Loggers are provided to display error or warning. Displaying too much message really \textbf{impacts the performances}.
	\item Check the logging level before creating arrays or strings.
	\item Avoid multiple calls to \texttt{Logger} functions. Prefer to build a \texttt{StringBuilder} then call the \texttt{Logger} function.
	\item Use the \texttt{Logger.log} function instead of building string in \texttt{Logger.info()}.
\end{itemize}

\subsection{Handlers.}\label{advanced:handlers}\hypertarget{advanced:handlers}{}
Logs are displayed on \texttt{System.out} but warnings and severe messages are also displayed on \texttt{System.err}.
\texttt{ChocoLogging.java} also provides utility functions to easily change handlers:
\begin{itemize}
	\item Functions \texttt{set...Handler} remove current handlers and replace them by a new handler.
	\item Functions \texttt{add...Handler} add new handlers but do not touch existing handlers.
\end{itemize}

% \subsubsection{Figure source}\label{advanced:figuresource}\hypertarget{advanced:figuresource}{}
% \begin{lstlisting}
%   digraph G {
%       node [style=filled, shape=box];
%       choco [shape=house,fillcolor=gold];
	
%       kernel [shape=house,fillcolor=gold];
%       engine [shape=octagon,fillcolor=indianred];
%       search [shape=octagon,fillcolor=darkorange];
%       branching [shape=octagon,fillcolor=indianred];
	
%       api [shape=house,fillcolor=gold];
%       model [fillcolor=indianred];
%       solver [fillcolor=indianred];
%       parser [fillcolor=darkorange];
      
%       user [fillcolor=darkorange];
%       samples [fillcolor=darkorange];
      
%       test [fillcolor=indianred];
      
%       choco -> kernel;
%       choco -> API;
%       choco -> user;
%       choco -> test;
      
%       kernel -> engine;
%       kernel -> search;
      
%       api -> model;
%       api -> solver
%       api -> parser;
      
%       user  -> samples;
      
%       search -> branching;
% 	}
% \end{lstlisting}
\label{doc:advanced}\hypertarget{doc:advanced}{}
\input{chapters/Choco_and_CP_Viz.tex}


\part{Elements of Choco}\label{part:elements}\hypertarget{part:elements}{}
\chapter{Variables (Model)}\label{ch:vars}\hypertarget{ch:vars}{}
This section describes the three kinds of \hyperlink{model:variables}{variables} that can be used within a Choco Model, and an object-variable.
\section{Integer variables}\label{integervariable}\hypertarget{integervariable}{}
\texttt{IntegerVariable} is a variable whose associated domain is made of integer values. 

\subsubsection{constructors:}
      \noindent\begin{tabular}{p{.8\linewidth}p{.15\linewidth}}
        Choco method & return type \\
        \hline
        \mylst{makeIntVar(String name, int lowB, int uppB, String... options)} &\texttt{IntegerVariable}\\
	\mylst{makeIntVar(String name, int[] values, String... options)} &\texttt{IntegerVariable}\\
	\mylst{makeIntVar(String name, List<Integer> values, String... options)} &\texttt{IntegerVariable}\\
	\mylst{makeIntVar(String name, TIntArrayList values, String... options)} &\texttt{IntegerVariable}\\
        \mylst{makeBooleanVar(String name, String... options)}  &\texttt{IntegerVariable}\\
        \mylst{makeIntVarArray(String name, int dim, int lowB, int uppB, String... options)} &\texttt{IntegerVariable[]}\\
        \mylst{makeIntVarArray(String name, int dim, int[] values, String... options)} &\texttt{IntegerVariable[]}\\
        \mylst{makeIntVarArray(String name, int dim, List<Integer> values, String... options)} &\texttt{IntegerVariable[]}\\
        \mylst{makeIntVarArray(String name, int dim, TIntArrayList values, String... options)} &\texttt{IntegerVariable[]}\\
        \mylst{makeBooleanVarArray(String name, int dim, String... options)}  &\texttt{IntegerVariable[]}\\
        \mylst{makeIntVarArray(String name, int dim1, int dim2, int lowB, int uppB, String... options)}  &\texttt{IntegerVariable[][]}\\
        \mylst{makeIntVarArray(String name, int dim1, int dim2, int[] values, String... options)}  &\texttt{IntegerVariable[][]}\\
        \mylst{makeIntVarArray(String name, int dim1, int dim2, List<Integer> values, String... options)}  &\texttt{IntegerVariable[][]}\\
         \mylst{makeIntVarArray(String name, int dim1, int dim2, TIntArrayList values, String... options)}  &\texttt{IntegerVariable[][]}\\
      \end{tabular}
% 	\begin{itemize}
% 		\item to create an \textbf{IntegerVariable} object:
% 		\begin{itemize}
% 			\item \mylst{makeIntVar(String name, int lowB, int uppB, String... options)}
% 			\item \mylst{makeIntVar(String name, List<Integer> values, String... options)}
% 			\item \mylst{makeIntVar(String name, int[] values, String... options)}
% 		\end{itemize}
% 		\item to create an \textbf{array of IntegerVariable} object:
% 		\begin{itemize}
% 			\item \mylst{makeIntVarArray(String name, int dim, int lowB, int uppB, String... options)}
% 			\item \mylst{makeIntVarArray(String name, int dim, int[] values, String... options)}
% 		\end{itemize}
% 		\item to create a \textbf{matrix of IntegerVariable} object:
% 		\begin{itemize}
% 			\item \mylst{makeIntVarArray(String name, int dim1, int dim2, int lowB, int uppB, String... options)}
% 			\item \mylst{makeIntVarArray(String name, int dim1, int dim2, int[] values, String... options)}
% 		\end{itemize}
% 		\item to create an \textbf{IntegerVariable} object with pre defined domain [0,1]:
% 		\begin{itemize}
% 			\item \mylst{makeBooleanVar(String name, String... options)}
% 		\end{itemize}
% 		\item to create an \textbf{array of IntegerVariable} object with pre defined domain [0,1]:
% 		\begin{itemize}
% 			\item \mylst{makeBooleanVarArray(String name, int dim, String... options)}
% 		\end{itemize}
% 	\end{itemize}
% 	\item \textbf{return type} : \texttt{IntegerVariable} \emph{or} \texttt{IntegerVariable[]} \emph{or} \texttt{IntegerVariable[][]}
\subsubsection{options:}
	\begin{itemize}
		\item \emph{no option} : equivalent to option \hyperlink{venum:venumoptions}{\tt Options.V\_ENUM}
		\item \hyperlink{venum:venumoptions}{\tt Options.V\_ENUM} : to force Solver to create enumerated domain for the variable.
		\item \hyperlink{vbound:vboundoptions}{\tt Options.V\_BOUND} : to force Solver to create bounded domain for the variable.
		\item \hyperlink{vlink:vlinkoptions}{\tt Options.V\_LINK} : to force Solver to create linked list domain for the variable.
		\item \hyperlink{vbtree:vbtreeoptions}{\tt Options.V\_BTREE} : to force Solver to create binary tree domain for the variable.
		\item \hyperlink{vblist:vblistoptions}{\tt Options.V\_BLIST} : to force Solver to create bipartite list domain for the variable.
		\item \hyperlink{vmakespan:vmakespanoptions}{\tt Options.V\_MAKEPSAN} : declare the current variable as makespan.
		\item \hyperlink{vnodecision:vnodecisionoptions}{\tt Options.V\_NO\_DECISION} : to force variable to be removed from the pool of decisional variables.
		\item \hyperlink{vobjective:vobjectiveoptions}{\tt Options.V\_OBJECTIVE} : to define the variable to be the one to optimize.
	\end{itemize}
\subsubsection{methods:}
      \begin{itemize}
      \item \mylst{removeVal(int val)}: remove value \emph{val} from the domain of the current variable
      \end{itemize}

A variable with $\{0,1\}$ domain is automatically considered as boolean domain.

\subsubsection{Example:}
\lstinputlisting{java/vintegervariable.j2t}

Integer variables are illustrated on the \hyperlink{model:example1:nqueenschoco}{n-Queens problem}. 

\section{Real variables}\label{realvariable}\hypertarget{realvariable}{}
\texttt{RealVariable} is a variable whose associated domain is made of real values. Only enumerated domain is available for real variables. 

Such domain are memory consuming. In order to minimize the memory use and to have the precision you need, the model offers a way to set a precision (default value is 1.0e-6):
\lstinputlisting{java/vprecision.j2t}

\subsubsection{constructor:}
      \noindent\begin{tabular}{p{.8\linewidth}p{.15\linewidth}}
        Choco method & return type \\
        \hline
        \mylst{makeRealVar(String name, double lowB, double uppB, String... options)} &\texttt{RealVariable}\\
      \end{tabular}
%	\begin{itemize}
%		\item to create a \textbf{RealVariable} object:
%		\begin{itemize}
%			\item \mylst{makeRealVar(String name, double lowB, double uppB, String... options)}
%		\end{itemize}
%	\end{itemize}
%	\item \textbf{return type} : \texttt{RealVariable}
\subsubsection{options:}
	\begin{itemize}
		\item \emph{no option} : no particular choice on decision or objective.
		\item \hyperlink{vnodecision:vnodecisionoptions}{\tt Options.V\_NO\_DECISION} : to force variable to be removed from the pool of decisional variables.
		\item \hyperlink{vobjective:vobjectiveoptions}{\tt Options.V\_OBJECTIVE} : to define the variable to be the one to optimize.
	\end{itemize}

\subsubsection{Example:}
\lstinputlisting{java/vrealvariable.j2t}

Real variables are illustrated on the \hyperlink{model:example3:thecyclohexaneproblemwithchoco}{CycloHexan problem}. 

\section{Set variables}\label{setvariable}\hypertarget{setvariable}{}
\texttt{SetVariable} is high level modeling tool. It allows to represent variable whose values are sets. A SetVariable on integer values between $[1,n]$ has $2^{n}$ values (every possible subsets of $\{1..n\}$). This makes an exponential number of values and the domain is represented with two bounds corresponding to the intersection of all possible sets (called the kernel) and the union of all possible sets (called the envelope) which are the possible candidate values for the variable. The consistency achieved on SetVariables is therefore a kind of bound consistency.

\subsubsection{constructors:}
      \noindent\begin{tabular}{p{.8\linewidth}p{.15\linewidth}}
        Choco method & return type \\
        \hline
        \mylst{makeSetVar(String name, int lowB, int uppB, String... options)} &\texttt{SetVariable}\\
        \mylst{makeSetVarArray(String name, int dim, int lowB, int uppB, String... options)} &\texttt{SetVariable[]}
      \end{tabular}
%	\begin{itemize}
%		\item to create an \textbf{SetVariable} object:
%		\begin{itemize}
%			\item \mylst{makeSetVar(String name, int lowB, int uppB, String... options)}
%		\end{itemize}
%		\item to create an \textbf{array of SetVariable} object:
%		\begin{itemize}
%			\item \mylst{makeSetVarArray(String name, int dim, int lowB, int uppB, String... options)}
%		\end{itemize}
%	\end{itemize}
%	\item \textbf{return type} : \texttt{SetVariable} \emph{or} \texttt{SetVariable[]}
\subsubsection{options:}
	\begin{itemize}
		\item \emph{no option} : equivalent to option \hyperlink{venum:venumoptions}{\tt Options.V\_ENUM}
		\item \hyperlink{venum:venumoptions}{\tt Options.V\_ENUM} : to force Solver to create \texttt{SetVariable} with enumerated domain for the caridinality variable.
		\item \hyperlink{vbound:vboundoptions}{\tt Options.V\_BOUND} : to force Solver to create \texttt{SetVariable} with bounded cardinality.
		\item \hyperlink{vnodecision:vnodecisionoptions}{\tt Options.V\_NO\_DECISION} : to force variable to be removed from the pool of decisional variables.
		\item \hyperlink{vobjective:vobjectiveoptions}{\tt Options.V\_OBJECTIVE} : to define the variable to be the one to optimize.
	\end{itemize}

The variable representing the cardinality can be accessed and constrained using method \texttt{getCard()} that returns an \hyperlink{integervariable}{\tt IntegerVariable} object.

\subsubsection{Example:}
\lstinputlisting{java/vsetvariable.j2t}

Set variables are illustrated on the \hyperlink{model:example2:ternarysteinerchoco}{ternary Steiner problem}. 



\section{Task variables}\label{taskvariable}\hypertarget{taskvariable}{}
\texttt{TaskVariable} is an object-variable composed of three \hyperlink{integervariable}{\tt IntegerVariable}: a starting time integer variable $start$, an ending time integer variable $end$ and a duration integer variable $duration$. To create a \texttt{TaskVariable}, one can creates the $start$, $end$ and $duration$ before, or indicates the earliest starting time ($int$), the latest completion time ($int$) and the duration({\tt int} or {\tt IntegerVariable}).

\subsubsection{constructors:}
      \noindent\begin{tabular}{p{.8\linewidth}p{.15\linewidth}}
        Choco method & return type \\
        \hline
        \mylst{makeTaskVar(String name, IntegerVariable start, IntegerVariable end, IntegerVariable duration, String... options)} &\texttt{TaskVariable}\\
		\mylst{makeTaskVar(String name, IntegerVariable start, IntegerVariable duration, String... options)} &\texttt{TaskVariable}\\
		\mylst{makeTaskVar(String name, int binf, int bsup, IntegerVariable duration, String... options)} &\texttt{TaskVariable}\\
        \mylst{makeTaskVar(String name, int binf, int bsup, int duration, String... options)}  &\texttt{TaskVariable}\\
        \mylst{makeTaskVar(String name, int bsup, IntegerVariable duration, String... options)} &\texttt{TaskVariable}\\
        \mylst{makeTaskVar(String name, int bsup, int duration, String... options)} &\texttt{TaskVariable}\\
        \mylst{makeTaskVarArray(String prefix, IntegerVariable[] starts, IntegerVariable[] ends, IntegerVariable[] durations, String... options)}  &\texttt{TaskVariable[]}\\
        \mylst{makeTaskVarArray(String name,  int binf, int bsup, IntegerVariable[] durations, String... options)}  &\texttt{TaskVariable[]}\\
        \mylst{makeTaskVarArray(String name,  int binf, int bsup, int[] durations, String... options)}  &\texttt{TaskVariable[][]}\\
        \mylst{makeTaskVarArray(String name,  int binf, int bsup, IntegerVariable[][] durations, String... options) }  &\texttt{TaskVariable[][]}\\
        \mylst{makeTaskVarArray(String name,  int binf, int bsup, int[][] durations, String... options)}  &\texttt{TaskVariable[][]}\\
      \end{tabular}

\subsubsection{options:}
Options are for the three \texttt{IntegerVariable}. See \hyperlink{integervariable}{\tt IntegerVariable} for more details about options.

\subsubsection{Example:}
\lstinputlisting{java/vtaskvariable.j2t} 

\chapter{Operators (Model)}\label{ch:operators}\hypertarget{ch:operators}{}
This section lists and details the \hyperlink{model:expressionvariables}{operators} that can be used within a Choco Model to combine variables in expressions.
\section{abs (operator)}\label{abs:absoperator}\hypertarget{abs:absoperator}{}
Returns an expression variable that represents the absolute value of the argument (\(|n|\)).

\begin{itemize}
	\item \textbf{API} : \mylst{abs(IntegerExpressionVariable n)}
	\item \textbf{return type} : IntegerExpressionVariable
	\item \textbf{options} : \emph{n/a}
	\item \textbf{favorite domain} : unknown
\end{itemize}

\textbf{Example}:
\lstinputlisting{java/oabs.j2t}

%%% Local Variables: 
%%% mode: latex
%%% TeX-master: t
%%% End: 

%\part{cos}
\label{cos}
\hypertarget{cos}{}

\section{cos (operator)}\label{cos:cosoperator}\hypertarget{cos:cosoperator}{}
Returns an expression variable corresponding to the cosinus value of the argument (\(cos(x)\)).

\begin{itemize}
	\item \textbf{API} : \mylst{cos(RealExpressionVariable exp)}
	\item \textbf{return type} : RealExpressionVariable
	\item \textbf{options} : \emph{n/a}
	\item \textbf{favorite domain} : real
\end{itemize}

\textbf{Example}:
\lstinputlisting{java/ocos.j2t}

%\part{div}
\label{div}
\hypertarget{div}{}

\section{div (operator)}\label{div:divoperator}\hypertarget{div:divoperator}{}
Returns an expression variable that represents the \(integer\) \(quotient\) of the division of the first argument variable by the second one (\(n_1/n_2\)).

\begin{itemize}
	\item \textbf{API} :
	\begin{itemize}
		\item \mylst{div(IntegerExpressionVariable n1, IntegerExpressionVariable n2)}
		\item \mylst{div(IntegerExpressionVariable n1, int n2)}
		\item \mylst{div(int n1, IntegerExpressionVariable n2)}
	\end{itemize}
	\item \textbf{return type} : IntegerExpressionVariable
	\item \textbf{options} : \emph{n/a}
	\item \textbf{favorite domain} : \emph{n/a}
\end{itemize}

\textbf{Example}:
\lstinputlisting{java/odiv.j2t}


\section{ifThenElse (operator)}\label{ifthenelse:ifthenelseoperator}\hypertarget{ifthenelse:ifthenelseoperator}{}
\texttt{ifThenElse}$(c,v_1,v_2)$ states that if the constraint $c$ is satisfied, it returns the second parameter $v_1$, otherwise it returns the third one $v_2$.

\begin{itemize}
	\item \textbf{API} : \mylst{ifThenElse(Constraint c, IntegerExpressionVariable v1, IntegerExpressionVariable v2)}
	\item \textbf{return type} : IntegerExpressionVariable
	\item \textbf{options} : \emph{n/a}
	\item \textbf{favorite domain} : unknown
\end{itemize}

\textbf{Example}:
\lstinputlisting{java/oifthenelse.j2t}

\section{max (operator)}\label{max:maxoperator}\hypertarget{max:maxoperator}{}
Returns an expression variable equals to the greater value of the argument (\(max(x_1, x_2, ..., x_n)\)).

\begin{itemize}
	\item \textbf{API} :
	\begin{itemize}
		\item \mylst{max(IntegerExpressionVariable x1, IntegerExpressionVariable x2)}
		\item \mylst{max(int x1, IntegerExpressionVariable x2)}
		\item \mylst{max(IntegerExpressionVariable x1, int x2)}
		\item \mylst{max(IntegerExpressionVariable[] x)}
	\end{itemize}
	\item \textbf{return type}: IntegerExpressionVariable
	\item \textbf{options} : \emph{n/a}
	\item \textbf{favorite domain} : \emph{to complete}
\end{itemize}

\textbf{Example}:
\lstinputlisting{java/omax.j2t}

%%% Local Variables: 
%%% mode: latex
%%% TeX-master: t
%%% End: 


\section{min (operator)}\label{min:minoperator}\hypertarget{min:minoperator}{}
Returns an expression variable equals to the smaller value of the argument (\(min(x_1, x_2, ..., x_n)\)).

\begin{itemize}
	\item \textbf{API} :
	\begin{itemize}
		\item \mylst{min(IntegerExpressionVariable x1, IntegerExpressionVariable x2)}
		\item \mylst{min(int x1, IntegerExpressionVariable x2)}
		\item \mylst{min(IntegerExpressionVariable x1, int x2)}
		\item \mylst{min(IntegerExpressionVariable[] x)}
	\end{itemize}
	\item \textbf{return type}: IntegerExpressionVariable
	\item \textbf{options} : \emph{n/a}
	\item \textbf{favorite domain} : \emph{to complete}
\end{itemize}

\textbf{Example}:
\lstinputlisting{java/omin.j2t}

%%% Local Variables: 
%%% mode: latex
%%% TeX-master: t
%%% End: 

%\part{minus}
\label{minus}
\hypertarget{minus}{}

\section{minus (operator)}\label{minus:minusoperator}\hypertarget{minus:minusoperator}{}
Returns an expression variable that corresponding to the difference between the two arguments (\(x-y\)).

\begin{itemize}
	\item \textbf{API} :
	\begin{itemize}
		\item \mylst{minus(IntegerExpressionVariable x, IntegerExpressionVariable y)}
		\item \mylst{minus(IntegerExpressionVariable x, int y)}
		\item \mylst{minus(int x, IntegerExpressionVariable y)}
		\item \mylst{minus(RealExpressionVariable x, RealExpressionVariable y)}
		\item \mylst{minus(RealExpressionVariable x, double y)}
		\item \mylst{minus(double x, RealExpressionVariable y)}
	\end{itemize}
	\item \textbf{return type} :
	\begin{itemize}
		\item \texttt{IntegerExpressionVariable}, if parameters are \texttt{IntegerExpressionVariable}
		\item \texttt{RealExpressionVariable}, if parameters are \texttt{RealExpressionVariable}
	\end{itemize}
	\item \textbf{options} : \emph{n/a}
	\item \textbf{favorite domain} : \emph{to complete}
\end{itemize}

\textbf{Example}
\lstinputlisting{java/ominus.j2t}


\section{mod (operator)}\label{mod:modoperator}\hypertarget{mod:modoperator}{}
Returns an expression variable that represents the integer remainder of the division of the first argument variable by the second one (\(x_1\%x_2\)).

\begin{itemize}
	\item \textbf{API}:
	\begin{itemize}
		\item \mylst{mod(IntegerExpressionVariable x1, IntegerExpressionVariable x2)}
		\item \mylst{mod(int x1, IntegerExpressionVariable x2)}
		\item \mylst{mod(IntegerExpressionVariable x1, int x2)}
	\end{itemize}
	\item \textbf{return type} : \texttt{IntegerExpressionVariable}
	\item \textbf{options} : \emph{n/a}
	\item \textbf{favorite domain} : \emph{n/a}
\end{itemize}

\textbf{Example}:
\lstinputlisting{java/omod.j2t}

%%% Local Variables: 
%%% mode: latex
%%% TeX-master: t
%%% End: 

%\part{mult}
\label{mult}
\hypertarget{mult}{}

\section{mult (operator)}\label{mult:multoperator}\hypertarget{mult:multoperator}{}
Returns an expression variable that corresponding to the product of variables in argument (\(x*y\)).

\begin{itemize}
	\item \textbf{API} :
	\begin{itemize}
		\item \mylst{mult(IntegerExpressionVariable x, IntegerExpressionVariable y)}
		\item \mylst{mult(IntegerExpressionVariable x, int y)}
		\item \mylst{mult(int x, IntegerExpressionVariable y)}
		\item \mylst{mult(RealExpressionVariable x, RealExpressionVariable y)}
		\item \mylst{mult(RealExpressionVariable x, double y)}
		\item \mylst{mult(double x, RealExpressionVariable y)}
	\end{itemize}
	\item \textbf{return type} :
	\begin{itemize}
		\item \texttt{IntegerExpressionVariable}, if parameters are \texttt{IntegerExpressionVariable}
		\item \texttt{RealExpressionVariable}, if parameters are \texttt{RealExpressionVariable}
	\end{itemize}
	\item \textbf{options} : \emph{n/a}
	\item \textbf{favorite domain} : \emph{to complete}
\end{itemize}

\textbf{Example}
\lstinputlisting{java/omult.j2t}

%\part{neg}
\label{neg}
\hypertarget{neg}{}

\section{neg (operator)}\label{neg:negoperator}\hypertarget{neg:negoperator}{}

Returns an expression variable that is the opposite of the expression integer variable in argument (\(-x\)).

\begin{itemize}
	\item \textbf{API} : \mylst{neg(IntegerExpressionVariable x)}
	\item \textbf{return type} : \texttt{IntegerExpressionVariable}
	\item \textbf{options} : \emph{n/a}
	\item \textbf{favorite domain} : \emph{n/a}
\end{itemize}

\textbf{Example}:
\lstinputlisting{java/oneg.j2t}

%\part{plus}
\label{plus}
\hypertarget{plus}{}



\section{plus (operator)}\label{plus:plusoperator}\hypertarget{plus:plusoperator}{}
Returns an expression variable that corresponding to the sum of the two arguments (\(x+y\)).

\begin{itemize}
	\item \textbf{API} :
	\begin{itemize}
		\item \mylst{plus(IntegerExpressionVariable x, IntegerExpressionVariable y)}
		\item \mylst{plus(IntegerExpressionVariable x, int y)}
		\item \mylst{plus(int x, IntegerExpressionVariable y)}
		\item \mylst{plus(RealExpressionVariable x, RealExpressionVariable y)}
		\item \mylst{plus(RealExpressionVariable x, double y)}
		\item \mylst{plus(double x, RealExpressionVariable y)}
	\end{itemize}
	\item \textbf{return type} :
	\begin{itemize}
		\item \texttt{IntegerExpressionVariable}, if parameters are \texttt{IntegerExpressionVariable}
		\item \texttt{RealExpressionVariable}, if parameters are \texttt{RealExpressionVariable}
	\end{itemize}
	\item \textbf{options} : \emph{n/a}
	\item \textbf{favorite domain} : \emph{to complete}
\end{itemize}

\textbf{Example}
% \begin{itemize}
% 	\item example1:
% \end{itemize}

\lstinputlisting{java/oplus.j2t}


%\part{power}
\label{power}
\hypertarget{power}{}

\section{power (operator)}\label{power:poweroperator}\hypertarget{power:poweroperator}{}
Returns an expression variable that represents the first argument raised to the power of the second argument (\(x^y\)).

\begin{itemize}
	\item \textbf{API} :
	\begin{itemize}
		\item \mylst{power(IntegerExpressionVariable x, IntegerExpressionVariable y)}
		\item \mylst{power(int x, IntegerExpressionVariable y)}
		\item \mylst{power(IntegerExpressionVariable x, int y)}
		\item \mylst{power(RealExpressionVariable x, int y)}
	\end{itemize}
	\item \textbf{return type}:
	\begin{itemize}
		\item \texttt{IntegerExpressionVariable}, if parameters are \texttt{IntegerExpressionVariable}
		\item \texttt{RealExpressionVariable}, if parameters are \texttt{RealExpressionVariable}
	\end{itemize}
	\item \textbf{option} : \emph{n/a}
	\item \textbf{favorite domain} : \emph{to complete}
\end{itemize}

\textbf{Example} : 
\lstinputlisting{java/opower.j2t}

%\part{scalar}
\label{scalar}
\hypertarget{scalar}{}

\section{scalar (operator)}\label{scalar:scalaroperator}\hypertarget{scalar:scalaroperator}{}
Return an integer expression that corresponds to the scalar product of coefficients array and variables array (\(c_1*x_1+c_2*x_2+...+c_n*x_n\)).

\begin{itemize}
	\item \textbf{API} :
	\begin{itemize}
		\item \mylst{scalar(int[] c, IntegerVariable[] x)}
		\item \mylst{scalar(IntegerVariable[] x, int[] c)}
	\end{itemize}
	\item \textbf{return type} : IntegerExpressionVariable
	\item \textbf{options} : \emph{n/a}
	\item \textbf{favorite domain} : \emph{to complete}
\end{itemize}

\textbf{Example}:

\lstinputlisting{java/oscalar.j2t}

%\part{sin}
\label{sin}
\hypertarget{sin}{}

\section{sin (operator)}\label{sin:sinoperator}\hypertarget{sin:sinoperator}{}
Returns a real variable that corresponding to the sinus value of the argument (\(sin(x)\)).

\begin{itemize}
	\item \textbf{API} : \mylst{sin(RealExpressionVariable exp)}
	\item \textbf{return type} : RealExpressionVariable
	\item \textbf{options} : \emph{n/a}
	\item \textbf{favorite domain} : real
\end{itemize}

\textbf{Example}:
\lstinputlisting{java/osin.j2t}

%\part{sum}
\label{sum}
\hypertarget{sum}{}

\section{sum (operator)}\label{sum:sumoperator}\hypertarget{sum:sumoperator}{}
Return an integer expression that corresponds to the sum of the variables given in argument (\(x_1+x_2+...+x_n\)).

\begin{itemize}
	\item \textbf{API}: \mylst{sum(IntegerVariable... lv)}
	\item \textbf{return type} : IntegerExpressionVariable
	\item \textbf{options} : \emph{n/a}
	\item \textbf{favorite domain} : \emph{to complete}
\end{itemize}

\textbf{Example} :
\lstinputlisting{java/osum.j2t}

%
\section{TRUE (operator)}\label{true:trueoperator}\hypertarget{true:trueoperator}{}
Returns an expression always equals to \emph{true}.

%%% Local Variables: 
%%% mode: latex
%%% TeX-master: t
%%% End: 

\chapter{Constraints (Model)}\label{ch:constraints}\hypertarget{ch:constraints}{}
This section lists and details the \hyperlink{model:constraints}{constraints} currently available in Choco.
%\part{abs}
%\label{abs}\hypertarget{abs}{}
\section{abs (constraint)}\label{abs:absconstraint}\hypertarget{abs:absconstraint}{}
\begin{notedef}
  \texttt{abs}$(x,y)$ states that $x$ is the absolute value of $y$:
$$x = |y|$$
\end{notedef}

\begin{itemize}
	\item \textbf{API} : \mylst{abs(IntegerVariable x, IntegerVariable y)}
	\item \textbf{return type} : \texttt{Constraint}
	\item \textbf{options} : \emph{n/a}
	\item \textbf{favorite domain} : enumerated
\end{itemize}

\textbf{Example}:
\lstinputlisting{java/cabs.j2t}


%\part{alldifferent}
\label{alldifferent}
\hypertarget{alldifferent}{}

\section{allDifferent (constraint)}\label{alldifferent:alldifferentconstraint}\hypertarget{alldifferent:alldifferentconstraint}{}
\begin{notedef}
  \texttt{allDifferent}$(\collec{x_1}{x_n})$ states that the arguments have pairwise distinct values:
 $$x_i \neq x_j,\quad \forall\ i\neq j$$  
\end{notedef}
This constraint is the basis of any matching problems.
Notice that the filtering algorithm (AC~\cite{ReginAAAI94} or BC~\cite{LopezIJCAI03}) depends on the nature (enumerated or bounded) of variables $x$. 

\begin{itemize}
	\item \textbf{API} :
	\begin{itemize}
		\item \mylst{allDifferent(IntegerVariable... x)}
		\item \mylst{allDifferent(String options, IntegerVariable... x)}
	\end{itemize}
	\item \textbf{return type} : \texttt{Constraint}
	\item \textbf{options} :
	\begin{itemize}
		\item \emph{no option}: if the domains of $x$ are \emph{enumerated}, the constraint refers to the alldifferent of \cite{ReginAAAI94};
if they are \emph{bounded}, a dedicated algorithm~\cite{LopezIJCAI03} for bound propagation is used instead. 
%the  clever choice made on domains of given variables
		\item \hyperlink{calldiffac:calldiffacoptions}{\tt Options.C\_ALLDIFFERENT\_AC} for \cite{ReginAAAI94} implementation of arc consistency
		\item \hyperlink{calldiffbc:calldiffbcoptions}{\tt Options.C\_ALLDIFFERENT\_BC} for \cite{LopezIJCAI03} implementation of bound consistency
		\item \hyperlink{calldiffclique:calldiffcliqueoptions}{\tt Options.C\_ALLDIFFERENT\_CLIQUE} for propagating the clique of differences
	\end{itemize}
	\item \textbf{favorite domain} : \emph{enumerated} for arc consistency, \emph{bounded} for bound consistency.
	\item \textbf{references} :
      \begin{itemize}
      \item  \cite{ReginAAAI94}: \emph{A filtering algorithm for constraints of difference in CSPs}
      \item  \cite{LopezIJCAI03}: \emph{A fast and simple algorithm for bounds consistency of the alldifferent constraint}
      \item global constraint catalog: \href{http://www.emn.fr/x-info/sdemasse/gccat/Calldifferent.html}{\tt alldifferent}
      \end{itemize}
\end{itemize}



\textbf{Example}:
\lstinputlisting{java/calldifferent.j2t}

\label{among}
\hypertarget{among}{}

\section{among (constraint)}\label{among:amongconstraint}\hypertarget{among:amongconstraint}{}

% \subsection{among values}\label{among:amongvalues}\hypertarget{among:amongvalues}{}

% \begin{notedef}
%   \texttt{among}$(x, V)$ states that the variable $x$ takes its value in $V$:
%  $$x \subseteq V$$  
% \end{notedef}

% \begin{itemize}
% 	\item \textbf{API}: \mylst{among(IntegerVariable x, int[] v)}
% 	\item \textbf{return type}: \texttt{Constraint}
% \end{itemize}

% \textbf{Example}:
% \lstinputlisting{java/camong1.j2t}

%\subsection{among values with counter}\label{among:amongvaluescounter}\hypertarget{among:amongvaluescounter}{}

\begin{notedef}
\texttt{among}$(z, \collec{x_1}{x_n}, s)$ states that $z$ is the number of $x_i$ belonging to set $s$:
 $$ z = \vert\lbrace i\ |\ x_i \in s \rbrace\vert $$  
\end{notedef}

\begin{itemize}
\item \textbf{API}: 
  \begin{itemize}
  \item \mylst{among(IntegerVariable z, IntegerVariable[] x, int[] v)}
  \item \mylst{among(IntegerVariable z, IntegerVariable[] x, SetVariable s)}
  \end{itemize}
	\item \textbf{return type}: \texttt{Constraint}
	\item  \cite{Bessiere05ERCIM}: \emph{\textit{Among}, \textit{common} and \textit{disjoint} Constraints}
	\item  \cite{Bessiere06ERCIM}: \emph{\textit{Among}, \textit{common} and \textit{disjoint} Constraints}
	\item global constraint catalog: \href{http://www.emn.fr/x-info/sdemasse/gccat/Camong.html}{\tt among}
\end{itemize}

\textbf{Example}:

among with a collection of values
\lstinputlisting{java/camong2.j2t}

among with a set variable
\lstinputlisting{java/camong3.j2t}

%\part{and}
\label{and}
\hypertarget{and}{}

\section{and (constraint)}\label{and:andconstraint}\hypertarget{and:andconstraint}{}
\begin{notedef}
  \texttt{and}$(\collec{C_1}{C_n})$ states that constraints in arguments are all satisfied:
$$ C_1 \land C_2 \land\ldots\land C_n$$

  \texttt{and}$(\collec{b_1}{b_n})$ states that booleans in arguments are all true:
$$ (b_1=1)\ \land\ (b_2=1)\ \land\ \ldots\ \land\ (b_n=1)$$
\end{notedef}

\begin{itemize}
\item \textbf{API} : 
\begin{itemize}
\item \mylst{and(Constraint... c)}
\item \mylst{and(IntegerVariable... b)}
\end{itemize}
\item \textbf{return type} : \texttt{Constraint}
\item \textbf{options} : \emph{n/a}
\item \textbf{favorite domain} : \emph{n/a}
\item \textbf{references} :\\
  global constraint catalog: \href{http://www.emn.fr/x-info/sdemasse/gccat/Cand.html}{\tt and}
\end{itemize}

\textbf{Examples:}
\begin{itemize}
	\item example1:
\end{itemize}
\lstinputlisting{java/cand1.j2t}
\begin{itemize}
	\item example2
\end{itemize}

\lstinputlisting{java/cand2.j2t}


%\part{atmostnvalue}
\label{atmostnvalue}
\hypertarget{atmostnvalue}{}

\section{atMostNValue (constraint)}\label{atmostnvalue:atmostnvalueconstraint}\hypertarget{atmostnvalue:atmostnvalueconstraint}{}
\begin{notedef}
\texttt{atMostNValue}$(z, \collec{x_1}{x_n})$ states that the number of distinct values occurring in collection $x$ is at most $z$:
$$z\ge|\collec{x_1}{x_n}|$$  
\end{notedef}

\begin{itemize}
	\item \textbf{API} : \mylst{atMostNValue(IntegerVariable z, IntegerVariable[] x)}
	\item \textbf{return type} : \texttt{Constraint}
	\item \textbf{options} : \emph{n/a}
	\item \textbf{favorite domain} : \emph{n/a}
	\item \textbf{references} :
      \begin{itemize}
      \item  \cite{BessiereCPAIOR05} \emph{Filtering algorithms for the NValue constraint}
      \item global constraint catalog: \href{http://www.emn.fr/x-info/sdemasse/gccat/Catmost_nvalue.html}{\tt atmost\_nvalue}
      \end{itemize}
    \end{itemize}

\textbf{Example}:
\lstinputlisting{java/catmostnvalue.j2t}

%\part{boolchanneling}
\label{boolchanneling}
\hypertarget{boolchanneling}{}

\section{boolChanneling (constraint)}\label{boolchanneling:boolchannelingconstraint}\hypertarget{boolchanneling:boolchannelingconstraint}{}
\begin{notedef}  
\texttt{boolChanneling}$(b,x,v)$ states that boolean $b$ is true if and only if $x$ has value $v$:
$$(b=1)\quad\iff\quad (x=v)$$ 
\end{notedef}

$b$ is an indicator variable acting as an observer of value $v$. 
See also \hyperlink{domainchanneling}{\texttt{domainChanneling}} for observing all the values of $x$.
%Imagine a bin packing problem where variable $x$ tells you on which a given bin object is placed. By stating the boolean channeling, $b$ is true if and only if the object is placed on bin $v$, the knapsack constraint for bin $v$ can then be easily stated as a scalar of the boolean variables.
\begin{itemize}
	\item \textbf{API} : \mylst{boolChanneling(IntegerVariable b, IntegerVariable x, int v)}
	\item \textbf{return type} : \texttt{Constraint}
	\item \textbf{options} : \emph{n/a}
	\item \textbf{favorite domain} : enumerated for $x$
\end{itemize}

\textbf{Example}:
\lstinputlisting{java/cboolchanneling.j2t}

\label{clause}
\hypertarget{clause}{}

\section{clause (constraint)}\label{clause:clauseconstraint}\hypertarget{clause:clauseconstraint}{}
\begin{notedef}
  \texttt{clause}$(\collec{b^+_1}{b^+_n}, \collec{b^-_1}{b^-_m})$ states that at least one boolean $b^+_i$ is true or one boolean $b^-_j$ is false.
% $$(x_1=1) \vee (x_2=1) \vee \dots \vee (x_n=1) \vee (y_1=0) \vee (y_2=0) \vee \dots \vee (y_m=0)$$  
 $$\bigvee_{i=1}^n (b^+_i=1) \vee \bigvee_{j=1}^m (b^-_j=0)$$
\end{notedef}

\begin{itemize}
	\item \textbf{API} :
	\begin{itemize}
		\item \mylst{clause(IntegerVariable[] bpos, IntegerVariable[] bneg)}
		\item \mylst{clause(String options, IntegerVariable[] bpos, IntegerVariable[] bneg)}
	\end{itemize}
	\item \textbf{return type} : \texttt{Constraint}
	\item \textbf{options} :
	\begin{itemize}
		\item \emph{no option} default filtering
		\item \hyperlink{cclause:cclauseoptions}{\tt Options.C\_CLAUSES\_ENTAIL} ensures quick entailment tests
	\end{itemize}
	\item \textbf{favorite domain} : \emph{n/a}.
	\item \textbf{references} : global constraint catalog: \href{http://www.emn.fr/x-info/sdemasse/gccat/Cclause_or.html}{\tt clause\_or}
\end{itemize}



\textbf{Example}:
\lstinputlisting{java/cclause.j2t}

\hypertarget{costregular}{}

\section{costRegular (constraint)}\label{costregular:costregularconstraint}\hypertarget{costregular:costregularconstraint}{}
\begin{notedef}
  \texttt{costRegular}$(z, \collec{x_1}{x_n},\mathcal{L}(\Pi), \coll{c_{i,j}})$ states that sequence \collec{x_1}{x_n} is a word belonging to the regular language $\mathcal{L}(\Pi)$ and that $z$ is its cost computed as the sum of the individual symbol weights $c_{i,x_i}$:
$$ \collec{x_1}{x_n} \in \mathcal{L}(\Pi)\quad\land\quad \sum_{i=1}^n c_{i,x_i} = z.$$
\end{notedef}

Like \hyperlink{regular}{\texttt{regular}}, this constraint is useful for modelling sequencing rules in personnel scheduling and rostering problems. Furthermore it allows to handle a linear counter (or cost) on the sequence. See \hyperlink{multicostregular}{\texttt{multiCostRegular}} to simultaneously handle several linear counters.

\texttt{costRegular} is the optimization variant of the \hyperlink{regular:regularconstraint}{\texttt{regular}} constraint. Enforcing GAC is NP-Hard, then the implemented algorithm~\cite{DemasseyC06} achieves an intermediate AC-BC level of consistency. Let $\mathcal{L}_x=\mathcal{L}(\Pi)\cap (D_1\times\ldots\times D_n)$ be the set of words of the language $\mathcal{L}(\Pi)$ that can be matched by \collec{x_1}{x_n} according to their current domains \collec{D_1}{D_n}, then:  
      \begin{itemize}
      \item Arc Consistency is enforced over $x$ regarding the language and the lower and upper bounds of $z$: for each value $v\in D_i$, there exists a word in $\mathcal{L}_x$, with $v$ as its $i$-th symbol and whose cost is between the bounds of $z$.
      \item Bound Consistency is enforced over $z$ regarding $x$ and the language: the lower and upper bounds of $z$ are set as the minimum and maximum costs of any words in $\mathcal{L}_x$.
      \end{itemize}
In summary, \texttt{costRegular}$(z, x,\mathcal{L}(\Pi), c)$ dominates its decomposition \hyperlink{regular}{\texttt{regular}}$(x,\mathcal{L}(\Pi))\land$\hyperlink{equation}{\texttt{equation}}$(z, x, c)$. Another decomposition proposed in~\cite{BeldiceanuC05} can easily be generated by introducing intermediary cost variables \collec{z_1}{z_n} and state variables \collec{q_0}{q_n}, then posting constraints in extension on each tuple $(q_{i-1},x_i,q_i,z_i)$ with the $\Pi$ transition table, and one linear sum $z=z_1+\cdots+z_n$. In terms of consistency, the two approaches are incomparable (words with costs out of the bounds of $z$ may not be filtered by the decomposition, see examples in~\cite{MenanaCPAIOR09}).  

Several API exists for defining the regular language:
\begin{itemize}
\item With a deterministic finite automaton (DFA) $\Pi$ weighted by a cost table $c$ with two dimensions, then $c[i][j]$ is the cost of any transition in $\Pi$ labeled by $j$ when processing the $i$-th symbol of a word: it models the cost of assigning variable $x_i$ to value $j$. The constraint ensures that $z=\sum_i c[i][x_i]$.
\item With a DFA $\Pi$ weighted by a cost table $c$ with three dimensions, then $c[i][j][s]$ is the cost of the transition in $\Pi$ outgoing from state $s$ and labeled by $j$ when processing the $i$-th symbol of a word: it models the cost of assigning variable $x_i$ to value $j$ if assignment sequence $\collec{x_1}{x_{i-1}}$ reaches state $s$ when processed by $\Pi$.  The constraint ensures that $z=\sum_i c[i][x_i][s_i]$ where $\collec{s_0, s_1}{s_n}$ is the sequence of states encountered when recognizing $\collec{x_1}{x_{n}}$ in $\Pi$.
\item With a weighted valued multi-graph $G(\Pi)$ and a node $s$, then $G(\Pi)$ must be a layered graph with $n+1$ layers and $s$ be the unique node in layer 0. Such a graph defines a valued DFA, by setting the arcs as the transitions, the arc values as the transition labels, the arc weights as the transition costs, the nodes as the states, and the nodes in the last layer as the accepting states. Note that this DFA recognizes only words of length $n$. The constraint ensures that $z$ is the total weight of the path in $G(\Pi)$ produced when recognizing \collec{x_1}{x_n}.  
\end{itemize}

Automaton $\Pi$ is encoded as an object of class \texttt{FiniteAutomaton}, whose API contains:
\begin{lstlisting}
  FiniteAutomaton();
  FiniteAutomaton(String regularExpression);
  int addState();
  void setInitialState(int state); 
  void setFinal(int state); 
  void addTransition(int state1, int state2, int.. labels);
  FiniteAutomaton union(FiniteAutomaton a);
  FiniteAutomaton intersection(FiniteAutomaton a);
  FiniteAutomaton complement();
  void minimize();
  int getNbStates();
  void toDotty(String dotFileName);
\end{lstlisting}

\begin{itemize}
	\item \textbf{API} :
	\begin{itemize}
		\item \mylst{costRegular(IntegerVariable z, IntegerVariable[] x, FiniteAutomaton pi, int[][] c)}\\
based on the  \mylst{ConstraintType.COSTREGULAR} implementation
		\item \mylst{costRegular(IntegerVariable z, IntegerVariable[] x, FiniteAutomaton pi, int[][][] c)}\\
based on the  \mylst{ConstraintType.FASTCOSTREGULAR} implementation
		\item \mylst{costRegular(IntegerVariable z, IntegerVariable[] x, DirectedMultigraph<Node,Arc> g, Node s)}\\
based on the  \mylst{ConstraintType.FASTCOSTREGULAR} implementation
	\end{itemize}
	\item \textbf{return type} : \texttt{Constraint}
	\item \textbf{options} : \emph{n/a}
	\item \textbf{favorite domain} : \emph{n/a}
	\item \textbf{references} : \cite{DemasseyC06}: \emph{A \texttt{Cost-Regular} based hybrid column generation approach}
\end{itemize}



\textbf{Example}:

Build the \texttt{FiniteAutomaton} manually by adding states and transitions:
\lstinputlisting{java/ccostregular.j2t}

Build the \texttt{FiniteAutomaton} from a combination of several regular expressions: 
\lstinputlisting{java/ccostregular2.j2t}

%\part{cumulative}
\label{cumulative}
\hypertarget{cumulative}{}

\section{cumulative (constraint)}\label{cumulative:cumulativeconstraint}\hypertarget{cumulative:cumulativeconstraint}{}
\todo{to be cleaned.}

\begin{notedef}
  \texttt{cumulative(start,duration,height,capacity)} states that a set of tasks (defined by their starting times, finishing dates, durations and heights (or consumptions)) are executed on a cumulative resource of limited capacity. That is, the total height of the tasks which are executed at any time $t$ does not exceed the capacity of the resource:
$$\sum_{\{i\ |\ \mathtt{start}[i]\le t < \mathtt{start}[i]+\mathtt{duration}[i]\}} \mathtt{height}[i] \le \mathtt{capacity},\quad (\forall \text{ time } t)$$
\end{notedef}

The notion of task does not exist yet in Choco. The \texttt{cumulative} takes therefore as input, several arrays of integer variables (of same size $n$) denoting the starting, duration, and height of each task. When the array of finishing times is also specified, the constraint ensures that \texttt{start[i] + duration[i] = end[i]} for all task $i$.
As usual, a task is executed in the interval \texttt{[start,end-1]}.

For further informations, see the section devoted to this constraint in the Choco Tutorial document. 
%A tutorial on the use of this constraint is available \hyperlink{schedulinganduseofthecumulative:schedulinganduseofthecumulativeconstraint}{here}

\begin{itemize}
	\item \textbf{API} :
	\begin{itemize}
		\item \mylst{cumulative(IntegerVariable[] start, IntegerVariable[] end, IntegerVariable[] duration, IntegerVariable[] height, IntegerVariable capa, String... options)}
		\item \mylst{cumulative(IntegerVariable[] start, IntegerVariable[] end, IntegerVariable[] duration, int[] height, int capa, String... options)}
		\item \mylst{cumulative(IntegerVariable[] start, IntegerVariable[] duration, IntegerVariable[] height, IntegerVariable capa, String... options)}
	\end{itemize}
	\item \textbf{return type} : \texttt{Constraint}
	\item \textbf{options} :
	\begin{itemize}
		\item \emph{no option}
		\item \hyperlink{ccumulativeti:ccumulativetioptions}{SettingType.TASK\_INTERVAL.getOptionName()} for fast task intervals
		\item \hyperlink{ccumulativesti:ccumulativestioptions}{SettingType.SLOW\_TASK\_INTERVAL.getOptionName()} for slow task intervals
		\item \hyperlink{ccumulativecef:ccumulativecefoptions}{SettingType.VILIM\_CEF\_ALGO.getOptionName()} for Vilim theta lambda tree + lazy computation of the inner maximization of the edge finding rule of Van hentenrick and Mercier
		\item \hyperlink{ccumulativescef:ccumulativescefoptions}{SettingType.VHM\_CEF\_ALGO\_N2K.getOptionName()} for Simple $n^2 * k$ algorithm (lazy for R) (CalcEF -- Van Hentenrick)
	\end{itemize}
	\item \textbf{favorite domain} : \emph{n/a}
	\item \textbf{references} :
      \begin{itemize}
      \item  \cite{BeldiceanuCP02} \emph{A new multi-resource cumulatives constraint with negative heights}
      \item global constraint catalog: \href{http://www.emn.fr/x-info/sdemasse/gccat/Ccumulative.html}{\tt cumulative}
      \end{itemize}
\end{itemize}

\textbf{Example}:
\lstinputlisting{java/ccumulative.j2t} 

\section{cumulativeMax (constraint)}\label{cumulativemax:cumulativemaxconstraint}\hypertarget{cumulativemax:cumulativemaxconstraint}{}
Specific case of \hyperlink{cumulative:cumulativeconstraint}{Cumulative}, where the \textbf{consumption} is equal to 0.

\begin{itemize}
	\item \textbf{API} :
	\begin{itemize}
		\item \mylst{cumulativeMax(String name, TaskVariable[] tasks, IntegerVariable[] heights, IntegerVariable[] usages, IntegerVariable capacity, String... options)}
		\item \mylst{cumulativeMax(String name, TaskVariable[] tasks, IntegerVariable[] heights, IntegerVariable capacity, String... options)}
		\item \mylst{cumulativeMax(TaskVariable[] tasks, int[] heights, int capacity, String... options)}
	\end{itemize}
	\item \textbf{return type} : \texttt{Constraint}
	\item \textbf{options} :
	\begin{itemize}
		\item \emph{no option}
		\item \hyperlink{ccumulativeti:ccumulativetioptions}{SettingType.TASK\_INTERVAL.getOptionName()} for fast task intervals
		\item \hyperlink{ccumulativesti:ccumulativestioptions}{SettingType.SLOW\_TASK\_INTERVAL.getOptionName()} for slow task intervals
		\item \hyperlink{ccumulativecef:ccumulativecefoptions}{SettingType.VILIM\_CEF\_ALGO.getOptionName()} for Vilim theta lambda tree + lazy computation of the inner maximization of the edge finding rule of Van hentenrick and Mercier
		\item \hyperlink{ccumulativescef:ccumulativescefoptions}{SettingType.VHM\_CEF\_ALGO\_N2K.getOptionName()} for Simple $n^2 * k$ algorithm (lazy for R) (CalcEF -- Van Hentenrick)
	\end{itemize}
	\item \textbf{favorite domain} : \emph{n/a}
\end{itemize}
\section{cumulativeMin (constraint)}\label{cumulativemin:cumulativeminconstraint}\hypertarget{cumulativemin:cumulativeminconstraint}{}
Specific case of \hyperlink{cumulative:cumulativeconstraint}{Cumulative}, where the \textbf{capacity} is infinite.

\begin{itemize}
	\item \textbf{API} :
	\begin{itemize}
		\item \mylst{cumulativeMin(String name, TaskVariable[] tasks, IntegerVariable[] heights, IntegerVariable[] usages, IntegerVariable consumption, String... options)}
		\item \mylst{cumulativeMin(String name, TaskVariable[] tasks, IntegerVariable[] heights, IntegerVariable consumption, String... options)}
		\item \mylst{cumulativeMin(TaskVariable[] tasks, int[] heights, int consumption, String... options)}
	\end{itemize}
	\item \textbf{return type} : \texttt{Constraint}
	\item \textbf{options} :
	\begin{itemize}
		\item \emph{no option}
		\item \hyperlink{ccumulativeti:ccumulativetioptions}{SettingType.TASK\_INTERVAL.getOptionName()} for fast task intervals
		\item \hyperlink{ccumulativesti:ccumulativestioptions}{SettingType.SLOW\_TASK\_INTERVAL.getOptionName()} for slow task intervals
		\item \hyperlink{ccumulativecef:ccumulativecefoptions}{SettingType.VILIM\_CEF\_ALGO.getOptionName()} for Vilim theta lambda tree + lazy computation of the inner maximization of the edge finding rule of Van hentenrick and Mercier
		\item \hyperlink{ccumulativescef:ccumulativescefoptions}{SettingType.VHM\_CEF\_ALGO\_N2K.getOptionName()} for Simple $n^2 * k$ algorithm (lazy for R) (CalcEF -- Van Hentenrick)
	\end{itemize}
	\item \textbf{favorite domain} : \emph{n/a}
\end{itemize}
\label{disjoint}
\hypertarget{disjoint}{}

\section{disjoint (constraint)}\label{disjoint:disjointconstraint}\hypertarget{disjoint:disjointconstraint}{}

\begin{notedef}
  \texttt{disjoint}$(\collec{T^1_1}{T^1_n}, \collec{T^2_1}{T^2_m})$ states that each pair of tasks $(T^1_i, T^2_j)$ is in disjunction i.e., the processings of the two tasks do not overlap in time:
 $$T^1_i.end \le T^2_j.start\quad\lor\quad  T^2_j.end \le T^1_i.start,\quad \forall i=1..n, j=1..m $$  
\end{notedef}

CHOCO only provides a decomposition with reified precedences because the coloured cumulative is not available.

\begin{itemize}
	\item \textbf{API}: \mylst{disjoint(TaskVariable[] t1, TaskVariable[] t2)}
	\item \textbf{return type}: \texttt{Constraint[]}
	\item \textbf{favorite domain} : \emph{n/a}.
	\item \textbf{references} :
      \begin{itemize}
      \item global constraint catalog: \href{http://www.emn.fr/x-info/sdemasse/gccat/Cdisjoint_tasks.html}{\tt disjoint\_tasks}
      \end{itemize}
\end{itemize}
\textbf{Example}:
\lstinputlisting{java/cdisjoint2.j2t}

%\part{disjunctive}
\label{disjunctive}
\hypertarget{disjunctive}{}

\section{disjunctive (constraint)}\label{disjunctive:disjunctiveconstraint}\hypertarget{disjunctive:disjunctiveconstraint}{}
\todo{to be cleaned.}

\begin{notedef}
  \texttt{disjunctive(start,duration)} states that a set of tasks (defined by their starting times and durations) are executed on a ddisjunctive resource, i.e. they do not overlap in time:
$$|\{i\ |\ \mathtt{start}[i]\le t < \mathtt{start}[i]+\mathtt{duration}[i]\}| \le 1,\quad (\forall \text{ time } t)$$
\end{notedef}

\todo{The notion of task does not exist yet in Choco.} The \texttt{disjunctive} takes therefore as input arrays of integer variables (of same size $n$) denoting the starting and duration of each task. When the array of finishing times is also specified, the constraint ensures that \texttt{start[i] + duration[i] = end[i]} for all task $i$.
As usual, a task is executed in the interval \texttt{[start,end-1]}.

\begin{itemize}
	\item \textbf{API} :
	\begin{itemize}
		\item \mylst{disjunctive(IntegerVariable[] start, int[] duration, String...options)}
		\item \mylst{disjunctive(IntegerVariable[] start, IntegerVariable[] duration, String... options)}
		\item \mylst{disjunctive(IntegerVariable[] start, IntegerVariable[] end, IntegerVariable[] duration, String... options)}
		\item \mylst{disjunctive(IntegerVariable[] start, IntegerVariable[] end, IntegerVariable[] duration, IntegerVariable uppBound, String... options)}
	\end{itemize}
	\item \textbf{return type} : \texttt{Constraint}
	\item \textbf{options} :
	\begin{itemize}
		\item \emph{no option}
		\item \hyperlink{cdisjunctiveoc:cdisjunctiveocoptions}{SettingType.OVERLOAD\_CHECKING.getOptionName()} overload checking rule ( O(n*log(n)), Vilim), also known as task interval
		\item \hyperlink{cdisjunctivenfnl:cdisjunctivenfnloptions}{SettingType.NF\_NL.getOptionName()} NotFirst/NotLast rule ( O(n*log(n)), Vilim) (recommended)
		\item \hyperlink{cdisjunctivedp:cdisjunctivedpoptions}{SettingType.DETECTABLE\_PRECEDENCE.getOptionName()} Detectable Precedence rule ( O(n*log(n)), Vilim)
		\item \hyperlink{cdisjunctiveef:cdisjunctiveefoptions}{SettingType.EDGE\_FINDING\_D.getOptionName()} disjunctive Edge Finding rule ( O(n*log(n)), Vilim) (recommended)
		\item \hyperlink{cdisjunctivedf:cdisjunctivedfoptions}{SettingType.DEFAULT\_FILTERING.getOptionName()} use filtering algorithm proposed by Vilim. nested loop, each rule is applied until it reach it fixpoint		
		\item \hyperlink{cdisjunctivedf:cdisjunctivedfoptions}{SettingType.VILIM\_FILTERING.getOptionName()} use filtering algorithm proposed by Vilim. nested loop, each rule is applied until it reach it fixpoint
		\item \hyperlink{cdisjunctivesrf:cdisjunctivesrfoptions}{SettingType.SINGLE\_RULE\_FILTERING.getOptionName()} use filtering algorithm proposed by Vilim. nested loop, each rule is applied until it reach it fixpoint. A single filtering rule (debug only).					
	\end{itemize}
	\item \textbf{favorite domain} : \emph{n/a}
	\item \textbf{references} :\\
      global constraint catalog: \href{http://www.emn.fr/x-info/sdemasse/gccat/Cdisjunctive.html}{\tt disjunctive}
\end{itemize}

\textbf{Example}:
%\lstinputlisting{java/cdisjunctive.j2t}
\mylst{//TODO: complete} 

%\part{distanceeq}
\label{distanceeq}
\hypertarget{distanceeq}{}

\section{distanceEQ (constraint)}\label{distanceeq:distanceeqconstraint}\hypertarget{distanceeq:distanceeqconstraint}{}
\begin{notedef}
  \texttt{distanceEQ}$(x_1,x_2,x_3,c)$ states that $x_3$ plus an offset $c$ (by default $c=0$) is equal to the distance between $x_1$ and $x_2$:
$$ x_3 + c = | x_1 - x_2 |$$
\end{notedef}

\begin{itemize}
	\item \textbf{API} :
	\begin{itemize}
		\item \mylst{distanceEQ(IntegerVariable x1, IntegerVariable x2, int x3)}
		\item \mylst{distanceEQ(IntegerVariable x1, IntegerVariable x2, IntegerVariable x3)}
		\item \mylst{distanceEQ(IntegerVariable x1, IntegerVariable x2, IntegerVariable x3, int c)}
	\end{itemize}
	\item \textbf{return type}: \texttt{Constraint}
	\item \textbf{options} : \emph{n/a}
	\item \textbf{favorite domain} : \emph{to complete}
	\item \textbf{references} :\\
      global constraint catalog: \href{http://www.emn.fr/x-info/sdemasse/gccat/Call_min_dist.html}{\tt all\_min\_dist} (variant)
\end{itemize}

\textbf{Example}:
\lstinputlisting{java/cdistanceeq.j2t}

%\part{distancegt}
\label{distancegt}
\hypertarget{distancegt}{}

\section{distanceGT (constraint)}\label{distancegt:distancegtconstraint}\hypertarget{distancegt:distancegtconstraint}{}
\begin{notedef}
  \texttt{distanceGT}$(x_1,x_2,x_3,c)$ states that $x_3$ plus an offset $c$ (by default $c=0$) is strictly greater than the distance between $x_1$ and $x_2$:
$$ x_3 + c > | x_1 - x_2 |$$
\end{notedef}

\begin{itemize}
	\item \textbf{API} :
	\begin{itemize}
		\item \mylst{distanceGT(IntegerVariable x1, IntegerVariable x2, int x3)}
		\item \mylst{distanceGT(IntegerVariable x1, IntegerVariable x2, IntegerVariable x3)}
		\item \mylst{distanceGT(IntegerVariable x1, IntegerVariable x2, IntegerVariable x3, int c)}
	\end{itemize}
	\item \textbf{return type}: \texttt{Constraint}
	\item \textbf{options} : \emph{n/a}
	\item \textbf{favorite domain} : \emph{to complete}
	\item \textbf{references} :\\
      global constraint catalog: \href{http://www.emn.fr/x-info/sdemasse/gccat/Call_min_dist.html}{\tt all\_min\_dist} (variant)
\end{itemize}

\textbf{Example}:
\lstinputlisting{java/cdistancegt.j2t}

%\part{distancelt}
\label{distancelt}
\hypertarget{distancelt}{}

\section{distanceLT (constraint)}\label{distancelt:distanceltconstraint}\hypertarget{distancelt:distanceltconstraint}{}
\begin{notedef}
  \texttt{distanceLT}$(x_1,x_2,x_3,c)$ states that $x_3$ plus an offset $c$ (by default $c=0$) is strictly smaller than the distance between $x_1$ and $x_2$:
$$ x_3 + c < | x_1 - x_2 |$$
\end{notedef}

\begin{itemize}
	\item \textbf{API} :
	\begin{itemize}
		\item \mylst{distanceLT(IntegerVariable x1, IntegerVariable x2, int x3)}
		\item \mylst{distanceLT(IntegerVariable x1, IntegerVariable x2, IntegerVariable x3)}
		\item \mylst{distanceLT(IntegerVariable x1, IntegerVariable x2, IntegerVariable x3, int c)}
	\end{itemize}
	\item \textbf{return type}: \texttt{Constraint}
	\item \textbf{options} : \emph{n/a}
	\item \textbf{favorite domain} : \emph{to complete}
\end{itemize}

\textbf{Example}:
\lstinputlisting{java/cdistancelt.j2t}

%\part{distanceneq}
\label{distanceneq}
\hypertarget{distanceneq}{}

\section{distanceNEQ (constraint)}\label{distanceneq:distanceneqconstraint}\hypertarget{distanceneq:distanceneqconstraint}{}
\begin{notedef}
  \texttt{distanceNEQ}$(x_1,x_2,x_3,c)$ states that $x_3$ plus an offset $c$ (by default $c=0$) is not equal to the distance between $x_1$ and $x_2$:
$$ x_3 + c \neq | x_1 - x_2 |$$
\end{notedef}

\begin{itemize}
	\item \textbf{API} :
	\begin{itemize}
		\item \mylst{distanceNEQ(IntegerVariable x1, IntegerVariable x2, int x3)}
		\item \mylst{distanceNEQ(IntegerVariable x1, IntegerVariable x2, IntegerVariable x3)}
		\item \mylst{distanceNEQ(IntegerVariable x1, IntegerVariable x2, IntegerVariable x3, int c)}
	\end{itemize}
	\item \textbf{return type}: \texttt{Constraint}
	\item \textbf{options} : \emph{n/a}
	\item \textbf{favorite domain} : \emph{to complete}
	\item \textbf{references} :\\
      global constraint catalog: \href{http://www.emn.fr/x-info/sdemasse/gccat/Call_min_dist.html}{\tt all\_min\_dist} (variant)
\end{itemize}

\textbf{Example}:
\lstinputlisting{java/cdistanceneq.j2t}

%\part{domainchanneling}
\label{domainchanneling}
\hypertarget{domainchanneling}{}

\section{domainChanneling (constraint)}\label{domainchanneling:domainchannelingconstraint}\hypertarget{domainchanneling:domainchannelingconstraint}{}
\begin{notedef}  
\texttt{domainChanneling}$(x, \collec{b_1}{b_n})$ states that boolean $b_j$ is true if and only if $x$ has value $j$:
$$b_j=1\quad\iff\quad x=j,\qquad\forall j=1..n$$ 
\end{notedef}

It makes the link between a domain variable $x$ and those 0-1 variables $b$ that are associated with each potential value of $x$: the 0-1 variable $b_j$ associated with the value $j$ taken by $x$ is equal to 1, while the remaining 0-1 variables $b_i$ ($i\neq j$) are all equal to 0.

\begin{itemize}
	\item \textbf{API} : \mylst{domainChanneling(IntegerVariable x, IntegerVariable[] b)}
	\item \textbf{return type} : \texttt{Constraint}
	\item \textbf{options} : \emph{n/a}
	\item \textbf{favorite domain} : enumerated for $x$
	\item \textbf{references} :\\
	  global constraint catalog: \href{http://www.emn.fr/x-info/sdemasse/gccat/Cdomain_constraint.html}{\tt domain\_constraint}
\end{itemize}

\textbf{Example}:
\lstinputlisting{java/cdomainchanneling.j2t}

\section{element (constraint)}\label{element:elementconstraint}\hypertarget{element:elementconstraint}{}
See \hyperlink{nth:nthconstraint}{nth}.

\section{endsAfter (constraint)}\label{endsafter:endsafterconstraint}\hypertarget{endsafter:endsafterconstraint}{}
\begin{notedef}
  \texttt{endsAfter}$(T,c)$ states that task $T$ ends after time $c$:
  $$T.end \ge c$$
\end{notedef}

\begin{itemize}
	\item \textbf{API} :\mylst{endsAfter(TaskVariable t, int c) }
	\item \textbf{return type} : \texttt{Constraint}
	\item \textbf{options} : \emph{n/a}
	\item \textbf{favorite domain} : \emph{n/a}.
\end{itemize}

\textbf{Examples:}
%\lstinputlisting{java/ceq1.j2t}
\emph{to complete}

\section{endsAfterBegin (constraint)}\label{endsafterbegin:endsafterbeginconstraint}\hypertarget{endsafterbegin:endsafterbeginconstraint}{}

\begin{notedef}
\texttt{endsAfterBegin}$(T_1,T_2,c)$ states that task $T_1$ ends after the start time of $T_2$ minus $c$:
  $$T_{1}.end \ge T_{2}.start - c$$
\end{notedef}

\begin{itemize}
	\item \textbf{API} :\mylst{endsAfterBegin(TaskVariable t1, TaskVariable t2, int c)}
	\item \textbf{return type} : \texttt{Constraint}
	\item \textbf{options} : \emph{n/a}
	\item \textbf{favorite domain} : \emph{n/a}.
\end{itemize}

\textbf{Examples:}
%\lstinputlisting{java/ceq1.j2t}
\emph{to complete}

\section{endsAfterEnd (constraint)}\label{endsafterend:endsafterendconstraint}\hypertarget{endsafterend:endsafterendconstraint}{}
\begin{notedef}
\texttt{endsAfterEnd}$(T_1,T_2,c)$  states that task $T_1$ ends after the end time of $T_2$ minus $c$:
  $$T_{1}.end \ge T_{2}.end - c$$
\end{notedef}

\begin{itemize}
	\item \textbf{API} :\mylst{endsAfterEnd(TaskVariable t1, TaskVariable t2, int c)}
	\item \textbf{return type} : \texttt{Constraint}
	\item \textbf{options} : \emph{n/a}
	\item \textbf{favorite domain} : \emph{n/a}.
\end{itemize}

\textbf{Examples:}
%\lstinputlisting{java/ceq1.j2t}
\emph{to complete}

\section{endsBefore (constraint)}\label{endsbefore:endsbeforeconstraint}\hypertarget{endsbefore:endsbeforeconstraint}{}
\begin{notedef}
  \texttt{endsBefore}$(T,c)$ states that task $T$ ends before time $c$:
  $$T.end \le c$$
\end{notedef}

\begin{itemize}
	\item \textbf{API} :\mylst{endsBefore(final TaskVariable t, final int c) }
	\item \textbf{return type} : \texttt{Constraint}
	\item \textbf{options} : \emph{n/a}
	\item \textbf{favorite domain} : \emph{n/a}.
\end{itemize}

\textbf{Examples:}
%\lstinputlisting{java/ceq1.j2t}
\emph{to complete}

\section{endsBeforeBegin (constraint)}\label{endsbeforebegin:endsbeforebeginconstraint}\hypertarget{endsbeforebegin:endsbeforebeginconstraint}{}
\begin{notedef}
\texttt{endsBeforeBegin}$(T_1,T_2,c)$  states that task $T_1$ ends before the start time of $T_2$ minus $c$:
  $$T_{1}.end \le T_{2}.start - c$$
\end{notedef}

\begin{itemize}
	\item \textbf{API} :\mylst{endsBeforeBegin(TaskVariable t1, TaskVariable t2, int c)}
	\item \textbf{return type} : \texttt{Constraint}
	\item \textbf{options} : \emph{n/a}
	\item \textbf{favorite domain} : \emph{n/a}.
\end{itemize}

\textbf{Examples:}
%\lstinputlisting{java/ceq1.j2t}
\emph{to complete}

\section{endsBeforeEnd (constraint)}\label{endsbeforeend:endsbeforeendconstraint}\hypertarget{endsbeforeend:endsbeforeendconstraint}{}
\begin{notedef}
\texttt{endsBeforeEnd}$(T_1,T_2,c)$  states that task $T_1$ ends before the end time of $T_2$ minus $c$:
  $$T_{1}.end \le T_{2}.end - c$$
\end{notedef}

\begin{itemize}
	\item \textbf{API} :\mylst{endsBeforeEnd(TaskVariable t1, TaskVariable t2, int c)}
	\item \textbf{return type} : \texttt{Constraint}
	\item \textbf{options} : \emph{n/a}
	\item \textbf{favorite domain} : \emph{n/a}.
\end{itemize}

\textbf{Examples:}
%\lstinputlisting{java/ceq1.j2t}
\emph{to complete}

\section{endsBetween (constraint)}\label{endsbetween:endsbetweenconstraint}\hypertarget{endsbetween:endsbetweenconstraint}{}
\begin{notedef}
  \texttt{endsBetween}$(T, c_1, c_2)$ states that task $T$ ends between times $c_1$ and $c_2$:
  $$c_1 \le T.end \le c_2$$
\end{notedef}

\begin{itemize}
	\item \textbf{API} :\mylst{endsBetween(TaskVariable t, int min, int max)}
	\item \textbf{return type} : \texttt{Constraint}
	\item \textbf{options} : \emph{n/a}
	\item \textbf{favorite domain} : \emph{n/a}.
\end{itemize}

\textbf{Examples:}
%\lstinputlisting{java/ceq1.j2t}
\emph{to complete}

%\part{eq}
\label{eq}
\hypertarget{eq}{}

\section{eq (constraint)}\label{eq:eqconstraint}\hypertarget{eq:eqconstraint}{}
\begin{notedef}
  \texttt{eq}$(x,y)$ states that the two arguments are equal:
$$x = y$$
\end{notedef}

\begin{itemize}
	\item \textbf{API} :
	\begin{itemize}
		\item \mylst{eq(IntegerExpressionVariable x, IntegerExpressionVariable y)}
		\item \mylst{eq(IntegerExpressionVariable x, int y)}
		\item \mylst{eq(int x, IntegerExpressionVariable y)}
		\item \mylst{eq(SetVariable x, SetVariable y)}
		\item \mylst{eq(RealExpressionVariable x, RealExpressionVariable y)}
		\item \mylst{eq(RealExpressionVariable x, double y)}
		\item \mylst{eq(double x, RealExpressionVariable y)}
		\item \mylst{eq(IntegerVariable x, RealVariable y)}
		\item \mylst{eq(RealVariable x, IntegerVariable y)}
	\end{itemize}
	\item \textbf{return type} : \texttt{Constraint}
	\item \textbf{options} : \emph{n/a}
	\item \textbf{favorite domain} : \emph{to complete}.
	\item \textbf{references} :\\
      global constraint catalog: \href{http://www.emn.fr/x-info/sdemasse/gccat/Ceq.html}{\tt eq} (on domain variables) and \href{http://www.emn.fr/x-info/sdemasse/gccat/Ceq_set.html}{\tt eq\_set} (on set variables). 
\end{itemize}

\textbf{Examples:}
\begin{itemize}
	\item example1:
\end{itemize}
\lstinputlisting{java/ceq1.j2t}
\begin{itemize}
	\item example2
\end{itemize}

\lstinputlisting{java/ceq2.j2t}

%\part{eqcard}
\label{eqcard}
\hypertarget{eqcard}{}

\section{eqCard (constraint)}\label{eqcard:eqcardconstraint}\hypertarget{eqcard:eqcardconstraint}{}
\begin{notedef}
  \texttt{eqCard}$(s,z)$ states that the cardinality of set $s$ is equal to $z$:
$$|s| = z$$
\end{notedef}

\begin{itemize}
	\item \textbf{API} :
	\begin{itemize}
		\item \mylst{eqCard(SetVariable s, IntegerVariable z)}
		\item \mylst{eqCard(SetVariable s, int z)}
	\end{itemize}
	\item \textbf{return type} : \texttt{Constraint}
	\item \textbf{options} : \emph{n/a}
	\item \textbf{favorite domain} : \emph{to complete}
\end{itemize}

\textbf{Example}:
\lstinputlisting{java/ceqcard.j2t}

%\part{equation}
\label{equation}
\hypertarget{equation}{}

\section{equation (constraint)}\label{equation:equationconstraint}\hypertarget{equation:equationconstraint}{}
\begin{notedef}
  \texttt{equation}$(z, \collec{x_1}{x_n},\collec{c_1}{c_n})$ states that $z$ is the weighted sum of $x$ by $c$:
$$c_1x_1+c_2x_2+...+c_nx_n = z$$
Restrictions: $\{\forall c_i \in c, c_i \ge 0\} \wedge z \ge 0$
\end{notedef}
See also \hyperlink{knapsackproblem:knapsackproblemconstraint}{knapsackProblem}.

\begin{itemize}
	\item \textbf{API} :
	\begin{itemize}
		\item equation(int z, IntegerVariable[] x, int[] c)
		\item equation(String option, int z, IntegerVariable[] x, int[] c)
		\item equation(IntegerVariable z, IntegerVariable[] x, int[] c)
		\item equation(String option, IntegerVariable z, IntegerVariable[] x, int[] c)
	\end{itemize}
	\item \textbf{return type} : \texttt{Constraint}
	\item \textbf{options} :
	\begin{itemize}
		\item \emph{no option} or \texttt{"cp:ac"}: to enforce GAC  using \hyperlink{regular}{\texttt{regular}}
		\item \texttt{"cp:bc"}: to enforce bound consistency using \hyperlink{eq}{\texttt{eq}}$(z,$\hyperlink{scalar}{\texttt{scalar}}$(x,c))$
	\end{itemize}
	\item \textbf{favorite domain} : \emph{to complete}
	\item \textbf{global constraint catalog} : \href{http://www.emn.fr/z-info/sdemasse/gccat/Cscalar_product.html}{scalar\_product}
\end{itemize}

\textbf{Example}:
\lstinputlisting{java/cequation.j2t}

\section{exactly (constraint)}\label{exactly:exactlyconstraint}\hypertarget{exactly:exactlyconstraint}{}
Deprecated, see \hyperlink{occurrence:occurrenceconstraint}{\texttt{occurrence}}.

%\part{false}
%\label{false}\hypertarget{false}{}
\section{FALSE (constraint)}\label{false:falseconstraint}\hypertarget{false:falseconstraint}{}
\(FALSE\) always returns \emph{false}.

\input{chapters/constraints/Cfeaspairac.tex}
%\part{feastupleac}
\label{feastupleac}
\hypertarget{feastupleac}{}

\section{feasTupleAC (constraint)}\label{feastupleac:feastupleacconstraint}\hypertarget{feastupleac:feastupleacconstraint}{}
\begin{notedef}
  \texttt{feasTupleAC}$(\collec{x_1}{x_n},feasTuples)$ states an extensional constraint on \collec{x_1}{x_n} defined by the table $feasTuples$ of compatible tuples of values, and then enforces arc consistency:
      $$\exists \text{ tuple } i\ |\quad \collec{x_1}{x_n}=feasTuples[i]$$
\end{notedef}

The API is duplicated to define options.
\begin{itemize}
	\item \textbf{API} :
	\begin{itemize}
		\item \mylst{feasTupleAC(List<int[]> feasTuples, IntegerVariable... x)}
		\item \mylst{feasTupleAC(String options, List<int[]> feasTuples, IntegerVariable... x)}
	\end{itemize}
	\item \textbf{return type}: \texttt{Constraint}
	\item \textbf{options} :
	\begin{itemize}
		\item \emph{no option}: use AC32 (default arc consistency)
		\item \hyperlink{cext32:cext32options}{\tt Options.C\_EXT\_AC32}: to get AC3rm algorithm (maintaining the current support of each value in a non backtrackable way)
		\item \hyperlink{cext2001:cext2001options}{\tt Options.C\_EXT\_AC2001}: to get AC2001 algorithm (maintaining the current support of each value)
		\item \hyperlink{cext2008:cext2008options}{\tt Options.C\_EXT\_AC2008}: to get AC2008 algorithm (maintained by STR)
	\end{itemize}
	\item \textbf{favorite domain} : \emph{to complete}
	\item \textbf{references} :\\
      global constraint catalog: \href{http://www.emn.fr/x-info/sdemasse/gccat/Cin_relation.html}{in\_relation}
\end{itemize}

\textbf{Example}:
\lstinputlisting{java/cfeastupleac.j2t}

%\part{feastuplefc}
\label{feastuplefc}
\hypertarget{feastuplefc}{}

\section{feasTupleFC (constraint)}\label{feastuplefc:feastuplefcconstraint}\hypertarget{feastuplefc:feastuplefcconstraint}{}
\begin{notedef}
  \texttt{feasTupleFC}$(\collec{x_1}{x_n},feasTuples)$ states an extensional constraint on \collec{x_1}{x_n} defined by the table $feasTuples$ of compatible tuples of values, and then performs Forward Checking:
      $$\exists \text{ tuple } i\ |\quad \collec{x_1}{x_n}=feasTuples[i]$$
\end{notedef}


\begin{itemize}
	\item \textbf{API} : \mylst{feasTupleFC(List<int[]> tuples, IntegerVariable... x)}
	\item \textbf{return type}: \texttt{Constraint}
	\item \textbf{options} : \emph{n/a}
	\item \textbf{favorite domain}: \emph{to complete}
	\item \textbf{references} :\\
      global constraint catalog: \href{http://www.emn.fr/x-info/sdemasse/gccat/Cin_relation.html}{in\_relation}
\end{itemize}

\textbf{Example}:
\lstinputlisting{java/cfeastuplefc.j2t}

\section{forbiddenInterval (constraint)}\label{forbiddeninterval:forbiddenintervalconstraint}\hypertarget{forbiddeninterval:forbiddenintervalconstraint}{}
\todo{to be detailed}

\begin{notedef}
  \texttt{forbiddenInterval}$(\collec{t_1}{t_2})$ applies additionnal search tree reduction based on time intervals in which no operation can start or end in an optimal solution. \emph{The tasks must all belong to one disjunctive resource and have fixed durations}.
\end{notedef}

\begin{itemize}
	\item \textbf{API} :
	\begin{itemize}
		\item \mylst{forbiddenInterval(String name, TaskVariable[] tasks)}
		\item \mylst{forbiddenInterval(TaskVariable[] tasks)}
	\end{itemize}
	\item \textbf{return type} : \texttt{Constraint}
	\item \textbf{options} : \emph{n/a}
	\item \textbf{favorite domain} : \emph{n/a}
\end{itemize}

\textbf{Example}:
\emph{to complete}
%\lstinputlisting{java/ceqcard.j2t}


%\part{geost}
\label{geost}
\hypertarget{geost}{}

\section{geost (constraint)}\label{geost:geostconstraint}\hypertarget{geost:geostconstraint}{}
\todo{to be cleaned}

\begin{notedef}
\texttt{geost} is a global constraint that generically handles a variety of geometrical placement problems. 
It handles geometrical constraints (non-overlapping, distance, etc.) between polymorphic objects (ex: polymorphism can be used for representing rotation) in any dimension.
%The \texttt{geost}$(K, O, S, C)$ constraint is given set of parameters which will define the environment of \texttt{geost}. The parameters are as follows:
The parameters of \texttt{geost}$(dim, objects, shiftedBoxes, eCtrs)$ are respectively:
the space dimension, the list of geometrical objects, the set of boxes that compose the shapes of the objects, the set of geometrical constraints.
The greedy mode should be used without external constraints to have safe results, because it excludes external constraints from its exploration and look for instanciation of variables involved in geost which respect the geost constraint.
For further informations, see the section devoted to this constraint in the Choco Tutorial document. 
%visit the following \hyperlink{geostdescription:placementanduseofthegeostconstraint}{page}.
\end{notedef}

\begin{itemize}
	\item \textbf{API} :\\
\mylst{geost(int dim, Vector<GeostObject> objects, Vector<ShiftedBox> shiftedBoxes, Vector<ExternalConstraint> eCtrs)}\\
\mylst{geost(int dim, Vector<GeostObject> objects, Vector<ShiftedBox> shiftedBoxes, Vector<ExternalConstraint> eCtrs, Vector<int[]> ctrlVs)}
	\item \textbf{return type} : \texttt{Constraint}
	\item \textbf{options} :\emph{n/a}
	\item \textbf{favorite domain} : \emph{to complete}
	\item \textbf{references} :\\
      global constraint catalog: \href{http://www.emn.fr/x-info/sdemasse/gccat/Cgeost.html}{geost}
\end{itemize}

The geost constraint requires the creation of different objects:

\centerline{\begin{tabular}{p{3cm}p{5cm}p{6cm}}
parameter &type &description \\
\hline
\emph{objects} &\texttt{Vector<GeostObject>} &geometrical objects\\
\emph{shiftedBoxes} &\texttt{Vector<ShiftedBox>} &boxes that compose the object shapes\\
\emph{eCtrs} &\texttt{Vector<ExternalConstraint>} &geometrical constraints\\
\emph{ctrlVs} &\texttt{Vector<int[]>} &controlling vectors (for greedy mode)\\[1em]
\end{tabular}}

\noindent Where a \texttt{\bf GeostObject} is defined by:

\centerline{\begin{tabular}{p{3cm}p{4cm}p{7cm}}
attribute &type &description \\
\hline
\emph{dim} &\texttt{int} &dimension\\
\emph{objectId} &\texttt{int} &object id\\
\emph{shapeId} &\texttt{IntegerVariable} &shape id\\
\emph{coordinates} &\texttt{IntegerVariable[$dim$]} &coordinates of the origin\\
\emph{startTime} &\texttt{IntegerVariable} &starting time\\
\emph{durationTime} &\texttt{IntegerVariable} &duration\\
\emph{endTime} &\texttt{IntegerVariable} &finishing time\\[1em]
\end{tabular}}

\noindent Where a \texttt{\bf ShiftedBox} is a $dim$-box defined by the shape it belongs to, its origin (the coordinates of the lower left corner of the box) and its lengths in every dimensions:

\centerline{\begin{tabular}{p{3cm}p{4cm}p{7cm}}
attribute &type &description \\
\hline
\emph{sid} &\texttt{int} &shape id\\
\emph{offset} &\texttt{int[$dim$]} &coordinates of the offset (lower left corner)\\
\emph{size} &\texttt{int[$dim$]} &lengths in every dimensions\\[1em]
\end{tabular}}

\noindent Where an \texttt{\bf ExternalConstraint} contains informations and functionality common to all external constraints and is defined by:

\centerline{\begin{tabular}{p{3cm}p{4cm}p{7cm}}
attribute &type &description \\
\hline
 \emph{ectrID} &\texttt{int} &constraint id\\
 \emph{dimensions} &\texttt{int[]} &list of dimensions that the external constraint is active for\\
 \emph{objectIdentifiers} &\texttt{int[]} &list of object ids that this external constraint affects.\\[1em]
\end{tabular}}

\textbf{Example}:
\lstinputlisting{java/cgeost.j2t}

%\part{geq}
\label{geq}
\hypertarget{geq}{}

\section{geq (constraint)}\label{geq:geqconstraint}\hypertarget{geq:geqconstraint}{}
\begin{notedef}
  \texttt{geq}$(x,y)$ states that $x$ is greater than or equal to $y$:
$$x\ge y$$
\end{notedef}

\begin{itemize}
	\item \textbf{API} :
	\begin{itemize}
		\item \mylst{geq(IntegerExpressionVariable x, IntegerExpressionVariable y)}
		\item \mylst{geq(IntegerExpressionVariable x, int y)}
		\item \mylst{geq(int x, IntegerExpressionVariable y)}
		\item \mylst{geq(RealExpressionVariable x, RealExpressionVariable y)}
		\item \mylst{geq(RealExpressionVariable x, double y)}
		\item \mylst{geq(double x, RealExpressionVariable y)}
	\end{itemize}
	\item \textbf{return type} : \texttt{Constraint}
	\item \textbf{options} : \emph{n/a}
	\item \textbf{favorite domain} : \emph{to complete}.
	\item \textbf{references} :\\
      global constraint catalog: \href{http://www.emn.fr/x-info/sdemasse/gccat/Cgeq.html}{geq}
\end{itemize}

\textbf{Examples:}
\begin{itemize}
	\item example1:
\end{itemize}

\lstinputlisting{java/ceq1.j2t}

\begin{itemize}
	\item example2
\end{itemize}

\lstinputlisting{java/ceq2.j2t}

%\part{geqcard}
\label{geqcard}
\hypertarget{geqcard}{}

\section{geqCard (constraint)}\label{geqcard:geqcardconstraint}\hypertarget{geqcard:geqcardconstraint}{}
\begin{notedef}
  \texttt{geqCard}$(s,z)$ states that the cardinality of set $s$ is greater than or equal to $z$:
$$|s| \ge z$$
\end{notedef}

\begin{itemize}
	\item \textbf{API} :
	\begin{itemize}
		\item \mylst{geqCard(SetVariable s, IntegerVariable z)}
		\item \mylst{geqCard(SetVariable s, int z)}
	\end{itemize}
	\item \textbf{return type} : \texttt{Constraint}
	\item \textbf{options} : \emph{n/a}
	\item \textbf{favorite domain} : \emph{to complete}
\end{itemize}

\textbf{Example}:
\lstinputlisting{java/cgeqcard.j2t}

%\part{globalcardinality}
\label{globalcardinality}
\hypertarget{globalcardinality}{}

\section{globalCardinality (constraint)}\label{globalcardinality:globalcardinalityconstraint}\hypertarget{globalcardinality:globalcardinalityconstraint}{}
\begin{notedef}
  \texttt{globalCardinality}$(\collec{x_1}{x_n},\collec{l_0}{l_{M-m}}, \collec{u_0}{u_{M-m}}, m)$ states lower bounds $l$ and upper bounds $u$ on the occurrence numbers of the values in collection $x$ according to offset $m$: 
$$l_{j-m}\ \le\ |\{i=1..n\ ,\ x_i=j\}|\ \le\ u_{j-m},\quad\forall j=m..M$$   
  \texttt{globalCardinality}$(\collec{x_1}{x_n},\collec{z_0}{z_{M-m}}, o)$ states that $z$ are the occurrence numbers of the values in collection $x$ according to offset $m$: 
$$z_{j-m} = |\{i=1..n\ ,\ x_i=j\}|,\quad\forall j=m..M$$   
\end{notedef}
Note that the length of the bound tables $l$ and $u$ are equal to $M+m-1$ where $m$, the offset, is the minimum counted value and $M$ is the maximum counted value.

%offset is the minimum value over all variables in $x$
Several APIs exist:
\begin{itemize}
	\item \emph{constant bounds on cardinalities} \collec{l_0}{l_{M-m}} and \collec{u_0}{u_{M-m}}: use the propagator of \cite{ReginAAAI96} or of \cite{QuimperCP03} depending on the set options and the nature of the domain variables.
	\item \emph{variable cardinalities} \collec{z_0}{z_{M-m}}: use the propagator of \cite{QuimperCP03} that:      
      \begin{itemize}
      \item enforces Bound Consistency over $x$ regarding the lower and upper bounds of $z$, 
      \item maintains the upper bound of $z_j$ by counting the variables that may be instantiated to $j$, 
      \item maintains the lower bound of $z_j$ by counting the variables instantiated to $j$, 
      \item enforces $z_0 + \cdots + z_{M-m} = n$
      \end{itemize}
\end{itemize}

The APIs are duplicated to define options. 

\begin{itemize}
	\item \textbf{API} :
      \begin{itemize}
	\item \mylst{globalCardinality(IntegerVariable[] x, int[] low, int[] up, int offset)}
	\item \mylst{globalCardinality(String options, IntegerVariable[] x, int[] low, int[] up, int offset)}
	\item \mylst{globalCardinality(IntegerVariable[] x, int[] values, int[] low, int[] up)}
	\item \mylst{globalCardinality(IntegerVariable[] x, IntegerVariable[] card, int offset)}
	\item \mylst{globalCardinality(IntegerVariable[] x, int[] values, IntegerVariable[] card)}
      \end{itemize}
	\item \textbf{return type} : \texttt{Constraint}
	\item \textbf{options}:
	\begin{itemize}
		\item \emph{no option}: 
          if $x$ have \emph{bounded} domains or if the cardinalities are variable $z$, use the propagator of~\cite{QuimperCP03} for BC, otherwise use the propagator of \cite{ReginAAAI96};
		\item \hyperlink{cgccac:cgccacoptions}{\tt Options.C\_GCC\_AC} : for \cite{ReginAAAI96} implementation of arc consistency
		\item \hyperlink{cgccbc:cgccbcoptions}{\tt Options.C\_GCC\_BC} : for  \cite{QuimperCP03} implementation of bound consistency
	\end{itemize}
	\item \textbf{favorite domain} : \emph{enumerated} for arc consistency, \emph{bounded} for bound consistency.
	\item \textbf{references} :
      \begin{itemize}
      \item \cite{ReginAAAI96}: \emph{Generalized arc consistency for global cardinality constraint},
      \item \cite{QuimperCP03}: \emph{An efficient bounds consistency algorithm for the global cardinality constraint}
      \item global constraint catalog: \href{http://www.emn.fr/x-info/sdemasse/gccat/Cglobal_cardinality.html}{global\_cardinality}
      \end{itemize}
\end{itemize}

\textbf{Examples:}
\begin{itemize}
	\item example1:
\end{itemize}

\lstinputlisting{java/cglobalcardinality1.j2t}

\begin{itemize}
	\item example2:
\end{itemize}

\lstinputlisting{java/cglobalcardinality2.j2t}

\begin{itemize}
	\item example3:
\end{itemize}

\lstinputlisting{java/cglobalcardinality3.j2t}

\begin{itemize}
	\item example4:
\end{itemize}

\lstinputlisting{java/cglobalcardinality4.j2t}

%\part{gt}
\label{gt}
\hypertarget{gt}{}

\section{gt (constraint)}\label{gt:gtconstraint}\hypertarget{gt:gtconstraint}{}
\begin{notedef}
  \texttt{gt}$(x,y)$ states that $x$ is strictly greater than $y$:
$$x>y$$
\end{notedef}

\begin{itemize}
	\item \textbf{API} :
	\begin{itemize}
		\item \mylst{gt(IntegerExpressionVariable x, IntegerExpressionVariable y)}
		\item \mylst{gt(IntegerExpressionVariable x, int y)}
		\item \mylst{gt(int x, IntegerExpressionVariable y)}
	\end{itemize}
	\item \textbf{return type} : \texttt{Constraint}
	\item \textbf{options} : \emph{n/a}
	\item \textbf{favorite domain} : \emph{to complete}.
	\item \textbf{references} :\\
      global constraint catalog: \href{http://www.emn.fr/x-info/sdemasse/gccat/Cgt.html}{gt}
\end{itemize}

\textbf{Example:}
\lstinputlisting{java/cgt.j2t}

%\part{ifonlyif}
\label{ifonlyif}
\hypertarget{ifonlyif}{}

\section{ifOnlyIf (constraint)}\label{ifonlyif:ifonlyifconstraint}\hypertarget{ifonlyif:ifonlyifconstraint}{}
\begin{notedef}
  \texttt{ifOnlyIf}$(c_1,c_2)$ states that $c_1$ holds if and only if $c_2$ holds:
$$c_1\iff c_2$$
\end{notedef}

\begin{itemize}
	\item \textbf{API} : \mylst{ifOnlyIf(Constraint c1, Constraint c2)}
	\item \textbf{return type} : \texttt{Constraint}
	\item \textbf{options} : \emph{n/a}
	\item \textbf{favorite domain} : \emph{n/a}
\end{itemize}

\textbf{Example}:
\lstinputlisting{java/cifonlyif.j2t}

%\part{ifthenelse}
\label{ifthenelse}
\hypertarget{ifthenelse}{}

\section{ifThenElse (constraint)}\label{ifthenelse:ifthenelseconstraint}\hypertarget{ifthenelse:ifthenelseconstraint}{}
\begin{notedef}
  \texttt{ifThenElse}$(c_1,c_2,c_3)$ states that if $c_1$ holds then $c_2$ holds, otherwise $c_3$ holds:
  $$(c_1\land c_2) \lor (\neg c_1 \land c_3)$$
\end{notedef}

\begin{itemize}
	\item \textbf{API} :\mylst{ifThenElse(Constraint c1, Constraint c2, Constraint c3)}
	\item \textbf{return type} : \texttt{Constraint}
	\item \textbf{options} : \emph{n/a}
	\item \textbf{favorite domain} : \emph{n/a}
\end{itemize}

\textbf{Example}:
\lstinputlisting{java/cifthenelse.j2t}

%\part{implies}
\label{implies}
\hypertarget{implies}{}

\section{implies (constraint)}\label{implies:impliesconstraint}\hypertarget{implies:impliesconstraint}{}
\begin{notedef}
  \texttt{implies}$(c_1,c_2)$ states that if $c_1$ holds then $c_2$ holds:
$$c_1\implies c_2$$
\end{notedef}

\begin{itemize}
	\item \textbf{API} : \mylst{implies(Constraint c1, Constraint c2)}
	\item \textbf{return type} : \texttt{Constraint}
	\item \textbf{options} : \emph{n/a}
	\item \textbf{favorite domain} : \emph{n/a}
\end{itemize}

\textbf{Example}:
\lstinputlisting{java/cimplies.j2t}

%\part{increasingnvalue}
\label{increasingnvalue}
\hypertarget{increasingnvalue}{}

\section{increasingNValue (constraint)}\label{increasingnvalue:increasingnvalueconstraint}\hypertarget{increasingnvalue:increasingnvalueconstraint}{}
\begin{notedef}
  \texttt{increasingNValue}$(z, \collec{x_1}{x_n})$ states that \collec{x_1}{x_n} is sorted in increasing order and that $z$ is the number of distinct values occurring in $x$.
$$z=|\{x_1,\ldots,x_n\}|\quad\land\quad x_i\le x_{i+1},\ \forall i=1..n.$$  
\end{notedef}

\begin{itemize}
	\item \textbf{API} : 
      \begin{itemize}
      \item \mylst{increasingNValue(IntegerVariable z, IntegerVariable[] x)}
      \item \mylst{increasingNValue(String option, IntegerVariable z, IntegerVariable[] x)}
      \end{itemize}
	\item \textbf{return type} : \texttt{Constraint}
	\item \textbf{options} :
	\begin{itemize}
		\item \emph{no option} filter on lower bound and on upper bound
		\item \hyperlink{cinvatleast:cinvatleastoptions}{\tt Options.C\_INCREASING\_NVALUE\_ATLEAST} filter on lower bound only
		\item \hyperlink{cinvatmost:cinvatmostoptions}{\tt Options.C\_INCREASING\_NVALUE\_ATMOST} filter on upper bound only
		\item \hyperlink{cinvboth:cinvbothoptions}{\tt Options.C\_INCREASING\_NVALUE\_BOTH} \textit{--default value--} filter on lower bound and on upper bound
	\end{itemize}
	\item \textbf{favorite domain} : \emph{to complete}
	\item \textbf{references} :\\
      global constraint catalog: \href{http://www.emn.fr/x-info/sdemasse/gccat/Cincreasing_nvalue.html}{increasing\_nvalue}
\end{itemize}

\textbf{Example}:
\lstinputlisting{java/cincreasingnvalue.j2t} 

%!TEX root = ../../content-doc.tex
%\part{inversechanneling}
\label{increasingsum}
\hypertarget{increasingsum}{}

\section{increasingSum (constraint)}\label{increasingsum:increasingsumconstraint}\hypertarget{increasingsum:increasingsumconstraint}{}
\begin{notedef}
  \texttt{increasingSum}$(\collec{x_1}{x_n},s)$ states that \collec{x_1}{x_n} is sorted in increasing order and that $s$ is equal to the sum of $x$.
$$\forall i=[1,n-1] \quad x_i \le x_{i+1}\quad\wedge \quad \sum_{i =[1,n]}x_i = s $$
%$$x_i = j \wedge j \le |y| \quad\iff\quad y_j = i \wedge i \le |x|,\qquad i \in[1,n],  j\in[1,m]$$
\end{notedef}
\begin{itemize}
	\item \textbf{API} : \mylst{increasingSum(IntegerVariable[] x, IntegerVariable s)}
	\item \textbf{return type} : \texttt{Constraint}
	\item \textbf{options} : \emph{no options}
	\item \textbf{favorite domain} : bounded for x and s
	\item \textbf{references} : --
      %\\ global constraint catalog: \href{http://www.emn.fr/z-info/sdemasse/gccat/Cinverse_within_range.html}{inverse\_within\_range}
\end{itemize}

\textbf{Example}:
\lstinputlisting{java/cincreasingsum.j2t}

%\part{infeaspairac}
\label{infeaspairac}
\hypertarget{infeaspairac}{}

\section{infeasPairAC (constraint)}\label{infeaspairac:infeaspairacconstraint}\hypertarget{infeaspairac:infeaspairacconstraint}{}
\begin{notedef}
  \texttt{infeasPairAC}$(x,y,infeasTuples)$ states an extensional binary constraint on $(x,y)$ defined by the table $infeasTuples$ of forbidden pairs of values, and then enforces arc consistency. Two APIs are available to define the forbidden pairs:
\begin{itemize}
	\item if $infeasTuples$ is encoded as a list of pairs \texttt{List<int[2]>}, then:
      $$\forall \text{ tuple } i\ |\quad (x,y)\neq infeasTuples[i]$$
	\item if $infeasTuples$ is encoded as a boolean matrix \texttt{boolean[][]}, let $\underline{x}$ and  $\underline{y}$ be the initial minimum values of $x$ and $y$, then:
      $$\forall (u,v)\ |\quad (x,y)=(u+\underline{x},v+\underline{y})\ \lor\ \neg infeasTuples[u][v]$$
\end{itemize}
\end{notedef}

The two APIs are duplicated to allow definition of options.
\begin{itemize}
	\item \textbf{API} :
	\begin{itemize}
		\item \mylst{infeasPairAC(IntegerVariable x, IntegerVariable y, List<int[]> infeasTuples)}
		\item \mylst{infeasPairAC(String options, IntegerVariable x, IntegerVariable y, List<int[]> infeasTuples)}
		\item \mylst{infeasPairAC(IntegerVariable x, IntegerVariable y, boolean[][] infeasTuples)}
		\item \mylst{infeasPairAC(String options, IntegerVariable x, IntegerVariable y, boolean[][] infeasTuples)}
	\end{itemize}
	\item \textbf{return type} : \texttt{Constraint}
	\item \textbf{options} :
	\begin{itemize}
		\item \emph{no option}: use AC3 (default arc consistency)
		\item \hyperlink{cext3:cext3options}{\tt Options.C\_EXT\_AC3}: to get AC3 algorithm (searching from scratch for supports on all values)
		\item \hyperlink{cext2001:cext2001options}{\tt Options.C\_EXT\_AC2001}: to get AC2001 algorithm (maintaining the current support of each value)
		\item \hyperlink{cext32:cext32options}{\tt Options.C\_EXT\_AC32}: to get AC3rm algorithm (maintaining the current support of each value in a non backtrackable way)
		\item \hyperlink{cext322:cext322options}{\tt Options.C\_EXT\_AC322}: to get AC3 with the used of \texttt{BitSet} to know if a support still exists
	\end{itemize}
	\item \textbf{favorite domain} : \emph{to complete}
\end{itemize}

\textbf{Example}:
\lstinputlisting{java/cinfeaspairac.j2t}

%\part{infeastupleac}
\label{infeastupleac}
\hypertarget{infeastupleac}{}

\section{infeasTupleAC (constraint)}\label{infeastupleac:infeastupleacconstraint}\hypertarget{infeastupleac:infeastupleacconstraint}{}
\begin{notedef}
  \texttt{infeasTupleAC}$(\collec{x_1}{x_n},feasTuples)$ states an extensional constraint on \collec{x_1}{x_n} defined by the table $infeasTuples$ of compatible tuples of values, and then enforces arc consistency:
      $$\forall \text{ tuple } i\ |\quad \collec{x_1}{x_n}\neq infeasTuples[i]$$
\end{notedef}

The API is duplicated to define options.
\begin{itemize}
	\item \textbf{API} :
	\begin{itemize}
		\item \mylst{infeasTupleAC(List<int[]> infeasTuples, IntegerVariable... x)}
		\item \mylst{infeasTupleAC(String options, List<int[]> infeasTuples, IntegerVariable... x)}
	\end{itemize}
	\item \textbf{return type}: \texttt{Constraint}
	\item \textbf{options} :
	\begin{itemize}
		\item \emph{no option}: use AC32 (default arc consistency)
		\item \hyperlink{cext32:cext32options}{\tt Options.C\_EXT\_AC32}: to get AC3rm algorithm (maintaining the current support of each value in a non backtrackable way)
		\item \hyperlink{cext2001:cext2001options}{\tt Options.C\_EXT\_AC2001}: to get AC2001 algorithm (maintaining the current support of each value)
		\item \hyperlink{cext2008:cext2008options}{\tt Options.C\_EXT\_AC2008}: to get AC2008 algorithm (maintained by STR)
	\end{itemize}
	\item \textbf{favorite domain} : \emph{to complete}
\end{itemize}

\textbf{Example}:
\lstinputlisting{java/cinfeastupleac.j2t}

%\part{infeastuplefc}
\label{infeastuplefc}
\hypertarget{infeastuplefc}{}

\section{infeasTupleFC (constraint)}\label{infeastuplefc:infeastuplefcconstraint}\hypertarget{infeastuplefc:infeastuplefcconstraint}{}
\begin{notedef}
  \texttt{infeasTupleFC}$(\collec{x_1}{x_n},feasTuples)$ states an extensional constraint on \collec{x_1}{x_n} defined by the table $infeasTuples$ of compatible tuples of values, and then performs Forward Checking:
      $$\forall \text{ tuple } i\ |\quad \collec{x_1}{x_n}\neq infeasTuples[i]$$
\end{notedef}

\begin{itemize}
	\item \textbf{API} : \mylst{infeasTupleFC(List<int[]> infeasTuples, IntegerVariable... x)}
	\item \textbf{return type}: \texttt{Constraint}
	\item \textbf{options} : \emph{n/a}
	\item \textbf{favorite domain}: \emph{to complete}
\end{itemize}

\textbf{Example}:
\lstinputlisting{java/cinfeastuplefc.j2t}

%\part{intdiv}
\label{intdiv}
\hypertarget{intdiv}{}

\section{intDiv (constraint)}\label{intdiv:intdivconstraint}\hypertarget{intdiv:intdivconstraint}{}
\begin{notedef}
  \texttt{intDiv}$(x,y,z)$ states that $z$ is equal to the integer quotient of $x$ by $y$:
$$z = \lfloor x / y \rfloor$$
\end{notedef}

\begin{itemize}
	\item \textbf{API}: \mylst{intDiv(IntegerVariable x, IntegerVariable y, IntegerVariable z)}
	\item \textbf{return type} : \texttt{Constraint}
	\item \textbf{option} : \emph{n/a}
	\item \textbf{favorite domain}: bound
\end{itemize}

\textbf{Example}:
\lstinputlisting{java/cintdiv.j2t}

%\part{inversechanneling}
\label{inversechanneling}
\hypertarget{inversechanneling}{}

\section{inverseChanneling (constraint)}\label{inversechanneling:inversechannelingconstraint}\hypertarget{inversechanneling:inversechannelingconstraint}{}
\begin{notedef}
  \texttt{inverseChanneling}$(\collec{x_1}{x_n},\collec{y_1}{y_m})$ states 
%a channeling between two arrays  $x$ and $y$ of integer variables with the same domain.It enforces 
that $x_i$ has value $j$ if and only if $y_j$ has value $i$:
$$x_i = j\quad\iff\quad y_j = i,\qquad\forall i=1..n, j=1..m$$
\end{notedef}
\begin{itemize}
	\item \textbf{API} : \mylst{inverseChanneling(IntegerVariable[] x, IntegerVariable[] y)}
	\item \textbf{return type} : \texttt{Constraint}
	\item \textbf{options} : \emph{no options}
	\item \textbf{favorite domain} : enumerated for x and y
	\item \textbf{references} :\\
      global constraint catalog: \href{http://www.emn.fr/x-info/sdemasse/gccat/Cinverse.html}{inverse}
\end{itemize}

\textbf{Example}:
\lstinputlisting{java/cinversechanneling.j2t}

%!TEX root = ../../content-doc.tex
%\part{inversechanneling}
\label{inversechannelingwithinrange}
\hypertarget{inversechannelingwithinrange}{}

\section{inverseChannelingWithinRange (constraint)}\label{inversechannelingwithinrange:inversechannelingconstraintwithinrange}\hypertarget{inversechannelingwithinrange:inversechannelingconstraintwithinrange}{}
\begin{notedef}
  \texttt{inverseChannelingWithinRange}$(\collec{x_1}{x_n},\collec{y_1}{y_m})$ states that
%a channeling between two arrays  $x$ and $y$ of integer variables with the same domain.It enforces 
if $x_i$ is assigned to $j$ and if $j$ is less than or equal to the number of items of the collection Y then $y_j$ is assigned to $i$.
Conversely, if $y_j$ is assigned to $i$ and if $i$ is less than or equal to the number of items of the collection X then $x_i$ is assigned to $j$.

$$x_i = j \wedge j < |y| \quad\iff\quad y_j = i \wedge i < |x|,\qquad i \in[1,n],  j\in[1,m]$$
\end{notedef}
\begin{itemize}
	\item \textbf{API} : \mylst{inverseChanneling(IntegerVariable[] x, IntegerVariable[] y)}
	\item \textbf{return type} : \texttt{Constraint}
	\item \textbf{options} : \emph{no options}
	\item \textbf{favorite domain} : enumerated for x and y
	\item \textbf{references} :\\
      global constraint catalog: \href{http://www.emn.fr/z-info/sdemasse/gccat/Cinverse_within_range.html}{inverse\_within\_range}
\end{itemize}

\textbf{Example}:
\lstinputlisting{java/cinversechannelingwithinrange.j2t}

%\part{inversechanneling}
\label{inverseset}
\hypertarget{inverseset}{}

\section{inverseSet (constraint)}\label{inverseset:inversesetconstraint}\hypertarget{inverseset:inversesetconstraint}{}
\begin{notedef}
  \texttt{inverseSet}$(\collec{x_1}{x_n},\collec{y_1}{y_m})$ states
that $x_i$ has (if an integer variable or contains, if a set variable) value $j$ if and only if $y_j$ contains value $i$:

If $x$ is a collection of integer variables:
$$x_i = j\quad\iff\quad i\in s_j,\qquad\forall i=0..n-1,j=0..m-1$$
Notice that this version induces that $y$ becomes a partition of the set of the indices of collection $x$.


If $x$ is a collection of set variables:
$$j \in x_i \quad\iff\quad i\in s_j,\qquad\forall i=0..n-1,j=0..m-1$$




\end{notedef}
\begin{itemize}
	\item \textbf{API} :
	\begin{itemize}
		\item \mylst{inverseSet(IntegerVariable[] x, SetVariable[] y)}
		\item \mylst{inverseSet(SetVariable[] x, SetVariable[] y)}
	\end{itemize}
	\item \textbf{return type} : \texttt{Constraint}
	\item \textbf{options} : \emph{no options}
	\item \textbf{favorite domain} : enumerated for x
	\item \textbf{references} :\\
      global constraint catalog: \href{http://www.emn.fr/x-info/sdemasse/gccat/Cinverse_set.html}{inverse\_set}
\end{itemize}

\textbf{Example}:
\lstinputlisting{java/cinverseset.j2t}

%\part{isincluded}
\label{isincluded}
\hypertarget{isincluded}{}

\section{isIncluded (constraint)}\label{isincluded:isincludedconstraint}\hypertarget{isincluded:isincludedconstraint}{}
\begin{notedef}
  \texttt{isIncluded}$(s_1,s_2)$ states that set $s_1$ is included in set $s_2$:
 $$s_1\subseteq s_2$$
\end{notedef}

\begin{itemize}
	\item \textbf{API} : \mylst{isIncluded(SetVariable s1, SetVariable s2)}
	\item \textbf{return type} : \texttt{Constraint}
	\item \textbf{options} :\emph{n/a}
	\item \textbf{favorite domain} : \emph{to complete}
\end{itemize}

\textbf{Example}:
\lstinputlisting{java/cisincluded.j2t}

%\part{isnotincluded}
\label{isnotincluded}
\hypertarget{isnotincluded}{}

\section{isNotIncluded (constraint)}\label{isnotincluded:isnotincludedconstraint}\hypertarget{isnotincluded:isnotincludedconstraint}{}
\begin{notedef}
  \texttt{isNotIncluded}$(s_1,s_2)$ states that set $s_1$ is not included in set $s_2$:
 $$s_1\not\subseteq s_2$$
\end{notedef}

\begin{itemize}
	\item \textbf{API} : \mylst{isNotIncluded(SetVariable s1, SetVariable s2)}
	\item \textbf{return type} : \texttt{Constraint}
	\item \textbf{options} :\emph{n/a}
	\item \textbf{favorite domain} : \emph{to complete}
\end{itemize}

\textbf{Example}:
\lstinputlisting{java/cisnotincluded.j2t} 

\section{knapsackProblem (constraint)}\label{knapsackproblem:knapsackproblemconstraint}\hypertarget{knapsackproblem:knapsackproblemconstraint}{}
\begin{notedef}
  \texttt{knapsackProblem}$(z^1, z^2, \collec{x_1}{x_n}, \collec{c^1_1}{c^1_n}, \collec{c^2_1}{c^2_n})$ states that $z^1$ (respectively, $z^2$) is the sum of the $x$ weighted by the costs $c^1$ (respectively, $c^2$):
$$\sum_{i=1}^{n}x_{i}c^1_{i}=z^1\quad \wedge\quad \sum_{i=1}^{n}x_{i}c^2_{i}=z^2$$
\end{notedef}
The knaspack problem can be modeled using only this constraint and the objective \mylst{maximize(z1)}: $x_i$ is the number of items of type $i$ and each item of type $i$ has a value $c^1_i$ and a weight $c^2_i$.
Based on \hyperlink{costregular:costregularconstraint}{\texttt{costRegular}}, this propagator simulates the dynamic programming approach of~\cite{TrickAOR03}.
It dominates the filtering of the decomposition in two \hyperlink{equation}{\texttt{equation}} constraints.

\begin{itemize}
	\item \textbf{API} : \mylst{knapsackProblem(IntegerVariable z1, IntegerVariable z2, IntegerVariable[] x, int[] c1, int[] c2)}
	\item \textbf{return type} : \texttt{Constraint}
	\item \textbf{options} : \emph{n/a}
	\item \textbf{favorite domain} : \emph{n/a}
	\item \textbf{references} : 
		\cite{TrickAOR03}: \emph{A Dynamic Programming Approach for Consistency and Propagation for Knapsack Constraints}
\end{itemize}

\textbf{Example}:
\lstinputlisting{java/cknapsack.j2t}


%\part{leq}
\label{leq}
\hypertarget{leq}{}

\section{leq (constraint)}\label{leq:leqconstraint}\hypertarget{leq:leqconstraint}{}
\begin{notedef}
  \texttt{leq}$(x,y)$ states that $x$ is less than or equal to $y$:
$$x \le y$$
\end{notedef}

\begin{itemize}
	\item \textbf{API} :
	\begin{itemize}
		\item \mylst{leq(IntegerExpressionVariable x, IntegerExpressionVariable y)}
		\item \mylst{leq(IntegerExpressionVariable x, int y)}
		\item \mylst{leq(int x, IntegerExpressionVariable y)}
		\item \mylst{leq(RealExpressionVariable x, RealExpressionVariable y)}
		\item \mylst{leq(RealExpressionVariable x, double y)}
		\item \mylst{leq(double x, RealExpressionVariable y)}
	\end{itemize}
	\item \textbf{return type} : \texttt{Constraint}
	\item \textbf{options} : \emph{n/a}
	\item \textbf{favorite domain} : \emph{to complete}.
	\item \textbf{references} :\\
      global constraint catalog: \href{http://www.emn.fr/x-info/sdemasse/gccat/Cleq.html}{leq}
\end{itemize}

\textbf{Example:}
\lstinputlisting{java/cleq.j2t}

%\part{leqcard}
\label{leqcard}
\hypertarget{leqcard}{}

\section{leqCard (constraint)}\label{leqcard:leqcardconstraint}\hypertarget{leqcard:leqcardconstraint}{}
\begin{notedef}
  \texttt{leqCard}$(s,z)$ states that the cardinality of set $s$ is less than or equal to $z$:
$$|s| \le z$$
\end{notedef}

\begin{itemize}
	\item \textbf{API} :
	\begin{itemize}
		\item \mylst{leqCard(SetVariable s, IntegerVariable z)}
		\item \mylst{leqCard(SetVariable s, int z)}
	\end{itemize}
	\item \textbf{return type} : \texttt{Constraint}
	\item \textbf{options} : \emph{n/a}
	\item \textbf{favorite domain} : \emph{to complete}
\end{itemize}

\textbf{Example}:
\lstinputlisting{java/cleqcard.j2t}

%\part{lex}
\label{lex}
\hypertarget{lex}{}

\section{lex (constraint)}\label{lex:lexconstraint}\hypertarget{lex:lexconstraint}{}
\begin{notedef}
  \texttt{lex}$(\collec{x_1}{x_n},\collec{y_1}{y_n})$ states a strict lexicographic ordering  $x <_{lex} y$:
$$\exists\ j=1..n\ |\qquad x_j<y_j\quad \land\quad x_i=y_i\ (\forall\  i<j)$$
\end{notedef}

\begin{itemize}
	\item \textbf{API} : \mylst{lex(IntegerVariable[] x, IntegerVariable[] y)}
	\item \textbf{return type} : \texttt{Constraint}
	\item \textbf{options} :\emph{n/a}
	\item \textbf{favorite domain} : \emph{to complete}
	\item \textbf{references} :
      \begin{itemize}
      \item \cite{FrischCP02}: \emph{Global Constraints for Lexicographic Orderings}
      \item global constraint catalog: \href{http://www.emn.fr/x-info/sdemasse/gccat/Clex_less.html}{lex\_less}
      \end{itemize}
\end{itemize}

\textbf{Example}:
\lstinputlisting{java/clex.j2t}

%\part{lexchain}
\label{lexchain}
\hypertarget{lexchain}{}

\section{lexChain (constraint)}\label{lexchain:lexchainconstraint}\hypertarget{lexchain:lexchainconstraint}{}
\begin{notedef}
  \texttt{lexChain}$(\collec{x^1_1}{x^1_n},\ldots,\collec{x^p_1}{x^p_n})$ states a strict lexicographic ordering on a chain of $p$ integer vectors:
$$x^1 <_{lex} x^2 <_{lex}\cdots <_{lex} x^p$$
%where $X^1$ contains up to $n$ variables. 
\end{notedef}

\begin{itemize}
	\item \textbf{API} : \mylst{lexChain(IntegerVariable[]... x)}
	\item \textbf{return type} : \texttt{Constraint}
	\item \textbf{options} : \emph{n/a}
	\item \textbf{favorite domain} : \emph{to complete}
	\item \textbf{references} :
      \begin{itemize}
      \item \cite{BeldiceanuSICS02} \emph{Arc-Consistency for a chain of Lexicographic Ordering Constraints} 
      \item global constraint catalog: \href{http://www.emn.fr/x-info/sdemasse/gccat/Clex_chain_less.html}{lex\_chain\_less}
      \end{itemize}
\end{itemize}

\textbf{Example}:
\lstinputlisting{java/clexchain.j2t}

%\part{lexchaineq}
\label{lexchaineq}
\hypertarget{lexchaineq}{}

\section{lexChainEq (constraint)}\label{lexchaineq:lexchaineqconstraint}\hypertarget{lexchaineq:lexchaineqconstraint}{}
\begin{notedef}
  \texttt{lexChainEq}$(\collec{x^1_1}{x^1_n},\ldots,\collec{x^p_1}{x^p_n})$ states a lexicographic ordering on a chain of $p$ integer vectors:
$$x^1 \le_{lex} x^2 \le_{lex}\cdots\le_{lex} x^p$$
%where $X^1$ contains up to $n$ variables. 
\end{notedef}

\begin{itemize}
	\item \textbf{API} : \mylst{lexChainEq(IntegerVariable[]... x)}
	\item \textbf{return type} : \texttt{Constraint}
	\item \textbf{options} : \emph{n/a}
	\item \textbf{favorite domain} : \emph{to complete}
	\item \textbf{references} :
      \begin{itemize}
      \item \cite{BeldiceanuSICS02} \emph{Arc-Consistency for a chain of Lexicographic Ordering Constraints} 
      \item global constraint catalog: \href{http://www.emn.fr/x-info/sdemasse/gccat/Clex_chain_lesseq.html}{lex\_chain\_lesseq}
      \end{itemize}
\end{itemize}

\textbf{Example}:
\lstinputlisting{java/clexchaineq.j2t}

%\part{lexeq}
\label{lexeq}
\hypertarget{lexeq}{}

\section{lexEq (constraint)}\label{lexeq:lexeqconstraint}\hypertarget{lexeq:lexeqconstraint}{}
\begin{notedef}
  \texttt{lexEq}$(\collec{x_1}{x_n},\collec{y_1}{y_n})$ states a lexicographic ordering  $x \le_{lex} y$:
$$\exists\ j=1..n\ |\qquad x_j\le y_j\quad \land\quad x_i=y_i\ (\forall\  i<j)$$
\end{notedef}

\begin{itemize}
	\item \textbf{API} : \mylst{lexEq(IntegerVariable[] x, IntegerVariable[] y)}
	\item \textbf{return type} : \texttt{Constraint}
	\item \textbf{options} :\emph{n/a}
	\item \textbf{favorite domain} : \emph{to complete}
	\item \textbf{references} :
      \begin{itemize}
      \item \cite{FrischCP02}: \emph{Global Constraints for Lexicographic Orderings}
      \item global constraint catalog: \href{http://www.emn.fr/x-info/sdemasse/gccat/Clex_lesseq.html}{lex\_lesseq}
      \end{itemize}
\end{itemize}

\textbf{Example}:
\textbf{Example}:
\lstinputlisting{java/clexeq.j2t}

%\part{leximin}
\label{leximin}
\hypertarget{leximin}{}

\section{leximin (constraint)}\label{leximin:leximinconstraint}\hypertarget{leximin:leximinconstraint}{}

\todo{check the specifications of the implemented version.}

\begin{notedef}
\texttt{leximin}(\collec{x_1}{x_n}, \collec{y_1}{y_n}) states a strict lexicographic ordering $x'<_{lex} y'$, where $x'$ and $y'$ are the permutations of $x$ and $y$ respectively sorted in increasing order.
$$\hyperlink{sorting}{\texttt{sorting}}(\collec{x_1}{x_n},\collec{x'_1}{x'_n})\ \land\ \hyperlink{sorting}{\texttt{sorting}}(\collec{y_1}{y_n},\collec{y'_1}{y'_n})\ \land\ \hyperlink{lex}{\texttt{lex}}(\collec{x'_1}{x'_n},\collec{y'_1}{y'_n})$$

  \end{notedef}

\begin{itemize}
	\item \textbf{API} :
	\begin{itemize}
		\item \mylst{leximin(IntegerVariable[] x, IntegerVariable[] y)}
		\item \mylst{leximin(int[] x, IntegerVariable[] y)}
	\end{itemize}
	\item \textbf{return type} : \texttt{Constraint}
	\item \textbf{options} :\emph{n/a}
	\item \textbf{favorite domain} : \emph{to complete}
	\item \textbf{references} :
      \begin{itemize}
      \item \cite{FrischIJCAI03}: \emph{Multiset ordering constraints} 
      \item global constraint catalog: \href{http://www.emn.fr/x-info/sdemasse/gccat/Clex_lesseq_allperm.html}{lex\_lesseq\_allperm} (variant)
      \end{itemize}
\end{itemize}

\textbf{Example}:
\lstinputlisting{java/cleximin.j2t}

%\part{lt}
\label{lt}
\hypertarget{lt}{}

\section{lt (constraint)}\label{lt:ltconstraint}\hypertarget{lt:ltconstraint}{}
\begin{notedef}
  \texttt{lt}$(x,y)$ states that $x$ is strictly smaller than $y$:
$$x<y$$
\end{notedef}

\begin{itemize}
	\item \textbf{API} :
	\begin{itemize}
		\item \mylst{lt(IntegerExpressionVariable x, IntegerExpressionVariable y)}
		\item \mylst{lt(IntegerExpressionVariable x, int y)}
		\item \mylst{lt(int x, IntegerExpressionVariable y)}
	\end{itemize}
	\item \textbf{return type} : \texttt{Constraint}
	\item \textbf{options} : \emph{n/a}
	\item \textbf{favorite domain} : \emph{to complete}.
	\item \textbf{references} :\\
      global constraint catalog: \href{http://www.emn.fr/x-info/sdemasse/gccat/Clt.html}{lt}
\end{itemize}

\textbf{Example}:
\lstinputlisting{java/clt.j2t}

%\part{max}
\label{max}
\hypertarget{max}{}

\section{max (constraint)}\label{max:maxconstraint}\hypertarget{max:maxconstraint}{}

\subsection{max of a list}\label{max:maxofalist}\hypertarget{max:maxofalist}{}

\begin{notedef}
\texttt{max}$(x,z)$ states that $z$ is equal to the greater element of vector $x$:
$$z = \max(x_1, x_2, ..., x_n)$$  
\end{notedef}

\begin{itemize}
	\item \textbf{API}:
	\begin{itemize}
		\item \mylst{max(IntegerVariable[] x, IntegerVariable z)}
		\item \mylst{max(IntegerVariable x1, IntegerVariable x2, IntegerVariable z)}
		\item \mylst{max(int x1, IntegerVariable x2, IntegerVariable z)}
		\item \mylst{max(IntegerVariable x1, int x2, IntegerVariable z)}
	\end{itemize}
	\item \textbf{return type}: \texttt{Constraint}
	\item \textbf{options} : \emph{n/a}
	\item \textbf{favorite domain} : \emph{to complete}
	\item \textbf{references} :\\
      global constraint catalog: \href{http://www.emn.fr/x-info/sdemasse/gccat/Cmaximum.html}{maximum}
\end{itemize}

\textbf{Example}:
\lstinputlisting{java/cmax1.j2t}

\subsection{max of a set}\label{max:maxofaset}\hypertarget{max:maxofaset}{}

\begin{notedef}
\texttt{max}$(s,x,z)$ states that $z$ is equal to the greater element of vector $x$ whose index is in set $s$:
$$z = \max_{i\in s}( x_i )$$
  \end{notedef}

\begin{itemize}
	\item \textbf{API}:
	\begin{itemize}
		\item \mylst{max(SetVariable s, IntegerVariable[] x, IntegerVariable z)}
	\end{itemize}
	\item \textbf{return type}: \texttt{Constraint}
	\item \textbf{options} : \emph{n/a}
	\item \textbf{favorite domain} : \emph{to complete}
\end{itemize}

\textbf{Example}:
\lstinputlisting{java/cmax2.j2t}

%\part{member}
\label{member}
\hypertarget{member}{}

\section{member (constraint)}\label{member:memberconstraint}\hypertarget{member:memberconstraint}{}

\begin{notedef}
  \texttt{member}$(x,s)$ states that integer $x$ belongs to set $s$:
$$x\in s$$
\end{notedef}

\begin{itemize}
	\item \textbf{API} :
	\begin{itemize}
		\item \mylst{member(int x, SetVariable s)}
		\item \mylst{member(SetVariable s, int x)}
		\item \mylst{member(SetVariable s, IntegerVariable x)}
		\item \mylst{member(IntegerVariable x, SetVariable s)}
		\item \mylst{member(member(SetVariable sv, IntegerVariable... vars)}
		\item \mylst{member(IntegerVariable x, int[] s)}
		\item \mylst{member(IntegerVariable x, int lower, int upper)}
	\end{itemize}
	\item \textbf{return type} : \texttt{Constraint}
	\item \textbf{options} :\emph{n/a}
	\item \textbf{favorite domain} : \emph{to complete}
	\item \textbf{references} :\\
      global constraint catalog: \href{http://www.emn.fr/x-info/sdemasse/gccat/Cin_set.html}{in\_set}
\end{itemize}

\textbf{Examples}:
1. using a set variable
\lstinputlisting{java/cmember.j2t}

2. using an array of integers
\lstinputlisting{java/camong1.j2t}

%\part{min}
\label{min}
\hypertarget{min}{}

\section{min (constraint)}\label{min:minconstraint}\hypertarget{min:minconstraint}{}

\subsection{min of a list}\label{min:minofalist}\hypertarget{min:minofalist}{}

\begin{notedef}
  \texttt{min}$(x,z)$ states that $z$ is equal to the smaller element
  of vector $x$:
$$z = \min(x_1, x_2, ..., x_n).$$
\end{notedef}
\begin{itemize}
	\item \textbf{API}:
	\begin{itemize}
		\item \mylst{min(IntegerVariable[] x, IntegerVariable z)}
		\item \mylst{min(IntegerVariable x1, IntegerVariable x2, IntegerVariable z)}
		\item \mylst{min(int x1, IntegerVariable x2, IntegerVariable z)}
		\item \mylst{min(IntegerVariable x1, int x2, IntegerVariable z)}
	\end{itemize}
	\item \textbf{return type}: \texttt{Constraint}
	\item \textbf{options} : \emph{n/a}
	\item \textbf{favorite domain} : \emph{to complete}
	\item \textbf{references} :\\
      global constraint catalog: \href{http://www.emn.fr/x-info/sdemasse/gccat/Cminimum.html}{minimum}
\end{itemize}

\textbf{Example}:
\lstinputlisting{java/cmin1.j2t}

\subsection{min of a set}\label{min:minofaset}\hypertarget{min:minofaset}{}

\begin{notedef}
  \texttt{min}$(s,x,z)$ states that $z$ is equal to the smaller
  element of vector $x$ whose index is in set $s$:
$$z = \min_{i\in s}( x_i ).$$
\end{notedef}
\begin{itemize}
	\item \textbf{API}:
	\begin{itemize}
		\item \mylst{min(SetVariable s,IntegerVariable[] x, IntegerVariable z)}
	\end{itemize}
	\item \textbf{return type}: \texttt{Constraint}
	\item \textbf{options} : \emph{n/a}
	\item \textbf{favorite domain} : \emph{to complete}
\end{itemize}

\textbf{Example}:
\lstinputlisting{java/cmin2.j2t}

%\part{mod}
\label{mod}
\hypertarget{mod}{}

\section{mod (constraint)}\label{mod:modconstraint}\hypertarget{mod:modconstraint}{}
\begin{notedef}
  \texttt{mod}$(x_1,x_2,x_3)$ states that $x_1$ is congruent to $x_2$
  modulo $x_3$:
$$x_1 \equiv x_2 \mod x_3$$
\end{notedef}
\begin{itemize}
	\item \textbf{API} : \mylst{mod(IntegerVariable x1, IntegerVariable x2, int x3)}
	\item \textbf{return type} : \texttt{Constraint}
	\item \textbf{options} : \emph{n/a}
	\item \textbf{favorite domain} : \emph{n/a}
\end{itemize}

\textbf{Example}:
\lstinputlisting{java/clex.j2t}

%\part{multicostregular}
\label{multicostregular}
\hypertarget{multicostregular}{}

\section{multiCostRegular (constraint)}\label{multicostregular:multicostregularconstraint}\hypertarget{multicostregular:multicostregularconstraint}{}
\begin{notedef}
  \texttt{multiCostRegular}$(\collec{z_1}{z_p}, \collec{x_1}{x_n},\mathcal{L}(\Pi), \coll{c_{i,j,k}})$ states that sequence \collec{x_1}{x_n} is a word belonging to the regular language $\mathcal{L}(\Pi)$ and that each $z_k$ is its cost computed as the sum of the individual symbol weights $c_{i,x_i,k}$: 
$$ \collec{x_1}{x_n} \in \mathcal{L}(\Pi)\quad\land\quad \sum_{i=1}^n c_{i,x_i,k} = z_k,\ \forall k=1..p.$$ 
\end{notedef}
\texttt{multiCostRegular} models the conjunction of $p$ \hyperlink{costregular}{\texttt{costRegular}} constraints, or the conjunction of a \hyperlink{regular}{\texttt{regular}} constraint with $p$ assignment cost functions. Like them, it is useful for modelling sequencing rules in personnel scheduling and rostering problems. Furthermore it allows to handle together several linear counters and costs on the sequence $x$. For example, one may count all the value occurrences like with a \hyperlink{globalcardinality}{\texttt{globalCardinality}} constraint. Counters can also model assignment costs in optimization problems or violation costs in over-constrained problems. For example, counting the occurrence number of a pattern allows to determine the violation cost of a soft forbidden pattern rule.

The filtering algorithm~\cite{MenanaCPAIOR09} of \texttt{multiCostRegular} does not achieve GAC (as it would be NP-hard) but it dominates the decompositions in \hyperlink{regular}{\texttt{regular}} or \hyperlink{costregular}{\texttt{costRegular}} constraints. 

The accepting language is specified by a deterministic finite automaton (DFA) $\Pi$ encoded as an object of class \texttt{Automaton} (see \hyperlink{costregular}{\texttt{costRegular}} for a short API).
The cost functions are vectors of weights on the transitions of $\Pi$. They are encoded as one matrix \texttt{int c[n][m][p][pi.getNbStates()]} such that
\texttt{c[i][j][k][s]} is the cost of assigning variable $x_i$ to value $j$ at state $s$ on dimension $k$. When the transition costs are independent of their initial states, a second API allows to specify a cost matrix \texttt{int c[n][m][p]}.

\begin{itemize}
	\item \textbf{API} : 
      \begin{itemize}
      \item \mylst{multiCostRegular(IntegerVariable[] z, IntegerVariable[] x, FiniteAutomaton pi, int[][][] c)}
      \item \mylst{multiCostRegular(IntegerVariable[] z, IntegerVariable[] x, FiniteAutomaton pi, int[][][][] c)}
      \end{itemize}
	\item \textbf{return type} : \texttt{Constraint}
	\item \textbf{options} : n/a
%      \begin{itemize}
%      \item \texttt{MultiCostRegular.DATA\_STRUCT} is  \texttt{MultiCostRegular.BITSET} or \texttt{MultiCostRegular.LIST}: a parameter stating which backtrable data structure to use for storing the outgoing arcs of the layered digraph. The observed behaviour is until $1000$ arcs the bipartite list is much more efficient, afterwards the memory efficiency of the bitset representation allow faster operations. 
%      \item \texttt{MultiCostRegular.U0}, \texttt{MultiCostRegular.R0}, \texttt{MultiCostRegular.MAXNONIMPROVEITER}, and \texttt{MultiCostRegular.MAXBOUNDITER} are value parameters of the subgradient algorithm used for solving the lagrangean relaxation.
%      \item \texttt{MultiCostRegular.D\_PREC} is a double parameter stating the precision of float computation. It is set by default to $10^{-5}$.
%      \end{itemize}
	\item \textbf{favorite domain} : \emph{to complete}
	\item \textbf{references} :\\
       \cite{MenanaCPAIOR09}: \emph{Sequencing and Counting with the {\tt multicost-regular} Constraint}
\end{itemize}
%\begin{notedef}
%  For further informations, see the multicost-regular description.
%\end{notedef}

\textbf{Example}:
\lstinputlisting{java/cmulticosteregular_import.j2t}
\lstinputlisting{java/cmulticostregular.j2t}


%\part{and}
\label{nand}
\hypertarget{nand}{}

\section{nand (constraint)}\label{nand:nandconstraint}\hypertarget{nand:nandconstraint}{}
\begin{notedef}
  \texttt{nand}$(\collec{C_1}{C_n})$ states that at least one constraint in arguments is not satisfied:
$$ C_1 \uparrow C_2 \uparrow\ldots\uparrow C_n$$

  \texttt{nand}$(\collec{b_1}{b_n})$ states that at least one boolean in arguments is false:
$$ (b_1=1)\ \uparrow\ (b_2=1)\ \uparrow\ \ldots\ \uparrow\ (b_n=1)$$
\end{notedef}

\begin{itemize}
\item \textbf{API} : 
\begin{itemize}
\item \mylst{nand(Constraint... c)}
\item \mylst{nand(IntegerVariable... b)}
\end{itemize}
\item \textbf{return type} : \texttt{Constraint}
\item \textbf{options} : \emph{n/a}
\item \textbf{favorite domain} : \emph{n/a}
\item \textbf{references} :\\
  global constraint catalog: \href{http://www.emn.fr/z-info/sdemasse/gccat/Cnand.html}{\tt nand}
\end{itemize}

\textbf{Examples:}
\begin{itemize}
	\item example1:
\end{itemize}
\lstinputlisting{java/cnand1.j2t}
\begin{itemize}
	\item example2
\end{itemize}

\lstinputlisting{java/cnand2.j2t}


%\part{neq}
\label{neq}
\hypertarget{neq}{}

\section{neq (constraint)}\label{neq:neqconstraint}\hypertarget{neq:neqconstraint}{}

\begin{notedef}
  \texttt{neq} states that the two arguments are different:
$$x \neq y.$$
\end{notedef}
\begin{itemize}
	\item \textbf{API} :
	\begin{itemize}
		\item \mylst{neq(IntegerExpressionVariable x, IntegerExpressionVariable y)}
		\item \mylst{neq(IntegerExpressionVariable x, int y)}
		\item \mylst{neq(int x, IntegerExpressionVariable y)}
	\end{itemize}
	\item \textbf{return type} : \texttt{Constraint}
	\item \textbf{options} : \emph{n/a}
	\item \textbf{favorite domain} : \emph{to complete}.
	\item \textbf{references} :\\
      global constraint catalog: \href{http://www.emn.fr/x-info/sdemasse/gccat/Cneq.html}{neq}
\end{itemize}

\textbf{Examples:}
\begin{itemize}
	\item example1:
\end{itemize}

\lstinputlisting{java/cneq1.j2t}

\begin{itemize}
	\item example2
\end{itemize}

\lstinputlisting{java/cneq2.j2t}
%\part{neqcard}
\label{neqcard}
\hypertarget{neqcard}{}

\section{neqCard (constraint)}\label{neqcard:neqcardconstraint}\hypertarget{neqcard:neqcardconstraint}{}
\begin{notedef}
  \texttt{neqCard}$(s,z)$ states that the cardinality of set $s$ is not equal to $z$:
$$|s| \neq z$$
\end{notedef}

\begin{itemize}
	\item \textbf{API} :
	\begin{itemize}
		\item \mylst{neqCard(SetVariable s, IntegerVariable z)}
		\item \mylst{neqCard(SetVariable s, int z)}
	\end{itemize}
	\item \textbf{return type} : \texttt{Constraint}
	\item \textbf{options} : \emph{n/a}
	\item \textbf{favorite domain} : \emph{to complete}
\end{itemize}

\textbf{Example}:
\lstinputlisting{java/cneqcard.j2t}

%\part{not}
\label{not}
\hypertarget{not}{}

\section{not (constraint)}\label{not:notconstraint}\hypertarget{not:notconstraint}{}
\begin{notedef}
  \texttt{not}$(c)$ holds if and only if constraint $c$ does not hold:
$$\neg c$$
\end{notedef}
\begin{itemize}
	\item \textbf{API} : \mylst{not(Constraint c)}
	\item \textbf{return type} : \texttt{Constraint}
	\item \textbf{options} : \emph{n/a}
	\item \textbf{favorite domain} : \emph{n/a}
\end{itemize}

\textbf{Example} : 
\lstinputlisting{java/cnot.j2t}

%\part{notmember}
\label{notmember}
\hypertarget{notmember}{}

\section{notMember (constraint)}\label{notmember:notmemberconstraint}\hypertarget{notmember:notmemberconstraint}{}
\begin{notedef}
\texttt{notMember}$(x,s)$ states that integer $x$ does not belong to $s$:
$$x\not\in s$$  
\end{notedef}

\begin{itemize}
	\item \textbf{API} :
	\begin{itemize}
		\item \mylst{notMember(int x, SetVariable s)}
		\item \mylst{notMember(SetVariable s, int x)}
		\item \mylst{notMember(SetVariable s, IntegerVariable x)}
		\item \mylst{notMember(IntegerVariable x, SetVariable s)}
		\item \mylst{notMember(IntegerVariable x, int[] s)}
		\item \mylst{notMember(IntegerVariable x, int lower, int upper)}
	\end{itemize}
	\item \textbf{return type} : \texttt{Constraint}
	\item \textbf{options} :\emph{n/a}
	\item \textbf{favorite domain} : \emph{to complete}
\end{itemize}

\textbf{Example}:
\begin{itemize}
\item with a set variable
\end{itemize}
\lstinputlisting{java/cnotmember.j2t}

\begin{itemize}
\item with a collection of values
\end{itemize}
\lstinputlisting{java/cdisjoint1.j2t}

%\part{nor}
\label{nor}
\hypertarget{nor}{}

\section{nor (constraint)}\label{nor:norconstraint}\hypertarget{nor:norconstraint}{}
\begin{notedef}
  \texttt{nor}$(c_1,\ldots,c_n)$ states that at least one constraint in arguments is satisfied:
$$ c_1 \downarrow c_2 \downarrow\ldots\downarrow c_n$$

  \texttt{nor}$(b_1,\ldots,b_n)$ states that at least one boolean variable in argument is true:
$$ (b_1=1) \downarrow (b_2=1) \downarrow\ldots\downarrow (b_n=1)$$
\end{notedef}

\begin{itemize}
\item \textbf{API} : 
\begin{itemize}
\item \mylst{nor(Constraint... c)}
\item \mylst{nor(IntegerVariable... b)}
\end{itemize}
	\item \textbf{return type} : \texttt{Constraint}
	\item \textbf{options} : \emph{n/a}
	\item \textbf{favorite domain} : \emph{n/a}
	\item \textbf{references} :\\
  global constraint catalog: \href{http://www.emn.fr/z-info/sdemasse/gccat/Cnor.html}{\tt nor}
\end{itemize}

\textbf{Examples:}
\begin{itemize}
	\item example1:
\end{itemize}
\lstinputlisting{java/cnor1.j2t}
\begin{itemize}
	\item example2
\end{itemize}
\lstinputlisting{java/cnor2.j2t}

%\part{nth}
\label{nth}
\hypertarget{nth}{}

\section{nth (constraint)}\label{nth:nthconstraint}\hypertarget{nth:nthconstraint}{}
\texttt{nth} is the well known \emph{element} constraint.
Several APIs are available: 

\begin{notedef}
\begin{itemize}
\item \texttt{nth}$(i,\collec{x_0}{x_{n}},y)$ states that $y=x_i$
\item \texttt{nth}$(i,\collec{x_0}{x_{n}},y,o)$ ensures that $y=x_{i-o}$ ($o$ is an \emph{offset} for shifting values)
\item \texttt{nth}$(i,j,\coll{x_{i,j}},y)$ ensures that $y=x_{i,j}$
\end{itemize}
\end{notedef}

\begin{itemize}
	\item \textbf{API} :
	\begin{itemize}
		\item \mylst{nth(IntegerVariable i, int[] x, IntegerVariable y)}
		\item \mylst{nth(String option, IntegerVariable i, int[] x, IntegerVariable y)}
		\item \mylst{nth(IntegerVariable i, IntegerVariable[] x, IntegerVariable y)}
		\item \mylst{nth(IntegerVariable i, int[] x, IntegerVariable y, int offset)}
		\item \mylst{nth(String option, IntegerVariable i, int[] x, IntegerVariable y, int offset)}		
		\item \mylst{nth(IntegerVariable i, IntegerVariable[] x, IntegerVariable y, int offset)}
		\item \mylst{nth(String option, IntegerVariable i, IntegerVariable[] x, IntegerVariable y, int offset)}
		\item \mylst{nth(IntegerVariable i, IntegerVariable j, int[][] x, IntegerVariable y)}
	\end{itemize}
	\item \textbf{return type} : \texttt{Constraint}
	\item \textbf{options} :
	\begin{itemize}
		\item \emph{no option} 
		\item \hyperlink{cnthg:cnthgoptions}{\tt Options.C\_NTH\_G} for global consistency
	\end{itemize}
	\item \textbf{favorite domain} : \emph{to complete}
	\item \textbf{references} :\\
      global constraint catalog: \href{http://www.emn.fr/x-info/sdemasse/gccat/Celement.html}{element}
\end{itemize}

\textbf{Example}:
\lstinputlisting{java/cnth.j2t} 

%\part{occurrence}
\label{occurrence}
\hypertarget{occurrence}{}

\section{occurrence (constraint)}\label{occurrence:occurrenceconstraint}\hypertarget{occurrence:occurrenceconstraint}{}
\begin{notedef}
  \texttt{occurrence}$(v,z,x)$ states that $z$ is equal to the number of elements in $x$ with value $v$:
$$z=|\{i\ |\ x_i=v\}|$$   
\end{notedef}
  This is a specialization of \hyperlink{globalcardinality:globalcardinalityconstraint}{\tt globalCardinality} with only one value counter, and of \hyperlink{among:amongconstraint}{\tt among} with exactly one counted value.

\begin{itemize}
	\item \textbf{API}: 
      \begin{itemize}
      \item \mylst{occurrence(IntegerVariable z, IntegerVariable[] x, int v)}
      \item \mylst{occurrence(int z, IntegerVariable[] x, int v)}
      \item \mylst{occurrence(int v, IntegerVariable z, IntegerVariable... x)}
      \end{itemize}
	\item \textbf{return type} : \texttt{Constraint}
	\item \textbf{options} :\emph{n/a}
	\item \textbf{favorite domain} : \emph{to complete}
	\item \textbf{references} :\\
      global constraint catalog: \href{http://www.emn.fr/x-info/sdemasse/gccat/Ccount.html}{count}
\end{itemize}

\textbf{Example}:
\lstinputlisting{java/coccurrence.j2t}

%\part{occurrencemax}
\label{occurrencemax}
\hypertarget{occurrencemax}{}

\section{occurrenceMax (constraint)}\label{occurrencemax:occurrencemaxconstraint}\hypertarget{occurrencemax:occurrencemaxconstraint}{}
\begin{notedef}
  \texttt{occurrenceMax}$(v,z,x)$ states that $z$ is at least equal to the number of elements in $x$ with value $v$:
$$z\ge|\{i\ |\ x_i=v\}|$$   
\end{notedef}
  See also \hyperlink{occurrence:occurrenceconstraint}{\tt occurrence}.

\begin{itemize}
	\item \textbf{API}:
      \begin{itemize}
      \item \mylst{occurrenceMax(IntegerVariable z, IntegerVariable[] x, int v)}
      \item \mylst{occurrenceMax(int v, IntegerVariable z, IntegerVariable... x)}
      \end{itemize}
	\item \textbf{return type} : \texttt{Constraint}
	\item \textbf{options} :\emph{n/a}
	\item \textbf{favorite domain} : \emph{to complete}
	\item \textbf{references} :\\
      global constraint catalog: \href{http://www.emn.fr/x-info/sdemasse/gccat/Ccount.html}{count}
\end{itemize}

\textbf{Example}:
\lstinputlisting{java/coccurrencemax.j2t}

%\part{occurrencemin}
\label{occurrencemin}
\hypertarget{occurrencemin}{}

\section{occurrenceMin (constraint)}\label{occurrencemin:occurrenceminconstraint}\hypertarget{occurrencemin:occurrenceminconstraint}{}
\begin{notedef}
  \texttt{occurrenceMin}$(v,z,x)$ states that $z$ is at most equal to the number of elements in $x$ with value $v$:
$$z\le|\{i\ |\ x_i=v\}|$$   
\end{notedef}
  See also \hyperlink{occurrence:occurrenceconstraint}{\tt occurrence}.

\begin{itemize}
	\item \textbf{API}:
      \begin{itemize}
      \item \mylst{occurrenceMin(IntegerVariable z, IntegerVariable[] x, int v)}
      \item \mylst{occurrenceMin(int v, IntegerVariable z, IntegerVariable... x)}
      \end{itemize}
	\item \textbf{return type} : \texttt{Constraint}
	\item \textbf{options} :\emph{n/a}
	\item \textbf{favorite domain} : \emph{to complete}
	\item \textbf{references} :\\
      global constraint catalog: \href{http://www.emn.fr/x-info/sdemasse/gccat/Ccount.html}{count}
\end{itemize}

\textbf{Example}:
\lstinputlisting{java/coccurrencemin.j2t}

%\part{oppositesign}
\label{oppositesign}
\hypertarget{oppositesign}{}

\section{oppositeSign (constraint)}\label{oppositesign:oppositesignconstraint}\hypertarget{oppositesign:oppositesignconstraint}{}


\begin{notedef}
  \texttt{oppositeSign}$(x,y)$ states that the two arguments have opposite signs:
$$xy<0$$
\end{notedef}
Note that $0$ has both signs then constraint fails if $x$ or $y$ is equal to 0.

\begin{itemize}
	\item \textbf{API} : \mylst{oppositeSign(IntegerExpressionVariable x, IntegerExpressionVariable y)}
	\item \textbf{return type} : \texttt{Constraint}
	\item \textbf{options} :\emph{n/a}
	\item \textbf{favorite domain} : \emph{to complete}
\end{itemize}

\textbf{Example}:
\lstinputlisting{java/coppositesign.j2t} 

%\part{or}
\label{or}
\hypertarget{or}{}

\section{or (constraint)}\label{or:orconstraint}\hypertarget{or:orconstraint}{}
\begin{notedef}
  \texttt{or}$(c_1,\ldots,c_n)$ states that at least one constraint in arguments is satisfied:
$$ c_1 \lor c_2 \lor\ldots\lor c_n$$

  \texttt{or}$(b_1,\ldots,b_n)$ states that at least one boolean variable in argument is true:
$$ (b_1=1) \lor (b_2=1) \lor\ldots\lor (b_n=1)$$
\end{notedef}

\begin{itemize}
\item \textbf{API} : 
\begin{itemize}
\item \mylst{or(Constraint... c)}
\item \mylst{or(IntegerVariable... b)}
\end{itemize}
	\item \textbf{return type} : \texttt{Constraint}
	\item \textbf{options} : \emph{n/a}
	\item \textbf{favorite domain} : \emph{n/a}
	\item \textbf{references} :\\
  global constraint catalog: \href{http://www.emn.fr/z-info/sdemasse/gccat/Cor.html}{\tt or}
\end{itemize}

\textbf{Examples:}
\begin{itemize}
	\item example1:
\end{itemize}
\lstinputlisting{java/cor1.j2t}
\begin{itemize}
	\item example2
\end{itemize}
\lstinputlisting{java/cor2.j2t}

%\part{pack}
\label{pack}
\hypertarget{pack}{}

\section{pack (constraint)}\label{pack:packconstraint}\hypertarget{pack:packconstraint}{}

\begin{notedef}
  \texttt{pack(items, load, bin, size)} states that a collection of items is packed into different bins, such that the total size of the items in each bin does not exceed the bin capacity:
$$ \mathtt{load}[b] = \sum_{i\in\mathtt{items}[b]} \mathtt{size}[i],\quad\forall \text{ bin } b $$
%and
$$ i\in\mathtt{items}[b]\ \iff\ \mathtt{bin}[i]=b,\quad\forall \text{ bin } b,\ \forall \text{ item } i $$
\end{notedef}
%\texttt{pack}$(sizes, n, )$ states a collection of items (each of them having a specific size) is packed into different bins of given capacity such that the total weight of the items in each bin does not exceed the bin capacity.
\texttt{pack} is a \href{http://www.emn.fr/x-info/sdemasse/gccat/Cbin_packing.html}{bin packing constraint} based on \cite{ShawCP04}. 

\begin{itemize}
	\item \textbf{API} :
	\begin{itemize}
		\item \mylst{pack(SetVariable[] items, IntegerVariable[] load, IntegerVariable[] bin, IntegerConstantVariable[] size, String... options)}
		\item \mylst{pack(PackModeler modeler,String... options)}: PackModeler is a high-level modeling object.
		\item \mylst{pack(int[] sizes, int nbBins, int capacity, String... options)}: build instance with PackModeler.
	\end{itemize}
	\item \textbf{Variables}:
	\begin{itemize}
		\item \texttt{SetVariable[] items: items}$[b]$ is the set of items packed into bin $b$.
		\item \texttt{IntegerVariable[] load: load}$[b]$ is the total size of the items packed into bin $b$.
		\item \texttt{IntegerVariable[] bin: bin}$[i]$ is the bin where item $i$ is packed into.
		\item \texttt{IntegerConstantVariable[] size: size}$[i]$ is the size of item $i$.
	\end{itemize}
	\item \textbf{return type} : \texttt{Constraint}
	\item \textbf{options} : 	
      \begin{itemize}
      \item \hyperlink{cpackar:cpackaroptions}{SettingType.ADDITIONAL\_RULES.getOptionName()}: additional filtering rules \emph{recommended}
      \item \hyperlink{cpackdlb:cpackdlboptions}{SettingType.DYNAMIC\_LB.getOptionName()}: feasibility tests based on dynamic lower bounds for 1D-bin packing
      \item \hyperlink{cpackfill:cpackfilloptions}{SettingType.FILL\_BIN.getOptionName()}: dominance rule: fill a bin when an item fit into pertfectly equal-sized items and bins must be equivalent
      \item \hyperlink{cpacklbe:cpacklbeoptions}{SettingType.LAST\_BINS\_EMPTY.getOptionName()}: empty bins are the last ones 
      \end{itemize}
	\item \textbf{favorite domain} : \emph{to complete}
	\item \textbf{references} :
      \begin{itemize}
      \item \cite{ShawCP04}: \emph{A constraint for bin packing}
      \item global constraint catalog: \href{http://www.emn.fr/x-info/sdemasse/gccat/Cbin_packing.html}{bin\_packing} (variant)
      \end{itemize}
\end{itemize}

\textbf{Example}:

Take a look at \emph{samples.pack} to see advanced use of the constraint.
\lstinputlisting{java/cpack_import.j2t}
\lstinputlisting{java/cpack.j2t}

%\part{precedencereified}
\label{precedencereified}
\hypertarget{precedencereified}{}

\section{precedenceReified (constraint)}\label{precedencereified:precedencereifiedconstraint}\hypertarget{precedencereified:precedencereifiedconstraint}{}
\begin{notedef}
  \texttt{precedenceReified}$(x_1,d,x_2,b)$ states that $x_1$ plus duration $d$ is less than or equal to $x_2$ requires boolean $b$ to be true:
  $$b\quad\iff\quad x_1 + d \le x_2$$
\end{notedef}

\begin{itemize}
	\item \textbf{API} : \mylst{precedenceReified(IntegerVariable x1, int d, IntegerVariable x2, IntegerVariable b)}
	\item \textbf{return type} : \texttt{Constraint}
	\item \textbf{options} :\emph{n/a}
	\item \textbf{favorite domain} : \emph{to complete}
\end{itemize}

\textbf{Example}:
\lstinputlisting{java/cprecedencereified.j2t}

\section{precedenceimplied (constraint)}\label{precedenceimplied:precedenceimpliedconstraint}\hypertarget{precedenceimplied:precedenceimpliedconstraint}{}
\emph{To complete}

\section{precedence (constraint)}\label{precedence:precedenceconstraint}\hypertarget{precedence:precedenceconstraint}{}
\emph{To complete}

\section{precedencedisjoint (constraint)}\label{precedencedisjoint:precedencedisjointconstraint}\hypertarget{precedencedisjoint:precedencedisjointconstraint}{}
\emph{To complete}

%\part{regular}
\label{regular}
\hypertarget{regular}{}

\section{regular (constraint)}\label{regular:regularconstraint}\hypertarget{regular:regularconstraint}{}
\begin{notedef}
  \texttt{regular}$(x,\mathcal{L}(\Pi))$ states that sequence $x$ is a word belonging to the regular language $\mathcal{L}(\Pi)$:
% recognized by a deterministic finite automaton (DFA) or a regular expression $\Pi$:
$$(x_1,\ldots,x_n)\in\mathcal{L}(\Pi)$$
\end{notedef}

The accepting language can be specified either by a deterministic finite automaton (DFA), a list of feasible or infeasible tuples, or a regular expression:
\begin{description}
\item[DFA:] Automaton $\Pi$ is defined on a given \emph{alphabet} $\Sigma\subseteq\Z$ by a set $Q=\{0,\ldots,m\}$ of \emph{states}, a subset $A\subseteq Q$ of \emph{final} or \emph{accepting states} and a table $\Delta\subseteq Q\!\times\!\Sigma\!\times Q$ of \emph{transitions} between states. $\Delta$ is encoded as \texttt{List<Transition>} where a Transition object $\delta=\texttt{new Transition}(q_i,\sigma,q_j)$ is made of three integers expressing the ingoing state $q_i$, the label $\sigma$, and the outgoing state $q_j$.
Automaton $\Pi$ is a DFA if $\Delta$ is finite and if it has only one initial state (here, state $0$ is considered as the unique initial state) and no two transitions sharing the same ingoing state and the same label.
\item[FiniteAutomaton:] is another API for building a DFA (manually, or from a regular expression, or from a \mylst{dk.brics.Automaton}) and operating on them (intersection, union, complement) in a more flexible way. Using this API leads to another implementation of the constraint: \mylst{FastRegular}. See \hyperlink{costregular}{\texttt{costRegular}} for a short API of \texttt{FiniteAutomaton}. 
\item[feasible tuples:] \emph{regular} can be used as an extensional constraint. Given the list of \emph{feasible} tuples for sequence $x$, this API builds a DFA from the list, and then enforces GAC on the constraint. Using \texttt{regular} can be more efficient than a standard GAC algorithm on tables of tuples if the tuples are structured so that the resulting DFA is compact. The DFA is built from the list of tuples by computing incrementally the minimal DFA after each addition of tuple. 
\item[infeasible tuples:] An another API allows to specify the list of \emph{infeasible} tuples and then builds the corresponding feasible DFA. This operation requires to know the entire alphabet, hence this API has two mandatory table fields \emph{min} and \emph{max} defining the minimum and maximum values of each variable $x_i$.
\item[regular expression:] Finally, the \texttt{regular} constraint can be based on a \href{http://en.wikipedia.org/wiki/regularexpression}{regular expression}, such as \mylst{String regexp = "(1\|2)3\{4\}5*";}. This expression recognizes any sequences starting by one 1 or one 2, then four consecutive 3 followed by any (possibly empty) sequences of 5.
\end{description}

\todo{Warning ! DFA and FiniteAutomaton are both based on the dk.brics library. The construction of these objects is non-deterministic and the order the filtering occur (not the result) may vary at each execution. This may results in different first solutions when branching dynamically using  weighted degrees-base heuristics for example.}

\begin{itemize}
	\item \textbf{API} :
	\begin{itemize}
		\item \mylst{regular(IntegerVariable[] x, FiniteAutomaton pi)}
		\item \mylst{regular(IntegerVariable[] x, DFA pi)}
		\item \mylst{regular(IntegerVariable[] x, List<int[]> feasTuples)}
		\item \mylst{regular(IntegerVariable[] x, List<int[]> infeasTuples, int[] min, int[] max)}
		\item \mylst{regular(IntegerVariable[] x, String regexp)}
	\end{itemize}
	\item \textbf{return type} : \texttt{Constraint}
	\item \textbf{options} :\emph{n/a}
	\item \textbf{favorite domain} : \emph{to complete}
	\item \textbf{references} :\\
       \cite{PesantCP04}: \emph{A regular language membership constraint}
\end{itemize}

\textbf{Examples}:
\begin{itemize}
	\item example with \texttt{FiniteAutomaton}: see \hyperlink{costregular:costregularconstraint}{\texttt{costRegular}}.
	\item example 1 with DFA:
\end{itemize}
\lstinputlisting{java/cregular1_import.j2t}
\lstinputlisting{java/cregular1.j2t}

\begin{itemize}
	\item example 2 with feasible tuples:
\end{itemize}
\lstinputlisting{java/cregular2.j2t}

\begin{itemize}
	\item example 3 with regular expression:
\end{itemize}
\lstinputlisting{java/cregular3.j2t}

\section{reifiedAnd (constraint)}\label{reifiedand:reifiedandconstraint}\hypertarget{reifiedand:reifiedandconstraint}{}
\emph{To complete}

\section{reifiedConstraint (constraint)}\label{reifiedconstraint:reifiedconstraintconstraint}\hypertarget{reifiedconstraint:reifiedconstraintconstraint}{}
\begin{notedef}
  \begin{itemize}
  \item \texttt{reifiedConstraint}$(b,C)$ states that boolean $b$ is true if and only if constraint $C$ holds:
  $$(b=1)\ \iff\ C$$
  \item \texttt{reifiedConstraint}$(b,C_1,C_2)$ states that boolean $b$ is true if and only if $C_1$ holds, and $b$ is false if and only if $C_2$ holds ($C_2$ must be the negation of constraint of $C_1$):
$$(b\land C_1) \lor (\neg b \land C_2)$$
  \end{itemize}
\end{notedef}

\begin{itemize}
	\item \textbf{API} :
	\begin{itemize}
		\item \mylst{reifiedConstraint(IntegerVariable b, Constraint c)}
		\item \mylst{reifiedConstraint(IntegerVariable b, Constraint c1, Constraint c2)}
	\end{itemize}
	\item \textbf{return type} : \texttt{Constraint}
	\item \textbf{options} : \emph{n/a}
	\item \textbf{favorite domain} : \emph{n/a}
\end{itemize}

The constraint $C$ to reify has to provide its negation $\neg C$ (the negation is needed for propagation). 
Most basic constraints of Choco provides their negation by default, and can then be reified using the first API.
The second API attends to reify user-defined constraints as it allows the user to directly specify the negation constraint. 

The {\tt  reifiedConstraint} filter algorithm:
\begin{enumerate}
	\item if $b$ is instantiated to 1 (resp. to 0), then $C$ (resp. $\neg C$) is propagated
	\item otherwise
	\begin{enumerate}
		\item if $C$ is entailed, $b$ is set to 1
		\item else if $C$ is failed, $b$ is set to 0.
	\end{enumerate}
\end{enumerate}


\textbf{Example}:
\lstinputlisting{java/creifiedintconstraint.j2t}


%%\part{reifiedintconstraint}
\label{reifiedintconstraint}
\hypertarget{reifiedintconstraint}{}

\section{reifiedIntConstraint (constraint)}\label{reifiedintconstraint:reifiedintconstraintconstraint}\hypertarget{reifiedintconstraint:reifiedintconstraintconstraint}{}
\begin{notedef}
  \begin{itemize}
  \item \texttt{reifiedIntConstraint}$(b,c)$ states that boolean $b$ is true if and only if constraint $c$ holds:
  $$b\ \iff\ c$$
  \item \texttt{reifiedIntConstraint}$(b,c_1,c_2)$ states that boolean $b$ is true if and only if $c_1$ holds, and $b$ is false if and only if $c_2$ holds ($c_2$ must be the opposite constraint of $c_1$):
$$(b\land c_1) \lor (\neg b \land c_2)$$
  \end{itemize}
\end{notedef}

\begin{itemize}
	\item \textbf{API} :
	\begin{itemize}
		\item \mylst{reifiedIntConstraint(IntegerVariable b, Constraint c)}
		\item \mylst{reifiedIntConstraint(IntegerVariable b, Constraint c1, Constraint c2)}
	\end{itemize}
	\item \textbf{return type} : \texttt{Constraint}
	\item \textbf{options} : \emph{n/a}
	\item \textbf{favorite domain} : \emph{n/a}
\end{itemize}

Parameter \emph{b} is a boolean variable (enumerated domain with two values $\{0,1\}$) and \emph{c} is a constraint over Integer variables.

The constraint $c$ to reify has to provide its opposite (the opposite is needed for propagation). Most basic constraints of Choco provides their opposite by default, and can then be reified using the first API.
The second API attends to reify user-defined constraints as it allows the user to directly specify the opposite constraint.

\textbf{Example}:

\lstinputlisting{java/creifiedintconstraint.j2t}

\section{reifiedLeftImp (constraint)}\label{reifiedleftimp:reifiedleftimpconstraint}\hypertarget{reifiedleftimp:reifiedleftimpconstraint}{}
\emph{To complete}

\section{reifiedNot (constraint)}\label{reifiednot:reifiednotconstraint}\hypertarget{reifiednot:reifiednotconstraint}{}
\emph{To complete}

\section{reifiedOr (constraint)}\label{reifiedor:reifiedorconstraint}\hypertarget{reifiedor:reifiedorconstraint}{}
\emph{To complete}

\section{reifiedRightImp (constraint)}\label{reifiedrightimp:reifiedrightimpconstraint}\hypertarget{reifiedrightimp:reifiedrightimpconstraint}{}
\emph{To complete}

\section{reifiedXnor (constraint)}\label{reifiedxnor:reifiedxnorconstraint}\hypertarget{reifiedxnor:reifiedxnorconstraint}{}
\emph{To complete}

\section{reifiedXor (constraint)}\label{reifiedxor:reifiedxorconstraint}\hypertarget{reifiedxor:reifiedxorconstraint}{}
\emph{To complete}

%\part{relationpairac}
\label{relationpairac}
\hypertarget{relationpairac}{}

\section{relationPairAC (constraint)}\label{relationpairac:relationpairacconstraint}\hypertarget{relationpairac:relationpairacconstraint}{}
\begin{notedef}
  \texttt{relationPairAC}$(x,y,rel)$ states an extensional binary constraint on $(x,y)$ defined by the binary relation $rel$:
$$(x,y)\in rel$$
\end{notedef}
Many constraints of the same kind often appear in a model. Relations can therefore often be shared among many constraints to spare memory.

The API is duplicated to allow definition of options.

\begin{itemize}
	\item \textbf{API} :
	\begin{itemize}
		\item \mylst{relationPairAC(IntegerVariable x, IntegerVariable y, BinRelation rel)}
		\item \mylst{relationPairAC(String options, IntegerVariable x, IntegerVariable y, BinRelation rel)}
	\end{itemize}
	\item \textbf{return type} : \texttt{Constraint}
	\item \textbf{options} :
	\begin{itemize}
		\item \emph{no option} : use AC3 (default arc consistency)
		\item \hyperlink{cext3:cext3options}{\tt Options.C\_EXT\_AC3}: to get AC3 algorithm (searching from scratch for supports on all values)
		\item \hyperlink{cext2001:cext2001options}{\tt Options.C\_EXT\_AC2001}: to get AC2001 algorithm (maintaining the current support of each value)
		\item \hyperlink{cext32:cext32options}{\tt Options.C\_EXT\_AC32}: to get AC3rm algorithm (maintaining the current support of each value in a non backtrackable way)
		\item \hyperlink{cext322:cext322options}{\tt Options.C\_EXT\_AC322}: to get AC3 with the used of \texttt{BitSet} to know if a support still exists
	\end{itemize}
	\item \textbf{favorite domain} : \emph{to complete}
\end{itemize}

\textbf{Example}:
\lstinputlisting{java/crelationpairac_import.j2t}
\lstinputlisting{java/ccoupletest.j2t}
\lstinputlisting{java/crelationpairac.j2t}

%\part{relationtupleac}
\label{relationtupleac}
\hypertarget{relationtupleac}{}

\section{relationTupleAC (constraint)}\label{relationtupleac:relationtupleacconstraint}\hypertarget{relationtupleac:relationtupleacconstraint}{}
\begin{notedef}
  \texttt{relationTupleAC}$(x,rel)$ states an extensional constraint on $(x_1,\ldots,x_n)$ defined by the $n$-ary relation $rel$, and then enforces arc consistency:
$$(x_1,\ldots,x_n)\in rel$$
\end{notedef}
Many constraints of the same kind often appear in a model. Relations can therefore often be shared among many constraints to spare memory.
The API is duplicated to define options.

\begin{itemize}
	\item \textbf{API}:
	\begin{itemize}
		\item \mylst{relationTupleAC(IntegerVariable[] x, LargeRelation rel)}
		\item \mylst{relationTupleAC(String options, IntegerVariable[] x, LargeRelation rel)}
	\end{itemize}
	\item \textbf{return type}: \texttt{Constraint}
	\item \textbf{options} :
	\begin{itemize}
		\item \emph{no option}: use AC32 (default arc consistency)
		\item \hyperlink{cext32:cext32options}{\tt Options.C\_EXT\_AC32}: to get AC3rm algorithm (maintaining the current support of each value in a non backtrackable way)
		\item \hyperlink{cext2001:cext2001options}{\tt Options.C\_EXT\_AC2001}: to get AC2001 algorithm (maintaining the current support of each value)
		\item \hyperlink{cext2008:cext2008options}{\tt Options.C\_EXT\_AC2008}: to get AC2008 algorithm (maintained by STR)
	\end{itemize}
	\item \textbf{favorite domain} : \emph{to complete}
\end{itemize}

\textbf{Example} :
\lstinputlisting{java/cnotallequal.j2t}
\lstinputlisting{java/crelationtupleac.j2t}
%\part{relationtuplefc}
\label{relationtuplefc}
\hypertarget{relationtuplefc}{}

\section{relationTupleFC (constraint)}\label{relationtuplefc:relationtuplefcconstraint}\hypertarget{relationtuplefc:relationtuplefcconstraint}{}
\begin{notedef}
  \texttt{relationTupleFC}$(x,rel)$ states an extensional constraint on $(x_1,\ldots,x_n)$ defined by the $n$-ary relation $rel$, and then enforces forward checking:
$$(x_1,\ldots,x_n)\in rel$$
\end{notedef}
Many constraints of the same kind often appear in a model. Relations can therefore often be shared among many constraints to spare memory.

\begin{itemize}
	\item \textbf{API}: \mylst{relationTupleFC(IntegerVariable[] x, LargeRelation rel)}
	\item \textbf{return type}: \texttt{Constraint}
	\item \textbf{options} : \emph{n/a}
	\item \textbf{favorite domain} : \emph{to complete}
\end{itemize}

\textbf{Example} :
\lstinputlisting{java/cnotallequal.j2t}
\lstinputlisting{java/crelationtuplefc.j2t}

%\part{samesign}
\label{samesign}
\hypertarget{samesign}{}

\section{sameSign (constraint)}\label{samesign:samesignconstraint}\hypertarget{samesign:samesignconstraint}{}
\begin{notedef}
  \texttt{sameSign}$(x,y)$ states that the two arguments have the same sign:
$$xy\ge 0$$
\end{notedef}
Note that $0$ has both signs then constraint holds if $x$ or $y$ is equal to 0.

\begin{itemize}
	\item \textbf{API} : \mylst{sameSign(IntegerExpressionVariable x, IntegerExpressionVariable y)}
	\item \textbf{return type} : \texttt{Constraint}
	\item \textbf{options} :\emph{n/a}
	\item \textbf{favorite domain} : \emph{to complete}
\end{itemize}

\textbf{Example}:
\lstinputlisting{java/csamesign.j2t}

%\part{setdisjoint}
\label{setdisjoint}
\hypertarget{setdisjoint}{}

\section{setDisjoint (constraint)}\label{setdisjoint:setdisjointconstraint}\hypertarget{setdisjoint:setdisjointconstraint}{}
\begin{notedef}
  \texttt{setDisjoint}$(s_1,\ldots,s_n)$ states that the arguments are pairwise disjoint:
$$s_i\cap s_j=\emptyset, \quad \forall\ i\neq j $$
\end{notedef}

\begin{itemize}
	\item \textbf{API} : \mylst{setDisjoint(SetVariable[] sv)}
	\item \textbf{return type} : \texttt{Constraint}
	\item \textbf{options} :\emph{n/a}
	\item \textbf{favorite domain} : \emph{to complete}
\end{itemize}

\textbf{Example}:
\lstinputlisting{java/csetdisjoint.j2t}

%\part{setinter}
\label{setinter}
\hypertarget{setinter}{}

\section{setInter (constraint)}\label{setinter:setinterconstraint}\hypertarget{setinter:setinterconstraint}{}
\begin{notedef}
  \texttt{setInter}$(s_1,s_2,s_3)$ states that the third set $s_3$ is exactly the intersection of the two first sets:
$$s_1\cap s_2=s_3$$
\end{notedef}

\begin{itemize}
	\item \textbf{API} : \mylst{setInter(SetVariable s1, SetVariable s2, SetVariable s3)}
	\item \textbf{return type} : \texttt{Constraint}
	\item \textbf{options} :\emph{n/a}
	\item \textbf{favorite domain} : \emph{to complete}
\end{itemize}

\textbf{Example}:
\lstinputlisting{java/csetinter.j2t}
%\part{setunion}
\label{setunion}
\hypertarget{setunion}{}

\section{setUnion (constraint)}\label{setunion:setunionconstraint}\hypertarget{setunion:setunionconstraint}{}
\begin{notedef}
  \texttt{setUnion}$(sv,s_{union})$ states that the $s_{union}$ set is exactly the union of the sets $sv$:
$$sv_1\cup sv_2 \cup \ldots sv_i \cup sv_{i+1} \ldots \cup sv_n=s_{union}$$
\end{notedef}

\begin{itemize}
	\item \textbf{API} : 
	\begin{itemize}
		\item \mylst{setUnion(SetVariable s1, SetVariable s2, SetVariable union)}
		\item \mylst{setUnion(SetVariable[] sv, SetVariable union)}
	\end{itemize}
	\item \textbf{return type} : \texttt{Constraint}
	\item \textbf{options} :\emph{n/a}
	\item \textbf{favorite domain} : \emph{to complete}
\end{itemize}

\textbf{Example}:
\lstinputlisting{java/csetunion.j2t}
%\part{sorting}
\label{sorting}
\hypertarget{sorting}{}

\section{sorting (constraint)}\label{sorting:sortingconstraint}\hypertarget{sorting:sortingconstraint}{}
\begin{notedef}
  \texttt{sorting}$(x,y)$ holds on the set of variables being either in x or in y, and is satisfied by v if and only if v(y) is the sorted version of v(x) in increasing order.
$$y=x_sorted$$   
\end{notedef}

\begin{itemize}
	\item \textbf{API}: \mylst{sorting(IntegerVariable[] x, IntegerVariable[] y)}
	\item \textbf{return type} : \texttt{Constraint}
	\item \textbf{options} :\emph{n/a}
	\item \textbf{favorite domain} : \emph{to complete}
	\item \textbf{references} :
	\begin{itemize}
      \item \cite{BleuzenCP97}: \emph{Narrowing a block of sortings in quadratic time}
      \item \cite{MehlhornCP00}: \emph{Faster algorithms for bound-consistency of the Sortedness and the Alldifferent constraint}
      \item global constraint catalog: \href{http://www.emn.fr/x-info/sdemasse/gccat/Csort.html}{count}
      \end{itemize}
\end{itemize}

\textbf{Example}:
\lstinputlisting{java/csorting.j2t}
\section{startsAfter (constraint)}\label{startsafter:startsafterconstraint}\hypertarget{startsafter:startsafterconstraint}{}
\begin{notedef}
  \texttt{startsAfter}$(T,c)$ states that the task variable $t$ starts after time $c$:
  $$T.start \ge c$$
\end{notedef}

\begin{itemize}
	\item \textbf{API} :\mylst{startsAfter(final TaskVariable t, final int c) }
	\item \textbf{return type} : \texttt{Constraint}
	\item \textbf{options} : \emph{n/a}
	\item \textbf{favorite domain} : \emph{n/a}.
\end{itemize}

\textbf{Examples:}
%\lstinputlisting{java/ceq1.j2t}
\emph{to complete}

\section{startsAfterBegin (constraint)}\label{startsafterbegin:startsafterbeginconstraint}\hypertarget{startsafterbegin:startsafterbeginconstraint}{}
\begin{notedef}
\texttt{startsAfterBegin}$(T_1,T_2,c)$  states that task $T_1$ starts after the start time of $T_2$ minus $c$:
  $$T_{1}.start \ge T_{2}.start - c$$
\end{notedef}

\begin{itemize}
	\item \textbf{API} :\mylst{startsAfterBegin(TaskVariable t1, TaskVariable t2, int c)}
	\item \textbf{return type} : \texttt{Constraint}
	\item \textbf{options} : \emph{n/a}
	\item \textbf{favorite domain} : \emph{n/a}.
\end{itemize}

\textbf{Examples:}
%\lstinputlisting{java/ceq1.j2t}
\emph{to complete}

\section{startsAfterEnd (constraint)}\label{startsafterend:startsafterendconstraint}\hypertarget{startsafterend:startsafterendconstraint}{}
\begin{notedef}
\texttt{startsAfterEnd}$(T_1,T_2,c)$  states that task $T_1$ starts after the end time of $T_2$ minus $c$:
  $$T_{1}.start \ge T_{2}.end - c$$
\end{notedef}

\begin{itemize}
	\item \textbf{API} :\mylst{startsAfterEnd(TaskVariable t1, TaskVariable t2, int c)}
	\item \textbf{return type} : \texttt{Constraint}
	\item \textbf{options} : \emph{n/a}
	\item \textbf{favorite domain} : \emph{n/a}.
\end{itemize}

\textbf{Examples:}
%\lstinputlisting{java/ceq1.j2t}
\emph{to complete}

\section{startsBefore (constraint)}\label{startsbefore:startsbeforeconstraint}\hypertarget{startsbefore:startsbeforeconstraint}{}
\begin{notedef}
  \texttt{startsBefore}$(T,c)$ states that the task variable $T$ starts before time $c$:
  $$T.start \le c$$
\end{notedef}

\begin{itemize}
	\item \textbf{API} :\mylst{startsBefore(final TaskVariable t, final int c) }
	\item \textbf{return type} : \texttt{Constraint}
	\item \textbf{options} : \emph{n/a}
	\item \textbf{favorite domain} : \emph{n/a}.
\end{itemize}

\textbf{Examples:}
%\lstinputlisting{java/ceq1.j2t}
\emph{to complete}

\section{startsBeforeBegin (constraint)}\label{startsbeforebegin:startsbeforebeginconstraint}\hypertarget{startsbeforebegin:startsbeforebeginconstraint}{}
\begin{notedef}
\texttt{startsBeforeBegin}$(T_1,T_2,c)$  states that task $T_1$ starts before the start time of $T_2$ minus $c$:
  $$T_{1}.start \le T_{2}.start - c$$
\end{notedef}

\begin{itemize}
	\item \textbf{API} :\mylst{startsBeforeBegin(TaskVariable t1, TaskVariable t2, int c)}
	\item \textbf{return type} : \texttt{Constraint}
	\item \textbf{options} : \emph{n/a}
	\item \textbf{favorite domain} : \emph{n/a}.
\end{itemize}

\textbf{Examples:}
%\lstinputlisting{java/ceq1.j2t}
\emph{to complete}

\section{startsBeforeEnd (constraint)}\label{startsbeforeend:startsbeforeendconstraint}\hypertarget{startsbeforeend:startsbeforeendconstraint}{}
\begin{notedef}
\texttt{startsBeforeend}$(T_1,T_2,c)$  states that task $T_1$ starts before the end time of $T_2$ minus $c$:
  $$T_{1}.start \le T_{2}.end - c$$
\end{notedef}

\begin{itemize}
	\item \textbf{API} :\mylst{startsBeforeEnd(TaskVariable t1, TaskVariable t2, int c)}
	\item \textbf{return type} : \texttt{Constraint}
	\item \textbf{options} : \emph{n/a}
	\item \textbf{favorite domain} : \emph{n/a}.
\end{itemize}

\textbf{Examples:}
%\lstinputlisting{java/ceq1.j2t}
\emph{to complete}

\section{startsBetween (constraint)}\label{startsbetween:startsbetweenconstraint}\hypertarget{startsbetween:startsbetweenconstraint}{}
\begin{notedef}
  \texttt{startsBetween}$(T, c_1, c_2)$ states that task $T$ starts between times $c_1$ and $c_2$:
  $$c_1 \le T.start \le c_2$$
\end{notedef}

\begin{itemize}
	\item \textbf{API} :\mylst{startsBetween(TaskVariable t, int min, int max)}
	\item \textbf{return type} : \texttt{Constraint}
	\item \textbf{options} : \emph{n/a}
	\item \textbf{favorite domain} : \emph{n/a}.
\end{itemize}

\textbf{Examples:}
%\lstinputlisting{java/ceq1.j2t}
\emph{to complete}

%\part{stretchpath}
\label{stretchpath}
\hypertarget{stretchpath}{}

\section{stretchPath (constraint)}\label{stretchpath:stretchpathconstraint}\hypertarget{stretchpath:stretchpathconstraint}{}


\begin{notedef}
  A \emph{stretch} in a sequence $x$ is a maximum subsequence of (consecutive) identical values.  
  \texttt{stretchPath}$(param,x)$ enforces the minimal and maximal length of the stretches in sequence $x$ of any values given in $param$:
Consider the sequence $x$ as a concatenation of stretches $x^1.x^2\ldots x^k$ with $v^i$ and $l^i$ being respectively the value and the length of stretch $x^i$,
$$\forall i\in\{1,\ldots,k\},\ \forall j,\quad param[j][0]=v^i\quad\implies\quad param[j][1]\le l^i\le param[j][2]$$
\end{notedef}


Useful for Rostering Problems. \texttt{stretchPath} is implemented by a \hyperlink{regular}{\texttt{regular}} constraint that performs GAC. The bounds on the stretch lengths are defined by $param$ a list of triples of integers: $[value, min, max]$ specifying the minimal and maximal lengths of any stretch of the corresponding value. 

This API requires a Java library on automaton available on \href{http://www.brics.dk/automaton/}{http://www.brics.dk/automaton/}. (It is contained in the Choco jar file.)

\begin{itemize}
	\item \textbf{API} : \mylst{stretchPath(List<int[]> param, IntegerVariable... x)}
	\item \textbf{return type} : Constraint
	\item \textbf{options} :\emph{n/a}
	\item \textbf{favorite domain} : \emph{to complete}
	\item \textbf{references} :
      \begin{itemize}
      \item \cite{PesantCP04}: \emph{A regular language membership constraint}
      \item global constraint catalog: \href{http://www.emn.fr/x-info/sdemasse/gccat/Cstretch_path.html}{stretch\_path}
      \end{itemize}
\end{itemize}


\textbf{Example}:
\lstinputlisting{java/cstretchpath.j2t}

%\part{times}
\label{times}
\hypertarget{times}{}

\section{times (constraint)}\label{times:timesconstraint}\hypertarget{times:timesconstraint}{}
\begin{notedef}
  \texttt{times}$(x_1, x_2, x_3)$ states that the third argument is equal to the product of the two arguments:
$$x_3=x_1\times x_2.$$
\end{notedef}

\begin{itemize}
	\item \textbf{API}:
	\begin{itemize}
		\item \mylst{times(IntegerVariable x1, IntegerVariable x2, IntegerVariable x3)}
		\item \mylst{times(int x1, IntegerVariable x2, IntegerVariable x3)}
		\item \mylst{times(IntegerVariable x1, int x2, IntegerVariable x3)}
	\end{itemize}
	\item \textbf{return type} : \texttt{Constraint}
	\item \textbf{option} : \emph{n/a}
	\item \textbf{favorite domain}: bound
\end{itemize}

\textbf{Example}:
\lstinputlisting{java/ctimes.j2t}
%\part{tree}
\label{tree}
\hypertarget{tree}{}

\section{tree (constraint)}\label{tree:treeconstraint}\hypertarget{tree:treeconstraint}{}

Let $G=(V,A)$ be a digraph on $V=\{1,\ldots,n\}$. $G$ can be modeled by a sequence of domain variables $x=(x_1,\dots,x_n)\in V^n$ -- the \emph{successors} variables -- whose respective domains are given by $D_i=\{j\in V\ |\ (i,j)\in A\}$. Conversely, when instantiated, $x$ defines a subgraph $G_x=(V,A_x)$ of $G$ with $A_x=\{(i,x_i)\ |\ i\in V\}\subseteq A$. Such a subgraph has one particularity: any connected component of $G_x$ contains either no loop -- and then it contains a cycle -- or exactly one loop $x_i=i$ and then it is a \emph{tree} of root $i$ (literally, it is an anti-arborescence as there exists a path from each node to $i$ and $i$ has a loop).

\begin{notedef}
  \texttt{tree}$(x,restrictions)$ is a vertex-disjoint graph partitioning constraint. It states that $G_x$ is a forest (its connected components are trees) that satisfies some conditions specified by $restrictions$.
\texttt{tree} deals with several kinds of graph restrictions on:
\begin{itemize}
	\item the number of trees
	\item the number of proper trees (a tree is proper if it contains more than 2 nodes)
    \item the weigth of the partition: the sum of the weights of the edges
	\item incomparability: some nodes in pairs have to belong to distinct trees
	\item precedence: some nodes in pairs have to belong to the same tree in a given order
	\item conditional precedence: some nodes in pairs have to respect a given order if they belong to the same tree
	\item the in-degree of the nodes
	\item the time windows on nodes (given travelling times on arcs)
\end{itemize}
\end{notedef}

Many applications require to partition a graph such that each component contains exactly one \emph{resource} node and several \emph{task} nodes. A typical example is a routing problem where vehicle routes are paths (a path is a special case of tree) starting from a depot and delivering goods to several clients. Another example is a local network where each computer has to be connected to one shared printer. Last, one can cite the problem of reconstructing plylogeny trees.
The constraint \texttt{tree} can handle these kinds of problems with many additional constraints on the structure of the partition.

\begin{itemize}
	\item \textbf{API} : \mylst{tree(TreeParametersObject param)}
	\item \textbf{return type} : \texttt{Constraint}
	\item \textbf{options} :\emph{n/a}
	\item \textbf{favorite domain} : \emph{to complete}
	\item \textbf{references} :
      \begin{itemize}
      \item \cite{beldiceanuCONSTRAINTS08}: \emph{Combining tree partitioning, precedence, and incomparability constraints}
      \item global constraint catalog: \href{http://www.emn.fr/x-info/sdemasse/gccat/Cproper_forest.html}{proper\_forest} (variant)
      \end{itemize}

\end{itemize}

The tree constraint API requires a particular Model object, named \textbf{\tt TreeParametersObject}.
It can be created with the following parameters:

\begin{tabular}{p{3cm}p{3cm}p{7cm}}
parameter &type &description\\
\hline
$n$ &\texttt{int} &number of nodes in the initial graph $G$\\
$nTree$ &\texttt{IntegerVariable} &number of trees in the resulting forest $G_x$\\
$nProper$ &\texttt{IntegerVariable} &number of proper trees in $G_x$\\
$objective$ &\texttt{IntegerVariable} &(bounded) total \todo{cost} of $G_x$\\
%$objective$ &\texttt{IntegerVariable} &(bounded) total weight of $G_x$\\
$graphs$ &\texttt{List<BitSet[]>} &
\begin{minipage}[t]{7cm}
graphs encoded as successor lists,\\
  \texttt{graphs[0]} the initial graph $G$,\\
  \texttt{graphs[1]} a precedence graph,\\
  \texttt{graphs[2]} a conditional precedence graph,\\
  \texttt{graphs[3]} an incomparability graph
\end{minipage}\\
$matrix$ &\texttt{List<int[][]>} &\texttt{matrix[0]} the indegree of each node, and \texttt{matrix[1]} the starting time from each node\\
$travel$ &\texttt{int[][]} &the travel time of each arc
\end{tabular}

\textbf{Example}:
\lstinputlisting{java/ctree_import.j2t}
\lstinputlisting{java/ctree.j2t}

%\part{true}
\label{true}
\hypertarget{true}{}

\section{TRUE (constraint)}\label{true:trueconstraint}\hypertarget{true:trueconstraint}{}
\(TRUE\) always returns \emph{true}.

%\part{xnor}
\label{xnor}
\hypertarget{xnor}{}

\section{xnor (constraint)}\label{xnor:xnorconstraint}\hypertarget{xnor:xnorconstraint}{}
\begin{notedef}
    \texttt{xnor}$(b_1,b_2)$ states that two booleans are either both true, or both false:
$$ (b_1=1)\quad\iff\quad(b_2=1)$$
\end{notedef}

\begin{itemize}
    \item \textbf{API} : \mylst{xnor(IntegerVariable b1, IntegerVariable b2)}
	\item \textbf{return type} : \texttt{Constraint}
	\item \textbf{options} : \emph{n/a}
	\item \textbf{favorite domain} : \emph{n/a}
\end{itemize}

\textbf{Examples:}
\lstinputlisting{java/cxnor.j2t}

%\part{xor}
\label{xor}
\hypertarget{xor}{}

\section{xor (constraint)}\label{xor:xorconstraint}\hypertarget{xor:xorconstraint}{}
\begin{notedef}
    \texttt{xor}$(b_1,b_2)$ states that two booleans are the one true and the other one false:
$$ (b_1=1)\quad \iff\quad (b_2=0)$$
\end{notedef}

\begin{itemize}
    \item \textbf{API} : \mylst{xor(IntegerVariable b_1, IntegerVariable b_2)}
	\item \textbf{return type} : \texttt{Constraint}
	\item \textbf{options} : \emph{n/a}
	\item \textbf{favorite domain} : \emph{n/a}
\end{itemize}

\textbf{Examples:}
\lstinputlisting{java/cxor.j2t}


\chapter{Options (Model)}\label{ch:options}\hypertarget{ch:options}{}
This section lists and details the options that can be declared on variables or constraints within a Choco Model.
\input{chapters/Options.tex}

\chapter{Branching strategies (Solver)}\label{ch:branchstrat}\hypertarget{ch:branchstrat}{}
This section lists and details the \hyperlink{solver:searchstrategy}{branching strategies} currently available in Choco.

\section{AssignInterval (Branching strategy)}\label{assigninterval:assignintervalbranchstrat}\hypertarget{assigninterval:assignintervalbranchstrat}{}
\begin{notedef}
  \texttt{AssignInterval} is a \textbf{binary branching} assigning two distinct intervals to a real variable. Following the \emph{interval bisection rule}, the interval representing the domain of the selected variable is split into two parts at its midpoint. In the first branch, the variable upper bound is set to the midpoint; in the second branch, the variable lower bound is set to the midpoint.
$$B_1: x\in[\underline{x}, m],\quad B_2: x\in[m, \overline{x}],\qquad \text{with } m=\frac{\underline{x}+\overline{x}}{2}$$
\end{notedef}

\begin{itemize}
	\item \textbf{Constructor} :\mylst{AssignInterval(VarSelector<RealVar> varSel, ValIterator<RealVar> valIt)}
	\item \textbf{type of variable} : real
	\item \textbf{references} : \emph{n/a}
\end{itemize}

\textbf{Example}:
%\lstinputlisting{java/cabs.j2t}


\section{AssignOrForbidIntVarVal (Branching strategy)}\label{assignorforbidintvarval:assignorforbidintvarvalbranchstrat}\hypertarget{assignorforbidintvarval:assignorforbidintvarvalbranchstrat}{}
\begin{notedef}
  \texttt{AssignOrForbidIntVarVal} is a \textbf{binary branching} assigning a value to an integer variable. In the first branch, the selected value is assigned to the selected variable; in the second branch, the value is removed from the variable domain.
$$B_1: x=v,\quad B_2: x\neq v$$
\end{notedef}

\begin{itemize}
	\item \textbf{Constructor} :\mylst{AssignOrForbidIntVarVal(VarSelector<IntDomainVar> varSel, ValSelector<IntDomainVar> valSel)}
	\item \textbf{type of variable} : integer
	\item \textbf{references} : \emph{n/a}
\end{itemize}

\textbf{Example}:
%\lstinputlisting{java/cabs.j2t}


\section{AssignOrForbidIntVarValPair (Branching strategy)}\label{assignorforbidintvarvalpair:assignorforbidintvarvalpairbranchstrat}\hypertarget{assignorforbidintvarvalpair:assignorforbidintvarvalpairbranchstrat}{}
\begin{notedef}
\texttt{AssignOrForbidIntVarValPair} is a  \textbf{binary branching} assigning a value to an integer variable. In the first branch, the selected value is assigned to the selected variable; in the second branch, the value is removed from the variable domain.
It requires a \texttt{VarValPairSelector} selecting both variable and value at the same time.
$$B_1: x=v,\quad B_2: x\neq v$$
\end{notedef}


\begin{itemize}
	\item \textbf{Constructor} :\mylst{AssignOrForbidIntVarValPair(VarValPairSelector varValSel)}
	\item \textbf{type of variable} : integer
	\item \textbf{references} : \emph{n/a}
	\item \textbf{see also} : \hyperlink{assignorforbidintvarval:assignorforbidintvarvalbranchstrat}{\texttt{AssignOrForbidIntVarVal}} based on distinct variable/value selectors
\end{itemize}

\textbf{Example}:
%\lstinputlisting{java/cabs.j2t}


\section{AssignSetVar (Branching strategy)}\label{assignsetvar:assignsetvarbranchstrat}\hypertarget{assignsetvar:assignsetvarbranchstrat}{}
\begin{notedef}
  \texttt{AssignSetVar} is a \textbf{n-ary branching} assigning distinct values to a set variable. The selected variable is successively assigned, in each branch, to a next selected value.
$$B_1: x=v_1,\quad B_2: x= v_2,\quad\ldots,\quad B_m: x= v_m$$
\end{notedef}

\begin{itemize}
	\item \textbf{Constructor} :\mylst{AssignSetVar(VarSelector<SetVar> varSel, ValSelector<SetVar> valSel)}
	\item \textbf{type of variable} : set
	\item \textbf{references} : \emph{n/a}
\end{itemize}

\textbf{Example}:
%\lstinputlisting{java/cabs.j2t}


\section{AssignVar (Branching strategy)}\label{assignvar:assignvarbranchstrat}\hypertarget{assignvar:assignvarbranchstrat}{}
\begin{notedef}
  \texttt{AssignVar} is an \textbf{n-ary branching} assigning distinct values to an integer variable. The selected variable is successively assigned, in each branch, to a next selected value.
$$B_1: x=v_1,\quad B_2: x= v_2,\quad\ldots,\quad B_m: x= v_m$$
\end{notedef}

\begin{itemize}
	\item \textbf{Constructor} :
      \begin{itemize}
      \item \mylst{AssignVar(VarSelector<IntDomainVar> varSel, ValSelector<IntDomainVar> valSel)}
      \item \mylst{AssignVar(VarSelector<IntDomainVar> varSel, ValIterator<IntDomainVar> valIt)}
      \end{itemize}
	\item \textbf{type of variable} : integer
	\item \textbf{references} : \emph{n/a}
\end{itemize}

\textbf{Example}:
%\lstinputlisting{java/cabs.j2t}


\section{DomOverWDegBranchingNew (Branching strategy)}\label{domoverwdeg:domoverwdegbranchstrat}\hypertarget{domoverwdeg:domoverwdegbranchstrat}{}
\begin{notedef}
  \texttt{DomOverWDegBranchingNew} is a \textbf{n-ary branching} assigning distinct values to an integer variable. It maintains (incrementally or dynamically) on each constraint, the count of the failures caused by the constraint from the beginning of the search.
To each variable are then associated, at any time, three values: $dom$ the current domain size, $deg$ the current number of uninstantiated constraints involving the variable, and $w$ the sum of the counters associated with these $deg$ constraints.
The strategy selects the variable with the smallest ratio $r_i=dom/w*deg$.  Ties are randomly broken when \mylst{seed != null}.
The variable is then successively assigned, in each branch, to a next selected value.
% er a counter on each constraint. Whenever a dead-end occurs, the counter associated with the contraint that just failed is updated during search whenever a dead-end occurs. Then, the weight, $w$, of a variable $X_i$ is the sum of the weights of the constraints involving $X_i$ and at least one uninstantiated variable. And combining $w$ with the domain size, $dom$, and the degree, $deg$, of $X_i$, this strategy selects the variable with the smallest ratio $r_i=dom/w*deg$.
$$B_1: x=v_1,\quad B_2: x= v_2,\quad\ldots,\quad B_m: x= v_m,\qquad\text{ with } x=\arg\min\bigl(\frac{dom_x}{w_x*deg_x}\bigr)$$
\end{notedef}

\begin{itemize}
	\item \textbf{Constructor} :\mylst{DomOverWDegBranchingNew(Solver s, IntDomainVar[] vars, ValIterator valIt, Number seed)}
	\item \textbf{type of variable} : integer
	\item \textbf{references} : \cite{Boussemart04}: \emph{Boosting systematic search by weighting constraints}.
\end{itemize}

\textbf{Example}:
%\lstinputlisting{java/cabs.j2t}


\section{DomOverWDegBinBranchingNew (Branching strategy)}\label{domoverwdegbin:domoverwdegbinbranchstrat}\hypertarget{domoverwdegbin:domoverwdegbinbranchstrat}{}
\begin{notedef}
  \texttt{DomOverWDegBranchingNew} is a \textbf{binary branching} assigning distinct values to an integer variable. It maintains (incrementally or dynamically) on each constraint, the count of the failures caused by the constraint from the beginning of the search.
To each variable are then associated, at any time, three values: $dom$ the current domain size, $deg$ the current number of uninstantiated constraints involving the variable, and $w$ the sum of the counters associated with these $deg$ constraints.
The strategy selects the variable with the smallest ratio $r_i=dom/w*deg$.  Ties are randomly broken when \mylst{seed != null}.
The variable is then assigned, in the first branch, to the selected value; in the second branch, the value is removed from the variable domain.
$$B_1: x=v,\quad B_2: x\neq v,\qquad\text{ with } x=\arg\min\bigl(\frac{dom_x}{w_x*deg_x}\bigr)$$\end{notedef}

\begin{itemize}
	\item \textbf{Constructor}: \mylst{DomOverWDegBinBranchingNew(Solver s, IntDomainVar[] vars, ValSelector valSel, Number seed)}
	\item \textbf{type of variable} : integer
	\item \textbf{references} : \cite{Boussemart04}: \emph{Boosting systematic search by weighting constraints}.
\end{itemize}

\textbf{Example}:
%\lstinputlisting{java/cabs.j2t}


\section{ImpactBasedBranching (Branching strategy)}\label{impact:impactbranchstrat}\hypertarget{impact:impactbranchstrat}{}
\begin{notedef}
  \texttt{ImpactBasedBranching} is a \textbf{n-ary branching} assigning all distinct values to an integer variable. 
  The impact of a branching decision measures the reduction of the search space induced when the decision was posted and propagated since the beginning of the search.
  Here, branching decisions are variable-value assignments. 
The strategy selects an integer variable maximizing the total impact minus the domain size. The variable is then successively assigned, in each branch, to its possible values selected by decreasing order of their impacts. 
  \emph{Restarting search can dramatically improve performance.}
$$B_1: x=v_1,\quad B_2: x= v_2,\quad\ldots,\quad B_m: x= v_m,$$
with $x=\arg\max\sum_{v\in D(x)}(impact(x,v)-1)$  and $impact(x,v_i) > impact(x,v_j)$, $\forall j<i$.
\end{notedef}

\begin{itemize}
	\item \textbf{Constructor} :
	\begin{itemize}
	\item \mylst{ImpactBasedBranching(Solver s, IntDomainVar[] vars)}
	\item \mylst{ImpactBasedBranching(Solver s, IntDomainVar[] vars, AbstractImpactStrategy ibs)}
	\end{itemize}	
	\item \textbf{type of variable} : integer
	\item \textbf{references} : \cite{Refalo04}: \emph{Impact-Based Search Strategies for Constraint Programming}. 
\end{itemize}

\textbf{Example}:
%\lstinputlisting{java/cabs.j2t}


\section{PackDynRemovals (Branching strategy)}\label{packdynremovals:packdynremovalsbranchstrat}\hypertarget{packdynremovals:packdynremovalsbranchstrat}{}
\begin{notedef}
  \texttt{PackDynRemovals} is a \textbf{n-ary branching} for packing problems with identical bin capacities and without pre-assignments, placing an item in distinct bins. It is a specialization of \hyperlink{assignvar:assignvarbranchstrat}{AssignVar} for breaking symmetries: 
the selected item $x$ is successively placed, in each branch, in a next selected bin. At every backtrack, once a bin has been tried, all other bins having exactly the same residual space $rcap$ become unselectable.
$$B_1: bin(x)=v_1,\quad B_2: bin(x)=v_2,\quad \ldots,\quad B_n: bin(x)=v_n,\qquad\text{with } rcap(v_j)\neq rcap(v_i)\ \forall i<j$$
\end{notedef}

\begin{itemize}
	\item \textbf{Constructor}:\mylst{PackDynRemovals(VarSelector varSel, ValSelector valSel, PackSConstraint pack)}
	\item \textbf{type of variable} : integer \emph{(bin-to-item assignment variable)}
	\item \textbf{references} : \emph{n/a}
\end{itemize}

\textbf{Example}:
%\lstinputlisting{java/cabs.j2t}


\section{SetTimes (Branching strategy)}\label{settimes:settimesbranchstrat}\hypertarget{settimes:settimesbranchstrat}{}
\begin{notedef}
  \texttt{SetTimes} is a \textbf{n-ary branching} for scheduling problems with makespan minimization objective, fixing distinct task variables to their earliest starting time.
At each node, a set of available tasks is considered then, in each branch, one task variable is selected and its starting time is set to its smallest value. 
The task variables are selected in order according to a \mylst{TaskVarSelector}, or in the decreasing order defined by a \mylst{Comparator}, ties being randomly broken when boolean \mylst{rand} is set. 
%The selected variable is either chosen by a \mylst{TaskVarSelector} or as the maximal element according to a \mylst{Comparator}. Ties in the variable selection are randomly broken when boolean \mylst{rand} is set. 
\emph{The search is not complete: do not use within a} \texttt{solveAll}. 
$$B_1: T_1.start=\underline{T_1.start},\quad B_2: T_2.start=\underline{T_2.start},\quad \ldots,\quad B_n: T_n.start=\underline{T_n.start}$$
  \end{notedef}

\begin{itemize}
	\item \textbf{Constructor} :
      \begin{itemize}
      \item \mylst{SetTimes(Solver solver, List<TaskVar> tasks, TaskVarSelector varSel)}
      \item \mylst{SetTimes(Solver solver, List<TaskVar> tasks, Comparator<ITask> comp, boolean rand)}
      \end{itemize}
	\item \textbf{type of variable} : task
	\item \textbf{references} : \emph{n/a}
\end{itemize}

\textbf{Example}:
%\lstinputlisting{java/cabs.j2t}


\section{TaskOverWDegBinBranching (Branching strategy)}\label{taskdomoverwdeg:taskdomoverwdegbranchstrat}\hypertarget{taskdomoverwdeg:taskdomoverwdegbranchstrat}{}
\begin{notedef}
  \texttt{TaskOverWDegBinBranching} is a \textbf{binary branching} for scheduling problems, fixing the relative order between two task variables constrained by a precedence relation. Like in \hyperlink{domoverwdegbin:domoverwdegbinbranchstrat}{\texttt{DomOverWDegBranchingNew}}, the pair of task variables is selected according to the ratio of the size of the variable domains and the failure weight of the constraints involving these variables. 
  In the first branch, the selected pair of task variables is ordered according to the value returned by the \mylst{OrderingValSelector} ($1$ if $T_1$ precedes $T_2$ or $0$ if $T_2$ precedes $T_1$); in the second branch, the opposite relative order is enforced.
$$B_1: \texttt{precedence}(T_1,T_2,order),\quad B_2: \texttt{precedence}(T_1,T_2,1-order)$$
\end{notedef}

\begin{itemize}
\item \textbf{Constructor} :\mylst{TaskOverWDegBinBranching(Solver s, IPrecedenceRatio[] varRatios, OrderingValSelector valOrd, Number seed)}
\item \textbf{type of variable} : task \emph{(actually: precedence boolean indicator)}
\item \textbf{references} : \cite{Boussemart04}: \emph{Boosting systematic search by weighting constraints}.
\end{itemize}

\textbf{Example}:
% \lstinputlisting{java/cabs.j2t}



\chapter{Variable selectors (Solver)}\label{ch:varsel}\hypertarget{ch:varsel}{}
This section lists and details the \hyperlink{solver:variableselector}{variable selectors} currently available in Choco.

\section{CompositeIntVarSelector (Variable selector)}\label{compositeintvarselector:compositeintvarselectorvarselector}\hypertarget{compositeintvarselector:compositeintvarselectorvarselector}{}
\begin{notedef}
  \texttt{CompositeIntVarSelector}$(h_1,h_2)$ selects a constraint according to heuristic $h_1$, then selects an integer variable involved in the constraint according to heuristic $h_2$:
$$h_2(support(h_1))$$
  \end{notedef}

\begin{itemize}
	\item \textbf{Constructor} :\mylst{CompositeIntVarSelector(ConstraintSelector h1, HeuristicIntVarSelector h2)}
	\begin{itemize}
\item \texttt{ConstraintSelector} is an \texttt{interface}. No implementation provided. 
\item \texttt{HeuristicIntVarSelector} is an \texttt{abstract class}  implemented by: \hyperlink{mindomain:mindomainvarselector}{MinDomain}, \hyperlink{maxdomain:maxdomainvarselector}{MaxDomain}, \hyperlink{maxregret:maxregretvarselector}{MaxRegret}, \hyperlink{maxvaldomain:maxvaldomainvarselector}{MaxValueDomain}, \hyperlink{minvaldomain:minvaldomainvarselector}{MinValueDomain}, \hyperlink{mostconstrained:mostconstrainedvarselector}{MostConstrained}. \todo{list to complete}
\end{itemize}

	\item \textbf{type of variable} : integer
	\item \textbf{references} : \emph{n/a}
\end{itemize}

\textbf{Example}:
%\lstinputlisting{java/cabs.j2t}


\section{CyclicRealVarSelector (Variable selector)}\label{cyclicrealvarselector:cyclicrealvarselectorvarselector}\hypertarget{cyclicrealvarselector:cyclicrealvarselectorvarselector}{}
\begin{notedef}
  \texttt{CyclicRealVarSelector}$(x)$ selects the real variables in the order they appear in array $x$, in a cyclic way until they are all instantiated, i.e. with interval domain under the desired precision (\textbf{static}).
%  \texttt{CyclicRealVarSelector}$(x)$: since a dichotomy algorithm is used, cyclic assigning is needed to instantiate a real interval variable. A variable is selected several times to split its domain until it reaches the desired precision.
\end{notedef}

\begin{itemize}
	\item \textbf{Constructor} : 
	\begin{itemize}
	\item \mylst{CyclicRealVarSelector(Solver s)}
	\item \mylst{CyclicRealVarSelector(Solver s, RealVar[] vars)}
	\end{itemize}	
	\item \textbf{type of variable} : real
	\item \textbf{references} : \emph{n/a}
\end{itemize}

\textbf{Example}:
%\lstinputlisting{java/cabs.j2t}


\section{LexIntVarSelector (Variable selector)}\label{lexintvarselector:lexintvarselectorvarselector}\hypertarget{lexintvarselector:lexintvarselectorvarselector}{}
\begin{notedef}
  \texttt{LexIntVarSelector}$(h_1,h_2)$ selects the integer variable according to the first heuristic $h_1$, ties being broken by the second heuristic $h_2$:
%applies two heuristics lexicographically for selecting a variable: a first heuristic is applied finding the best variables, ties are broken with the second heuristic 
$$\begin{cases}
  h_1(x) &\quad\text{ if } |h_1(x)|=1,\\
  h_2(h_1(x)) &\quad\text{ otherwise}.
\end{cases}$$
\end{notedef}

\begin{itemize}
	\item \textbf{Constructor}: \mylst{LexIntVarSelector(TiedIntVarSelector h1, HeuristicIntVarSelector h2)}
	\begin{itemize}
\item \texttt{TiedIntVarSelector} is an \texttt{interface} implemented by : \hyperlink{mindomain:mindomainvarselector}{MinDomain}, \hyperlink{maxdomain:maxdomainvarselector}{MaxDomain}, \hyperlink{maxregret:maxregretvarselector}{MaxRegret}, \hyperlink{maxvaldomain:maxvaldomainvarselector}{MaxValueDomain}, \hyperlink{minvaldomain:minvaldomainvarselector}{MinValueDomain}, \hyperlink{mostconstrained:mostconstrainedvarselector}{MostConstrained}, \hyperlink{randomvarint:randomvarintvarselector}{RandomIntVarSelector}.
\item \texttt{HeuristicIntVarSelector} is an \texttt{abstract class}  implemented by: \hyperlink{mindomain:mindomainvarselector}{MinDomain}, \hyperlink{maxdomain:maxdomainvarselector}{MaxDomain}, \hyperlink{maxregret:maxregretvarselector}{MaxRegret}, \hyperlink{maxvaldomain:maxvaldomainvarselector}{MaxValueDomain}, \hyperlink{minvaldomain:minvaldomainvarselector}{MinValueDomain}, \hyperlink{mostconstrained:mostconstrainedvarselector}{MostConstrained}, \hyperlink{randomvarint:randomvarintvarselector}{RandomIntVarSelector}.
\end{itemize}
	\item \textbf{type of variable} : integer
	\item \textbf{references} : \emph{n/a}
\end{itemize}

\textbf{Example}:
%\lstinputlisting{java/cabs.j2t}


\section{MaxDomain (Variable selector)}\label{maxdomain:maxdomainvarselector}\hypertarget{maxdomain:maxdomainvarselector}{}
\begin{notedef}
  \texttt{MaxDomain}$(x)$ selects the integer variable with the largest domain (\textbf{dynamic}):
$$\max |D(x)|$$
\end{notedef}

\begin{itemize}
	\item \textbf{Constructor} : 
	\begin{itemize}
	\item \mylst{MaxDomain(Solver s)}
	\item \mylst{MaxDomain(Solver s, IntDomainVar[] vars)}
	\end{itemize}	
	\item \textbf{type of variable} : integer
	\item \textbf{references} : \emph{n/a}
\end{itemize}

\textbf{Example}:
%\lstinputlisting{java/cabs.j2t}


\section{MaxDomSet (Variable selector)}\label{maxdomset:maxdomsetvarselector}\hypertarget{maxdomset:maxdomsetvarselector}{}
\begin{notedef}
  \texttt{MaxDomSet}$(x)$ selects the set variable with the largest open domain (\textbf{dynamic}):
$$\max |Env(x)\setminus Ker(x)|$$
\end{notedef}

\begin{itemize}
	\item \textbf{Constructor} : 
	\begin{itemize}
	\item \mylst{MaxDomSet(Solver s)}
	\item \mylst{MaxDomSet(Solver s, SetVar[] vars)}
	\end{itemize}	
	\item \textbf{type of variable} : set
	\item \textbf{references} : \emph{n/a}
\end{itemize}

\textbf{Example}:
%\lstinputlisting{java/cabs.j2t}


\section{MaxRegret (Variable selector)}\label{maxregret:maxregretvarselector}\hypertarget{maxregret:maxregretvarselector}{}
\begin{notedef}
  \texttt{MaxRegret}$(x)$ selects the integer variable with the largest difference between the two smallest values in its domain (\textbf{dynamic}):
$$\max(\underline{x}_2 - \underline{x}),\qquad\text{with } \underline{x}_2=\min(D(x)\setminus\underline{x})$$
\end{notedef}

\begin{itemize}
	\item \textbf{Constructor} : 
	\begin{itemize}
	\item \mylst{MaxRegret(Solver s)}
	\item \mylst{MaxRegret(Solver s, IntDomainVar[] vars)}
	\end{itemize}	
	\item \textbf{type of variable} : integer
	\item \textbf{references} : \emph{n/a}
\end{itemize}

\textbf{Example}:
%\lstinputlisting{java/cabs.j2t}


\section{MaxRegretSet (Variable selector)}\label{maxregretset:maxregretsetvarselector}\hypertarget{maxregretset:maxregretsetvarselector}{}
\begin{notedef}
  \texttt{MaxRegretSet}$(x)$ selects the set variable with the largest difference between the two smallest values in its envelope (\textbf{dynamic}):
$$\max(\underline{x}_2 - \underline{x}),\qquad\text{with } \underline{x}=\min(Env(x)), \underline{x}_2=\min(Env(x)\setminus\underline{x})$$
\end{notedef}

\begin{itemize}
	\item \textbf{Constructor} : 
	\begin{itemize}
	\item \mylst{MaxRegretSet(Solver s)}
	\item \mylst{MaxRegretSet(Solver s, SetVar[] vars)}
	\end{itemize}	
	\item \textbf{type of variable} : set
	\item \textbf{references} : \emph{n/a}
\end{itemize}

\textbf{Example}:
%\lstinputlisting{java/cabs.j2t}


\section{MaxValueDomain (Variable selector)}\label{maxvaldomain:maxvaldomainvarselector}\hypertarget{maxvaldomain:maxvaldomainvarselector}{}
\begin{notedef}
  \texttt{MaxValueDomain}$(x)$ selects the integer variable with the largest value in its domain (\textbf{dynamic}):
$$\max(\bar{x})$$
\end{notedef}

\begin{itemize}
	\item \textbf{Constructor} : 
	\begin{itemize}
	\item \mylst{MaxValueDomain(Solver s)}
	\item \mylst{MaxValueDomain(Solver s, IntDomainVar[] vars)}
	\end{itemize}	
	\item \textbf{type of variable} : integer
	\item \textbf{references} : \emph{n/a}
\end{itemize}

\textbf{Example}:
%\lstinputlisting{java/cabs.j2t}


\section{MaxValueDomSet (Variable selector)}\label{maxvaldomset:maxvaldomsetvarselector}\hypertarget{maxvaldomset:maxvaldomsetvarselector}{}
\begin{notedef}
  \texttt{MaxValueDomSet}$(x)$ selects the set variable with the largest value in its envelope (\textbf{dynamic}):
$$\max(\bar{x}),\qquad\text{with } \bar{x}=\max(Env(x))$$
\end{notedef}

\begin{itemize}
	\item \textbf{Constructor} : 
	\begin{itemize}
	\item \mylst{MaxValueDomSet(Solver s)}
	\item \mylst{MaxValueDomSet(Solver s, SetVar[] vars)}
	\end{itemize}	
	\item \textbf{type of variable} : set
	\item \textbf{references} : \emph{n/a}
\end{itemize}

\textbf{Example}:
%\lstinputlisting{java/cabs.j2t}


\section{MinDomain (Variable selector)}\label{mindomain:mindomainvarselector}\hypertarget{mindomain:mindomainvarselector}{}
\begin{notedef}
  \texttt{MinDomain}$(x)$ selects the integer variable with the smallest domain (\textbf{dynamic}):
$$\min |D(x)|$$
\end{notedef}

\begin{itemize}
	\item \textbf{Constructor} : 
	\begin{itemize}
	\item \mylst{MinDomain(Solver s)}
	\item \mylst{MinDomain(Solver s, IntDomainVar[] vars)}
	\end{itemize}	
	\item \textbf{type of variable} : integer
	\item \textbf{references} : \emph{n/a}
\end{itemize}

\textbf{Example}:
%\lstinputlisting{java/cabs.j2t}


\section{MinDomSet (Variable selector)}\label{mindomset:mindomsetvarselector}\hypertarget{mindomset:mindomsetvarselector}{}
\begin{notedef}
  \texttt{MinDomSet}$(x)$ selects the set variable with the smallest open domain (\textbf{dynamic}):
$$\min |Env(x)\setminus Ker(x)|$$
\end{notedef}

\begin{itemize}
	\item \textbf{Constructor} : 
	\begin{itemize}
	\item \mylst{MinDomSet(Solver s)}
	\item \mylst{MinDomSet(Solver s, SetVar[] vars)}
	\end{itemize}	
	\item \textbf{type of variable} : set
	\item \textbf{references} : \emph{n/a}
\end{itemize}

\textbf{Example}:
%\lstinputlisting{java/cabs.j2t}


\section{MinValueDomain (Variable selector)}\label{minvaldomain:minvaldomainvarselector}\hypertarget{minvaldomain:minvaldomainvarselector}{}
\begin{notedef}
  \texttt{MinValueDomain}$(x)$ selects the integer variable with the smallest value in its domain (\textbf{dynamic}):
$$\min(\underline{x})$$
\end{notedef}

\begin{itemize}
	\item \textbf{Constructor} : 
	\begin{itemize}
	\item \mylst{MinValueDomain(Solver s)}
	\item \mylst{MinValueDomain(Solver s, IntDomainVar[] vars)}
	\end{itemize}	
	\item \textbf{type of variable} : integer
	\item \textbf{references} : \emph{n/a}
\end{itemize}

\textbf{Example}:
%\lstinputlisting{java/cabs.j2t}


\section{MinValueDomSet (Variable selector)}\label{minvaldomset:minvaldomsetvarselector}\hypertarget{minvaldomset:minvaldomsetvarselector}{}
\begin{notedef}
  \texttt{MinValueDomSet}$(x)$ selects the set variable with the smallest value in its envelope (\textbf{dynamic}):
$$\min(\underline{x}),\qquad\text{with } \underline{x}=\min(Env(x))$$
\end{notedef}

\begin{itemize}
	\item \textbf{Constructor} : 
	\begin{itemize}
	\item \mylst{MinValueDomSet(Solver s)}
	\item \mylst{MinValueDomSet(Solver s, SetVar[] vars)}
	\end{itemize}	
	\item \textbf{type of variable} : set
	\item \textbf{references} : \emph{n/a}
\end{itemize}

\textbf{Example}:
%\lstinputlisting{java/cabs.j2t}


\section{MostConstrained (Variable selector)}\label{mostconstrained:mostconstrainedvarselector}\hypertarget{mostconstrained:mostconstrainedvarselector}{}
\begin{notedef}
  \texttt{MostConstrained}$(x)$ selects the integer variable involved in the largest number of constraints initially present in the solver (\textbf{static}):
$$\max(initDeg(x))$$
\end{notedef}

\begin{itemize}
	\item \textbf{Constructor} : 
	\begin{itemize}
	\item \mylst{MostConstrained(Solver s)}
	\item \mylst{MostConstrained(Solver s, IntDomainVar[] vars)}
	\end{itemize}	
	\item \textbf{type of variable} : integer
	\item \textbf{references} : \emph{n/a}
\end{itemize}

\textbf{Example}:
%\lstinputlisting{java/cabs.j2t}


\section{MostConstrainedSet (Variable selector)}\label{mostconstrainedset:mostconstrainedsetvarselector}\hypertarget{mostconstrainedset:mostconstrainedsetvarselector}{}
\begin{notedef}
  \texttt{MostConstrainedSet}$(x)$ selects the set variable involved in the largest number of constraints initially present in the solver (\textbf{static}):
$$\max(initDeg(x))$$
\end{notedef}

\begin{itemize}
	\item \textbf{Constructor} : 
	\begin{itemize}
	\item \mylst{MostConstrainedSet(Solver s)}
	\item \mylst{MostConstrainedSet(Solver s, SetVar[] vars)}
	\end{itemize}	
	\item \textbf{type of variable} : set
	\item \textbf{references} : \emph{n/a}
\end{itemize}

\textbf{Example}:
%\lstinputlisting{java/cabs.j2t}


\section{RandomIntVarSelector (Variable selector)}\label{randomvarint:randomvarintvarselector}\hypertarget{randomvarint:randomvarintvarselector}{}
\begin{notedef}
  \texttt{RandomIntVarSelector}$(x)$ selects an integer variable randomly (\textbf{dynamic}). The random \texttt{seed} can be fixed.
\end{notedef}

\begin{itemize}
	\item \textbf{Constructor} : 
	\begin{itemize}
	\item \mylst{RandomIntVarSelector(Solver s)}
	\item \mylst{RandomIntVarSelector(Solver s, long seed)}
	\item \mylst{RandomIntVarSelector(Solver s, IntDomainVar[] vars, long seed)}
	\end{itemize}	
	\item \textbf{type of variable} : integer
	\item \textbf{references} : \emph{n/a}
\end{itemize}

\textbf{Example}:
%\lstinputlisting{java/cabs.j2t}


\section{RandomSetVarSelector (Variable selector)}\label{randomvarset:randomvarsetvarselector}\hypertarget{randomvarset:randomvarsetvarselector}{}
\begin{notedef}
  \texttt{RandomSetVarSelector}$(x)$ selects a set variable randomly (\textbf{dynamic}). The random \texttt{seed} can be fixed.
\end{notedef}

\begin{itemize}
	\item \textbf{Constructor} : 
	\begin{itemize}
	\item \mylst{RandomSetVarSelector(Solver s)}
	\item \mylst{RandomSetVarSelector(Solver s, long seed)}
	\item \mylst{RandomSetVarSelector(Solver s, SetVar[] vars, long seed)}
	\end{itemize}	
	\item \textbf{type of variable} : set
	\item \textbf{references} : \emph{n/a}
\end{itemize}

\textbf{Example}:
%\lstinputlisting{java/cabs.j2t}


\section{StaticSetVarOrder (Variable selector)}\label{staticsetvarorder:staticsetvarordervarselector}\hypertarget{staticsetvarorder:staticsetvarordervarselector}{}
\begin{notedef}
  \texttt{StaticSetVarOrder}$(x)$ selects the set variables in the order they appear in array $x$ (\textbf{static}).
$$x_1$$
\end{notedef}

\begin{itemize}
	\item \textbf{Constructor} : 
	\begin{itemize}
	\item \mylst{StaticSetVarOrder(Solver s)}
	\item \mylst{StaticSetVarOrder(Solver s, SetVar[] vars)}
	\end{itemize}	
	\item \textbf{type of variable} : set
	\item \textbf{references} : \emph{n/a}
\end{itemize}

\textbf{Example}:
%\lstinputlisting{java/cabs.j2t}


\section{StaticVarOrder (Variable selector)}\label{staticvarorder:staticvarordervarselector}\hypertarget{staticvarorder:staticvarordervarselector}{}
\begin{notedef}
  \texttt{StaticVarOrder}$(x)$ selects the integer variables in the order they appear in array $x$ (\textbf{static}).
$$x_1$$
\end{notedef}

\begin{itemize}
	\item \textbf{Constructor} : 
	\begin{itemize}
	\item \mylst{StaticVarOrder(Solver s)}
	\item \mylst{StaticVarOrder(Solver s, IntDomainVar[] vars)}
	\end{itemize}	
	\item \textbf{type of variable} : integer
	\item \textbf{references} : \emph{n/a}
\end{itemize}

\textbf{Example}:
%\lstinputlisting{java/cabs.j2t}



\chapter{Value iterators (Solver)}\label{ch:valite}\hypertarget{ch:valite}{}
This section lists and details the \hyperlink{solver:valueiterator}{value iterators} currently available in Choco.

\section{DecreasingDomain (Value iterator)}\label{decreasingdomain:decreasingdomainvaliterator}\hypertarget{decreasingdomain:decreasingdomainvaliterator}{}
\begin{notedef}
  \texttt{DecreasingDomain} selects the integer variable largest value:
$$\max(D(x))$$
\end{notedef}

\begin{itemize}
	\item \textbf{Constructor} : \mylst{DecreasingDomain()}
	\item \textbf{type of variable} : integer
%	\item \textbf{references} : \emph{n/a}
\end{itemize}

%\textbf{Example}:
%\lstinputlisting{java/cabs.j2t}


\section{IncreasingDomain (Value iterator)}\label{increasingdomain:increasingdomainvaliterator}\hypertarget{increasingdomain:increasingdomainvaliterator}{}
\begin{notedef}
  \texttt{IncreasingDomain} selects the integer variable smallest value:
$$\min(D(x))$$
\end{notedef}

\begin{itemize}
	\item \textbf{Constructor} : \mylst{IncreasingDomain()}
	\item \textbf{type of variable} : integer
%	\item \textbf{references} : \emph{n/a}
\end{itemize}

%\textbf{Example}:
%\lstinputlisting{java/cabs.j2t}


\section{RealIncreasingDomain (Value iterator)}\label{realincreasingdomain:realincreasingdomainvaliterator}\hypertarget{realincreasingdomain:realincreasingdomainvaliterator}{}
\begin{notedef}
  \texttt{RealIncreasingDomain} selects the real variable smallest value:
$$\min(D(x))$$
\end{notedef}

\begin{itemize}
	\item \textbf{Constructor} : \mylst{RealIncreasingDomain()}
	\item \textbf{type of variable} : real
%	\item \textbf{references} : \emph{n/a}
\end{itemize}

%\textbf{Example}:
%\lstinputlisting{java/cabs.j2t}



\chapter{Value selector (Solver)}\label{ch:valsel}\hypertarget{ch:valsel}{}
This section lists and details the \hyperlink{solver:valueselector}{value selectors} currently available in Choco.

\section{BestFit (Value selector)}\label{bestfit:bestfitvalselector}\hypertarget{bestfit:bestfitvalselector}{}
\begin{notedef}
  \texttt{BestFit}, associated with a \hyperlink{pack:packconstraint}{pack}(items, load, bin, size) constraint, selects the bin $v$ with the minimum residual space $rcap_v$ for the bin assignment integer variable $\mathtt{bin}[i]$ of item $i$:
$$v\in D(\mathtt{bin}[i]):\quad\min\{rcap_v\}\qquad\text{with } rcap_v=\overline{\mathtt{load}[v]} -\!\sum_{j\in Ker(\mathtt{items}[v])}\!\mathtt{size}[j]$$
\end{notedef}

\begin{itemize}
	\item \textbf{Constructor} : \mylst{BestFit(PackSConstraint cstr)}
	\item \textbf{type of variable} : integer (\emph{bin-to-item assignment variable} $\mathtt{bin}[i]$)
%	\item \textbf{references} : \emph{n/a}
\end{itemize}

%\textbf{Example}:
%\lstinputlisting{java/cabs.j2t}


\section{CostRegularValSelector (Value selector)}\label{costregularvalselector:costregularvalselectorvalselector}\hypertarget{costregularvalselector:costregularvalselectorvalselector}{}
\begin{notedef}
  \texttt{CostRegularValSelector}, associated with a \hyperlink{costregular:costregularconstraint}{\texttt{CostRegular}}$(z, \collec{x_1}{x_n},\mathcal{L}(\Pi), \coll{c_{i,j}})$ constraint, selects a value $v_i$ in the domain of the sequence integer variable $x_i$ that can be extended to a solution for the constraint $\collec{v_1}{v_n}\in\mathcal{L}(\Pi)$ of maximum cost (if \texttt{max}, resp. minimum cost if $\neg$\texttt{max}):
$$ v_i\in D(x_i):\quad \exists v_j\in D(x_j)\ \forall j\neq i, \text{ s.t. } \sum_{j=1}^n c_{j,v_j} = \bar{z}\ (\text{resp. } =\underline{z})$$   
\end{notedef}
Remember that the underlying data structure of a \texttt{CostRegular} constraint is a layered acyclic digraph, where each arc in the $i$-th layer corresponds to a possible assignment of variable $x_i$ to a value $v\in D(x_i)$ and has length $c_{iv}$, and where each path corresponds to a solution of the constraint, i.e. a word $\collec{v_1}{v_n}$ of language $\mathcal{L}(\Pi)$, whose cost $z$ is equal to the path length $\sum_{j=1}^n c_{j,v_j}$.

This heuristic works with the \mylst{ConstraintType.COSTREGULAR} implementation of the constraint: see \hyperlink{fcostregularvalselector:fcostregularvalselectorvalselector}{FCostRegularValSelector} for use with the \mylst{ConstraintType.FASTCOSTREGULAR} implementation (preferred).

\begin{itemize}
	\item \textbf{Constructor} : \mylst{CostRegularValSelector(CostRegular cr, boolean max)}
	\item \textbf{type of variable} : integer
%	\item \textbf{references} : \emph{n/a}
\end{itemize}

%\textbf{Example}:
%\lstinputlisting{java/cabs.j2t}


\input{chapters/strategies/vsfastcostregularvalselector}
\section{MaxVal (Value selector)}\label{maxval:maxvalvalselector}\hypertarget{maxval:maxvalvalselector}{}
\begin{notedef}
  \texttt{MaxVal} selects the largest value in the domain of the integer variable:
$$\max(D(x))$$
\end{notedef}

\begin{itemize}
	\item \textbf{Constructor} : \mylst{MaxVal()}
	\item \textbf{type of variable} : integer
%	\item \textbf{references} : \emph{n/a}
\end{itemize}

%\textbf{Example}:
%\lstinputlisting{java/cabs.j2t}


%\section{MCRValSelector (Value selector)}\label{mcrvalselector:mcrvalselectorvalselector}\hypertarget{mcrvalselector:mcrvalselectorvalselector}{}
\begin{notedef}
  \texttt{MCRValSelector}, for rostering problems associated with a table of \hyperlink{costregular:costregularconstraint}{\texttt{MultiCostRegular}} constraints$(z, \collec{x_1}{x_n},\mathcal{L}(\Pi), \coll{c_{i,j}})$ constraint, selects a value $v_i$ in the domain of the sequence integer variable $x_i$ that can be extended to a solution for the constraint $\collec{v_1}{v_n}\in\mathcal{L}(\Pi)$ of maximum cost (if \texttt{max}, resp. minimum cost if $\neg$\texttt{max}):
$$ v_i\in D(x_i):\quad \exists v_j\in D(x_j)\ \forall j\neq i, \text{ s.t. } \sum_{j=1}^n c_{j,v_j} = \bar{z}\ (\text{resp. } =\underline{z})$$   
\end{notedef}
Remember that the underlying data structure of a \texttt{CostRegular} constraint is a layered acyclic digraph, where each arc in the $i$-th layer corresponds to a possible assignment of variable $x_i$ to a value $v\in D(x_i)$ and has length $c_{iv}$, and where each path corresponds to a solution of the constraint, i.e. a word $\collec{v_1}{v_n}$ of language $\mathcal{L}(\Pi)$, whose cost $z$ is equal to the path length $\sum_{j=1}^n c_{j,v_j}$.

\begin{itemize}
	\item \textbf{Constructor} : \mylst{MCRValSelector(FastMultiCostRegular[] ct, boolean max)}
	\item \textbf{type of variable} : integer
	\item \textbf{references} : \emph{n/a}
\end{itemize}

\textbf{Example}:
%\lstinputlisting{java/cabs.j2t}


\section{MidVal (Value selector)}\label{midval:midvalvalselector}\hypertarget{midval:midvalvalselector}{}
\begin{notedef}
  \texttt{MidVal} selects the closest value (equal or greater) to the integer variable domain midpoint:
$$\min(v\ |\ v\ge \frac{\underline{x}+\overline{x}}{2})$$
\end{notedef}

\begin{itemize}
	\item \textbf{Constructor} : \mylst{MidVal()}
	\item \textbf{type of variable} : integer
%	\item \textbf{references} : \emph{n/a}
\end{itemize}

%\textbf{Example}:
%\lstinputlisting{java/cabs.j2t}


\section{MinEnv (Value selector)}\label{minenv:minenvvalselector}\hypertarget{minenv:minenvvalselector}{}
\begin{notedef}
  \texttt{MinEnv} selects the smallest value in the open domain of the set variable:
$$\min(Env(x)\setminus Ker(x))$$
\end{notedef}

\begin{itemize}
	\item \textbf{Constructor} : \mylst{MinEnv()}
	\item \textbf{type of variable} : set
%	\item \textbf{references} : \emph{n/a}
\end{itemize}

%\textbf{Example}:
%\lstinputlisting{java/cabs.j2t}


\section{MinVal (Value selector)}\label{minval:minvalvalselector}\hypertarget{minval:minvalvalselector}{}
\begin{notedef}
  \texttt{MinVal} selects the smallest value in the domain of the integer variable:
$$\min(D(x))$$
\end{notedef}

\begin{itemize}
	\item \textbf{Constructor} : \mylst{MinVal()}
	\item \textbf{type of variable} : integer
%	\item \textbf{references} : \emph{n/a}
\end{itemize}

%\textbf{Example}:
%\lstinputlisting{java/cabs.j2t}


\section{RandomIntValSelector (Value selector)}\label{randomintvalselector:randomintvalselectorvalselector}\hypertarget{randomintvalselector:randomintvalselectorvalselector}{}
\begin{notedef}
  \texttt{RandomIntValSelector} selects a random value in the domain of the integer variable. The random \texttt{seed} can be specified:
$$rand(D(x))$$
In case of bounded domain, it selects one of the bounds of the domain. 

\end{notedef}

\begin{itemize}
	\item \textbf{Constructor} : 
		\begin{itemize}
		\item \mylst{RandomIntValSelector()}
		\item \mylst{RandomIntValSelector(long seed)}
		\end{itemize}
	\item \textbf{type of variable} : set
%	\item \textbf{references} : \emph{n/a}
\end{itemize}

%\textbf{Example}:
%\lstinputlisting{java/cabs.j2t}


\section{RandomSetValSelector (Value selector)}\label{randomsetvalselector:randomsetvalselectorvalselector}\hypertarget{randomsetvalselector:randomsetvalselectorvalselector}{}
\begin{notedef}
  \texttt{RandomSetValSelector} selects a random value in the open domain of the set variable. The random \texttt{seed} can be specified:
$$rand(Env(x)\setminus Ker(x))$$
\end{notedef}

\begin{itemize}
	\item \textbf{Constructor} : 
		\begin{itemize}
		\item \mylst{RandomSetValSelector()}
		\item \mylst{RandomSetValSelector(long seed)}
		\end{itemize}
	\item \textbf{type of variable} : set
%	\item \textbf{references} : \emph{n/a}
\end{itemize}

%\textbf{Example}:
%\lstinputlisting{java/cabs.j2t}



\chapter{Visualizer (Visualization)}\label{ch:visualizer}\hypertarget{ch:visualizer}{}
\section{Variable-oriented Visualizers}\label{vov:visualizers}\hypertarget{vov:visualizers}{}
This section lists and details the \hyperlink{chocoandcpviz:visualizers}{variable-oriented visualizers} currently available in Choco.
\subsection{BinaryMatrix (visualizer)}\label{binarymatrix:visu}\hypertarget{binarymatrix:visu}{}
\begin{notedef}
  \texttt{BinaryMatrix}$(bool)$ is a specialized visualizer for a matrix of 0/1 variables $bool$.
\end{notedef}

\begin{itemize}
	\item \textbf{API} : 
	\begin{itemize}
	\item \mylst{BinaryMatrix(IntDomainVar[][] bool, String display, int width, int height)}
	\item \mylst{BinaryMatrix(IntDomainVar[][] bool, String display, int x, int y, int width, int height, String group, int min, int max)}
	\end{itemize}
\end{itemize}

%\textbf{Example}:
%\lstinputlisting{java/cabs.j2t}


\subsection{BinaryVector (visualizer)}\label{binaryvector:visu}\hypertarget{binaryvector:visu}{}
\begin{notedef}
  \texttt{BinaryVector}$(bool)$  is a specialized visualizer for a vector of 0/1 variables $bool$.
\end{notedef}

\begin{itemize}
	\item \textbf{API} : 
	\begin{itemize}
	\item \mylst{BinaryVector(IntDomainVar[] bool, String display, int width, int height)}
	\item \mylst{BinaryVector(IntDomainVar[] bool, String display, int x, int y, int width, int height, String group, int min, int max)}
	\end{itemize}
\end{itemize}

%\textbf{Example}:
%\lstinputlisting{java/cabs.j2t}


\subsection{DomainMatrix (visualizer)}\label{domainmatrix:visu}\hypertarget{domainmatrix:visu}{}
\begin{notedef}
  \texttt{DomainMatrix}$(xs)$ is a specialized visualizer for a matrix of variables $xs$.
\end{notedef}

\begin{itemize}
	\item \textbf{API} : 
	\begin{itemize}
	\item \mylst{DomainMatrix(Var[][] vars, String display, int width, int height)}
	\item \mylst{DomainMatrix(Var[][] vars, String display, int x, int y, int width, int height, String group, int min, int max)}
	\end{itemize}
\end{itemize}

%\textbf{Example}:
%\lstinputlisting{java/cabs.j2t}


\subsection{Vector (visualizer)}\label{vector:visu}\hypertarget{vector:visu}{}
\begin{notedef}
  \texttt{Vector}$(vars)$ is a specialized visualizer for variables $vars$.
\end{notedef}

\begin{itemize}
	\item \textbf{API} : 
	\begin{itemize}
	\item \mylst{Vector(Var[] vars, String display, int width, int height)}
	\item \mylst{Vector(Var[] vars, String display, int x, int y, int width, int height, String group, int min, int max)}
	\end{itemize}
\end{itemize}

%\textbf{Example}:
%\lstinputlisting{java/cabs.j2t}


\subsection{VectorSize (visualizer)}\label{vectorsize:visu}\hypertarget{vectorsize:visu}{}
\begin{notedef}
  \texttt{VectorSize}$(vars)$ is specialized visualizer for the sum of cardinalities of variables $vars$.
\end{notedef}

\begin{itemize}
	\item \textbf{API} : 
	\begin{itemize}
	\item \mylst{VectorSize(Var[] vars, String display, int width, int height)}
	\item \mylst{VectorSize(Var[] vars, String display, int x, int y, int width, int height, String group, int min, int max)}
	\end{itemize}
\end{itemize}

%\textbf{Example}:
%\lstinputlisting{java/cabs.j2t}


\subsection{VectorWaterfall (visualizer)}\label{vectorwaterfall:visu}\hypertarget{vectorwaterfall:visu}{}
\begin{notedef}
  \texttt{VectorWaterfall}$(vars$ is a specialized visualizer for a waterfall representation of a vector of variables $vars$.
\end{notedef}

\begin{itemize}
	\item \textbf{API} : 
	\begin{itemize}
	\item \mylst{VectorWaterfall(Var[] vars, String display, int width, int height)}
	\item \mylst{VectorWaterfall(Var[] vars, String display, int x, int y, int width, int height, String group, int min, int max)}
	\end{itemize}
\end{itemize}

%\textbf{Example}:
%\lstinputlisting{java/cabs.j2t}



\section{Constraint-oriented Visualizers}\label{cov:visualizers}\hypertarget{cov:visualizers}{}
This section lists and details the \hyperlink{chocoandcpviz:visualizers}{constraint-oriented visualizers} currently available in Choco.
\subsection{AllDifferent (visualizer)}\label{alldiff:visu}\hypertarget{alldiff:visu}{}
\begin{notedef}
  \texttt{AllDifferent}$(vars)$ a specialized visualizer for the \mylst{AllDifferent} constraint, where $vars$ is an array of variables involved in one AllDifferent constraint.
\end{notedef}

\begin{itemize}
	\item \textbf{API} : 
	\begin{itemize}
	\item \mylst{AllDifferent(IntDomainVar[] vars, String display, int width, int height)}
	\item \mylst{AllDifferent(IntDomainVar[] vars, String display, int x, int y, int width, int height, String group, int min, int max)}
	\end{itemize}
\end{itemize}

%\textbf{Example}:
%\lstinputlisting{java/cabs.j2t}


\subsection{AllDifferentMatrix (visualizer)}\label{alldiffmatrix:visu}\hypertarget{alldiffmatrix:visu}{}
\begin{notedef}
  \texttt{AllDifferentMatrix}$(vars)$ a specialized visualizer for a list of \mylst{AllDifferent} constraints, where $vars$ is a matrix of variables involved in $n$ AllDifferent constraints.
\end{notedef}

\begin{itemize}
	\item \textbf{API} : 
	\begin{itemize}
	\item \mylst{AllDifferentMatrix(IntDomainVar[][] vars, String display, int width, int height)}
	\item \mylst{AllDifferentMatrix(IntDomainVar[][] vars, String type, String display, int x, int y, int width, int height, String group, int min, int max)}
	\end{itemize}
\end{itemize}

%\textbf{Example}:
%\lstinputlisting{java/cabs.j2t}


\subsection{BooleanChanneling (visualizer)}\label{boolchan:visu}\hypertarget{boolchan:visu}{}
\begin{notedef}
  \texttt{BooleanChanneling}$(x, bs, o)$ is a specialized visualizer for the boolean channeling constraint, where $x$ is a variable, $bs$ an array of boolean variables and $o$ the first value (in Choco: \mylst{domainChanneling}).
\end{notedef}

\begin{itemize}
	\item \textbf{API} : 
	\begin{itemize}
	\item \mylst{BoolChanneling(IntDomainVar x, IntDomainVar[] bs, int o, String display, int width, int height)}
	\item \mylst{BoolChanneling(IntDomainVar x, IntDomainVar[] bs, String display, int width, int height)}, where $o$ is computed from the lower bound $x$'s domain.
	\item \mylst{BoolChanneling(IntDomainVar var, IntDomainVar[] bool, int o, String display, int x, int y, int width, int height, String group, int min, int max)}
	\item \mylst{BoolChanneling(IntDomainVar var, IntDomainVar[] bool, String display, int x, int y, int width, int height, String group, int min, int max)}, where $o$ is computed from the lower bound $x$'s domain.
	\end{itemize}
\end{itemize}

%\textbf{Example}:
%\lstinputlisting{java/cabs.j2t}


\subsection{Cumulative (visualizer)}\label{cumulative:visu}\hypertarget{cumulative:visu}{}
\begin{notedef}
  \texttt{Cumulative}$(ts, l, e)$ is a specialized visualizer for the \mylst{Cumulative} constraint, where $ts$ is an array of task variables, $l$ the limit variable and $e$ the end variable.
\end{notedef}

\begin{itemize}
	\item \textbf{API} : 
	\begin{itemize}
	\item \mylst{Cumulative(TaskVar[] tasks, IntDomainVar limit, IntDomainVar end, String display, int width, int height)}, where \mylst{display} can take its value in \mylst{\{"compact", "expanded", "gantt"\}}
	\item \mylst{Cumulative(TaskVar[] tasks, IntDomainVar limit, IntDomainVar end, String display, int x, int y, int width, int height, String group, int min, int max)}, where \mylst{display} can take its value in \mylst{\{"compact", "expanded", "gantt"\}}
	\end{itemize}
\end{itemize}

%\textbf{Example}:
%\lstinputlisting{java/cabs.j2t}


\subsection{Element (visualizer)}\label{element:visu}\hypertarget{element:visu}{}
\begin{notedef}
  \texttt{Element}$(index, values, value)$ is a specialized visualizer for the \mylst{Element} constraint, where $index$ is a variable, $values$ an array of integer values and $value$ a variable.
\end{notedef}

\begin{itemize}
	\item \textbf{API} : 
	\begin{itemize}
	\item \mylst{Element(IntDomainVar index, int[] values, IntDomainVar value, String display, int width, int height)}
	\item \mylst{Element(IntDomainVar index, int[] values, IntDomainVar value, String display, int x, int y, int width, int height, String group, int min, int max)}
	\end{itemize}
\end{itemize}

%\textbf{Example}:
%\lstinputlisting{java/cabs.j2t}


\subsection{Gcc (visualizer)}\label{gcc:visu}\hypertarget{gcc:visu}{}
\begin{notedef}
  \texttt{Gcc}$(vars, values, low, high)$ is a specialized visualizer for the \mylst{Gcc} constraint, where $vars$ is an array of variables, $values$ an array of integer values, $low$ minimum number of occurrences of values and $high$ maximum number of occurrences of values.
\end{notedef}

\begin{itemize}
	\item \textbf{API} : 
	\begin{itemize}
	\item \mylst{Gcc(IntDomainVar[] vars, int[] values, int[] low, int[] high, String display, int width, int height)}
	\item \mylst{Gcc(IntDomainVar[] vars, int[] values, int[] low, int[] high, String display, int x, int y, int width, int height, String group, int min, int max) }
	\end{itemize}
\end{itemize}

%\textbf{Example}:
%\lstinputlisting{java/cabs.j2t}


\section{Inverse (visualizer)}\label{inverse:visu}\hypertarget{inverse:visu}{}
\begin{notedef}
  \texttt{Inverse}$(X, Y)$ is a specialized visualizer for the \mylst{Inverse} constraint, where $X$ is an array of variables and $Y$ is an other array of variables.
\end{notedef}

\begin{itemize}
	\item \textbf{API} : 
	\begin{itemize}
	\item \mylst{Inverse(IntDomainVar[] X, IntDomainVar[] Y, String display, int width, int height)}
	\item \mylst{Inverse(IntDomainVar[] X, IntDomainVar[] Y, String display, int x, int y, int width, int height, String group, int min, int max)}
	\end{itemize}
\end{itemize}

%\textbf{Example}:
%\lstinputlisting{java/cabs.j2t}


\subsection{LexLe (visualizer)}\label{lexle:visu}\hypertarget{lexle:visu}{}
\begin{notedef}
  \texttt{LexLe}$(X, Y)$ is a specialized visualizer for the \mylst{LexLe} constraint, where $X$ is an array of variables and $Y$ is an other array of variables.
\end{notedef}

\begin{itemize}
	\item \textbf{API} : 
	\begin{itemize}
	\item \mylst{LexLe(IntDomainVar[] X, IntDomainVar[] Y, String display, int width, int height)}
	\item \mylst{LexLe(IntDomainVar[] X, IntDomainVar[] Y, String display, int x, int y, int width, int height, String group, int min, int max)}
	\end{itemize}
\end{itemize}

%\textbf{Example}:
%\lstinputlisting{java/cabs.j2t}


\subsection{LexLt (visualizer)}\label{lexlt:visu}\hypertarget{lexlt:visu}{}
\begin{notedef}
  \texttt{LexLt}$(X, Y)$ is a specialized visualizer for the \mylst{LexLt} constraint, where $X$ is an array of variables and $Y$ is an other array of variables.
\end{notedef}

\begin{itemize}
	\item \textbf{API} : 
	\begin{itemize}
	\item \mylst{LexLt(IntDomainVar[] X, IntDomainVar[] Y, String display, int width, int height)}
	\item \mylst{LexLt(IntDomainVar[] X, IntDomainVar[] Y, String display, int x, int y, int width, int height, String group, int min, int max)}
	\end{itemize}
\end{itemize}

%\textbf{Example}:
%\lstinputlisting{java/cabs.j2t}



\part{Extras}\label{ch:extra}\hypertarget{ch:extra}{}
%\input{chapters/Choco_and_visu.tex}
%\part{sudoku and cp}
\label{sudokuandcp}
\hypertarget{sudokuandcp}{}

\chapter{Sudoku and Constraint Programming}\label{sudokuandcp:sudokuandconstraintprogramming}\hypertarget{sudokuandcp:sudokuandconstraintprogramming}{}

\section{Sudoku ?!?}\label{sudokuandcp:sudoku!}\hypertarget{sudokuandcp:sudoku!}{}

\insertGraphique{5cm}{media/sudokuillustration.jpg}{A sudoku grid}

Everybody knows those grids that appeared last year in the subway, in wating lounges, on colleague's desks, etc. In Japanese \emph{su} means digit and \emph{doku}, unique. But this game has been discovered by an American ! The first grids appeared in the USA in 1979 (they were hand crafted). \href{http://en.wikipedia.org/wiki/sudoku}{Wikipedia} tells us that they were designed by Howard Garns a retired architect. He died in 1989 well before the success story of sudoku initiated by Wayne Gould, a retired judge from Hong-Kong. The rules are really simple: a 81 cells square grid is divided in 9 smaller blocks of 9 cells (3 x 3). Some of the 81 are filled with one digit. The aim of the puzzle is to fill in the other cells, using digits except 0, such as each digit appears once and only once in each row, each column and each smaller block. The solution is unique.

\subsection{Solving sudokus}\label{sudokuandcp:solvingsudokus}\hypertarget{sudokuandcp:solvingsudokus}{}

Many computer techniques exist to quickly solve a sudoku puzzle. Mainly, they are based on backtracking algorithms. The idea is the following: give a free cell a value and continue as long as choices remain consistent. As soon as an inconsistency is detected, the computer program backtracks to its earliest past choice et tries another value. If no more value is available, the program keeps backtracking until it can go forward again. This systematic technique make it sure to solve a sudoku grid. However, no human player plays this way: this needs too much memory ! 

\begin{note}
see \href{http://en.wikipedia.org/wiki/sudoku}{Wikipedia} for a panel of solving techniques.
\end{note}

\section{Sudoku and Artificial Intelligence}\label{sudokuandcp:sudokuandartificialintelligence}\hypertarget{sudokuandcp:sudokuandartificialintelligence}{}

Many techniques and rules have been designed and discovered to solve sudoku grids. Some are really simple, some need to use some useful tools: pencil and eraser. 

\subsection{Simple rules: single candidate and single position}\label{sudokuandcp:simplerules:singlecandidateandsingleposition}\hypertarget{sudokuandcp:simplerules:singlecandidateandsingleposition}{}

\insertGraphique{5cm}{media/sudokuillustrationscsp.jpg}{Simple rules: single candidates and single position} 

Let consider the grid on Figure~\ref{fig:media/sudokuillustrationscsp.jpg} and the cell with the red dot. In the same line, we find: 3, 4, 6, 7, and 9. In the same column: 2, 3, 5, and 8. In the same block: 2, 7, 8, and 9. There remain only one possibility: \textbf{1}. This is the \textbf{single candidate} rule. This cell should be filled in with \textbf{1}. 

Now let consider a given digit: let's say 4. In the block with a blue dot, there is no 4. Where can it be ? The 4's in the surrounding blocks heavily constrain the problem. There is a \textbf{single position} possible: the blue dot. This another simple rule to apply.

Alternatively using these two rules allows a player to fill in many cells and even solve the simplest grids. 
But, limits are easily reached. More subtle approaches are needed: but an important tool is now needed ... an eraser ! 

\subsection{Human reasoning principles}\label{sudokuandcp:humanreasoningprinciples}\hypertarget{sudokuandcp:humanreasoningprinciples}{}

\insertGraphique{7cm}{media/sudokuillustrationmarks.jpg}{Introducing marks}

Many techniques do exist but a vaste amount of them rely on simple principles. The first one is: do not try to find the value of a cell but instead focus on values that \textbf{will never be assigned} to it. The space of possibility is then reduced. This is where the eraser comes handy. Many players marks the remaining possibilities as in the grid on the left. 

Using this information, rather subtle reasoning is possible. For example, consider the seventh column on the grid on the left. Two cells contain as possible values the two values 5 and 7. This means that those two values cannot appear elsewhere in that very same column. Therefore, the other unassigned cell on the column can only contain a 6. We have \emph{deduced} something.

\noindent\begin{minipage}[b]{0.8\linewidth}
This was an easy to spot inference. This is not always the case. Consider the part of the grid on the right. Let us consider the third column. For cells 4 and 5, only two values are available: 4 and 8. Those values cannot be assigned to any other cell in that column. Therefore, in cell 6 we have a 3, and thus and 7 in cell 2 and finally a 1 in cell 3. This can be a very powerful rule.

Such a reasoning (sometimes called \emph{Naked Pairs}) is easily generalized to any number of cells (always in the same region: row, column or block) presenting this same configuration. This local reasoning can be applied to any region of the grid. It is important to notice that the inferred information can (and should) be used from a region to another.   
\end{minipage}%
\begin{minipage}[m]{0.2\linewidth}
~~\Graph{media/sudokuillustrationpart.jpg}{width=3cm}
\end{minipage}

\noindent The following principles of \emph{human} reasoning can be listed: 
\begin{itemize}
	\item reasoning on \emph{possible} values for a cell (by erasing impossible ones)
	\item systematically applying an evolved local reasoning (such as the \emph{Naked Pairs} rule)
	\item transmitting inferred information from a region to another related through a given a set of cells
\end{itemize}

\subsection{Towards Constraint Programming}\label{sudokuandcp:towardsconstraintprogramming}\hypertarget{sudokuandcp:towardsconstraintprogramming}{}

Those three principles are at the core of \textbf{constraint programming} a recent technique coming from both \emph{artificial intelligence} and \emph{operations research}.

\begin{itemize}
	\item The first principle is called \textbf{domain reduction} or \emph{filtering}
	\item The second considers its region as a \textbf{constraint} (a relation to be verified by the solution of the problem): here we consider an \emph{all different} constraint (all the values must be different in a given region). Constraints are considered \textbf{locally} for reasoning
	\item The third principle is called \textbf{propagation}: constraints (regions) communicate with one another through the available values in variables (cells)
\end{itemize}

Constraint programming is able to solve this problem as a human would do. Moreover, a large majority of the rules and techniques described on the Internet amount to a well-known problem: the alldifferent problem. A \textbf{constraint solver} (as \textbf{Choco}) is therefore able to reason on this problem allowing the solving of sudoku grid as a human would do although it has not be specifically designed to.

Ideally, iterating local reasoning will lead to a solution. However, for exceptionnaly hard grids, an enumerating phase (all constraint solvers provide tools for that) relying on backtracking may be necessary.

\section{See also}\label{sudokuandcp:seealso}\hypertarget{sudokuandcp:seealso}{}

\begin{itemize}
	\item \href{http://njussien.e-constraints.net/sudoku/eng-jouer.html}{SudokuHelper} a sudoku solver and helper applet developed with \emph{Choco}.
	\item \href{http://www.palmsudoku.com}{PalmSudoku} a rather complete list of rules and tips for solving sudokus
\end{itemize}



\backmatter

\printglossaries
\addcontentsline{toc}{part}{Glossary}

\addcontentsline{toc}{part}{Bibliography}
\bibliographystyle{apalike}
\bibliography{choco-doc}

\input{../shared/fdl-1.3}

\end{document}

%%% Local Variables: 
%%% mode: latex
%%% TeX-master: t
%%% End: 
