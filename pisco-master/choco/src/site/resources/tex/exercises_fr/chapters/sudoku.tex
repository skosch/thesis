%\part{exercises}
\label{sudoku}
\hypertarget{sudoku}{}

\section{Sudoku}\label{debutant:sudoku}\hypertarget{debutant:sudoku}{}

\url{http://fr.wikipedia.org/wiki/Sudoku}
\begin{quotation}
Le sudoku est un jeu en forme de grille, inspir� du carr� latin. Le but du jeu est de remplir la grille avec une s�rie de chiffres (ou de lettres ou de symboles) tous diff�rents, qui ne se trouvent jamais plus d?une fois sur une m�me ligne, dans une m�me colonne ou dans une m�me sous-grille. La plupart du temps, les symboles sont des chiffres allant de 1 � 9, les sous-grilles �tant alors des carr�s de 3 x 3. Quelques symboles sont d�j� dispos�s dans la grille, ce qui autorise une r�solution progressive du probl�me complet.
\end{quotation}

\begin{center} 
\begin{sudoku-block}
| | |7|5| | |3| | |. | |4| | |2| |1| | |. |1| | | |7| | |5| |. | | |3|1|4| |2| |6|. |4| | | |6|2|7| | |. | |6|5| |3| | | |8|. | |7|1| | | |6| | |. |8| | | | | | | | |. | |5| |7| | | |4|1|. 
\end{sudoku-block}

\end{center}
 \begin{itemize}
\item Comment mod�liser ce probl�me?
\item Une propagation suffit-elle?
\item Mod�liser ce probl�me avec des contraintes $allDifferent$. Quelles sont les diff�rences notables?
\end{itemize}

\lstinputlisting{java/sudoku1.j2t}

\begin{quote}
\`A retenir
\begin{itemize}
\item Variables temporaires
\item Diff�rence entre contraintes locales / contraintes globales
\item Statistiques : temps d'ex�cution, nombre de noeuds, nombre de backtracks, ...
\item propagation -- r�solution
\end{itemize}
\end{quote}