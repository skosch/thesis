%!TEX root = ../content-tut.tex
%\part{exercises}
\label{exercises}
\hypertarget{exercises}{}

\chapter{Exercises}\label{exercises:exercises}\hypertarget{exercises:exercises}{}

\section{I'm new to CP}\label{exercises:i'mnewtocp}\hypertarget{exercises:i'mnewtocp}{}

\begin{note}
\textbf{The goal of this practical work is twofold :}
\begin{itemize}
	\item \textbf{problem modelling} with the help of variables and constraints ;
	\item \textbf{mastering the syntax} of Choco in order to tackle basic problems.
\end{itemize}

\textbf{Warning}, constraint modelling should not be solver specific. That is why you are strongly advised to write down your model before starting its implementation within Choco. %The constraint factories in Choco are static methods of the Choco API.

\end{note}

\subsection{Exercise 1.1 (A soft start)}\label{exercises:exercise1.1}\hypertarget{exercises:exercise1.1}{}

Algorithm 1 (below) describes a problem which fits the minimum choco syntax requirements.
\begin{description}
	\item[Question 1] describe the constraint network modelled in Algorithm 1.
	\item[Question 2] give the variable domains after constraint propagation.
\end{description}
\begin{center}
\textbf{Algorithm 1:} Mysterious model.
\lstinputlisting{java/tmysterious.j2t}
\end{center}
(\hyperlink{solutions:solutionofexercise1.1}{Solution})

\subsection{Exercise 1.2 (DONALD +  GERALD = ROBERT)}\label{exercises:exercise1.2}\hypertarget{exercises:exercise1.2}{}
Associate a different digit to every letter so that the equation DONALD + GERALD = ROBERT is verified.

(\hyperlink{solutions:solutionofexercise1.2}{Solution})

\subsection{Exercise 1.3 (A famous example. . . a sudoku grid)}\label{exercises:exercise1.3}\hypertarget{exercises:exercise1.3}{}
A sudoku grid is a square composed of nine squares called \emph{blocks}. Each block is itself composed of 3x3 cells
(see figure 1). The purpose of the game is to fill the grid so that each block, column and row contains all the
numbers from 1 to 9 once and only once

\begin{description}
	\item[Question 1] propose a way to model the sudoku problem with difference constraints. Implement your model with Choco.
	\item[Question 2] which global constraint can be used to model such a problem ? Modify your code accordingly.
	\item[Question 3] Test, for both models, the initial propagation step (use Choco \texttt{propagate()} method). What can be noticed ? What is the point in using global constraints ?
\end{description}

\insertGraphique{.5\linewidth}{media/sudoku-grid.jpeg}{An exemple of a Sudoku grid}

(\hyperlink{solutions:solutionofexercise1.3}{Solution})

\subsection{Exercise 1.4 (The knapsack problem)}\label{exercises:exercise1.4}\hypertarget{exercises:exercise1.4}{}
Let us organise a trek. Each hiker carries a knapsack of capacity 34 and can store 3 kinds of food which respectively supply energetic values (6,4,2) for a consumed capacity of (7,5,3). The problem is to find which food is to be put in the knapsack so that the energetic value is maximal.

\begin{description}
	\item[Question 1] In the first place, we will not consider the idea of maximizing the energetic value. Try to find a satisfying solution by modelling and implementing the problem within choco.
	\item[Question 2] Find and use the choco method to \textbf{maximise} the energetic value of the knapsack.
	\item[Question 3] Propose a Value selector heuristic to improve the efficiency of the model.
\end{description}

(\hyperlink{solutions:solutionofexercise1.4}{Solution})

\subsection{Exercise 1.5 (The n-queens problem)}\label{exercises:exercise1.5}\hypertarget{exercises:exercise1.5}{}
The $n$-queens problem aims to place $n$ queens on a chessboard of size $n$ so that no queen can attack one another.
\begin{description}
\item[Question 1] propose and implement a model based on one $L_{i}$ variable for every row. The value of $L_{i}$ indicates the column where a queen is to be put. Use simple difference constraints and confirm that 92 solutions are obtained for $n= 8$.
\item[Question 2] Add a redundant model by considering variables on the columns ($C_{i}$). Continue to use simple difference constraints.
\item[Question 3] Compare the number of nodes created to find the solutions with both models. How can you explain such a difference ?
\item[Question 4] Add to the previously implemented model the following heuristics:
  \begin{itemize}
  \item Select first the line variable $L_I$ which has the smallest domain ;
  \item Select the value $j\in L_i$ so that the associated column variable $C_j$ has the smallest domain.
  \end{itemize}
  Again, compare both approaches in term of number of nodes and solving time to find ONE solution for $n = 75, 90, 95, 105$.
\item[Question 5] what changes are caused by the use of the global constraint \texttt{alldifferent} ?
\end{description}

\insertGraphique{.3\linewidth}{media/nqueen.png}{A solution of the n-queens problem for $n = 8$}

(\hyperlink{solutions:solutionofexercise1.5}{Solution})

\section{I know CP}\label{exercises:iknowcp}\hypertarget{exercises:iknowcp}{}

\subsection{Exercise 2.1 (Bin packing, cumulative and search strategies)}\label{exercises:exercise2.1}\hypertarget{exercises:exercise2.1}{}

Can $n$ objects of a given size fit in $m$ bins of capacity $C$ ? The problem is here stated has a satisfaction problem for the sake of simplicity. Your model and heuristics will be checked by generating random instances for given $n$ and $C$. The random generation must be reproducible.
\begin{description}
	\item[Question 1] Propose a boolean model (0/1 variables).
	\item[Question 2] Let us turn this satisfaction problem into an optimization one. Use your previously stated model but increase regularly the number of containers until a feasible solution is found.
	\item[Question 3] Implement a naive lower bound. This can be done by considering the occupied size globally.
	\item[Question 4] Propose a model with integer variables based on the cumulative constraint (see choco user guide/API for details). Define an objective function to minimize the number of used bins.
	\item[Bonus question] Compare different search strategies (variables/values selector) on this model for $n$ between 10 and 15.
\end{description}

Take a look at the following exercise in the old version of Choco and try to transpose it on new version of Choco.

\begin{lstlisting}
	int[] instance = getRandomPackingPb(n, capaBin, seed); 
	QuickSort sort = new QuickSort(instance); //Sort objects in increasing order
	sort.sort(); 
	Problem pb = new Problem(); 
	IntDomainVar[] debut = new IntDomainVar[n]; 
	IntDomainVar[] duree = new IntDomainVar[n]; 
	IntDomainVar[] fin = new IntDomainVar[n]; 
	 
	int nbBinMin = computeLB(instance, capaBin); 
	for (int i = 0; i < n; i++) { 
	    debut[i] = pb.makeEnumIntVar("debut " + i, 0, n); 
	    duree[i] = pb.makeEnumIntVar("duree " + i, 1, 1); 
	    fin[i] = pb.makeEnumIntVar("fin " + i, 0, n); 
	} 
	IntDomainVar obj = pb.makeEnumIntVar("nbBin ", nbBinMin, n); 
	pb.post(pb.cumulative(debut, fin, duree, instance, capaBin)); 
	for (int i = 0; i < n; i++) { 
	    pb.post(pb.geq(obj, debut[i])); 
	} 
	 
	IntDomainVar[] branchvars = new IntDomainVar[n + 1]; 
	System.arraycopy(debut, 0, branchvars, 0, n); 
	branchvars[n] = obj; 
	 
	//long tps = System.currentTimeMillis(); 
	pb.getSolver().setVarSelector(new StaticVarOrder(branchvars)); 
	Solver.setVerbosity(Solver.SOLUTION); 
	pb.minimize(obj, false); 
	Solver.flushLogs(); 
	// print solution 
	System.out.println("------------------------ " + (obj.getVal() + 1) + " bins"); 
	if (pb.isFeasible() == Boolean.TRUE) { 
	    for (int j = 0; j <= obj.getVal(); j++) { 
	    System.out.print("Bin " + j + ": "); 
	    int load = 0; 
	    for (int i = 0; i < n; i++) { 
	        if (debut[i].isInstantiatedTo(j)) { 
	        System.out.print(i + " "); 
	        load += instance[i]; 
	        } 
	    } 
	    System.out.println(" - load " + load); 
	    } 
	    //System.out.println("tps " + tps + " node " 
        //     + ((NodeLimit) pb.getSolver().getSearchSolver().limits.get(1)).getNbTot()); 
	}
\end{lstlisting}

(\hyperlink{solutions:solutionofexercise2.1}{Solution})

\subsection{Exercise 2.2 (Social golfer)}\label{exercises:exercise2.2}\hypertarget{exercises:exercise2.2}{}

A group of golfers play once a week and are splitted into $k$ groups of size $s$ (there are therefore $ks$ golfers in the club). The objective is to build a game scheduling on $w$ weeks so that no golfer play in the same group than another one more than once (hence the name of the problem: \emph{social golfers}). However, it may happen that two golfers will never play together. The point is only that once they have played together, they cannot play together anymore. 

\begin{note}
You can test your model with the parameters $(w, s, g)$ set to: $\{(11, 6, 2), (13, 7, 2), (9, 8, 8), (9, 8, 4), (4, 7, 3), (3, 6, 4)\}.$
\end{note}

\begin{description}
	\item[Question 1] Propose a boolean model for this problem. Use an heuristic that consists in scheduling a golfer on every week before scheduling a new one. More precisely, a golfer can be put in the first available group of each week before considering the next golfer.
	\item[Question 2] Identify some symmetries of the problem by using every similar elements of the problem. Try to improve your model by breaking those symmetries.
\end{description}

\begin{table}[htbp]
\centering
 	\begin{tabular}{ c c c c c}
		  &  group 1 &  group 2 &  group 3 &  group 4 \\
          \cline{2-5}
		 week 1 &  1 2 3 &  4 5 6 &  7 8 9 &  10 11 12 \\
		 week 1 &  1 4 7 &  10 2 5 &  8 11 3 &  6 9 12 \\
		 week 1 &  1 5 9 &  10 2 6 &  7 11 3 &  4 8 12 \\
	\end{tabular}
\caption{A valid configuration with 4 groups of 3 golfers on 3 weeks.}
\end{table}

\begin{lstlisting}
  Problem pb = new Problem();
  int numplayers = g * s;

  // golfmat[i][j][k] : is golfer k playing week j in group i ?
  IntDomainVar[][][] golfmat = new IntDomainVar[g][w][numplayers];
  for (int i = 0; i < g; i++) {
      for (int j = 0; j < w; j++)
      	  for (int k = 0; k < numplayers; k++)
          	  golfmat[i][j][k] = pb.makeEnumIntVar("("+i+"_"+j+"_"+k+")", 0, 1);
  }

  //every week, every golfer plays in one group
  for (int i = 0; i < w; i++) {
      for (int j = 0; j < numplayers; j++) {
          IntDomainVar[] vars = new IntDomainVar[g];
          for (int k = 0; k < g; k++) {
              vars[k] = golfmat[k][i][j];
          }
          pb.post(pb.eq(pb.scalar(vars, getOneMatrix(g)), 1));
      }
  }
	
  //every group is of size s
  for (int i = 0; i < w; i++) {
      for (int j = 0; j < g; j++) {
          IntDomainVar[] vars = new IntDomainVar[numplayers];
          System.arraycopy(golfmat[j][i], 0, vars, 0, numplayers);
          pb.post(pb.eq(pb.scalar(vars, getOneMatrix(numplayers)), s));
      }
  }
	
  //every pair of players only meets once
  // Efficient way : use of a ScalarAtMost
  for (int i = 0; i < numplayers; i++) {
      for (int j = i + 1; j < numplayers; j++) {
          IntDomainVar[] vars = new IntDomainVar[w * g * 2];
          int cpt = 0;
          for (int k = 0; k < w; k++) {
              for (int l = 0; l < g; l++) {
                  vars[cpt] = golfmat[l][k][i];
                  vars[cpt + w * g] = golfmat[l][k][j];
                  cpt++;
              } 
          }
          pb.post(new ScalarAtMostv1(vars, w * g, 1));
      }
  }
	
  //break symetries among weeks
  //enforce a lexicographic ordering between every pairs of week
  for (int i = 0; i < w; i++) {
      for (int j = i + 1; j < w; j++) {
          IntDomainVar[] vars1 = new IntDomainVar[numplayers * g];
          IntDomainVar[] vars2 = new IntDomainVar[numplayers * g];
          int cpt = 0;
          for (int k = 0; k < numplayers; k++) {
              for (int l = 0; l < g; l++) {
                  vars1[cpt] = golfmat[l][i][k];
                  vars2[cpt] = golfmat[l][j][k];
                  cpt++;
              }
          }
          pb.post(pb.lex(vars1, vars2));
      }
  }
	
  //break symetries among groups
  for (int i = 0; i < numplayers; i++) {
      for (int j = i + 1; j < numplayers; j++) {
          IntDomainVar[] vars1 = new IntDomainVar[w * g];
          IntDomainVar[] vars2 = new IntDomainVar[w * g];
          int cpt = 0;
          for (int k = 0; k < w; k++) {
              for (int l = 0; l < g; l++) {
                  vars1[cpt] = golfmat[l][k][i];
                  vars2[cpt] = golfmat[l][k][j];
                  cpt++;
              }
          }
          pb.post(pb.lex(vars1, vars2));
      }
  }
	
  //break symetries among players
  for (int i = 0; i < w; i++) {
      for (int j = 0; j < g; j++) {
          for (int p = j + 1; p < g; p++) {
              IntDomainVar[] vars1 = new IntDomainVar[numplayers];
              IntDomainVar[] vars2 = new IntDomainVar[numplayers];
              int cpt = 0;
              for (int k = 0; k < numplayers; k++) {
                  vars1[cpt] = golfmat[j][i][k];
                  vars2[cpt] = golfmat[p][i][k];
                  cpt++;
              }
              pb.post(pb.lex(vars1, vars2));
          }
      }
  }
	
  //gather branching variables
  IntDomainVar[] staticvars = new IntDomainVar[g * w * numplayers];
  int cpt = 0;
  for (int i = 0; i < numplayers; i++) {
      for (int j = 0; j < w; j++) {
          for (int k = 0; k < g; k++) {
              staticvars[cpt] = golfmat[k][j][i];
              cpt++;
          }
      }
  }
  pb.getSolver().setVarSelector(new StaticVarOrder(staticvars));
	
  pb.getSolver().setTimeLimit(120000);
  Solver.setVerbosity(Solver.SOLUTION);
  pb.solve();
  Solver.flushLogs();
\end{lstlisting}

(\hyperlink{solutions:solutionofexercise2.2}{Solution})

\subsection{Exercise 2.3 (Golomb rule)}\label{exercises:exercise2.3}\hypertarget{exercises:exercise2.3}{}

\emph{under development}

(\hyperlink{solutions:solutionofexercise2.3}{Solution})

\section{I know CP and Choco}\label{exercises:iknowcpandchoco}\hypertarget{exercises:iknowcpandchoco}{}

\subsection{Exercise 3.1 (Hamiltonian Cycle Problem Traveling Salesman Problem)}\label{exercises:exercise3.1}\hypertarget{exercises:exercise3.1}{}

Given a graph $G = (V,E)$, an \emph{Hamiltonian cycle} is a cycle that goes through every nodes of G once and only once. This exercise first intorduces a naive model to solve the Hamiltonian Cycle Problem. A second part tackles with the well known Traveling Salesman Problem.

Let $V = \{2,...,n\}$ be a set of cities index to cover, and let $d$ be a single warehouse duplicated into two indices $1$ and $n+1$. Notice the duplication distinguishes the source from the sink while there is only one warehouse. Finally, let us denote by $V_{d} = V \cup \{1,n+1\}$ the set of nodes to cover by a tour. Thus, the two following problems are defined:
\begin{itemize}
	\item find an Hamiltonian path covering all the cities of $V$
	\item find an Hamiltonian cycle of minimum cost that covers all the cities of $V$.
\end{itemize}

\subsubsection{Question 1 [Hamiltonian Cycle Problem]:} We first consider the satisfaction problem. Formally, a directed graph $G = (V,E)$ represents the topology of the cities and the unfolded warehouse. There is an arc $(i,j)\in E$ iff there exists a directed road from $i\in V$ to $j\in V$. Furthermore, every arc $(1,i)$ and $(i,n+1)$, with $i\in V$, belongs to E. Such a problem has to respect the following constraints:
\begin{itemize}
	\item each node of $V_{d}$ is reached exactly once,
	\item there is no subcycle containing nodes of $V$. In other words, the single cycle involved in $G$ is hamiltonian and contains arc $(n+1,1)$.
\end{itemize}
\begin{description}
\item[Question 1.a] The first constraint can directly be modelled using those proposed by Choco. On the other way, the second one requires to implement a constraint. This can be done through the following steps (see the provided skeleton):
\begin{itemize}
	\item strictly specify your constraint signature,
	\item formalise the underlying subproblemand information that need to be maintained,
	\item ignore in a first time the Choco event based mechanism and implement your filtering algorithm directly within the \texttt{propagate()} method,
	\item once your algorithm has been checked, try to reformulate your constraint through an event based implementation with the following methods: \texttt{awakeOnInst()}, \texttt{awakeOnSup()}, \texttt{awakeOnInf()}, \texttt{awakeOnBounds()}, \texttt{awakeOnRem()}, \texttt{awakeOnRemovals()}.
\end{itemize}
\item[Question 1.b] Now, propose a search heuristic (both on variables and values) that incrementaly builds the searched path from the source node. For this purpose, you have to respectively implement java classes that inherit from \texttt{IntVarSelector} and \texttt{ValSelector}.
\end{description}

\subsubsection{Question 2 [Traveling Salesman Problem]:} We now consider the optimisation view of the Hamiltonian Cycle Problem. A quantitative information is now associated with each arc of $G$ given by a cost function $f: E \leftarrow\Z_+$. Then, the graph $G$ is now defined by the triplet $(V_d,E,f)$ and we have to find an Hamiltonian path of minimum cost in $G$. 

For this purpose, we provide a skeleton of a Choco global constraint that dynamically maintains a lower bound evaluation of the searched path cost. Here, an evaluation of a minimum spanning tree of $G$ is proposed. \emph{Be careful}: take into account the partial assignment of the variables associated with the cities. 

\begin{description}
	\item[Question 2.a] find an upper bound on the cost of the Hamiltonian path,
	\item[Question 2.b] back-propagate lower/upper bounds informations on the required/infeasible arcs of $G$.
\end{description}

(\hyperlink{solutions:solutionofexercise3.1}{Solution})

\subsection{Exercise 3.2 (Shop scheduling)}\label{exercises:exercise3.2}\hypertarget{exercises:exercise3.2}{}

Given a set of $n$ tasks $T$ and $m$ disjunctive resources $R$, the problem is to find a plan to assign tasks to resources so that for every instant $t$, each resource $r\in R$ executes at most one task. Each task $T_i\in T$ is defined by:
\begin{itemize}
	\item a starting date $s_i = [s^-_i, s^+_i]\in\Z_+$,
	\item an ending date  $e_i = [e^-_i, e^+_i]\in\Z_+$,
	\item a duration  $d_i = e_i-s_i$, %$[d^-_i, d^+_i]\in\Z_+$,
	\item a resource $r_i = \{res_{1},\ldots, res_{m}\}\subseteq R$,
	\item a set of tasks, $preds_i\subseteq T$ that need to be processed before the start of $T_i$.
\end{itemize}

Let us consider the following satisfaction problem : Can one find a schedule of tasks $T$ on the resources $R$ that
\begin{itemize}
\item satisfies all the precedence constraints:
  $$ e_j\le s_i,\qquad (\forall T_i\in T, \forall T_j\in preds_i)$$
\item the last processed task ends before a given date $D$ ?:
  $$ e_i\le D,\qquad (\forall T_i\in T)$$
\end{itemize}

Then consider the optimization version : We now aim at finding the scheduling that satisfies all the constraints and that minimizes the date of the last task processed.
You will be given for this :
\begin{itemize}
\item a class structure where you have to describe your model (\texttt{AssignmentProblem}),
\item a class structure describing a Task (\texttt{Task}),
\item a class structure (\texttt{BinaryNonOverlapping}) which defines a Choco constraint. This constraint takes two tasks as parameters and has to verify whether at any time $t$ those tasks will be processed by the same resource or not.
\item a class structure (\texttt{MandatoryInterval}) which describes for a given task, the time window it has to be processed in.
\end{itemize}

\begin{description}
	\item[Question 1] How would you model the job scheduling problem ? Make use of the constraint \texttt{BinaryNonOverlapping}.
	\item[Question 2] Implement your model as if the constraint \texttt{BinaryNonOverlapping} was implemented.
	\item[Question 3] Sketch the mandatory processing interval of a task.
	\item[Question 4] Implement the constraint \texttt{BinaryNonOverlapping}:
	\begin{itemize}
		\item Implement the following reasoning : if two tasks have to be processed on the same resource and their mandatory intervals intersect, throw a failure.
		\item Now, implement the condition : if two tasks have a mandatory interval intersection, they must be scheduled on different resources.
		\item Finally, implement the following reasoning : If two tasks have to be processed by the same resource, then the starting and ending dates of every task ought to be updated functions to their mandatory intervals.
	\end{itemize}
	\item[Question 5] Implement an variable selection heuristic on the decision variable of the problem.
	\item[Question 6] Propose a model which minimize the end date of the last assigned task.
	\item[Question 7] Can you find a way to improve the BinaryNonOverlapping constraint.
	\item[Bonus Question] Find a lower bound on the end date of the last processed task.
\end{description}

\hyperlink{solutions:solutionofexercise3.2}{Solution}
