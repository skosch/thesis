%!TEX root = ../content-tut.tex
%\part{solutions}
\label{solutions}
\hypertarget{solutions}{}

\chapter{Solutions}\label{solutions:solutions}\hypertarget{solutions:solutions}{}

\section{I'm new to CP}\label{solutions:i'mnewtocp}\hypertarget{solutions:i'mnewtocp}{}

\subsection{Solution of Exercise 1.1 (A soft start)}\label{solutions:solutionofexercise1.1}\hypertarget{solutions:solutionofexercise1.1}{}
(\hyperlink{exercises:exercise1.1}{Problem})

\noindent\emph{\textbf{Question 1}: describe the constraint network related to code}

The model is defined as :
\begin{itemize}
	\item $V = \{x_1, x_2, x_3\}$: the set of variables,
	\item $D = \{[0,5], [0,5], [0,5]\}$: the set of domain
	\item $C = \{x_1>x_2, x_1\neq x_3, x_2>x_3\}$: the set of constraints.
\end{itemize}

\noindent\emph{\textbf{Question 2}: give the variable domains after constraint propagation.}

\begin{itemize}
	\item From $x_1 = [0,5]$ and $x_2 = [0,5]$ and $x_1>x_2$, we can deduce tha : the domain of $x_1$ can be reduce to $[1,5]$ and the domain of $x_2$ can be reduce to $[0,4]$.
	\item Then, from $x_2 = [0,4]$ and $x_3 = [0,5]$ and $x_2>x_3$, we can deduce that : the domain of $x_2$ can be reduce to $[1,4]$ and the domain of $x_3$ can be reduce to $[0,3]$.
	\item Then, from $x_1 = [1,5]$ and $x_2 = [1,4]$ and $x_1>x_2$, we can deduce that : the domain of $x_1$ can be reduce to $[2,5]$.
\end{itemize}

We cannot deduce anything else, so we have reached a \textbf{fix point}, and here is the domain of each variables:
$$x_{1} : [2,5],\quad x_{2} : [1,4],\quad x_{3} : [0,3].$$


\subsection{Solution of Exercise 1.2 (DONALD + GERALD = ROBERT)}\label{solutions:solutionofexercise1.2}\hypertarget{solutions:solutionofexercise1.2}{}

(\hyperlink{exercises:exercise1.2}{Problem})

\lstinputlisting{java/tdonald.j2t}

\subsection{Solution of Exercise 1.3 (A famous example. . . a sudoku grid)}\label{solutions:solutionofexercise1.3}\hypertarget{solutions:solutionofexercise1.3}{}

(\hyperlink{exercises:exercise1.3}{Problem})

\noindent\emph{\textbf{Question 1}: propose a way to model the sudoku problem with difference constraints. Implement your model with choco solver.}

\lstinputlisting{java/tsudokunaive.j2t}

\noindent\emph{\textbf{Question 2}: which global constraint can be used to model such a problem ? Modify your code to use this constraint.}

The \emph{allDifferent} constraint can be used to remplace every disequality constraint on the first Sudoku model. It improves the efficient of the model and make it more ``readable''.

\lstinputlisting{java/tsudokualldiff.j2t}
\noindent\emph{\textbf{Question 3}: Test for both model the initial propagation step (use choco} \texttt{propagate()} \emph{method). What can be noticed ? What is the point in using global constraints ?}

The sudoku problem can be solved just with the propagation. \todo{FIXME explanation.}
The global constraint provides a more efficient filter algorithm, due to more complex deduction.

\subsection{Solution of Exercise 1.4 (The knapsack problem)}\label{solutions:solutionofexercise1.4}\hypertarget{solutions:solutionofexercise1.4}{}

(\hyperlink{exercises:exercise1.4}{Problem})

\noindent\emph{\textbf{Question 1} : In the first place, we will not consider the idea of maximizing the energetic value. Try to find a satisfying solution by modelling and implementing the problem within choco.}

\lstinputlisting{java/tknapsack1.j2t}

\noindent\emph{\textbf{Question 2} : Find and use the choco method to maximize the energetic value of the knapsack.}
Replace the common \mylst{s.solve()} with:
\lstinputlisting{java/tknapsack2.j2t}
The first argument defines the objective variable to optimize, the second indicates if the search must restart from the root node after finding each solution.
Choco doesn't allow the optimization of integer expression. By adding an equality constraint between the expression and a integer variable, one can bypass this restriction.

\noindent\emph{\textbf{Question 3} : Propose a Value selector heuristic to improve the efficiency of the model.}

It can be improved using the following value selector strategy. It iterates over decreasing values of every domain variables: 
\lstinputlisting{java/tknapsack3.j2t}

\subsection{Solution of Exercise 1.5 (The n-queens problem)}\label{solutions:solutionofexercise1.5}\hypertarget{solutions:solutionofexercise1.5}{}
(\hyperlink{exercises:exercise1.5}{Problem})

\noindent\emph{\textbf{Question 1} : propose and implement a model based on one} $L_{i}$ \emph{variable for every row...}
\lstinputlisting{java/tqueensrow.j2t}

\noindent\emph{\textbf{Question 2} : Add a redundant model by considering variable on the columns ($C_i$). Continue to use simple difference constraints.}

\lstinputlisting{java/tqueensredund.j2t}

\noindent\emph{\textbf{Question 3} : Compare the number of nodes created to find the solutions with both models. How can you explain such a difference ?}

The channeling permit to reduce more nodes from the tree search... \todo{FIXME}

\noindent\emph{\textbf{Question 4} : Add to the previous implemented model the following heuristics,
\begin{itemize}
	\item Select first the line variable ($L_i$) which has the smallest domain ;
	\item Select the value $j\in L_i$ so that the associated column variable $C_j$ has the smallest domain.
\end{itemize}
Again, compare both approaches in term of nodes number and solving time to find ONE solution for $n = 75, 90, 95, 105$.}

Replace the following lines in your program :
\begin{lstlisting}
	s.addGoal(new AssignOrForbidIntVarVal(new MinDomain(s, s.getVar(queens)), new MinVal()));
\end{lstlisting}
with
\begin{lstlisting}
	s.addGoal(new AssignOrForbidIntVarVal(new MinDomain(s, s.getVar(queens)), new NQueenValueSelector(s.getVar(queensdual))));
\end{lstlisting}
The variable selector strategy (\texttt{MinDomain}) already exists in Choco. It iterates over variables given and returns the variable ordering by creasing domain size. 
The value selector strategy has to be created as follow:
\lstinputlisting{java/tqueensel.j2t}

\noindent\emph{\textbf{Question 5} : what changes are caused by the use of the global constraint \textbf{alldifferent} ?}

\lstinputlisting{java/tqueensalldiff.j2t}

\section{I know CP}\label{solutions:iknowcp}\hypertarget{solutions:iknowcp}{}

\subsection{Solution of Exercise 2.1 (Bin packing, cumulative and search strategies)}\label{solutions:solutionofexercise2.1}\hypertarget{solutions:solutionofexercise2.1}{}
(\hyperlink{exercises:exercise2.1}{Problem})

\subsection{Solution of Exercise 2.2 (Social golfer)}\label{solutions:solutionofexercise2.2}\hypertarget{solutions:solutionofexercise2.2}{}
(\hyperlink{exercises:exercise2.2}{Problem})

\subsection{Solution of Exercise 2.3 (Golomb rule)}\label{solutions:solutionofexercise2.3}\hypertarget{solutions:solutionofexercise2.3}{}

\emph{under development}

(\hyperlink{exercises:exercise2.3}{Problem})

\section{I know CP and Choco2.0}\label{solutions:iknowcpandchoco2.0}\hypertarget{solutions:iknowcpandchoco2.0}{}

\subsection{Solution of Exercise 3.1 (Hamiltonian Cycle Problem Traveling Salesman Problem)}\label{solutions:solutionofexercise3.1}\hypertarget{solutions:solutionofexercise3.1}{}

(\hyperlink{exercises:exercise3.1}{Problem})

\subsection{Solution of Exercise 3.2 (Shop scheduling)}\label{solutions:solutionofexercise3.2}\hypertarget{solutions:solutionofexercise3.2}{}

(\hyperlink{exercises:exercise3.2}{Problem})

\emph{under development}
