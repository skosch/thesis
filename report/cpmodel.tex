\section{CP model}
The constraint programming model is based on a set of decision variables $B_j =
\{k_1, \dots, k_{n_k} \}$, where each variable stands for the batch job $j$ is
assigned to. The complete model is given in Model \ref{model:cpmodel}.

\begin{model}

\begin{alignat}{2}
\mathrm{Min.}\quad & \Lmax && \\
\mathrm{s.t.}\quad & \label{cpcs:pack} \mathtt{pack}(J, K, b) && \\
& \label{cpcs:cumul} \mathtt{cumul}(J, b) && \\
& \label{cpcs:pk} P_k = \underset{j}{\max} \; p_j \quad && \forall \{j \in J
| B_j = k\}, \forall k \in K \\
& \label{cpcs:dk} D_k = \underset{j}{\min} \; d_j \quad && \forall \{j \in J
| B_j = k\}, \forall k \in K \\
& \label{cpcs:ck} C_k + P_{k+1} = C_{k+1} \quad && \forall k \in K \\
& \label{cpcs:lmax} \Lmax \geq \underset{k}{\max} \; (C_k - D_k) && \\
& \label{cpcs:empty} \mathtt{IfThen}(P_k = 0, P_{k+1} = 0) \quad && \forall k
\in \{ k_1, \dots, k_{n_k - 2}\} \\
& \label{cpcs:pp} B_j \leq k \quad && \forall \{ j \in J, k \in K | j > k \}
\end{alignat}
\caption{Constraint programming model}
\label{model:cpmodel}
\end{model}

Constraint \eqref{cpcs:pack} makes sure the jobs are distributed into the
batches such that no batch exceeds the capacity $b$. Constraint
\eqref{cpcs:cumul}, a global cumulative constraint, keeps jobs from overlapping
on the resource -- while redundant with \eqref{cpcs:pack}, this speeds up
propagation slightly. Constraints \eqref{cpcs:pk}, \eqref{cpcs:dk},
\eqref{cpcs:ck} and \eqref{cpcs:lmax} define $P_k$, $D_k$, $C_k$ (batch
completion time) and $\Lmax$, respectively.



\subsection{Temporal constraints on a job's start date} Given any partial
assignment of jobs and an open job $j$, we can reason that \begin{alist}
\item{if the first batch with a due date later than the job is $k$, then the job
cannot be part of a batch after $k$ -- this would result in a non-EDD sequence
of batches.} \item{if the first batches up to $k-1$ offer not enough capacity
for $j$ due to the given partial assignment, then the job cannot be part of a
batch before $k$.} \end{alist} Since batches are \textit{not} dynamically
created like in Malapert's solution but fixed from the start, any partial
assignment that fails due to these constraints cannot be part of an optimal
solution.

This constraint is redundant with both the $(C_{k+1}\geq C_k)$ and
\texttt{packing} constraints, but may help accelerate the propagation in some
cases.

\subsection{Grouping empty batches} Just like in the MIP model, we can force
empty batches to the back and thus establish dominance of certain solutions. The
implementation is much easier than in the MIP model: \begin{alignat}{2} &
\mathtt{IfThen}( C_{k+2} > C_{k}, C_{k+1} > C_{k} ) \quad && \forall k \in
\{k_1, \dots, k_{n_k-2}\} \end{alignat}

\subsection{No postponing of jobs to later batches} Just like in the MIP model,
jobs should never go into a batch with an index greater than their own:
\begin{alignat}{2} & x_{jk} = 0 \quad && \forall j,k : j > k \end{alignat} This
is implemented as $\mathtt{assignments}_j \leq k$. 

{\color{darkred} This constraint is analogous to the concept of finding an upper
bound on $n_k$ -- essentially, it would be very helpful to find an upper bound
on the latest batch for \textit{every} job.}


