\section{Decomposition approach}

Instead of solving the entire problem using one model, we can solve the problem
step by step, using the best techiques available for each subproblem. This
approach is inspired by a method called \textit{Benders decomposition}.

\subsection{Branch-and-bound by batch in chronological
order}\label{sec:mipdecomp}
A basic version of this approach uses branch-and-bound to transverse the search
tree. At each node on level $\ell$, a single MIP and/or CP model is run to
assign jobs to batch $k = \ell$. The remaining jobs are passed to the children
nodes, which assign jobs to the next batch, and so on -- until a solution, and
thus a new upper bound on $\Lmax$, is found. Several constraints are used to
prune parts of the search tree that are known to offer only solutions worse than
this upper bound. Figure \ref{fig:decomp_diagram1} shows an example in which a
MIP model is used to assign jobs to the batch at every node.
\begin{algorithm}[h]
\fontsize{9pt}{11.5pt}\selectfont
\begin{algorithmic}
\State update \textit{currentAssignments} \Comment{this keeps track of where we are
in the tree}
\If{no jobs given to this node} \Comment{if this is a leaf node, i.e. all jobs
are assigned to batches}
  \State calculate $L_{\text{max,current}}$ based on \textit{currentAssignments}
  \If{$L_{\text{max,current}}$ < $L_{\text{max,incumbent}}$}
    \State $L_{\text{max,incumbent}} \gets L_{\text{max,current}}$
    \State \textit{bestAssignments} $\gets$ \textit{currentAssignments}
  \EndIf
  \State return to parent node
\EndIf
\State set up MIP model \Comment{as described below}
\Repeat
  \State $x_j \gets$ model.solve($x_j$) \Comment{let model assign jobs to the
  batch}
  \State spawn and run child node with all $\{j | x_j = 0\}$ \Comment{pass
  unassigned jobs to children}
  \State add constraint to keep this solution from recurring \Comment{this
  happens once the child node returns}
\Until{model has no more solutions}
\State return to parent node
\end{algorithmic}
\caption{MIP node class code overview}
\label{alg:bbnode_mip1}
\end{algorithm}
Algorithm \ref{alg:bbnode_mip1} outlines what happens at each node: the model
finds the best jobs to assign to the batch according to some rule, lets the
children handle the remaining jobs, and tries the next best solution once the
first child has explored its subtree and backtracked.
\subsubsection{Using MIP and cumulative packing after the batch}
\tikzset{
  treenode/.style = {align=center, inner sep=4pt, text centered,
    font=\sansfont\fontsize{11pt}{12pt}\selectfont},
  root/.style = {treenode},% arbre rouge noir, noeud noir
  nchild/.style = {treenode,
     },% arbre rouge noir, noeud rouge
  leaf/.style = {treenode}% arbre rouge noir, nil
}

\begin{figure}
\centering
\begin{tikzpicture}[->,>=stealth',level/.style={sibling distance = 3cm/#1,
  level distance = 1.5cm}, scale=0.7]
\node (Root) [root] {MIP}
 child{ node [nchild] {MIP} 
            child{ node [nchild] {MIP}}
            child{ node [nchild] {MIP}
              child{ node [leaf] {Solution 1}}
						}                            
    }
    child{ node [nchild] {MIP} }
    child{ node [nchild] {MIP} 
            child{ node [nchild] {MIP} %2, left
                    child{ node [leaf] {Solution 2}} 
                  }
            child{ node [nchild] {MIP} }
		};
  \begin{scope}[every node/.style={right}]
    \path (Root -| Root-3-2) ++(1.5cm,0) node {\fontsize{10pt}{10pt}\selectfont $k=1$};
    \path (Root-1 -| Root-3-2) ++(1.5cm,0) node
    {\fontsize{10pt}{10pt}\selectfont $k=2$};
    \path (Root-1-1 -| Root-3-2) ++(1.5cm,0) node
    {\fontsize{10pt}{10pt}\selectfont $k=3$};
  \end{scope}

   \end{tikzpicture}
\caption{Batch-by-batch decomposition using MIP only}\label{fig:decomp_diagram1}
\end{figure}

The first version of the branch-and-bound batch-by-batch decomposition uses a
MIP model at each node to assign jobs to the respective batch. The remaining
jobs are packed such as to minimize their $\Lmax$, with a relaxation of the
batching requirement, i.e., as if on a cumulative resource. 
\begin{model}[h!]
\begin{alignat}{3}
\text{Min.}\quad & L_{\text{max,cumul}} && \\ 
\text{s. t.}\quad & \label{dc:eq1} \sum_j s_j x_j \leq b \quad && \forall j \in J \\
& P_k \geq p_j x_j \quad && \forall j \in J \\
& \label{dc:eq3} \sum_j x_j \geq 1 \quad && \forall \{j \in J | d_j = \min(d_j)\} \\
& \label{dc:eq4} P_k + \frac{1}{b} \sum_{i} s_i p_i \leq d_j +
L_{\text{max,incmb}} - 1 - v_k \quad && \forall j \in J, \forall \{i \in J | d_i
\leq d_j\} \\[2ex]
& \label{dc:eq5} \sum_t u_{jt} = 1 \quad && \forall j \in J \\
& \label{dc:eq6} \sum_j \sum_{t' \in T_{jt}} s_j u_{jt'} \leq b \quad && \forall t \in \mathcal{H} \\
& \label{dc:eq7} (v_k + t + p_j) u_{jt} \leq d_j + L_{\text{max,incmb}} - 1 \quad && \forall j \in J, \forall t \in \mathcal{H} \\
& \label{dc:eq8} L_{\text{max,cumul}} \geq (v_k + t + p_j) u_{jt} - d_j \quad && \forall j \in J, \forall t \in \mathcal{H} \\
& \label{dc:eq9} u_{j,t=0} = x_j \quad && \forall j \in J \\
& \label{dc:eq10} u_{it} \leq (1 - x_j) \quad && \forall i,j \in J, \forall t
\in \{1, \dots, p_j - 1\} \\[2ex]
& \label{dc:eq11} b - \sum_{i \in J} s_i x_i \leq (b w_j + 1) s_j \quad && \forall j \in J \\
& \label{dc:eq12} P_k + 2w_j n_t \geq p_j + n_t x_j \quad && \forall j \in J\\
& \label{dc:eq13} P_k - 2(1 - w_j)n_t \leq p_j +n_t x_j - 1 \quad && \forall j
\in J
\end{alignat}
\caption{MIP model in batch-by-batch branch-and-bound}
\label{model:decomp_mip}
\end{model}

\begin{table}[h]
\begin{tabular}{l p{5in}}
$x_j$ & is 1 iff job $j$ is assigned to the batch \\
$u_{jt}$ & is 1 iff job $j$ starts in time slot $t$ \\
$T_{jt}$ & is the set of all time slots occupied by job $j$ if it ended at time
$t$, that is $T_{jt} = \{t - p_j + 1, \dots, t\}$ \\
$v_k$ & is the start time of the batch at the given node in the search tree \\
$L_{\text{max,incmb}}$ & is the incumbent (known best) value of and thus an
upper bound on $\Lmax$ \\
$\mathcal{H}$ & is the set of all indexed time points $\{0, \dots, n_t - 1\}$ \\
$w_j$ & is 1 iff job $j$ is either longer than the batch ($p_j > P_k$) or
already part of the batch ($x_j = 1$). Neither condition must be fulfilled for constraint
\eqref{dc:eq11} to have an effect on job $j$
\end{tabular}
\caption{Notation used in the decomposition model}
\end{table}

Model \ref{model:decomp_mip} implements a time-indexed cumulative constraint on the
non-batched jobs. Constraints \eqref{dc:eq1} through \eqref{dc:eq3} ensure that the
batch stays below capacity, define the duration of the batch $P_k$ and force at
least one of the earliest-due jobs into the batch.

Constraints \eqref{dc:eq4} express the interval relaxation used to ensure that no
jobs exceed their latest allowable finish date given any batch assignment. Even
jobs that are assigned to the batch have to fulfill this requirement.

Constraints \eqref{dc:eq5} and \eqref{dc:eq6} implement the cumulative nature of the
post-batch assignments by ensuring that each job starts only once, and no jobs
overlap on a given resource at any time.

Constraints \eqref{dc:eq7} again limit the possible end dates of a
job, but unlike \eqref{dc:eq4}, they use the time assignments on the cumulative
resource to determine end dates. Constraints \eqref{dc:eq8} define the value of
$L_{\text{max,cumul}}$, the maximum lateness of any job in the non-batched set.

Constraints \eqref{dc:eq9} force batched jobs to start at $t = 0$, while
\eqref{dc:eq10} force \textit{all} jobs to start either at $t = 0$ or after the
last batched job ends.

Constraints \eqref{dc:eq11} enforce a dominance rule: jobs must be assigned to
the batch such that the remaining capacity, $b_r = b - \sum_j s_j x_j$, is less
than the size $s_j$ of the \textit{smallest} job from the set of non-batched
jobs that are \textit{not longer} than the current batch. That is, if there
exists an non-batched job $j$ with $p_j \leq P_k$ and $s_j \leq b_r$, then the
current assignment of jobs is infeasible in the model. The reasoning goes as
follows: given any feasible schedule, assume there is a batch $k_a$ with $b_r$
remaining capacity and a later batch $k_b$ containing a job $j$ such that $s_j
\leq b_r$ and $p_j \leq P_{k_a}$. Then job $j$ can always be moved to batch
$k_a$ without negatively affecting the quality of the solution: if the schedule
was optimal, then moving $j$ will not affect $\Lmax$ at all (otherwise, it was
no optimal schedule); if the schedule was not optimal, then $\Lmax$ will stay
constant (unless $j$ was the longest job in $k_b$ and $\Lmax$ occured in or in a
batch after $k_b$, in which case $\Lmax$ will be improved).

This rule is implemented by means of a binary variable $w_j$, which, as defined
by constraints \eqref{dc:eq11} and \eqref{dc:eq12}, is 1 iff $p_j > P_k \lor x_j
= 1$. These are the cases in which a job $j$ is \textit{not} to be considered in
\eqref{dc:eq10}, and so $w_j$ is used to scale $s_j$ to a value insignificantly
large in the eyes of the constraint's less-than relation.

After a solution is found, a child node in the search tree has run the subtree
and returned, a constraint of the form 
\begin{alignat}{2}
& \sum_j x_j + \sum_i (1-x_i) \leq n_j - 1 \quad && \forall \{j \in J | x_j =
1\}, \forall \{i \in J | x_i = 0 \}
\end{alignat}\marginpar{\em\small I still can't think of anything tighter ...}
is added to the model before the solver is called again, to exclude the last
solution from the set of feasible solutions. 

\subsubsection{Using CP and cumulative packing after the
batch}\label{sec:cpdecomp}
This approach is equivalent, but we now use CP to select the batch assignments,
again based on a minimized $\Lmax$ among the non-batched jobs. Model
\ref{model:decomp_cp} uses interval variables $j$ to represent the jobs;
time constraints on the jobs (``est'', ``lft'' etc.) are represented as
equations involving the terms $\startOf(j)$ and $\endOf(j)$.

Contraints \eqref{dcp:eq1} through \eqref{dcp:eq4} bi-directionally define the
relationship between a job's $x_j$ and $\startOf(j)$: $x_j = 1$ is equivalent
with a start time of $t = 0$, and $x_j = 0$ is equivalent with a start time of
$t \geq P_k$. The term $\sum_{i \in J} p_i$ is used as a large constant as it is
greater than any job's start time.

Constraint \eqref{dcp:eq8} is equivalent to constraints \eqref{dc:eq11} through
\eqref{dc:eq13} in Model \ref{model:decomp_mip} above. The cumulative constraint
\eqref{dcp:eq9} ensures that jobs do not overlap. 

\begin{model}[h]
\begin{alignat}{2}
\mathrm{Min.} \quad & \Lmax \quad && \\
\mathrm{s.t.} \quad 
& \label{dcp:eq1} \startOf(j) \geq (1-x_j) P_k \quad && \forall j \in J \\
& \startOf(j) \leq (1-x_j) \sum_{i \in J} p_i \quad && \forall j \in J \\
& x_j \geq \frac{ \frac{1}{2} - \startOf(j) }{ \sum_{i \in J} p_i } \quad &&
\forall j \in J \\
& \label{dcp:eq4} x_j \leq 1+\frac{ \frac{1}{2} - \startOf(j)}{\sum_{i \in J} p_i } \quad &&
\forall j \in J \\
& P_k \geq \underset{j \in J}{\max}(x_j p_j) \quad && \\
& \endOf(j) = d_j + L_{\text{max,incmb}} - 1 \quad && \forall j \in J \\
& \Lmax \geq \underset{j \in J}{\max}(\endOf(j) - d_j) \quad &&  \\
& \label{dcp:eq8} \mathtt{IfThen}(p_j \leq P_k \land x_j = 0, b - \sum_{j \in J} s_j
x_j \leq s_j) \quad && \forall j \in J \\
& P_k + \frac{1}{b} \sum_i s_i p_i \leq d_j + L_{\text{max,incmb}} - 1 - v_k
\quad && \forall j \in J, \forall \{i \in J | d_i \leq d_j\} \\
& \sum_j x_j \geq 1 \quad && \forall \{ j \in J | d_j = \min(d_j) \} \\
& \label{dcp:eq9} \mathtt{cumul}(J, b) \quad & &  
\end{alignat}
\caption{CP model in batch-by-batch branch-and-bound}
\label{model:decomp_cp}
\end{model}



