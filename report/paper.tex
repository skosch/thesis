\documentclass[13pt, letterpaper, twoside]{book}
\usepackage{graphicx}
\usepackage[driver=xetex,paperwidth=8.5in,paperheight=11in,left=1.4in, right=1.74in,top=1.4in, bottom=1.74in]{geometry}
\usepackage[no-math]{fontspec}

\usepackage{sectsty}
\usepackage{multicol,scalefnt}
\usepackage{amsmath, amssymb, amsfonts,  titlesec}
\usepackage[urw-garamond]{mathdesign}
\usepackage{fancyhdr,  booktabs, multirow}
\usepackage[font=small,format=plain,labelfont=it,up,textfont=it,up]{caption}
\usepackage{listings}

\usepackage{enumitem}


%========== DEFINITIONS ==========

\newlist{alist}{itemize}{1}
\setlist[alist]{label=--,labelindent=2in,leftmargin=9pt,labelsep=6pt, itemsep=0pt}

\def\Lmax{L_{\text{max}}}
\def\est{\mathtt{est}}
\def\lst{\mathtt{lst}}
\def\eft{\mathtt{eft}}
\def\lft{\mathtt{lft}}
\def\startOf{\mathtt{startOf}}
\def\endOf{\mathtt{endOf}}

%=========== TIKZ STUFF =========

\usetikzlibrary{fit}
\makeatletter
\tikzset{
  fn/.style={
    inner sep=0pt,
    fill=none,
    draw=none,
    reset transform,
    fit={(\pgf@pathminx,\pgf@pathminy) (\pgf@pathmaxx,\pgf@pathmaxy)},
  },
  reset transform/.code={\pgftransformreset}
}
\makeatother

%========= ALGO STUFF ============

%========= FONT SPECS ============
\tolerance 8000

\defaultfontfeatures{Mapping=tex-text, Ligatures=Common}

%\renewcommand\refname{references} % this sets the name of
\def\labelitemi{--}

\def\sansfont{\fontspec[Script=Latin,LetterSpace=2.6, Mapping=tex-text]{DIN 1451 Mittelschrift}}
\def\sansitalicfont{\fontspec[Script=Latin,LetterSpace=2.6, FakeSlant=0.2, Mapping=tex-text]{DIN 1451 Mittelschrift}}

\def\monofont{\fontspec[Script=Latin,Mapping=tex-text,Scale=0.91, AutoFakeBold]{Inconsolata}}

\renewcommand{\texttt}[1]{{\monofont #1}}
\renewcommand{\mathtt}[1]{{\text{\monofont #1}}\,}

\renewcommand{\normalsize}{\fontsize{12pt}{16pt}\selectfont}

%========== COLOR STUFF ===============

\definecolor{darkred}{rgb}{0.6, 0, 0.00}


%============ LISTINGS ==============
\lstset{
aboveskip=2\medskipamount, belowskip=2\medskipamount,
basicstyle=\monofont,
language=python,
numbers=left, numberstyle=\tiny,  numbersep=9pt,
xleftmargin=.4in, frame=l, xrightmargin=1.74in
}


%============= PAGE LAYOUT ============

%\titleformat{\section}{\huge\sansnormalfont}{\protect\makebox[0pt][r]{\thesection\quad}}{0em}{}
\titleformat{\chapter}{\fontsize{32pt}{36pt}\selectfont\sansfont}{}{0em}{}
\titleformat{\section}{\fontsize{18pt}{22pt}\selectfont\sansfont}{\protect\makebox[0pt][r]{\thesection\quad}}{0em}{}
\titleformat{\subsection}{\fontsize{12pt}{16pt}\selectfont\sansfont}{\protect\makebox[0pt][r]{\thesubsection\fontsize{18pt}{22pt}\selectfont\quad}}{0em}{}
%\titleformat{\paragraph}{\fontsize{12pt}{16pt}\selectfont}{}{}{}


\fancyhead[LE]{\sansfont\small Scheduling non-identical jobs on a batch resource \normalsize}
\fancyhead[RE]{}
\fancyhead[LO]{}
\fancyhead[RO]{\sansfont\small \nouppercase\rightmark}
\fancyfoot[C]{\sansfont\thepage}

\fancypagestyle{plain}{
\fancyhf{}
\renewcommand{\headrulewidth}{0pt}
\fancyfoot[C]{\sansfont\thepage}
}

%====== CHAR REPLACEMENTS ======%
\let\oldemptyset\emptyset
\let\emptyset\varnothing

%===== LONG EQUATION THINGS =====%
\DeclareFloatingEnvironment[
  fileext=los,
  listname=List of Models,
  name=Model,
  placement=tbhp,
  within=section
]{model}




\begin{document}
\frontmatter

\fontsize{12pt}{16pt}\selectfont
\thispagestyle{empty}
\pagestyle{fancy}

\baselineskip=16.8pt plus 0pt
\frenchspacing

\begin{centering}
\vspace{3em}
\LARGE\sansfont{Scheduling non-identical jobs on a batch resource}

\vspace{2em}
\large
\sansfont Sebastian Kosch\\

\vfill
\normalfont
\fontsize{12pt}{16pt}\selectfont
A thesis submitted in conformity with the requirements

for the degree of \textit{Bachelor of Applied Science}

\vspace{1em}
Supervisor: Prof. J. Christopher Beck, MIE

\vspace{2em}

\textmd Division of Engineering Science\\
University of Toronto\\

April 2013

\end{centering}
\pagebreak

%\include{firstpage}

\tableofcontents

\mainmatter
\pagebreak
\vskip 4em
\chapter{Introduction}
This will include about two to three pages on the history of the type of problem I'm considering.

Lorem ipsum dolor sit amet, consectetur adipiscing elit. Sed arcu sapien, porta et rutrum at, tincidunt quis quam. Pellentesque quis sapien nec elit tempor semper. Proin consectetur convallis venenatis. Nam in libero nec mi ullamcorper pulvinar. Fusce id augue in tellus semper convallis eu ac massa. Nunc purus elit, luctus non imperdiet sed, cursus dictum ante. Quisque et erat id elit porta suscipit. Aenean volutpat, leo ac commodo consequat, nibh mauris pretium metus, eu ullamcorper nulla nibh at mauris. Donec congue, eros sed dapibus volutpat, nisi magna rhoncus ante, at posuere odio nunc id odio. Phasellus placerat leo et mauris luctus pulvinar sit amet id enim. Quisque scelerisque, neque a viverra facilisis, tortor purus pharetra neque, ac consequat magna tellus non velit. Pellentesque vestibulum libero nisl. Donec congue dictum aliquet. Curabitur dictum tempus vulputate.

Duis volutpat justo accumsan orci sodales pulvinar. Nunc eu tortor quam, ut fringilla odio. Ut eu massa nulla. Proin lorem dui, imperdiet ut gravida non, scelerisque in nibh. Nam tortor quam, faucibus quis congue sit amet, varius id nibh. Mauris ac lacinia neque. Nullam faucibus porttitor elit, eu dictum lacus auctor nec. Nulla eget quam nunc. Etiam vulputate mi nisi. Class aptent taciti sociosqu ad litora torquent per conubia nostra, per inceptos himenaeos.

Ut nec neque libero. Nulla facilisi. Cum sociis natoque penatibus et magnis dis parturient montes, nascetur ridiculus mus. Nunc eros nisl, fermentum sed condimentum quis, fringilla vel tellus. Maecenas cursus mauris eros, sed sollicitudin erat. Duis sit amet egestas eros. Proin condimentum aliquam posuere. Quisque pretium hendrerit urna ut fringilla. Pellentesque eget eleifend ipsum. Quisque eu sapien quam. Maecenas vitae nibh quis erat posuere sollicitudin. Integer eu purus in nisi tempus lacinia malesuada nec leo. Etiam ac enim a libero faucibus vehicula. Nullam varius, nulla et scelerisque ornare, arcu elit posuere turpis, et volutpat justo ipsum vitae nisl. Integer nibh massa, scelerisque a tincidunt et, elementum sed justo.

Phasellus ut nisi quis massa varius sagittis. Nullam quis orci nec metus volutpat dapibus vel ac est. Donec vulputate, purus ut mollis varius, elit odio porttitor neque, quis iaculis risus lorem at orci. Praesent adipiscing lectus in risus tincidunt hendrerit. Curabitur vel dignissim ligula. Suspendisse neque neque, faucibus ac auctor ut, tristique quis eros. Morbi eu tellus vel velit placerat ullamcorper vel rutrum ipsum. Fusce condimentum lobortis euismod. Curabitur ornare felis vitae tellus rhoncus congue. Etiam quis felis id elit varius dictum vulputate sed sem. Cras fermentum ante sed est suscipit egestas. Mauris non est eu arcu feugiat semper.

Nullam id dui nulla. Maecenas condimentum enim vel odio egestas non aliquam sapien adipiscing. Duis hendrerit faucibus lorem, in ultricies nibh volutpat id. Curabitur eleifend eleifend tempor. Proin quis augue et neque sollicitudin pellentesque et eu est. Phasellus eleifend, quam pulvinar varius blandit, mi justo pharetra turpis, ut rutrum nibh felis vel est. Fusce ac diam sapien, ac ultricies nisi. In facilisis metus sit amet mauris porttitor vel tempor libero laoreet. Nulla orci risus, laoreet mollis tristique a, hendrerit eu erat. Sed vestibulum mattis erat, quis condimentum massa commodo nec. Proin dolor ante, vulputate quis tincidunt a, dignissim nec risus. Curabitur accumsan, quam et congue imperdiet, ligula nisl fermentum nisl, vitae luctus dolor quam a augue. Vestibulum ornare leo id nibh bibendum sed viverra arcu tincidunt. 



Lorem ipsum dolor sit amet, consectetur adipiscing elit. Sed arcu sapien, porta et rutrum at, tincidunt quis quam. Pellentesque quis sapien nec elit tempor semper. Proin consectetur convallis venenatis. Nam in libero nec mi ullamcorper pulvinar. Fusce id augue in tellus semper convallis eu ac massa. Nunc purus elit, luctus non imperdiet sed, cursus dictum ante. Quisque et erat id elit porta suscipit. Aenean volutpat, leo ac commodo consequat, nibh mauris pretium metus, eu ullamcorper nulla nibh at mauris. Donec congue, eros sed dapibus volutpat, nisi magna rhoncus ante, at posuere odio nunc id odio. Phasellus placerat leo et mauris luctus pulvinar sit amet id enim. Quisque scelerisque, neque a viverra facilisis, tortor purus pharetra neque, ac consequat magna tellus non velit. Pellentesque vestibulum libero nisl. Donec congue dictum aliquet. Curabitur dictum tempus vulputate.

Duis volutpat justo accumsan orci sodales pulvinar. Nunc eu tortor quam, ut fringilla odio. Ut eu massa nulla. Proin lorem dui, imperdiet ut gravida non, scelerisque in nibh. Nam tortor quam, faucibus quis congue sit amet, varius id nibh. Mauris ac lacinia neque. Nullam faucibus porttitor elit, eu dictum lacus auctor nec. Nulla eget quam nunc. Etiam vulputate mi nisi. Class aptent taciti sociosqu ad litora torquent per conubia nostra, per inceptos himenaeos.

Ut nec neque libero. Nulla facilisi. Cum sociis natoque penatibus et magnis dis parturient montes, nascetur ridiculus mus. Nunc eros nisl, fermentum sed condimentum quis, fringilla vel tellus. Maecenas cursus mauris eros, sed sollicitudin erat. Duis sit amet egestas eros. Proin condimentum aliquam posuere. Quisque pretium hendrerit urna ut fringilla. Pellentesque eget eleifend ipsum. Quisque eu sapien quam. Maecenas vitae nibh quis erat posuere sollicitudin. Integer eu purus in nisi tempus lacinia malesuada nec leo. Etiam ac enim a libero faucibus vehicula. Nullam varius, nulla et scelerisque ornare, arcu elit posuere turpis, et volutpat justo ipsum vitae nisl. Integer nibh massa, scelerisque a tincidunt et, elementum sed justo.

Phasellus ut nisi quis massa varius sagittis. Nullam quis orci nec metus volutpat dapibus vel ac est. Donec vulputate, purus ut mollis varius, elit odio porttitor neque, quis iaculis risus lorem at orci. Praesent adipiscing lectus in risus tincidunt hendrerit. Curabitur vel dignissim ligula. Suspendisse neque neque, faucibus ac auctor ut, tristique quis eros. Morbi eu tellus vel velit placerat ullamcorper vel rutrum ipsum. Fusce condimentum lobortis euismod. Curabitur ornare felis vitae tellus rhoncus congue. Etiam quis felis id elit varius dictum vulputate sed sem. Cras fermentum ante sed est suscipit egestas. Mauris non est eu arcu feugiat semper.



Nullam id dui nulla. Maecenas condimentum enim vel odio egestas non aliquam sapien adipiscing. Duis hendrerit faucibus lorem, in ultricies nibh volutpat id. Curabitur eleifend eleifend tempor. Proin quis augue et neque sollicitudin pellentesque et eu est. Phasellus eleifend, quam pulvinar varius blandit, mi justo pharetra turpis, ut rutrum nibh felis vel est. Fusce ac diam sapien, ac ultricies nisi. In facilisis metus sit amet mauris porttitor vel tempor libero laoreet. Nulla orci risus, laoreet mollis tristique a, hendrerit eu erat. Sed vestibulum mattis erat, quis condimentum massa commodo nec. Proin dolor ante, vulputate quis tincidunt a, dignissim nec risus. Curabitur accumsan, quam et congue imperdiet, ligula nisl fermentum nisl, vitae luctus dolor quam a augue. Vestibulum ornare leo id nibh bibendum sed viverra arcu tincidunt. 



Lorem ipsum dolor sit amet, consectetur adipiscing elit. Sed arcu sapien, porta et rutrum at, tincidunt quis quam. Pellentesque quis sapien nec elit tempor semper. Proin consectetur convallis venenatis. Nam in libero nec mi ullamcorper pulvinar. Fusce id augue in tellus semper convallis eu ac massa. Nunc purus elit, luctus non imperdiet sed, cursus dictum ante. Quisque et erat id elit porta suscipit. Aenean volutpat, leo ac commodo consequat, nibh mauris pretium metus, eu ullamcorper nulla nibh at mauris. Donec congue, eros sed dapibus volutpat, nisi magna rhoncus ante, at posuere odio nunc id odio. Phasellus placerat leo et mauris luctus pulvinar sit amet id enim. Quisque scelerisque, neque a viverra facilisis, tortor purus pharetra neque, ac consequat magna tellus non velit. Pellentesque vestibulum libero nisl. Donec congue dictum aliquet. Curabitur dictum tempus vulputate.

Duis volutpat justo accumsan orci sodales pulvinar. Nunc eu tortor quam, ut fringilla odio. Ut eu massa nulla. Proin lorem dui, imperdiet ut gravida non, scelerisque in nibh. Nam tortor quam, faucibus quis congue sit amet, varius id nibh. Mauris ac lacinia neque. Nullam faucibus porttitor elit, eu dictum lacus auctor nec. Nulla eget quam nunc. Etiam vulputate mi nisi. Class aptent taciti sociosqu ad litora torquent per conubia nostra, per inceptos himenaeos.
\section{This is the first section of introduction.}
Ut nec neque libero. Nulla facilisi. Cum sociis natoque penatibus et magnis dis parturient montes, nascetur ridiculus mus. Nunc eros nisl, fermentum sed condimentum quis, fringilla vel tellus. Maecenas cursus mauris eros, sed sollicitudin erat. Duis sit amet egestas eros. Proin condimentum aliquam posuere. Quisque pretium hendrerit urna ut fringilla. Pellentesque eget eleifend ipsum. Quisque eu sapien quam. Maecenas vitae nibh quis erat posuere sollicitudin. Integer eu purus in nisi tempus lacinia malesuada nec leo. Etiam ac enim a libero faucibus vehicula. Nullam varius, nulla et scelerisque ornare, arcu elit posuere turpis, et volutpat justo ipsum vitae nisl. Integer nibh massa, scelerisque a tincidunt et, elementum sed justo.

Phasellus ut nisi quis massa varius sagittis. Nullam quis orci nec metus volutpat dapibus vel ac est. Donec vulputate, purus ut mollis varius, elit odio porttitor neque, quis iaculis risus lorem at orci. Praesent adipiscing lectus in risus tincidunt hendrerit. Curabitur vel dignissim ligula. Suspendisse neque neque, faucibus ac auctor ut, tristique quis eros. Morbi eu tellus vel velit placerat ullamcorper vel rutrum ipsum. Fusce condimentum lobortis euismod. Curabitur ornare felis vitae tellus rhoncus congue. Etiam quis felis id elit varius dictum vulputate sed sem. Cras fermentum ante sed est suscipit egestas. Mauris non est eu arcu feugiat semper.

\subsection{This is the first subsection.}
Nullam id dui nulla. Maecenas condimentum enim vel odio egestas non aliquam sapien adipiscing. Duis hendrerit faucibus lorem, in ultricies nibh volutpat id. Curabitur eleifend eleifend tempor. Proin quis augue et neque sollicitudin pellentesque et eu est. Phasellus eleifend, quam pulvinar varius blandit, mi justo pharetra turpis, ut rutrum nibh felis vel est. Fusce ac diam sapien, ac ultricies nisi. In facilisis metus sit amet mauris porttitor vel tempor libero laoreet. Nulla orci risus, laoreet mollis tristique a, hendrerit eu erat. Sed vestibulum mattis erat, quis condimentum massa commodo nec. Proin dolor ante, vulputate quis tincidunt a, dignissim nec risus. Curabitur accumsan, quam et congue imperdiet, ligula nisl fermentum nisl, vitae luctus dolor quam a augue. Vestibulum ornare leo id nibh bibendum sed viverra arcu tincidunt. 

\chapter{Fundamentals}
Some introductory paragraphs on minimization/maximization problems.
\section{Linear programming models}
An introduction to and illustration of LP models.
\subsection{Problem motivation}
\subsection{Graphical explanation}
\section{Integer programming models}
An introduction to and illustration of MIP models
\subsection{Solution technique}
\subsection{Other methods}
\section{Constraint programming models}
\subsection{Concept}
\subsection{Solution}
\subsection{Global constraints}

\chapter{Problem description}
\section{Characteristics of the problem}
\section{Summary of Malapert's approach}
\section{Possible ways of attack}
\section{Test data for evaluation}
Malapert uses benchmark data by Daste. For design purposes, I created my own set of randomized job lists, with $s_j, p_j \in [1, 20]$ and $d_j \in [1, 10n]$ where $n$ is the number of jobs.

CSV files can be read in with \texttt{IloCsvReader} and \texttt{IloCsvLine}.


\chapter{My solution}
\section{MIP formulation improvements}
\section{CP formulation improvements}

\chapter{Discussion}


\pagebreak

\vskip 4em
\section[Intro]{Introduction}
\vspace{6.6em}

Books to read: Model Building in Mathematical Programming by HP Williams

Mathematical programming models all involve optimization. They want to minimize or maximize some objective function.

There are linear programming models, non-linear programming models and integer programming models.

Constraint programming and Integer Programming are twins and can usually be translated into one another. In CP each variable has a finite domain of possible values. Constraints connect/restrict the possible combinations of values which the variables cantake. These constraints are of a richer variety then the linear constraints of IP and are usually expressed in the form of predicates, such as ``all\_different(x1, x2, x3)''. The same condition could be imposed via IP, but it's more troublesome to formulate. Once one of the variables has been set to one of the values in its domain (either temporarily or permanently), this predicate would imply that this value must be taken out of the domain of all other variables (``propagation''). In this way constraint propagation is used until a feasible set of values is found (or not).

CP is useful where a problem function has no objective function and we are just looking for a feasible solution. We can't prove optimality, although we could using IP.

Comparisons and connectionsbetween IP and CLP are discussed by Barth (1995), Bockmayr and Kasper (1998), Brailsford
et al.
(1996), Prolland Smith (1998), Darby-Dowman and Little (1998), Hooker (1998) and Wilson and Williams (1998). Theformulation of the all\_different predicate using IP is discussed by Williams and Yan (1999).

The first approach to solving a multi-objective problem is to solve the model a number of times with each objective function in turn. The result may give an idea of what to do next.


\subsection{Chapter 8: Integer Programming}

The real power of IP as a method of modelling is that of binary constraints, where variables can take on values of 0 or 1. Maybe it should be called ``discrete programming.'' Precise definitions of those problems that can be formulated by IP models are given by Meyer (1975) and Jeroslow (1987). 

Problems with discrete inputs and outputs are the most obvious candidates for IP modelling (``lumpy inputs and outputs''). Sometimes solving an LP and rounding to the nearest integer works well, but sometimes it doesn't, as demonstrated in Williams, first example in 8.2. The smaller the variables (say, <5), the greater significance those rounding problems will have. 

When input variables have small domains, say, machine capacities, then again this may rule out LP relaxations. A good example of an IP problem is the knapsack assignment problem. A single constraint is that the capacity of the knapsack cannot be exceeded. Any LP formulation would always fill the knapsack 100\%, ignoring the discrete size of the objects. Two other well-known types of problems include \textit{set partitioning problems} and \textit{aircrew scheduling problems}. 

Integer programming models cannot be solved directly, but need to be brute-forced in a tree search manner. The search space, however, can be greatly reduced by a number of bounding techniques often depending on the nature of the problem. That is why the general method of IP solving is called \textit{branch-and-bound}.

So-called \textit{cutting planes methods} usually start by solving an IP problem as LP. If the resulting solution is integral, we're happy. Otherwise extra constraints (cutting planes) are added to the problem, further constraining it until an integer solution is found (or, if none can be found, we're out of luck). Cutting planes make for nice illustrations they are not very efficient with large problems. Cutting planes were invented by Gomory (1958).

Enumerative methods are generally applied to pure binary problems, where the search tree is pruned. The best known of these methods is Balas's additive algorithm described by Balas (1965). Other methods are givenby Geoffrion (1969). A good overall exposition is given in Chapter 4 of Garfinkel and Nemhauser (1972).

There are so-called \textit{pseudo-boolean methods} to solve pure binary problems that take boolean constraints as inputs. That may be comfortable for the user sometimes, but is rarely used in practice.

Generally, branch-and-bound methods first solve the LP relaxation to check whether we're lucky enough to find an integer solution. If not, a tree search is performed. 

\subsection{Chapter 9: Building IP models I.}

Binary variables are often called 01-variables. Decision variables could also have a domain like $\{0,1,2\}$ or $\{4,12,102\}$. Decision variables, especially the 01 kind, can be linked to the state of continuous variables like this: $x-M\delta \leq 0 \leftrightarrow x>0 \rightarrow \delta = 1$, where we know that $x < M$ is always true.

To use a 01 variable to indicate whether the following is satisfied:
\begin{align}
2x_1 + 3x_2 &\leq 1\\
x_1 &\leq 1\\
x_2 &\leq 1,
\end{align}
so, mathematically speaking,
\begin{align}
\delta = 1 &\rightarrow 2x_1 + 3x_2 \leq 1\\
2x_1 + 3x_2 \leq 1 &\rightarrow \delta = 1,
\end{align}
Then we can argue that at most, $2x_1 + 3x_2 = 5$, so
\[
2x_1 + 3x_2 + 4\delta \leq 5
\]
will ensure that $\delta = 1$ forces the equation to be true: use $M = 2 + 3 - 1$ to find $4$. In order to enforce $\delta = 1$ if the equation is true, use $m = 0 + 0 -1$ and write
\[
2x_1 + 3x_2 + \delta \geq 1
\]

All kinds of logical conditions can be modelled using 01 variables, although it's not always obvious how to capture them in that format. 

Logical conditions are sometimes expressed within a Constraint Logic Programming language as discussed inSection 2.4. The tightest way of expressing certain examples using linear programming constraints is described byHooker and Yan (1999) and Williams and Yan (1999). There is a close relationship between logic and 01 integerprogramming, which is explained in Williams (1995), Williams and Brailsford (1999) and Chandru and Hooker(1999).

\paragraph*{Special Ordered Sets}
Two very common types of restriction arise in mathematical programming, so two concepts (SOS1 and SOS2) have been developed. An SOS1 is a set of variables within which exactly one variable must be non-zero and the rest zero. An SOS2 is a set where at most two can be non-zero, and the two variables must be adjacent in the input ordering. Using a branch-and-bound algorithm specialized for SOS1 or SOS2 sets can speed things up greatly.

\paragraph*{Disjunctive constraints}
It is possible to define a set of constraints and postulate that at least one of them be satisfied.

\subsection{Special kinds of IP models}
Most practical IP models do not all into any of these categories but arise as MIP models often extending an existing LP model. Here are some examples.

\paragraph*{Set covering problems} We have a set $S = \{1,2,3,\dots,m\}$. We have a bunch of subsets $\mathcal{S}$ of subsets, each associated with a cost. Now cover all members of $S$ using the least-cost members of $\mathcal{S}$.

\paragraph*{Knapsack problem} These are the really simple ones with just one constraint, namely the constraint of not being able to take more items with you than the knapsack can carry while maximizing the value of the taken objects.

\paragraph*{Quadratic assignment problem} Two sets of objects $S$ and $T$ of the same size require the objects to be matched pairwise. There are costs associated with pairs of pairs, that is the cost $c_{ij,kl}$ is the cost of assigning $i$ to $j$ while also assigning $k$ to $l$. This cost will be incurred if both $\delta_{ij}$ and $\delta_{kl}$ are \texttt{true}, i.e. $\delta_{ij}\delta_{kl} = 1$. The objective function is a quadratic expression in 01 variables:
\begin{align}
\mathrm{Minimize}\;\sum^n_{\substack{i,j,k,l=1\\k>l}} c_{ij,kl}\delta_{ij}\delta_{kl}
\end{align}
The quadratic version is practically unsolveable, so to be able to enumerate the possible assignments the above objective function has to be split up into separate constraints, which obviously means there will be an explosion in problem size as the number of variables grow.

\subsection{How to formulate a good model}
According to Williams, it is easy to build IP models that, while correct, are really inefficient. Fortunately, with some knowledge of what happens behind the scenes (and some practice), sucky models can often be improved. One good method to know (although it doesn't help by itself) is to turn a general integer variable into a bunch of 01 variables. Say $\gamma$ is a general, non-negative integer variable with a known upper bound of $u$, then we can replace it with $\delta_0 + 2\delta_1 + 4\delta_2 + 8\delta_3 + \dots + 2^n\delta_n$. 

\paragraph*{Example problem} 
http://www.scribd.com/doc/49547850/Model-Building-in-Mathematical-Programming

\pagebreak
\section{How this could be solved}
We're trying to create an mixed integer linear programming model (MILP). In the original paper, they had one non-linear constraint:
\begin{align}
  (d_{max}-d_j)(1-x_{jk}) \geq D_k-d_j \;\;\;\forall j \in J, \forall k \in K
\end{align}
This can easily be turned into
\begin{align}
  x_{jk}d_j-D_k &\geq 0\;\;\;\forall j \in J, \forall k \in K\\
  d_{max}-D_k &\geq 0\;\;\;\forall k \in K
\end{align}
\subsection{Possible improvements}
We cannot modify the solution process itself -- that's the whole point of the exercise, after all. Maybe we can preprocess to limit the domains of some decision variables. This is where ``preordering'' may come into play.
\paragraph{Limiting batch due date}
Since we start with as many batches as jobs, the solver is free to put every job into a batch of the same number. To make things more efficient, it can also put jobs into batches before, but it would never make sense to put a job into a batch with a higher number, so we have a limit that
\begin{align}
  x_{jk} = 0 \;\;\; \forall j,k: j > k
\end{align}


\end{document}

