\documentclass[12pt, landscape]{article}
\usepackage{graphicx}
\usepackage[driver=xetex,paperwidth=8.5in,paperheight=11in,left=1.1in, right=1.3in,top=1.0in, bottom=1.2in]{geometry}
\usepackage[no-math]{fontspec}

\usepackage{sectsty}
\usepackage{multicol,scalefnt}
\usepackage{amsmath, amssymb, amsfonts,  titlesec}
\usepackage[urw-garamond]{mathdesign}
\usepackage{fancyhdr,  booktabs, multirow}
\usepackage[font=small,format=plain,labelfont=it,up,textfont=it,up]{caption}
\usepackage{listings}

\usepackage{enumitem}

\newlist{alist}{itemize}{1}
\setlist[alist]{label=--,labelindent=2in,leftmargin=9pt,labelsep=6pt, itemsep=-4pt}

%========= FONT SPECS ============
\tolerance 8000

\defaultfontfeatures{Mapping=tex-text, Ligatures=Common}

\renewcommand\refname{References} % this sets the name of
\def\labelitemi{--}

\def\sansfont{\fontspec[Script=Latin,LetterSpace=2.6, Mapping=tex-text]{DIN 1451 Mittelschrift}}
\def\sansitalicfont{\fontspec[Script=Latin,LetterSpace=2.6, FakeBold=1.5, FakeSlant=0.2, Mapping=tex-text]{DIN 1451 Mittelschrift}}

\def\monofont{\fontspec[Script=Latin,Mapping=tex-text,Scale=0.91]{Inconsolata}}

\renewcommand{\texttt}[1]{{\monofont #1}}

\lstset{
aboveskip=2\medskipamount, belowskip=2\medskipamount,
basicstyle=\monofont,
language=python,
numbers=left, numberstyle=\tiny,  numbersep=9pt,
xleftmargin=.4in, frame=l, xrightmargin=.25in
}

%\titleformat{\section}{\huge\sansnormalfont}{\protect\makebox[0pt][r]{\thesection\quad}}{0em}{}
\titleformat{\section}{\fontsize{18pt}{21pt}\selectfont\sansfont}{}{0em}{}
\titleformat{\subsection}{\fontsize{13pt}{19pt}\selectfont\sansfont}{\protect\makebox[0pt][r]{\quad}}{0em}{}


\fancyhead[LO, LE]{\sansfont\small ESC499 Proposal -- Sebastian Kosch \normalsize}
\fancyhead[RO, RE]{\sansfont\small \nouppercase\leftmark}
\fancyfoot[C]{\sansfont\thepage}

\begin{document}
\fontsize{12pt}{16pt}\selectfont
\thispagestyle{empty}
\pagestyle{fancy}

\baselineskip=16.8pt plus 0pt
\frenchspacing
\noindent{\bfseries ESC499 Proposal -- Sebastian Kosch, 997 241 024 -- Supervisor: Chris Beck (MIE)}
\section*{\Large\sansfont An improved CP/MIP formulation\\for batch processing of non-identical jobs}
\vskip 1.2em

Efficiency in manufacturing often boils down to managing limited resources in clever ways in order to perform the given jobs as quickly as possible. One such type of problem, 1|\textit{p-batch; n$<$b; non-identical}|$L_{\text{max}}$, is concerned with jobs of non-identical sizes, processing times and due dates that are to be assigned to a resource (e.g. an oven) in batches. The following constraints then apply: batch sizes must not exceed the resource's capacity; each batch's processing time equals that of the greatest of all the processing times of its jobs; and each batch's lateness $L$ is the difference between its completion time and the the earliest of all the due dates of its jobs.

The objective is to batch jobs together such as to minimize the greatest individual batch lateness $L_{\text{max}}$; the problem is NP-hard as shown in \cite{bruckner}.

\subsection{Research goal}
This problem was recently examined by Malapert \cite{Malapert}, who suggested a global constraint approach implemented as three filtering rules applied each time a job is assigned to a batch. This technique was compared with a simple Mixed Integer Programming (MIP) model to show its superior performance.

However, the MIP model given in \cite{Malapert} can likely be improved. Using \cite{cplex}, I will first re-model the given MIP formulation and compare it to an equivalent Constraint Programming (CP) formulation. I will then experiment with additional constraints in both CP and MIP to speed up the search for feasible batch assignments, particularly bounds on $L_{\text{max}}$ and $x_{jk}$. Such constraints will be taken from considerations discussed in prior conversations, existing literature, and potentially from examinations of how the solver handles the given models. The goal is to tighten the CP/MIP models until their performance match or exceed that of the global constraint implementation in \cite{Malapert}, if possible.

\subsection{Existing literature}
An extensive overview over CP/MIP techniques in scheduling is given in \cite{grossmann}. General literature reviews are provided in \cite{floudas1} and \cite{floudas2}. Keha et al. \cite{keha} discuss various MIP formulations for single-machine scheduling problems. The same problem as in this paper, but with a different objective was examined in \cite{azizoglu}, and optimality criteria and a heuristic are proposed in \cite{sabouni}. 

\pagebreak
\begin{thebibliography}{5}
\bibitem{bruckner} Brucker, P., Gladky, A., Hoogeveen, H., et al.: Scheduling a batching machine. Journal of Scheduling 1, 31–54 (1998)
\bibitem{Malapert} Malapert, A., Guéret C., Rousseau L.M.: A constraint programming approach for a batch processing problem with non-identical job sizes. European Journal of Operational Research 221, 533-545 (2012)
\bibitem{grossmann} Grossmann, I.E.: Mixed-integer optimization techniques for the design and scheduling of batch processes. In: CMU Engineering Design Research Center -- Dept. of Chem. Engineering Paper 203 (1992)
\bibitem{floudas1} Floudas, C.A., Lin, X.: Continuous-time versus discrete-time approaches for scheduling of chemical processes: a review. Computers and Chemical Engineering 28, 2109-2129 (2004)
\bibitem{floudas2} Floudas, C.A., Lin, X.: Mixed Integer Linear Programming in Process Scheduling: Modeling, Algorithms, and Applications. Annals of Operations Research 139, 131-162 (2005)
\bibitem{keha} Keha, A.B., Khowala, K., Fowler, J.W.: Mixed integer programming formulations for single machine scheduling problems. Computers \& Industrial Engineering 56, 357-367 (2009)
\bibitem{cplex} IBM Corp.: Cplex Optimizer and CP-Optimizer. Software package, <http://www-01.ibm.com/software/integration/optimization/cplex-optimizer/> (2011)
\bibitem{azizoglu} Azizoglu, M., Webster, S.: Scheduling a batch processing machine with non-identical job sizes. International Journal of Production Research 38/10, 2173-2184 (2000)
\bibitem{sabouni} Sabouni, M.T.Y., Jolai, F.: Optimal methods for batch processing problem with makespan and maximum lateness objectives. Applied Mathematical Modelling, 34, 314-324 (2010)


\end{thebibliography}
 

\end{document}

