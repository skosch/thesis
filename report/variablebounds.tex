\section{Bounds on variables known a priori}

\subsection{Lower bound on {\sansitalicfont L}\textsubscript{max}}
A lower bound on $\Lmax$ can be found using the lower bound on the completion
date of each bucket $q$, where a bucket is defined as the set of batches with
due date $\delta_q$:
\begin{alignat}{2}
& \Lmax \geq C_{\text{max},q} - \delta_q \quad && \forall q
\end{alignat}
This works because the buckets up to bucket $q$ are guaranteed to contain all
jobs with due dates $d \leq \delta_q$, since the batches within the buckets are
ordered by earliest due date (EDD) in the optimal solution. The buckets up to
bucket $q$ will likely also contain some later ($d > \delta_q$) jobs in the
optimal solution but this does not affect the validity of the lower bound.

%To find a lower bound on $C_{\text{max},q}$, we can exploit the fact that every
%batch needs to span its jobs. If we order the jobs up to bucket $q$ by non-increasing processing
%time $p$, then we can use algorithm \ref{alg:findcmax} to find
%$C_{\text{max},q}$:
%\begin{algorithm}
%\begin{algorithmic}
%\State $J^{\star} \gets J$ \Comment{initialize all jobs as unassigned jobs}
%\State $n_k \gets 1$; $S_k \gets \{0\}$; $P_k \gets \{0\}$ \Comment{Create one
%empty batch of size and length zero}
%\State sort $J^{\star}$ by processing time, non-increasing
%\Repeat
%  \State $j \gets J^{\star}$.pop() \Comment{select job for assignment, longest job
%first}
%  \Loop $\;$ through all $n_k$ existing batches $k$, first batch first
%    \State $k_p \gets \emptyset$ \Comment{no feasible batch}
%    \State $c_\text{min} = b$ \Comment{currently known minimum remaining
%    capacity}
%    \If{$s_j < b-S_k$ and $b-S_k < c_\text{min}$}
%      \State $k_p \gets k_p$; $c_\text{min} \gets b-S_k$
%    \EndIf
%  \EndLoop
%  \If{$|k_p| = 1$}
%      \State $S_{k_p} \gets S_{k_p} + s_j$ \Comment{assign job $j$ to batch $k_p$}
%  \Else
%    \State $n_k \gets n_k + 1$\Comment{open new batch}
%    \State $S_{n_k} \gets s_j$; $P_{n_k} \gets p_j$ \Comment{assign $s_j$ and $p_j$ to the new batch}
%  \EndIf
%\Until{$J^{\star}$ is empty}
%\end{algorithmic}
%\caption{Finding $C_{\text{max},q}$}
%\label{alg:findcmax}
%\end{algorithm}
%
%The algorithm is based on the following reasoning: no matter the assignment, one
%batch will have the length $P_k$ of the longest job $\max(p_j)$.
%Since we ordered and assigned the jobs by $p_j$, this will be the first batch
%$k_1$.
%Continue filling $k_1$ until some job $j_c$ exceeds capacity. This job
%must go into another batch $k_2$. Swapping $j_l$ with any of the already assigned jobs
%in $k_1$ (capacity allowing) increases $C_\text{max}$. Even if ``leaving
%capacity'' in $k_1$ for later, better-fitting open jobs may seem desirable --
%these other open jobs will necessarily be shorter than $j_c$, so such a move
%will again serve to increase $C_\text{max}$. Open jobs that fit into
%previous batches with some remaining capacity should be assigned to the batch
%with the \textit{minimum} feasible remaining capacity $c_\text{min}$. This
%avoids unnecessarily precluding shorter (still open) but slightly larger jobs
%from being assigned to previous batches wherever possible. 
%
%A combination of shorter jobs may be a better fit to a previous batch than a
%single, longer job in terms of using remaining capacity, but what matters is
%only the processing time.
%

Now we need to find $C_{\text{max},q}$, or at least a lower bound on it, in
polynomial time. The simplest approach simply considers the ``total elastic area'',
i.e. the sum of all $s_j p_j$ products:
\begin{alignat}{2}
& C_{\text{max},q} \geq \big\lceil\frac{1}{b} \sum_{j : d_j \leq \delta_q} s_j
p_j\big\rceil \quad
&& \forall q
\end{alignat}
A better lower bound on $C_{\text{max},q}$ would be given by a
preemptive-cumulative schedule. Unfortunately, minimizing $C_{\text{max}}$ for
such problems is equivalent to solving a standard bin-packing problem, which
requires exponential time. 

\subsection{Upper bound on {\sansitalicfont L}\textsubscript{max}}
An upper bound on $\Lmax$ can be found by using a dispatch rule to find a
feasible, if not optimal, schedule. A good approach could be the ``best-fit''
heuristic proposed in the original paper. {\color{darkred} This has not been
implemented yet.}

\subsection{Bounding the number of batches \sansitalicfont n\textsuperscript{k}}
Initially, the number of batches needed is assumed to be equal to the number of jobs: $n_k = n_j$. Reducing $n_k$ by pre-computing the maximum number of batches needed shrinks the $x_{jk}$ matrix.

Unfortunately, we cannot make a general statement that optimal solutions never have more batches than other feasible solutions -- a simple counterexample is shown in figure \ref{fig:bnk1}.\footnote{To be more precise, we cannot state that at least one optimal solution is in the subset of feasible solutions that uses the fewest number of batches -- a dominance situation that could be exploited, were it true.}

. . .

. . .
\vfil

\begin{figure}
  \centering
  \begin{subfigure}[b]{0.4\textwidth}
    \centering
    \begin{tikzpicture}[scale=0.2, font=\scriptsize]

      \draw [<->,thick] (0,12) node (yaxis) [above] {\sansitalicfont s}
        |- (25,0) node (xaxis) [right] {\sansitalicfont t};
      \draw[dotted] (0,10) -- (25,10);
        \draw (0,0) rectangle (5,7) node[fn] {$p = 10$\\$d = 10$\\$L = 0$};
        \draw (5,0) rectangle (20, 3) node[fn] {$p = 30, d=20, L=10$};
        
    \end{tikzpicture}
  \end{subfigure}
  \begin{subfigure}[b]{0.4\textwidth}
    \centering
    \begin{tikzpicture}[scale=0.2, font=\scriptsize]

      \draw [<->,thick] (0,12) node (yaxis) [above] {\sansitalicfont s}
        |- (25,0) node (xaxis) [right] {\sansitalicfont t};
      \draw[dotted] (0,10) -- (25,10);
        \draw (0,0) rectangle (5,7) node[fn] {$p = 10$\\$d = 10$\\$L = 20$};
        \draw (0,7) rectangle (15, 10) node[fn] {$p = 30, d=20, L=0$};
    
    \end{tikzpicture}
  \end{subfigure}
\caption{Overzealous batch elimination can increase $\Lmax$}\label{fig:bnk1}
\end{figure}

{\color{darkred} I have a long list of ideas here, none of which I've been able
to prove right or wrong, so I've commented them out and made this page end here.}

%It is perhaps possible, however, to generate a feasible solution that is likely to use $n_k < n_j$ and to guarantee that such a solution will never use fewer batches than the optimal solution. To do this, let the initial solution $\pi_0$ be a schedule in which every job is assigned to one batch (i.e. $n_k = n_j$) and the jobs are ordered by non-decreasing due date (i.e. $d_n \leq d_{n+1}$), as in figure \ref{fig:bnk2}.
%
%\input{figure showing all the jobs in EDD sequence}
%
%Capacity permitting, jobs are now moved into earlier batches to improve the schedule, eliminating the batch they were placed in initially. Every such move reduces $n_k$ by one.
%
%Two types of moves are possible: \textit{safe} moves and \textit{risky} moves. A move is safe when a job is moved into an earlier batch of longer processing time. A move is risky when a job is moved into an earlier batch of shorter processing time, thus increasing the lateness of that batch.
%
%Moving a job has three effects: 
%
%\begin{alist}
%\item{Effect on the batches after}
%\item{Effect on the batches between, including the new job}
%\item{Effect on the job itself}
%\end{alist}
%
%\begin{table}
%\centering
%\begin{tabular}{r c c}
%\toprule
%              & safe                 & risky \\
%\midrule
%batches after & lateness improves by $p_j$ & lateness improves by $p_a$ \\
%batches between & --- & lateness worsens by $p_j - p_a$ \\
%job itself & lateness improves & lateness improves\\ 
%\bottomrule
%\end{tabular}
%\caption{Lateness effects of safe and risky moves}
%\end{table}
%
%We can easily generate a solution that only contains safe moves.
%We now need to prove that our relaxed solution $\pi_{\text{edd}}$ will never use fewer batches than the optimal solution $\pi_{\text{opt}}$.
%
%For this to be true, we need to show that if there is ever a situation in which we could \textit{either} make $n$ unsafe moves \textit{or} $m>n$ safe ones, using the unsafe ones will never be beneficial if the remaining $m-n$ safe candidates cannot be moved somewhere else.
%
%We're trying to find a relaxation of the problem that is guaranteed to generate as least as many batches as the optimal solution.
%
%
%Whenever there is an alternative between $n$ unsafe moves and $>n$ safe ones, perform $n$ safe ones. Whenever only an unsafe is available, eliminate none and proceed to the next batch.
%
%Unfortunately, we cannot generally state that optimal solutions never have more batches than other feasible solutions: when a job $j_b$ is moved into a prior batch holding $j_a$ and $d_b > d_a$ as well as $p_a < p_b$, it can sometimes happen that $\Lmax$ is increased.
%
%\subsubsection{Some stuff to try}
%
%We can try only considering jobs \textit{before} $\Lmax$, because anything after that job wouldn't be moved in an optimal solution. 
%
%

